\documentclass{article}
\usepackage[a4paper,left=3cm,right=3cm,top=1cm,bottom=2cm]{geometry}
\usepackage{amsmath}
\usepackage{amssymb}
\usepackage{hyperref}

\setlength{\parindent}{0mm}

\usepackage{fontspec}
\setmainfont{Linux Libertine O}
\usepackage{unicode-math}
\setmathfont{Cambria Math}

\title{
\textit{\small{Георгий Потошин, 2024}}\\
\vspace{0.3ex}
\textit{\huge{Топология I, листочек 1}}\vspace{1ex}
}

\date{\vspace{-10ex}}

\begin{document}
\maketitle

\begin{enumerate}
    \item \textbf{Докажите, что постоянное отображение непрерывно.} 
        Пусть $X$ и $Y$ суть два топологических пространства и стрелка
        $\varphi: X \longrightarrow Y$ постоянна и образа $X$ состоит из
        одного $y_0$. Возьмём открытое множество $U$ в $Y$. Если оно содержит
        $y_0$, то его прообраз всё X, и стало быть прообраз открыт. Если оно не
        содержит $y_0$, то его прообраз пуст, а значит открыт. Тогда прообраз
        любого открытого множества открыт и стрелка $\varphi$ непрерывна.

    \item \textbf{Опишите все топологии на множествах, состоящим из не более
        чем трех элементов.}\par
        Описание с точностью до изоморфизма:\par
        \textbf{0 элементов:} $(\varnothing, \{\varnothing\})$\par
        \textbf{1 элемент:} Пусть $X = \{a\}$, тогда $(X, \{\varnothing, X\})$
        \par
        \textbf{2 элемента:} Пусть $X = \{a, b\}$
        \begin{itemize}
            \item антидискретная топология
            \item $(X, \{\varnothing, X, \{a\}\})$
            \item дискретная топология
        \end{itemize}\par
        \textbf{3 элемента:} Пусть $X = \{a, b, c\}$
        \par Мы начнем с топологий, где нет 1-элементных открытых множеств.
        Таких всего два типа, либо топология антидискретна, либо мы добавляем
        одно 2-элементное множество. Ещё одно мы добавить не сможем, так как
        их 1-элемнтное пересечение тоже будет открытым.
        \begin{itemize}
            \item антидискретная топология
            \item $(X, \{\varnothing, X, \{a, b\}\})$
        \end{itemize}
        Теперь пусть 1-элементное открытое множество одно. Тогда количество
        2-элементных сможет варьировать от 0 до 2. Если их 3, то выйдет
        дискретная топология. Если существует всего одно 2-элементное множество,
        то оно либо является надмножеством 1-элементного, либо нет. Если же
        есть два 2-элементных множества, то они обязаны быть надмножествами
        1-элементного, иначе же было бы два 1-элементных множеств.
        \begin{itemize}
            \item $(X, \{\varnothing, X, \{a\}\})$
            \item $(X, \{\varnothing, X, \{a\}, \{a, b\}\})$
            \item $(X, \{\varnothing, X, \{a\}, \{b, c\}\})$
            \item $(X, \{\varnothing, X, \{a\}, \{a, b\}, \{a, c\}\})$
        \end{itemize}
        Теперь у нас будет два 1-элементных открытых множества, а значит, что
        их объединение обязано быть открытым. Поэтому 2-элементных множеств
        либо одно, либо два.
        \begin{itemize}
            \item $(X, \{\varnothing, X, \{a\}, \{b\}, \{a, b\}\})$
            \item $(X, \{\varnothing, X, \{a\}, \{b\}, \{a, b\}, \{a, c\}\})$
        \end{itemize}
        \begin{itemize}
            \item дискретная топология
        \end{itemize}

    \item \textbf{Докажите, что вещественное пространство $\mathbb{R}^n$ c
        естественной топологией обладает счетной базой.} Возьмём $\mathcal{T}_0
        =\{\mathcal{B}_x^\varepsilon\;|\;x\in\mathbb{Q}^n\text{ и }\varepsilon
        \in\mathbb{Q}\}$. Мощность $\#\mathcal{T}_0 = \#\mathbb{Q}^{n+1}$, так
        как $\mathcal{T}_0$ естественно биективно отбражается в $\mathbb{Q}^n$ по
        средством $\mathcal{B}_x^\varepsilon\mapsto(x,\varepsilon)\in\mathbb{Q}
        ^n\times\mathbb{Q}\mapsto(x_1, ..., x_n, \varepsilon)\in\mathbb{Q}^{n+1}$,
        а значит $\mathcal{T}_0$ - счетно. Теперь пусть $U\subset\mathbb{R}^n$
        тогда U вместе с каждой своей $x$ точкой содержит некий шар с центром в
        ней, по плотности $\mathbb{Q}$ в $\mathbb{R}$ можно выбрать подшар с
        рациональным радиусом, содержащий точку $x$ и имеющий центр в
        рациональной точке. Каждый такой шар будет лежать в $\mathcal{T}_0$ и
        пересечение по всем ним составит $U$. Это значит, что $\mathcal{T}_0$ -
        это счетная база канонического топологического пространства
        $\mathbb{R}^n$.
    \item \textbf{Опишите все базы для топологического пространства X с
        дискретной топологией. Антидискретной топологией.} Пусть $X$ снабжено
        дискретной топологией и $\mathcal{B}$ является базой этой топологии без
        пустого множества. Тогда для всякого $x\in X$ существует подмножество
        $F\subset\mathcal{B}$, что объединение по нему равно $\{x\}$. Тогда
        $U\in F \Rightarrow U\subseteq\{x\}$ и так как $U$ по предположению не
        пусто, то $U = \{x\}$. Что означает вложенность $F_0=\{\{x\}\;|\;x\in X
        \}$ в $B$. Причем стоит отметить, что $F_0$ само по себе является базой
        топологии. А значит, что любая база образована объединением $F_0$ и
        некоторого подмножества $S$.\par
        Пусть теперь $X$ снабжено антидискретной топологией. Тогда баз будет
        всего две $\{\varnothing, X\}$ и $\{X\}$, так как два оставшихся
        подмножества $\mathcal{T}$ не будут базами.
    \item \textbf{Пусть $\mathbb{R}$ – вещественная прямая, $X$ –
        топологическое пространство с дискретной топологией, $X'$ –
        топологическое пространство с антидискретной топологией. Опишите
        множества отображений:}\par
        \textbf{из $\mathbb{R}$ в $X$:} Пусть $f\in\text{Mor}(\mathbb{R}, X)$ и
        его образ состоит более чем из одной точки. Пусть $x\in\text{im}(f)$, 
        тогда $f^{-1}(x)$ и $f^{-1}[X\backslash\{x\}]$ оба открыты и не пусты и
        $f^{-1}(x)\sqcup f^{-1}[X\backslash\{x\}]=\mathbb{R}$. Тогда $f^{-1}(x)$
        и $f^{-1}[X\backslash\{x\}]$ оба замкнуты, так как являются дополнениями
        к открытым. Возьмем две точки, одна из которых будет лежать в одном
        замкнутом множестве, а вторая в другом. Поделив отрезок с концами в
        этих точках пополам, мы получим точку, что лежит в одном из замкнутых,
        а значит отрезок можно уменьшить вдвое. Если рассмотреть
        последовательность точек каждого из замкнутых множеств, то мы получим
        две фундаментальные последовательности, что сходятся в $\mathbb{R}$ к
        одой и той же точке. Значит она принадлежит каждому из замкнутых, а
        значит их объединение не дизъюнктивно. А значит образ $f$ состоит из
        одной точке, отображение постоянно и непрерывно.\par 
        \textbf{из $X$ в $\mathbb{R}$:} Пусть $f \in \text{Map}(X, \mathbb{R})$.
        Тогда для любого открытого подмножества $\mathbb{R}$ его прообраз будет
        подмножества $X$ с дискретной топологией, а значит будет открытым.
        Тогда $\text{Mor}(X, \mathbb{R})=\text{Map}(X, \mathbb{R})$.\par
        \textbf{из $X'$ в $\mathbb{R}$:} Пусть $f\in\text{Mor}(X',
        \mathbb{R})$, тогда прообраз каждого замкнутого множества $\mathbb{R}$
        либо $X$, либо $\varnothing$, так как только они замкнуты в дискретном
        топологическом пространстве. Пусть $x\in\mathbb{R}$, тогда либо в $x$
        не идет ни одной стрелки, либо если есть хотя бы одна стрелка, и так как
        $\{x\}$ – замкнуто в $\mathbb{R}$, то его прообраз – непустое замкнутое
        множество, а значит весь $X$. При чем существует хотя бы одна стрелка,
        а значит все отображения постоянны и непрерывны.\par
        \textbf{из $\mathbb{R}$ в $X'$:} Пусть $f\in\text{Map}(\mathbb{R}, X')$.
        Тогда $f^{-1}[X'] = \mathbb{R}$ и $f^{-1}[\varnothing]=\varnothing$
        прообразы всех открытых множеств открыты, а значит $\text{Map}(
        \mathbb{R}, X')=\text{Mor}(\mathbb{R}, X')$.
    \item \textbf{Назовем отображение топологических пространств открытым,
        если образ любого открытого множества открыт. Назовем отображение
        топологических пространств замкнутым, если образ любого
        замкнутого множества замкнут. Приведите примеры непрерывного
        отображения топологических пространств, не являющегося открытым;
        замкнутым.} Пусть $(X = \{a,b,c\}, T = \{\varnothing, \{a\}, X\})$ –
        топологическое пространство, тогда $f:x\in X\mapsto c$ - непрерывное 
        постоянное отображение, что не переводит открытое $\{a\}$ в открытое
        множество. Если под $T$ понимать все замкнутые множества, что имеет
        смысл для конечного $X$, то это же отображение не будет переводить
        замкнутые в замкнутые.
    \item \textbf{Пусть $Y$ – подмножество топологического пространства $X$.
        Замыканием $\overline{Y}$ множества $Y$ в $X$ назовем пересечение всех
        замкнутых множеств, содержащих $Y$. Докажите, что $Y$ замкнуто.}
        Множество замкнутых множеств замкнуто относительно операции
        пересечения, а значит замыкание замкнуто.
    \item \textbf{Пусть $Y$ – подмножество топологического пространства $X$.
        Будем говорить, что $Y$ является всюду плотным в $X$, если
        пересечение $Y$ с любым открытым подмножеством $X$ непусто. Докажите,
        что $Y$ всюду плотно в $X$ тогда и только тогда, когда его замыкание
        совпадает с $X$.}\par
        $\Rightarrow:$ Пусть $Y$ всюду плотно в $X$. Заметим, что $\overline{Y}
        \subseteq X$, так как это подмножество топологического пространства.
        Напротив, пусть $x\in X$ и надмножество $Z\supseteq Y$ замкнуто, тогда
        любое открытое множество, содержащее $x$, имеет непустое пересечение с
        $Z$, и если бы $x$ принадлежало открытому дополнению $Z$, то тогдa
        пересечение $Z$ со своим дополнение было бы непусто, что ведет к
        противоречию, а значит наше предположение о том, что $x\in Z^c$ не
        верно. Значит $x\in Z$ содержится в любом замкнутом надмножестве $Y$, а
        значит что и пересечение по всем таким множествам будет содержать $x$.
        А это значит, что $\overline{Y}=X$.\par
        $\Leftarrow:$ Пусть $\overline{Y}=X$ и пусть $U\subseteq X$ открытое
        множество. Если $U\cap Y=\varnothing$, то значит $Y$ лежит в замкнутом
        $X\backslash U$, а значит $\overline{Y} \subseteq X\backslash U$. Тогда
        мы получим $X\subseteq X\backslash U\subseteq X$, что эквивалентно
        $U=\varnothing$. Значит Y всюду плотно в X.
    \item \textbf{Опишите все всюду плотные подмножества в пространстве с
        дискретной топологией; с антидискретной топологией.}\par
        \textbf{Дискретная топология:} Заметим, что любое множество в таком
        пространстве является своим же замыканием. Тогда единственное всюду
        плотное множество это только всё пространство.\par
        \textbf{Антидискретная топология:} Замыкание пустого множества пусто, а
        значит оно не всюду плотно, для непустого пространства. Если множество
        не пусто, то его замыканием является все пространство и оно всюду плотно.
    \item \textbf{Докажите, что множество рациональных чисел $\mathbb{Q}$ всюду
        плотно в $\mathbb{R}$.} В евклидовом пространстве множество открыто,
        если с каждой своей точкой содержит некий интервал с ней, что в
        частности означает, что непустое открытое множество содержит интервал.
        Тогда достаточно показать, что пересечение любого непустого интервала с
        $\mathbb{Q}$ непусто. Пусть $(a, b)\subset\mathbb{R}$ и $\varepsilon>0$
        его длина. Подмножество $\{n\in\mathbb{N}|n>\varepsilon\}$ имеет
        минимальный элемент $k$. По архимедовости поля $\mathbb{R}$, есть
        натуральное число $p$, что $p\varepsilon \leqslant k < (p+1)
        \varepsilon$, это значит, что $0 < k/(p+1) < \varepsilon$. Опять же по
        архимедовости есть целое число $l$, что $lk/(p+1) \leqslant a <
        (l+1)k/(p+1) < a + \varepsilon = b$. А значит, что в любом интервале
        найдется рациональное число и множество $\mathbb{Q}$ всюду плотно в
        $\mathbb{R}$.
\end{enumerate}

\end{document}
