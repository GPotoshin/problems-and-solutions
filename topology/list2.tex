\documentclass{article}
\usepackage[a4paper,left=3cm,right=3cm,top=1cm,bottom=2cm]{geometry}
\usepackage{amsmath}
\usepackage{amssymb}
\usepackage{hyperref}

\setlength{\parindent}{0mm}

\usepackage{fontspec}
\setmainfont{Linux Libertine O}
\usepackage{unicode-math}
\setmathfont{Cambria Math}

\title{
\textit{\small{Георгий Потошин, 2024}}\\
\vspace{0.3ex}
\textit{\huge{Топология I, листочек 2}}\vspace{1ex}
}

\date{\vspace{-10ex}}

\begin{document}
\maketitle

\begin{enumerate}
    \item \textbf{Докажите, что отрезок $(a, b)$ гомеоморфен прямой $\mathbb{R}$.}
        Для этого построим гомеоморфизм, но сначала докажем пару утверждений.\par
        \textbf{Утверждение 1: Непрерывность отображений метрических пространств
        эквивалентна непрерывности отображений соответственных топологических
        пространств.}\par
        \textbf{Доказательство:} Пусть $Y$ и $X$ – метрические
        пространства, с соответственными топологиями.\par
        $\Rightarrow:$ Пусть $f:X\longrightarrow Y$ – непрерывно с метрической
        точки зрения. То есть $\forall x\in X,\forall\varepsilon>0,\exists\delta$
        такое что $\rho(x, y) < \delta \Rightarrow \rho(f(x), f(y)) < \varepsilon$.
        Пусть $U\subseteq Y$ открыто и $x\in f^{-1}[U]$. Тогда точка $f(x)$
        лежит в $U$ вместе с некоторым шаром $\mathcal{B}_Y(f(x), \varepsilon)$
        с центром $f(x)$ и радиусом $\varepsilon$. И так как из непрерывности
        имеет место следствие $\rho(x,y)<\delta\Rightarrow\rho(f(x),f(y))<
        \varepsilon$ для некоторого $\delta$, то $\mathcal{B}_X(x,\delta)
        \subseteq f^{-1}[\mathcal{B}_Y(f(x),\varepsilon)]\subseteq f^{-1}[U]$,
        это означает что в прообразе $U$ вместе с каждой точкой $f(x)$ лежит
        некий шар с центром в ней, а значит прообраз открыт. Тогда прообраз
        любого открытого множества открыт и $f$ – непрерывнo с топологической
        точки зрения.
        \par
        $\Leftarrow:$ Пусть прообраз каждого открытого открыт, тогда прообраз
        $\mathcal{B}_Y(f(x), \varepsilon)$ открыт, а значит для $x$ существует
        $\delta(\varepsilon)$, такой что $\mathcal{B}_Y(x,\delta(\varepsilon))
        \subseteq f^{-1}[B(f(x), \varepsilon)]$, что означает метрическую
        непрерывность $f$.
        \par
        \textbf{Утверждение 2: Композиция гомеоморфизмов – гомеоморфизм.}\par
        \textbf{Доказательство:} Пусть $X,Y,Z$ – гомеоморфные топологические
        пространства и $f:X\longrightarrow Y$, $g:Y\longrightarrow Z$ –
        гомеоморфизмы. Тогда для открытого $U$ из $Z$, $(g\circ f)^{-1}[U]=
        f^{-1}[g^{-1}[U]]$ его прообраз открыт. Аналогично образ открытого $V$
        из $X$ тоже открыт, а значит $g\circ f$ – гомеоморфизм.\par
        Тогда $f:x\mapsto x/(1-x^2)$ - гомеоморфизм $(-1, 1)\longrightarrow
        \mathbb{R}$ и $g:x\mapsto\frac{b-a}{2}x + \frac{b+a}{2}$ – гомеоморфизм
        $(-1, 1)\longrightarrow(a, b)$. А значит $f\circ g^{-1}\in\text{Iso}(
        (a, b), \mathbb{R})$.
    \item \textbf{Пусть $f,g:X\longrightarrow Y$ – непрерывные отображения
        топологических пространств. \\Предположим, что $Y$ – хаусдорфово. Докажите,
        что множество $C=\{x\in X | f(x) = g(x)\}$ замкнуто в X.}\par
        \textbf{Утверждение 3: Произведение хаусдорфовых пространств само 
        хаусдорфово.}\par
        \textbf{Доказательство:} Пусть $X$ и $Y$ – два хаусдорфовых пространства.
        Пусть $(x, y), (x', y') \in X\times Y$ две разные точки, и пусть без
        потери общности первая координата в них различается. Тогда по
        хаусдорфовости $X$ и $Y$ существуют открытые множества $U_x, U_y,
        U_{x'},U_{y'}$, что $i\in U_i$ и при этом $U_x\cap U_{x'}=\varnothing$.
        Тогда если положить $U=U_x\times U_y$ и $V=U_{x'}\times U_{y'}$, то они
        будут открытыми в пространстве $X\times Y$, причем $U\cap V=\varnothing$.
        Тогда для любых 2 различных точек $X\times Y$ найдутся их
        непересекающиеся окрестности, а значит $X\times Y$ хаусдорфово.\par
        \textbf{Утверждение 4: Диагональ квадрата хаусдорфово пространства
        замкнута.}\par
        \textbf{Доказательство:} Пусть $Y$ – хаусдорфово пространство, и $\Delta
        = \{(y, y)|y\in Y\}$ – диагональ квадрата $Y^2$. Пусть $(x, y)\in\Delta^c$
        – точка его дополнения. Тогда $x\neq y$, и существуют непересекающиеся
        окрестности $U_x, U_y$ этих точек, а значит $U_x\times U_y\cap\Delta=
        \varnothing$. Тогда обозначим окрестность пары $(x, y)$ $U_{x,y}=U_x
        \times U_y$. Тогда будет иметь место следующее соотношение
        \[\Delta^c = \bigcup_{(x,y)\in\Delta^c}U_{x,y},\]
        а значит $\Delta^c$ – открыто, а $\Delta$ – замкнуто.\par
        \textbf{Утверждение 5: Пусть $f:X\longrightarrow Y, g:X'\longrightarrow
        Y'$ - непрерывные отображения топологических пространств, тогда
        отображение $f\times g: (x,y)\mapsto (f(x), g(y))$ непрерывно. В случае
        если если $X=X'$, то отображение $(f,g):x\mapsto(f(x), g(x))$ тоже
        непрерывно.}\par
        \textbf{Доказательство:} Пусть $U\subseteq Y\times Y'$ открытое
        множество, тогда $U=\bigcup_i V_i\times V_i'$, где $V_i$ и $V_i'$
        открытые множества соответственных топологических пространств. Это
        значит, что $(f\times g)^{-1}[U]=\bigcup_i f^{-1} [V_i]\times g^{-1}
        [V_i']$, что является объединением произведений отрытых множеств, а
        значит открыто и $(f\times g)$ непрерывно. Теперь пусть $X=X'$. Тогда
        $(f,g)^{-1}[U]=\bigcup_i f^{-1} [V_i]\cap g^{-1}[V_i']$ – очевидно
        открыто. А значит прообраз открытого при $(f, g)$ всегда открыт,
        значит $(f, g)$ непрерывно.\par

        Заметим, что $C = (f, g)^{-1}[\Delta]$ прообраз замкнутого множества
        при непрерывном отображении, а значит само $C$ замкнуто.\par
        \textbf{Докажите, что если $f:X\longrightarrow X$ – непрерывное
        отображение хаусдорфова пространства $X$ на себя, то множество
        неподвижных точек $C=\{x\in X|f(x)=x\}$ замкнуто в $X$.}\par
        Здесь $g=\text{id}_x$ – непрерывно, а значит по предыдущему заданию $C$
        замкнуто.
    \item \textbf{Пусть $\tau_1$ и $\tau_2$ – топологии на множестве X, причем
        $\tau_1\subseteq \tau_2$. Предположим, что $(X, \tau_2)$ компактно.
        Докажите, что $(X, \tau_1)$ тоже компактно.}\par
        Пусть $(U_i)_{i\in I}, U_i \in \tau_1$ – покрытие пространства  $X$.
        Так как $U_i\in\tau_2$, то это ещё и покрытие в топологии $\tau_2$. А
        значит оно содержит конечное подпокрытие, а значит $(X, \tau_1)$ -
        компактное пространство.
    \item \textbf{Приведите пример топологий $\tau_1$ и $\tau_2$ на множестве
        $X$ таких что, $\tau_1\nsubseteq\tau_2$ и $\tau_2\nsubseteq\tau_1$.}\par
        Если множество $X$ содержит как минимум 2 различных элемента $a$ и $b$,
        то топологии $\{\varnothing, \{a\}, X\}$ и $\{\varnothing, \{b\}, X\}$
        удовлетворяют условию. В противном случае топология единственна.
    \item \textbf{Докажите, что компактное хаусдорфово пространство регулярно
        (для любой точки и для любого замкнутого множества, не содержащего эту
        точку, существуют непересекающиеся открытые окрестности). Докажите, что
        оно нормально (любые два непересекающихся замкнутых множества имеют
        непересекающиеся открытые окрестности).}\par Пусть $x\in X$ – точка и
        $F\subseteq X\backslash\{x\}$ – замкнутое множество. Тогда по хаусдорфовости
        для каждого $f\in F$ найдется непересекающаяся пара окрестностей $U_f$ и
        $V_f$, где $x\in U_f$ и $f\in V_f$. Тогда $(F^c)\sqcup (V_f)_{f\in F}$ –
        покрытие $X$. Тогда по компактности можно выбрать конечное подпокрытие
        $(F^c) \sqcup (V_f)_{f\in J}$. Тогда $\bigcap_{i\in J}U_i$ будет
        окрестностью точки $x$ и $\bigcup_{i\in J}V_i$ будет окресностью
        множества $F$. Эти окрестности пересекаются по пустому множеству, а
        значит пространство $X$ регулярно.\par
        Пусть теперь $F_1, F_2$ - два непересекающихся замкнутых множества.
        Тогда для каждого $f\in F_1$ по регулярности найдется неперескающаяся
        пара окрестностей $U_f\ni f$ и $V_f\supset F_2$. Тогда $(F_1^c)\sqcup
        (U_f)_{f\in F_1}$ покрытие $X$ и в нем по компактности можно выделить
        конечное подпокрытие $(F_1^c)\sqcup (U_f)_{f\in J}$. Тогда $\bigcup_{f\in
        J}U_f$ – открытая окрестность $F_1$, а $\bigcap_{f\in J}V_f$ –
        открытая окрестность $F_2$, причем они не пересекаются, а значит
        пространства $X$ нормально.
    \item \textbf{Пусть $X,Y,Z$ – топологические пространства. Докажите, что
        отображение $f:Z\longrightarrow X\times Y$ непрерывно тогда и только
        тогда, когда композиции с естественными проекциями $pr_1\circ
        f:Z\longrightarrow X$ и $pr_2\circ f:Z\longrightarrow Y$
        непрерывны.}\par
        $\Rightarrow:$ Пусть $f$ непрерывно. Проекции в данном случае
        непрерывны, так относительно их непрерывности строится тихоновская
        топология.\par
        $\Leftarrow:$ Пусть $\text{pr}_1\circ f$ и $\text{pr}_2\circ f$
        непрерывны. Тогда пусть $U\subseteq X\times Z$, тогда $U$ представимо
        как объединение открытых $U_i\times V_i$. Тогда $f^{-1}[U] = \bigcup_i
        f^{-1}[U_i\times V_i]=\bigcup_i f^{-1}[U_i\times Y]\cap f^{-1}[X\times
        V_i]=\bigcup_i (\text{pr}_1\circ f)^{-1}[U_i]\cap(\text{pr}_2\circ f)^
        {-1}[V_i]$ – открытое множество, а значит отображение $f$ –
        непрерывно.
    \item \textbf{Пусть отображения топологических пространств $f:X_1
        \longrightarrow X_2$ и $g:Y_1\longrightarrow Y_2$ непрерывны.
        Определите естественное отображение $f\times g:X_1\times X_2
        \longrightarrow Y_1\times Y_2$ и покажите, что оно непрерывно.}
        Утверждение 5
    \item \textbf{Определим график $\Gamma_f$ непрерывного отображения
        топологических пространств $f:X\longrightarrow Y$ следующим образом:
        $X\times Y\supseteq \Gamma_f=\{(x,y)|y=f(x)\}$. Докажите, что
        ограничение естественной проекции $\text{pr}_1$ индуцирует гомеоморфизм
        $X\simeq\Gamma_f$.}\par Вспомним, что у отображений из каждой точки
        выходит ровно одна стрелка, а значит каждая такая стрелка однозначно
        определяется своим началом. Тогда ограничение проекции как раз и имеет
        смысл этого однозначного определения, тогда ограничение биективно.
        Причем любое отображения при индуцировании остается непрерывным.
        Проверим, что обратное ему тоже непрерывно. Пусть $U\subset\Gamma_f$ –
        открыто. Это значит, что $U=\Gamma_f\cap\bigcup_i U_i\times V_i$.
        Тогда если вспомнить, что проекция графика биективна, то $\text{pr}_1
        [U] = X\cap\bigcup_i U_i$, что открыто, так как является объединением
        открытых. 
    \item \textbf{Докажите, что топология на топологическом пространстве $X$
        дискретна тогда и только тогда, когда диагональ $\Delta=\{(x,x)|x\in X\}
        \subseteq X\times X$ открыта в $X\times X$.}\par
        $\Rightarrow:$ $X$ – дискретна, а значит $\{x\}$ открыто в $X$, тогда 
        $\{(x, x)\}$ в $X\times X$. Тогда $\Delta=\bigcup_{x\in X} \{(x,x)\}$
        открыто в $X$.\par
        $\Leftarrow:$ Пусть $\Delta$ открыты в $X\times X$. Тогда для любого
        $x\in X$ отображение $(x,\text{id}): a \mapsto (x, a)$ непрерывно
        (утверждение 5) и $(x,\text{id})^{-1}[\Delta]=\{x\}$ открыто в $X$.
    \item \textbf{Докажите, что топологическое пространство $X$ хаусдорфово
        тогда и только тогда, когда диагональ $\Delta\subseteq X\times X$
        замкнута в $X\times X$.}\par
        $\Rightarrow:$ Утверждение 4.\par
        $\Leftarrow:$ Если $\Delta$ замкнута, то $\Delta^c$ открыто в $X\times
        X$ и $\Delta^c=\bigcup_i U_i\times V_i$. Тогда пусть $a,b\in X$ –
        две разные точки. Из того, что $(a,b)\in\Delta^c$ следует, что
        существует индекс $i$, что $(a,b)\in U_i\times V_i \subseteq\Delta^c$.
        Причем $U_i$ и $V_i$ открыты, а их пересечение пусто, и они являются
        окрестностями $a$ и $b$. Это верно для любых различных $a$ и $b$, а
        значит $X$ – хаусдорфово.
    \item \textbf{Докажите, что произведение топологических пространств 
        $X\times Y$ компактно тогда и только тогда, когда $X$ и $Y$ компактны.}
        \par$\Rightarrow:$ Пусть $X\times Y$ компактно и $(U_i)_i$ – покрытие
        пространства $X$. Тогда $(U_i\times Y)_i$ – покрытие $X\times Y$, по
        компактности можно найти конечное подпокрытие $(U_i\times Y)_{i\in F}$,
        a значит $(U_i)_{i\in F}$ будет конечным подпокрытием X. Тогда X
        компактно.\par
        $\Leftarrow:$ Пусть теперь $X$ и $Y$ – компактны и $(O_i)_{i\in I}$
        покрытие $X\times Y$. Введём функцию поиска $f: X\times Y\longrightarrow
        I$, что $(x,y)\in O_{f(x,y)}$. Тогда $(x,y)\in O_{f(x,y)}=\bigcup_j U_j
        \times V_j$, а значит $(x,y)\in U_{(x,y)}\times V_{(x,y)}$. Зафиксируем
        $x_0\in X$. Тогда $(V_{(x_0,y)})_{y\in Y}$ – покрытие Y. Выберем
        конечное подпокрытие $(V_{(x_0, y)})_{y\in F_{x_0}\subseteq Y}$.
        Обозначим за $U_{x_0}=\bigcap_{y\in F_{x_0}}V_{(x_0,y)}$ открытую
        окрестность точки $x_0$. Тогда $(U_x)_{x\in X}$ – покрытие $X$. Выделим
        конечное подпокрытие $(U_x)_{x\in F\subseteq X}$. Тогда я утверждаю, что
        $(O_{f(x,y)})_{(x,y)\in\bigcup_{\chi\in F}\{\chi\}\times F_\chi}$ – 
        конечное подпокрытие, так как конечно и для точки $(x_0, y_0)$ мы
        найдем $U_{\chi},\chi\in F$, а также в $F_\chi$ мы найдем $\gamma$, что
        $(x_0, y_0)\in U_{(\chi,\gamma)}\times V_{(\chi,\gamma)}\subseteq O_{f(
        \chi,\gamma)}$.
    \item \textbf{Докажите теорему Тихонова: произведение непустого
        множества компактных пространств компактно.}\par\textbf{Общая структура:}
        Пусть $I$ – множество индексов и $E_i$ для $i\in I$ пространствo с
        топологией $\tau_i$. Посмотрим на $E=\prod_{i\in I}E_i$. Топология на
        $E$ – самая грубая, относительно которой проекция $\text{pr}_i$ – 
        непрерывна для каждого $i$. Тогда предбазой на $E$ будет
        $\{\text{pr}_i^{-1}[U]|i\in I,U\in \tau_i\}$, множество
        произведений открытых множеств одного пространства на остальные
        пространства. Все конечные пересечения множеств предбазы составят базу.
        Её элементами будут всевозможные произведения открытых множеств, где
        почти все сомножители имеют вид $E_i$.\par
        \textbf{Максимальное покрытие:} Заметим, что множество покрытий, что не
        содержат конечных подпокрытий, образуют частично упорядоченное множество
        по отношению вложенности. Причем очевидно, что у каждой цепи будет
        верхняя грань. Её легко найти, пусть есть некая цепь. Возьмём
        объединение по всем её элементам. Это будет покрытием пространства.
        Если бы оно содержало конечное покрытие, то тогда элементы этого
        конечного покрытия лежали бы в неких элементах цепи. Среди них
        можно найти максимальный, и он бы тогда содержал это конечное покрытие,
        но цепь состоит из элементов, что таких не содержат, непорядок. Тогда
        объединение по любой цепи этого множества лежит в ней же и является
        верхней гранью цепи. Тогда в силу леммы Цорна существует максимальный
        элемент множества.\par
        Пусть $X$ – топологическое пространство, а $M\subset \tau$ –
        максимальное покрытие, что не содержит конечного подпокрытия.\textbf{
        Тогда если $V\in M^c$, то существует $U_1, .., U_n\in M$, что $V\cup
        U_1 \cup ...\cup U_n = S$}. В противном случае $M$ не было бы
        максимальным, так как $M\sqcup\{V\}$ было бы более большим покрытием,
        что не содержит конечного подпокрытия. Обратное очевидно тоже верно.
        \textbf{Теперь пусть $U, V\in M^c$, тогда $U\cap V\in M^c$.} Это верно
        в силу того, что существуют $U_1, .., U_k\in M$ и $V_1, .., V_l\in M$,
        что $U\cup U_1\cup ...\cup U_k=x=V\cup V_1\cup ...\cup V_l$, а значит
        $(U\cap V)\cup U_1\cup ...\cup U_k\cup V_1\cup ...\cup V_l=x$. Это
        влечет, что $U\cap V\in M^c$.\par
        \textbf{(Лемма Александера о предбазе) Пусть $B$ предбаза
        топологического пространства $X$. Тогда если в любом покрытии $X$
        элементами предбазы найдется конечное подпокрытие, то $X$ компактно.}
        Докажем это. Пусть $X$ не компактно, а $M$ – максимальное покрытие
        открытыми множествами, что не содержит конечного подпокрытия. Тогда для
        любого $x\in X$ найдётся её окрестность $V_x$ в $M$. Тогда в эту
        окрестность вложен некий элемент базы вокруг точки $x$, $U_x=U_1\cap...
        \cap U_{n_x}$, что является пересечением конечного числа элементов
        предбазы. Заметим, что из максимальности $M$ следует, что само $U$
        лежит в $M$. Если бы все $U_i,i\in1..n_x$ не лежали бы в $M$, то и $U$
        не лежало бы там. А значит, что хотя бы один элемент предбазы из этого
        списка лежит в $M$, тогда для каждой точки $x$ есть её окрестность
        $P_x$ из предбазы, что лежит ещё и в $M$. $(P_x)_{x\in X}$ – покрытие
        $X$ как элементами $B$, так и элементами $M$, по предположению из неё
        можно выделить конечное подпокрытие. Тогда мы приходим к противоречию,
        так как это же конечное подпокрытие будет подмножеством $M$, а значит
        мы ошибались в нашем предположении, что $X$ не компактно.\par
        \textbf{Теперь уже решение номера:} Пусть $\mathcal{S}=(U_i)_{i\in I}$
        – покрытие
        произведения $E=\prod_{j\in J}E_j$ компактных пространств элементами
        предбазы. Пусть оно не содержит конечного покрытия. Для каждого
        индекса $j\in J$ положим $S_j:=\{\text{pr}_j^{-1}[V_{i,j}]=U_i|V_{i,j}
        \in\tau_j\,i\in I_j\}$. Тогда $(V_{i,j})_{i\in I_j}$ – не является
        покрытием $E_j$, так как в нём бы сразу конечное подпокрытие, а значит
        и в $E$ тоже. А значит можно выбрать точку $x_j$, что не лежит
        в $\bigcup_{i\in I_j}V_{i,j}$. Положим $\chi=(x_j)_{j\in J}$, тогда она
        не лежит ни в одном элементе $\mathcal{S}$, а значит $\mathcal{S}$ не
        покрытие. Мы пришли к противоречию, а значит $E$ всё же компактно.

    \item \textbf{Докажите, что график непрерывного отображения из связного
        пространства связен.} \par Пусть $X$ связно, и $f: X\mapsto Y$ непрерывно,
        тогда $\Gamma_f\subset X\times Y$ его график. Пусть $\Gamma_f=U_1\sqcup
        U_2$ не связно. Тогда $U_1=\Gamma_f\cap\bigcup_i A_i\times B_i$ и $U_2=
        \Gamma_f\cap\bigcup_iC_i\times D_i$. Заметим, что так как f – набор
        стрелок такой, что из каждой точки $X$ выходит ровно одна стрелка, то
        если положить $A:=\bigcup_i A_i$ и $C:=\bigcup_i C_i$, то $X=A\sqcup C$
        было бы не связно, что ведёт к противоречию, а значит график всё же
        связен.
    \item \textbf{Докажите, что произведение топологических пространств $X\times
        Y$ связно тогда и только тогда, когда $X$ и $Y$ связны.}\par
        $\Rightarrow:$ Пусть $X$ без потери общности не связно. Тогда $X=A\sqcup
        B$. Тогда $X\times Y = A\times Y\sqcup B\times Y$ не связно.\par
        $\Leftarrow:$ Пусть $F:X\times Y\longrightarrow \{0, 1\}$ непрерывно,
        тогда ограничение $F$ как на $\{x\}\times Y$, так и на $X\times \{y\}$
        непрерывны, а также $\{x\}\times Y$ и $X\times \{y\}$ гомеоморфны $X$ и
        $Y$ соответственно. Тогда в силу связности $X$ и $Y$ ограничения
        постоянны. Тогда для любой пары точек $(x,y)$ и $(x',y')$ справедливо
        $F(x,y)=F(x',y)=F(x',y')$. А значит $F$ постоянно, что эквивалентно
        связности $X\times Y$.
    \item \textbf{Докажите, что отрезок $[a, b]\subset \mathbb{R}$ связен.}\par
        Пусть $[a, b] = A\sqcup B$, где $A$ и $B$ не пусты. Тогда
        пусть $a_0\in A$ и $b_0\in B$. Поделим отрезок пополам. Его середина
        лежит в одном из множеств $A$ или $B$. Тогда мы сможем уменьшит отрезок
        вдвое оставив его концы в разных множествах. Продолжая эту процедуру мы
        получим сходящуюся последовательность, к некой точке $c$, что является
        предельной точкой обоих множеств, но должно лежать только в одном. Тогда
        оба множества не могут быть одновременно замкнутыми, а значит отрезок
        связен.
    \item \textbf{Опишите все связные открытые подмножества вещественной прямой
        $\mathbb{R}$.}\par
        Пусть $X$ – такое множество, что существует $c\notin X$, что оба
        множества $\{x\in X|x>c\}$ и $\{x\in X|x<c\}$ не пусты. Тогда $X=((-
        \infty,c)\cap X)\sqcup((c,+\infty)\cap X)$ не связно. Если это не так,
        то найдем будут верхнюю и нижнюю грань $X$ в $\mathbb{R}\cap\{
        -\infty,+\infty\}$. И $X$ будет содержать все точки между границами.
        Такими множествами являются прямая, лучи, открытые лучи, отрезки,
        интервалы и полуинтервалы. Они в силу рассуждений из предыдущего
        задания связны.

        
\end{enumerate}

\end{document}
