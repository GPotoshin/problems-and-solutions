\documentclass{article}
\usepackage[a4paper,left=3cm,right=3cm,top=1cm,bottom=2cm]{geometry}
\usepackage{amsmath}
\usepackage{amssymb}
\usepackage{hyperref}

\setlength{\parindent}{0mm}

\usepackage{fontspec}
\setmainfont{Linux Libertine O}
\usepackage{unicode-math}
\setmathfont{Cambria Math}

\title{
\textit{\small{Георгий Потошин, 2024}}\\
\vspace{0.3ex}
\textit{\huge{Топология I, листочек 2}}\vspace{1ex}
}

\date{\vspace{-10ex}}

\begin{document}
\maketitle

\begin{enumerate}
    \item \textbf{Докажите, что отрезок $(a, b)$ гомеоморфен прямой $\mathbb{R}$.}
        Для этого построим гомеоморфизм, но сначала докажем пару утверждений.\par
        \textbf{Утверждение 1: Непрерывность отображений метрических пространств
        эквивалентна непрерывности отображений соответственных топологических
        пространств.}\par
        \textbf{Доказательство:} Пусть $Y$ и $X$ – метрические
        пространства, с соответственными топологиями.\par
        $\Rightarrow:$ Пусть $f:X\longrightarrow Y$ – непрерывно с метрической
        точки зрения. То есть $\forall x\in X,\forall\varepsilon>0,\exists\delta$
        такое что $\rho(x, y) < \delta \Rightarrow \rho(f(x), f(y)) < \varepsilon$.
        Пусть $U\subseteq Y$ открыто и $x\in f^{-1}[U]$. Тогда точка $f(x)$
        лежит в $U$ вместе с некоторым шаром $\mathcal{B}_Y(f(x), \varepsilon)$
        с центром $f(x)$ и радиусом $\varepsilon$. И так как из непрерывности
        имеет место следствие $\rho(x, y)<\delta(\varepsilon)\Rightarrow\rho(f(x), 
        f(y))<\varepsilon$, то $\mathcal{B}_X(x,\varepsilon)\subseteq f^{-1}
        [\mathcal{B}_Y(f(x),\varepsilon)]\subseteq f^{-1}[U]$, это означает что
        в прообразе $U$ вместе с каждой точкой $f(x)$ лежит некий шар с центром
        в ней, а значит прообраз открыт. Тогда прообраз любого открытого
        множества открыт и $f$ – непрерывен с топологической точки зрения.
        \par
        $\Leftarrow:$ Пусть прообраз каждого открытого открыт, тогда прообраз
        $\mathcal{B}_Y(f(x), \varepsilon)$ открыт, а значит для $x$ существует
        $\delta(\varepsilon)$, такой что $\mathcal{B}_Y(x,\delta(\varepsilon))
        \subseteq f^{-1}[B(f(x), \varepsilon)]$, что означает метрическую
        непрерывность $f$.
        \par
        \textbf{Утверждение 2: Композиция гомеоморфизмов – гомеоморфизм.}\par
        \textbf{Доказательство:} Пусть X, Y, Z – гомеоморфные топологические
        пространства и $f:X\longrightarrow Y$, $g:Y\longrightarrow Z$ –
        гомеоморфизмы. Тогда для открытого $U$ из $Z$ $(g\circ f)^{-1}[U]=f^{-1}
        \circ g^{-1}[U]=f^{-1}[g^{-1}[U]]$ его прообраз открыт. Аналогично образ
        открытого $V$ из $X$ тоже открыт, а значит $g\circ f$ – гомеоморфизм.\par
        Тогда $f:x\mapsto x/(1-x^2)$ - гомеоморфизм $(-1, 1)\longrightarrow
        \mathbb{R}$ и $g:x\mapsto\frac{b-a}{2}x + \frac{b+a}{2}$ – гомеоморфизм
        $(-1, 1)\longrightarrow(a, b)$. А значит $f\circ g^{-1}\in\text{Iso}(
        (a, b), \mathbb{R})$. Тогда прообраз любого открытого множества при
        отображении $f\times g$ открыт, а значит оно непрерывно.
    \item \textbf{Пусть $f,g:X\longrightarrow Y$ – непрерывные отображения
        топологических пространств. \\Предположим, что $Y$ – хаусдорфово. Докажите,
        что множество $C=\{x\in X | f(x) = g(x)\}$ замкнуто в X.}\par
        \textbf{Утверждение 3: Произведение хаусдорфовых пространств само 
        хаусдорфово.}\par
        \textbf{Доказательство:} Пусть $X$ и $Y$ – два хаусдорфовых пространства.
        Пусть $(x, y), (x', y') \in X\times Y$ две разные точки, и пусть без
        потери общности первая координата в них различается. Тогда по
        хаусдорфовости $X$ и $Y$ существуют открытые множества $U_x, U_y,
        U_{x'},U_{y'}$, что $i\in U_i$ и при этом $U_x\cap U_{x'}=\varnothing$.
        Тогда если положить $U=U_x\times U_y$ и $V=U_{x'}\times U_{y'}$, то они
        будут открытыми в пространстве $X\times Y$, причем $U\cap V=\varnothing$.
        Тогда для любых 2 различных точек $X\times Y$ найдутся их
        непересекающиеся окрестности, а значит $X\times Y$ хаусдорфово.\par
        \textbf{Утверждение 4: Диагональ квадрата хаусдорфово пространства
        замкнута.}\par
        \textbf{Доказательство:} Пусть $Y$ – хаусдорфово пространство, и $\Delta
        = \{(y, y)|y\in Y\}$ – диагональ квадрата $Y^2$. Пусть $(x, y)\in\Delta^c$
        – точка его дополнения. Тогда $x\neq y$, и существуют непересекающиеся
        окрестности $U_x, U_y$ этих точек, а значит $U_x\times U_y\cap\Delta=
        \varnothing$. Тогда обозначим окрестность пары $(x, y)$ $U_{x,y}=U_x
        \times U_y$. Тогда будет иметь место следующее соотношение
        \[\Delta^c = \bigcup_{(x,y)\in\Delta^c}U_{x,y},\]
        а значит $\Delta^c$ – открыто, а $\Delta$ – замкнуто.\par
        \textbf{Утверждение 5: Пусть $f:X\longrightarrow Y, g:X'\longrightarrow
        Y'$ - непрерывные отображения топологических пространств, тогда
        отображение $f\times g: (x,y)\mapsto (f(x), g(y))$ непрерывно. В случае
        если если $X=X'$, то отображение $(f,g):x\mapsto(f(x), g(x))$ тоже
        непрерывно.}\par
        \textbf{Доказательство:} Пусть $U\subseteq Y\times Y'$ открытое
        множество, тогда $U=\bigcup\{V_i\times V_i'|i\in I\}$, где $V_i$ и $V_i'$
        открытые множества соответственных топологических пространств. Это
        значит, что $(f\times g)^{-1}[U]=\bigcup\{f^{-1} [V_i]\times g^{-1}[V_i']
        |i\in I\}$, что является объединением произведений отрытых множеств, а
        значит открыто. Теперь пусть $X=X'$. Тогда $(f,g)^{-1}[U]=\bigcup\{f^{-1}
        [V_i]\cap g^{-1}[V_i']|i\in I\}$ – очевидно открыто.
        А значит прообраз открытого при $(f, g)$ всегда открыт, значит (f, g)
        непрерывно.\par

        Заметим, что $C = (f, g)^{-1}[\Delta]$ прообраз замкнутого множества
        при непрерывном отображении, а значит само $C$ замкнуто.\par
        \textbf{Докажите, что если $f:X\longrightarrow X$ – непрерывное
        отображение хаусдорфова пространства $X$ на себя, то множество
        неподвижных точек $C=\{x\in X|f(x)=x\}$ замкнуто в $X$.}\par
        Здесь $g=\text{id}_x$ – непрерывно, а значит по предыдущему заданию $C$
        замкнуто.
    \item \textbf{Пусть $\tau_1$ и $\tau_2$ – топологии на множестве X, причем
        $\tau_1\subseteq \tau_2$. Предположим, что $(X, \tau_2)$ компактно.
        Докажите, что $(X, \tau_1)$ тоже компактно.}\par
        Пусть $(U_i)_{i\in I}, U_i \in \tau_1$ – покрытие пространства  $X$.
        Так как $U_i\in\tau_2$, то это ещё и покрытие в топологии $\tau_2$. А
        значит оно содержит конечное подпокрытие, а значит $(X, \tau_1)$ -
        компактное пространство.
    \item \textbf{Приведите пример топологий $\tau_1$ и $\tau_2$ на множестве
        $X$ таких что, $\tau_1\nsubseteq\tau_2$ и $\tau_2\nsubseteq\tau_1$.}\par
        Если множество $X$ содержит как минимум 2 различных элемента $a$ и $b$,
        то топологии $\{\varnothing, \{a\}, X\}$ и $\{\varnothing, \{b\}, X\}$
        удовлетворяют условию. В противном случае топология единственна.
    \item \textbf{Докажите, что компактное хаусдорфово пространство регулярно
        (для любой точки и для любого замкнутого множества, не содержащего эту
        точку, существуют непересекающиеся открытые окрестности). Докажите, что
        оно нормально (любые два непересекающихся замкнутых множества имеют
        непересекающиеся открытые окрестности).}\par Пусть $x\in X$ – точка и
        $F\subseteq X\backslash\{x\}$ – замкнутое множество. Тогда по хаусдорфовости
        для каждого $f\in F$ найдется непересекающаяся пара окрестностей $U_f$ и
        $V_f$, где $x\in U_f$ и $f\in V_f$. Тогда $(F^c)\sqcup (V_f)_{f\in F}$ –
        покрытие $X$. Тогда по компактности можно выбрать конечное подпокрытие
        $(F^c) \sqcup (V_f)_{f\in J}$. Тогда $\bigcap\{U_i|i\in J\}$ будет
        окрестностью точки $x$ и $\bigcup\{V_i|i\in J\}$ будет окресностью
        множества $F$. Эти окрестности пересекаются по пустому множеству, а
        значит пространство $X$ регулярно.\par
        Пусть теперь $F_1, F_2$ - два непересекающихся замкнутых множества.
        Тогда для каждого $f\in F_1$ по регулярности найдется неперескающаяся
        пара окрестностей $U_f\ni f$ и $V_f\supset F_2$. Тогда $(F_1^c)\sqcup
        (U_f)_{f\in F_1}$ покрытие $X$
\end{enumerate}

\end{document}
