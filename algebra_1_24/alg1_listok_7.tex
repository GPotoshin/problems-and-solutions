\documentclass{article}
\usepackage[a4paper,left=3cm,right=3cm,top=1cm,bottom=2cm]{geometry}
\usepackage{amsmath}
\usepackage{amssymb}
\usepackage{hyperref}
\usepackage[russian]{babel}

\usepackage{tikz-cd}
\usepackage{array}
\usepackage{graphicx}
\newcommand\mapsfrom{\mathrel{\reflectbox{\ensuremath{\mapsto}}}}
\setlength{\parindent}{0mm}

\usepackage{fontspec}
\setmainfont{Linux Libertine O}
\usepackage{unicode-math}
\setmathfont{Cambria Math}

\newcommand{\mymat}{\mathcal{M\mkern-3mu a\mkern-0.3mu t}}


\title{
\textit{\small{Георгий Потошин, 2024}}\\
\vspace{0.3ex}
\textit{\huge{Алгебра I, листочек 7}}\vspace{1ex}
}

\date{\vspace{-10ex}}

\begin{document}
\maketitle

\begin{enumerate}
    \item \textbf{Постройте базисы над полем $\mathbb{k}$ в алгебрах: матриц $\mymat_n(\mathbb{k})$;
        верхнетреугльных матриц; многочленов с коэффициентами в $\mathbb{k}$. Запишите законы умножения
        в этих базисах.}

        Очевидно, что на матрицы можно смотреть как на наборы чисел, а значит и как на элементы свободного модуля.
        Тогда матричные единицы $\{e_{i,j}\}$, в которых на одном месте стоит единица, а  на остальных нули,
        образуют базис алгебры, более того часто матрицы строят как свободная алгебра на матричных единицах. Умножение
        матричных единиц происоходит по следующему правилу $e_{i,j}e_{k,l}=\delta_{j,k}e_{i,l}$.

        Базис верхнетреугольных матриц состоит из матричных единиц $e_{i,j}$, для которых $i\leq j$. умножение остается
        таким же, проверим замкнутость по нему: пусть $i\leq j$ и $k\leq l$, если произведение $e_{i,j}e_{l,k}$ не нулевое,
        то $j=l$, тогда по транзитивности $\leq$ мы получим $i\leq l$, а значит результат произведения также врехнетреугольный.

        Так как полиномы имеют моном максимальной степени, то каждый полином расскладывается в линейную комбинацию $x^n$.
        Тогда $\{x^n\}$ – базис алгебры. Это нельзя формализовать из наивного определения полинов, но если из рассматривать
        как элементы группового (наверно правильнее говорить моноидального, так как $\mathbb{N}$ не группа) кольца
        $K[\mathbb{N}]$, то это утверждение верно по определению. Произведение ведёт себя следующим образом, $x^nx^m=x^{n+m}$.

    \item \textbf{Постройте канонические изоморфизмы}
        \begin{enumerate} 
            \item \textbf{$U+W\cong U\oplus W/(U\cap W)$ для подпространств $U,W\leq V$}

                Если прочитать это соотношение как $U+W\cong U\oplus (W/(U\cap W))$, то изоморфизм нельзя канонически
                построить, так как придётся выбирать базис, и поэтому мы пойдём по иному пути, который верен в более общем
                случае для модулей.

                $U+W\cong (U\oplus W)/(U\cap W)$. Построим точную последовательность
                \[
                    0\rightarrow U\cap W\rightarrow U\oplus W\rightarrow U+W\rightarrow 0
                \]
                где нетривиальные стрелки $i=a\mapsto (a,-a)$ и $\pi=(a,b)\mapsto a+b$ в том порядке, в котором они появлятся
                в последовательности. Её точность тривиальна, а тогда согласованно с ней искомое соотношение, котороe
                следует для теоремы об изоморфизме для $\pi$, то есть $U+W=\text{Im}(\pi)\cong U\oplus W/\text{Ker}(\pi)
                = U\oplus W/\text{Im}(i)\cong U\oplus W/U\cap W$, так как $i : U\cap W\rightarrow U\oplus W$ – вложение и
                факторизация происходит по нему. Сопутствующий изоморфизм будет слудующим:
                \[[(a,b)]\in U\oplus W/U\cap W \mapsto a+b\]

            \item \textbf{[Теорема Нетер об изоморфизме] $(U+W)/U\cong W/(U\cap W)$ для подмодулей $U,W\leq V$}

                Построим cюръективный морфизм $\phi = a\in W\mapsto a+U\in(U+W)/U$, ядро которого $W\cap U$. Применим
                теорему о гомоморфизме и получим нужное соотношение $W/(U\cap W)\cong (U+W)/U$. Сопутствующий изомрфизм
                $w+U\cap W\in W/(U\cap W)\mapsto w+U\in (U+W)/U$.

            \item \textbf{$V/(U+W)\cong (V/U)/(W/(U\cap W))$ для подмодулей $U,W\leq V$}

                Здесь правый фактор не происходит по стандартному вложению, так как одно не подмножество другого,
                поэтому это соотношение образовано из точной последовательности:
                \[0\rightarrow W/(U\cap W)\rightarrow V/U\rightarrow V/(U+W)\rightarrow 0\]
                где нетривиальные стрелки следующие $i=w+U\cap W\mapsto w+U$ и $\pi=v+U\mapsto v+U+W$. Первая
                инъективнаь так как для $w\in W$, $w+U=U$ означает, что $w\in U$, а тогда $w\in U\cap W$ и $w+U\cap W=
                U\cap W$. Вторая стрелка инъективна, так как для $v+U+W$ можно найти прообраз $v+U$. Последовательность
                точна, так как с одной стороны для $w\in W$ $w+U+W=U+W$, а значит $\text{Im}(i)\subseteq\text{Ker}(\pi)$,
                c другой стороны, если $v+U+W=U+W$, то $v\in U+W$, тогда $v=u+w$ для некоторых $u\in U$ и $w\in W$.
                тогда прообораз равен $v+U=w+u+U=w+U\in\text{Im}(i)$ и мы получили второе включение. Осталось использовать
                теорему о гомеоморфизме $V/(U+W)=\text{Im}(\pi)\cong(V/U)/\text{Ker}(\pi)=(V/U)/\text{Im}(i)\cong(V/U)/(W/(U\cap W))$.
                Сопутствующий изоморфизм следующий $[v+U]\in(V/U)/(W/(U\cap W))\mapsto v+U+W$. 

            \item \textbf{$V/U\cong(V/W)/(U/W)$ для подмодулей $W\leq U\leq V$}
                Построим точную последовательность
                \[0\rightarrow U/W\rightarrow V/W\rightarrow V/U\rightarrow 0\]
                где нетривиальные морфизмы $i=u+W\mapsto u+W$ и $\pi=v+W\mapsto U$. Единственная вещь достойная проверки
                – это точность посередине. С одной стороны для $u\in U$ $\pi(u+W)=U$, а значит $\text{Im}(i)\leq
                \text{Ker}(\pi)$, в другую сторону проверка также очевидна. Тогда согласно этой последовательности
                построим изоморфизм $V/U=\text{Im}(\pi)\cong (V/W)/\text{Ker}(\pi)=(V/W)/\text{Im}(i)\cong(V/W)/(U/W)$,
                сопутствующий изоморфизм $[v+W]\mapsto v+U$.
        \end{enumerate}
        \textit{Здесь во всех случаях корректность изоморфизма гарантирована теоремой о гомоморфизме.}

    \item \textbf{Постройте канонический изоморфизм $V\cong U\oplus V/U$, где $U\leq V$.}

        Этот случай совпадает с первым пунктом прошлого задания, а тогда я в кратце повторю шаги. Построим точную
        последовательность
        \[0\rightarrow U\rightarrow U\oplus V\rightarrow V\]
        где нетривиальные стрелки $i=u\mapsto (u,-u)$ и $\pi=(u,v)\mapsto u+v$. Точность гарантирует соотношение
        и индуцирует изоморфизм $[(u,v)]\mapsto u+v$.
\end{enumerate}

\end{document}
