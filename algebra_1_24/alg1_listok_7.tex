\documentclass{article}
\usepackage[a4paper,left=3cm,right=3cm,top=1cm,bottom=2cm]{geometry}
\usepackage{amsmath}
\usepackage{amssymb}
\usepackage{hyperref}
\usepackage[russian]{babel}

\usepackage{tikz-cd}
\usepackage{array}
\usepackage{graphicx}
\newcommand\mapsfrom{\mathrel{\reflectbox{\ensuremath{\mapsto}}}}
\setlength{\parindent}{0mm}

\usepackage{fontspec}
\setmainfont{Linux Libertine O}
\usepackage{unicode-math}
\setmathfont{Cambria Math}

\newcommand{\mymat}{\mathcal{M\mkern-3mu a\mkern-0.3mu t}}


\title{
\textit{\small{Георгий Потошин, 2024}}\\
\vspace{0.3ex}
\textit{\huge{Алгебра I, листочек 7}}\vspace{1ex}
}

\date{\vspace{-10ex}}

\begin{document}
\maketitle

\begin{enumerate}
    \item \textbf{Постройте базисы над полем $\mathbb{k}$ в алгебрах: матриц $\mymat_n(\mathbb{k})$;
        верхнетреугльных матриц; многочленов с коэффициентами в $\mathbb{k}$. Запишите законы умножения
        в этих базисах.}

        Очевидно, что на матрицы можно смотреть как на наборы чисел, а значит и как на элементы свободного модуля.
        Тогда матричные единицы $\{e_{i,j}\}$, в которых на одном месте стоит единица, а  на остальных нули,
        образуют базис алгебры, более того часто матрицы строят как свободная алгебра на матричных единицах. Умножение
        матричных единиц происходит по следующему правилу $e_{i,j}e_{k,l}=\delta_{j,k}e_{i,l}$.

        Базис верхнетреугольных матриц состоит из матричных единиц $e_{i,j}$, для которых $i\leq j$. Умножение остается
        таким же, проверим замкнутость по нему: пусть $i\leq j$ и $k\leq l$, если произведение $e_{i,j}e_{l,k}$ не нулевое,
        то $j=l$, тогда по транзитивности $\leq$ мы получим $i\leq l$, а значит результат произведения также врехнетреугольный.

        Так как полиномы имеют моном максимальной степени, то каждый полином раскладывается в линейную комбинацию $x^n$.
        Тогда $\{x^n\}$ – базис алгебры. Это нельзя формализовать из наивного определения полиномов, но если из рассматривать
        как элементы группового (наверно правильнее говорить моноидального, так как $\mathbb{N}$ не группа) кольца
        $K[\mathbb{N}]$, то это утверждение верно по определению. Произведение ведёт себя следующим образом, $x^nx^m=x^{n+m}$.

    \item \textbf{Постройте канонические изоморфизмы}
        \begin{enumerate} 
            \item \textbf{$U+W\cong U\oplus W/(U\cap W)$ для подпространств $U,W\leq V$}

                Если прочитать это соотношение как $U+W\cong U\oplus (W/(U\cap W))$, то изоморфизм нельзя канонически
                построить, так как придётся выбирать базис, и поэтому мы пойдём по иному пути, который верен в более общем
                случае для модулей.

                $U+W\cong (U\oplus W)/(U\cap W)$. Построим точную последовательность
                \[
                    0\rightarrow U\cap W\rightarrow U\oplus W\rightarrow U+W\rightarrow 0
                \]
                где нетривиальные стрелки $i=a\mapsto (a,-a)$ и $\pi=(a,b)\mapsto a+b$ в том порядке, в котором они появляются
                в последовательности. Её точность тривиальна, а тогда согласованно с ней искомое соотношение, котороe
                следует для теоремы об изоморфизме для $\pi$, то есть $U+W=\text{Im}(\pi)\cong U\oplus W/\text{Ker}(\pi)
                = U\oplus W/\text{Im}(i)\cong U\oplus W/U\cap W$, так как $i : U\cap W\rightarrow U\oplus W$ – вложение и
                факторизация происходит по нему. Сопутствующий изоморфизм будет следующим:
                \[[(a,b)]\in U\oplus W/U\cap W \mapsto a+b\]

            \item \textbf{[Теорема Нетер об изоморфизме] $(U+W)/U\cong W/(U\cap W)$ для подмодулей $U,W\leq V$}

                Построим cюръективный морфизм $\phi = a\in W\mapsto a+U\in(U+W)/U$, ядро которого $W\cap U$. Применим
                теорему о гомоморфизме и получим нужное соотношение $W/(U\cap W)\cong (U+W)/U$. Сопутствующий изомрфизм
                $w+U\cap W\in W/(U\cap W)\mapsto w+U\in (U+W)/U$.

            \item \textbf{$V/(U+W)\cong (V/U)/(W/(U\cap W))$ для подмодулей $U,W\leq V$}

                Здесь правый фактор не происходит по стандартному вложению, так как одно не подмножество другого,
                поэтому это соотношение образовано из точной последовательности:
                \[0\rightarrow W/(U\cap W)\rightarrow V/U\rightarrow V/(U+W)\rightarrow 0\]
                где нетривиальные стрелки следующие $i=w+U\cap W\mapsto w+U$ и $\pi=v+U\mapsto v+U+W$. Первая
                инъективна так как для $w\in W$, $w+U=U$ означает, что $w\in U$, а тогда $w\in U\cap W$ и $w+U\cap W=
                U\cap W$. Вторая стрелка инъективна, так как для $v+U+W$ можно найти прообраз $v+U$. Последовательность
                точна, так как с одной стороны для $w\in W$ $w+U+W=U+W$, а значит $\text{Im}(i)\subseteq\text{Ker}(\pi)$,
                c другой стороны, если $v+U+W=U+W$, то $v\in U+W$, тогда $v=u+w$ для некоторых $u\in U$ и $w\in W$.
                Тогда прообраз равен $v+U=w+u+U=w+U\in\text{Im}(i)$ и мы получили второе включение. Осталось использовать
                теорему о гомеоморфизме $V/(U+W)=\text{Im}(\pi)\cong(V/U)/\text{Ker}(\pi)=(V/U)/\text{Im}(i)\cong(V/U)/(W/(U\cap W))$.
                Сопутствующий изоморфизм следующий $[v+U]\in(V/U)/(W/(U\cap W))\mapsto v+U+W$. 

            \item \textbf{$V/U\cong(V/W)/(U/W)$ для подмодулей $W\leq U\leq V$}
                Построим точную последовательность
                \[0\rightarrow U/W\rightarrow V/W\rightarrow V/U\rightarrow 0\]
                где нетривиальные морфизмы $i=u+W\mapsto u+W$ и $\pi=v+W\mapsto U$. Единственная вещь достойная проверки
                – это точность посередине. С одной стороны для $u\in U$ $\pi(u+W)=U$, а значит $\text{Im}(i)\leq
                \text{Ker}(\pi)$, в другую сторону проверка также очевидна. Тогда согласно этой последовательности
                построим изоморфизм $V/U=\text{Im}(\pi)\cong (V/W)/\text{Ker}(\pi)=(V/W)/\text{Im}(i)\cong(V/W)/(U/W)$,
                сопутствующий изоморфизм $[v+W]\mapsto v+U$.
        \end{enumerate}
        \textit{Здесь во всех случаях корректность изоморфизма гарантирована теоремой о гомоморфизме.}

    \item \textbf{Постройте канонический изоморфизм $V\cong U\oplus V/U$, где $U\leq V$.}
        
        \textit{Я не рассматриваю в доказуемое соотношение правую часть как сумму пространства и фактор пространства,
        так как тогда изоморфизм не будет каноническим, так как придется выбирать базисы. С другой стороны разложение
        на компоненты}

        Этот случай совпадает с первым пунктом прошлого задания, а тогда я вкратце повторю шаги. Построим точную
        последовательность
        \[0\rightarrow U\rightarrow U\oplus V\rightarrow V\]
        где нетривиальные стрелки $i=u\mapsto (u,-u)$ и $\pi=(u,v)\mapsto u+v$. Точность гарантирует соотношение
        и индуцирует изоморфизм $[(u,v)]\mapsto u+v$.

    \item \textbf{Докажите, что}
        
        \textit{Здесь я изменю порядок пунктов, чтобы решение одних основывалось на предыдущих результатах.}
        \begin{enumerate}
            \item \textbf{$\textnormal{dim}(U\oplus W)=\textnormal{dim}(U)+\textnormal{dim}(W)$ для пространств $U,W$}

                Выберем базис $\{u_i\}_{i\in I}$ в $U$ и базис $\{w_j\}_{j\in J}$ в $W$, тогда базисом $U\oplus W$
                будет $\{(u_i,0)\}_{i\in I}\sqcup\{(0,w_j\}_{j\in J}$, так как очевидно порождает $U\oplus W$ и
                линейно независим, если $\sum(u_i, 0)k_i + \sum(0, w_j)l_j = (0,0)=(\sum u_ik_i, \sum w_jl_j)$, то
                в каждой координате линейная комбинация занулить, так как там нулевые линейные комбинация элементов
                базисов компонент суммы. Дизъюнктивная сумма по определению складывает кардиналы базисов, а значит
                формула суммы верна.

            \item \textbf{$\textnormal{dim}(V)=\textnormal{dim}(U)+\textnormal{dim}(V/U)$ для подпространства $U\leq V$}

                Для этого построим не канонический изоморфизм, выберем базис $\{u_i\}$ в $U$
                и дополним его элементами $\{v_i\}$ до базиса $V$. Тогда элемент $x=\sum u_ix_i
                +\sum v_jx_j$ мы отправим в $(\sum u_ix_i, \sum v_jx_j+U)$. Как нетрудно заметить,
                мы получим сюрьективный морфизм векторных пространств. Осталось проверить, его
                инъективность, она верна, так как если $(\sum u_ix_i, \sum v_jx_j+U)=(0,0)$,
                то по свойству базиса все координаты при $u_i$ занулятся, но и так как тогда
                $\sum v_jx_j\in U$, то $\sum v_jx_j=\sum u_jk_j$, но по свойству базиса все
                координаты должны занулится, а поэтому координаты при $v_j$ нулевые, ядро
                тривиально, морфизм инъективен, а значит теперь мы показали, что он изоморфизм.

                Тогда по прошлому пункту из $V\cong U\oplus V/U$ заключаем, что
                $\textnormal{dim}(V)=\textnormal{dim}(U)+\textnormal{dim}(V/U)$.
                
            \item \textbf{$\textnormal{dim}(U+W)=\textnormal{dim}(U)+\textnormal{dim}(W)−
                \textnormal{dim}(U\cap W)$ для подпространств $U,W\leq V$}

                По прошлому заданию мы знаем, что $U+W\cong(U\oplus W)/(U\cap W)$, а тогда
                $\text{dim}(U+W)=\text{dim}(U\oplus W)−\text{dim}(U\cap W)=
                \text{dim}(U)+\text{dim}(W)−\text{dim}(U\cap W)$
                
            \item \textbf{$\textnormal{dim}(V)−\textnormal{dim}(U)=\textnormal{dim}(V/
                (\textnormal{Im}(f)))−\textnormal{dim}(\textnormal{Ker}(f))$ для гомоморфизма
                $f:U\rightarrow V$}

                Mы знаем, что $\text{Im}(f)\cong U/\text{Ker}(f)$, а значит $\text{dim}(U)=
                \text{dim}(\text{Im}(f))+\text{dim}(\text{Ker}(f)$. C другой стороны
                $\text{dim}(V/\text{Im}(f)) = \dim(V)-\dim(\text{Im}(f))=\dim(V)-(\dim(U)-
                \dim(\text{Ker}(f))$, откуда мы и получаем искомое тождество.
        \end{enumerate}

    \item \textbf{Докажите, что целочисленными элементарными преобразованиями строк и
        столбцов любую целочисленную матрицу можно привести к диагональному виду с
        числами $d_1,\ldots,d_k$ на диагонали, так что $d_1|\ldots|d_k$.}

        Мы постараемся показать, что любую матрицу $A_0\in\mymat_{n\times m}(\mathbb{Z})$ можно привести к виду
        \[\left(\begin{array}{cc}
        a_1 & 0\ldots 0\\
        0 & \\
        \vdots & A_1\\
        0 & \end{array}\right)\]

        где $a_1$ делит все коэффициенты в $A_1$. Так как тогда можно продолжить
        алгоритм для матричного нижнего блока и так как целочисленные преобразования сохраняют наибольший общий делитель,
        то приведенный вида подматрицы $A_1$ на блоки $a_2$ и $A_2$ будет удовлетворять условию $a_1\vert a_2$.

        Этап 1. Для начала, если матрица ненулевая, то можно перестановками строк и столбцов добиться того, чтобы в верхнем левом угле
        стоял не нуль. В $\mathbb{Z}$ у каждого числа есть только конечный набор делителей, и мы будем этим активно пользоваться.

        Этап 2. Дальше мы добьёмся того, чтобы число $a$ в верхнем левом угле делило все числа в первой строке и в первом столбце. Мы этого
        добьёмся следующей процедурой, если некоторое число $b$ в первой строке или столбце не делится на $a$, то мы воспользуемся
        соотношением Безу и найдем целые числа $\alpha,\beta$, что $a\alpha+b\beta=\text{gcd}(a,b)=d$. Заменим $a'=a/d$ и $b'=b/d$.
        Тогда у нас будет $a'\alpha+b'\beta=1$. Этому соотношению будет соответствовать матрица c детерминантом 1.
        \[ P_0 = \left(\begin{array}{cc}\alpha & \beta\\-b' & a'\end{array}\right)\]
        Покажем, что эта матрица образована произведением элементарных матриц, соответствующих элементарным преобразованиям.
        Для этого обозначим $a=(1,0)$ и $b=(0,1)$, что можно записать в матричном виде
        \[\left(\begin{array}{cc|c}1 & 0 & a'\\0 & 1 & b'\end{array}\right)\]
        Применим алгоритм евклида по строкам, в котором каждое действие соответствует элементарному преобразованию по окончанию
        алгоритма мы получим
        \[\left(\begin{array}{cc|c}x & y & 1\\ z & w & 0\end{array}\right)\]
        где матрица имеет детерминант 1. Это означает, что $xa'+yb'=1$ и
        $za'+wb'=0$. Так как $\text{gdc}(a',b')=1$ и $\text{gdc}(z,w)=1$,
        иначе детерминант не был бы $1$, то нетрудно видеть, решив это диофантовое уравнение, что $z=-b'$ и $w=a'$. Тогда
        матрица слева это в точности матрица $P_0$. Она раскладывается в произведение элементарных.

        Теперь если $b$ лежит в первом столбце в строке $i$, то мы строим матрицу $P$ по матрице из единичной матрицы $I_m$,
        занулив в ней $1$ и $i$ строки и поместив в $(1,1)$ $(P_0)_{1,1}$, в $(i,1)$ $(P_0)_{2,1}$, в $(1,i)$ $(P_0)_{1,2}$ и
        в $(i,i)$ $(P_0)_{i,i}$. Очевидно, что эта матрица также получена теми же элементарными преобразованиями, что и $P_0$,
        но на иных строках $1$ и $i$. Теперь если $b$ 
        
        Домножение на $P$, как нетрудно убедится запишет в $(1,1)$ $\text{gcd}(a,b)$. Мы будем продолжать этот процесс, пока
        верхний левый коэфициент не будет делить все числа в первом столбце и в первой строке. Количество итераций ограничено
        количеством простых делителей верхнего левого коэффициента, так как каждая уменьшает их количество, а значит вычислимая
        ситуация наступит за конечное время.

        Дальше мы вычтим первую строку и первый столбец необходимое число раз, чтобы занулить все коэффициенты в первой строке
        и в первом столбце, кроме верхнего левого коэффициента, после этого, если наш верхний левый коэффициент не делит
        какой-нибудь коэффициент из нижней блочной матрицы, то мы добавим строку с этим коэффициентом в первую и начнем
        заново процедуру этапа 2. Это имеет смысл, так как такое добавление не изменит наш верхний левый коэффициент, потому
        как в первом столбце под первой строчкой везде нули, и это действие перенесет неделящийся коэффициент наверх,
        относительно которого вновь можно считать нод. По той же причине, что и в прошлый раз вычисления закончатся за
        конечное время.
        
        Теперь у нас будет нужный вид и когда вычисления закончатся у нас будет диагональная матрицаБ

    \item \textbf{Докажите, что подгруппа свободной? конечно порожденной абелевой группы свободна? и конечно порождена.
        Докажите, что любая свободная? конечно порожденная абелева группа изоморфна
            \[\mathbb{Z}/d_1\mathbb{Z}\oplus\ldots\oplus\mathbb{Z}/d_k\mathbb{Z}\]
        для некоторых $d1,\ldots,dk$, так что $d_1|\ldots|d_k$. Единственно ли такое разложеие?}

    \item \textbf{Верно ли, что подмодуль свободного модуля свободен?}
        Это конечно же не верно, так как в свободном $\mathbb{Z}/4\mathbb{Z}$-модуле $\mathbb{Z}/4\mathbb{Z}$ есть
        подмодуль $2\mathbb{Z}/4\mathbb{Z}$ и он конечно же не свободен, так как его порядок 2, что не является степенью
        порядка кольца $\mathbb{Z}/4\mathbb{Z}$, то есть не степень 4.

\end{enumerate}

\end{document}
