\documentclass{article}
\usepackage[a4paper,left=3cm,right=3cm,top=1cm,bottom=2cm]{geometry}
\usepackage{amsmath}
\usepackage{amssymb}
\usepackage{hyperref}
\usepackage[russian]{babel}

\usepackage{tikz-cd}
\usepackage{array}
\usepackage{graphicx}
\newcommand\mapsfrom{\mathrel{\reflectbox{\ensuremath{\mapsto}}}}
\setlength{\parindent}{0mm}

\usepackage{fontspec}
\setmainfont{Linux Libertine O}
\usepackage{unicode-math}
\setmathfont{Cambria Math}

\newcommand{\mymat}{\mathcal{M\mkern-3mu a\mkern-0.3mu t}}


\title{
\textit{\small{Георгий Потошин, 2024}}\\
\vspace{0.3ex}
\textit{\huge{Алгебра I, листочек 8}}\vspace{1ex}
}

\date{\vspace{-10ex}}

\begin{document}
\maketitle

\begin{enumerate}
    \item \textbf{Пусть $A,B\in\mymat_{n\times n}(K)$, $A$ и $B$ нильпотентны и $AB=BA$. Докажите,
        что $A+B$, $AB$ нильпотентны. Верно ли это без условия $AB = BA$?}

        Пусть $A^n=0$ и $B^m=0$, тогда в силу того, что они коммутируют мы получим $(AB)^{\max(m,n)}=A^{\max(m,n)}
        B^{\max(m,n)}=00=0$. Также $(A+B)^{m+n+1}=0$, так как по формуле бинома Ньютона в каждом слагаемом либо степень
        $A$, либо степень $B$ будет больше зануляющей степени, а значит будет сумма нулей. Формула бинома работает, так
        как слагаемые коммутируют. Без этого условия это не верно, так как например есть матричные единицы $e_{1,2}$ и
        $e_{2,1}$, они нильпотенты, но их сумма нет, так как в случае размера $2\times 2$ их сумма – обратимая матрица
        с детерминантом -1.
        
    \item \textbf{Докажите формулу для определителя матрицы Вандермонда:}
        \[\text{det}\left(\begin{array}{ccccc}
            1 & x_0 & x_0^2 & \ldots & x_0^n\\
            1 & x_1 & x_1^2 & \ldots & x_1^n\\
            \ldots & \ldots & \ldots & \ldots & \ldots\\
            1 & x_n & x_n^2 & \ldots & x_n^n\end{array}
            \right)=\prod_{0\leq j\le i\leq n}(x_i-x_j)\]

        Обозначим это определитель через $V(x_0,\ldots, x_n)$. Заметим, что
        \[\text{det}\left(\begin{array}{ccccc}
            1 & x_0 & x_0^2 & \ldots & x_0^n\\
            1 & x_1 & x_1^2 & \ldots & x_1^n\\
            \ldots & \ldots & \ldots & \ldots & \ldots\\
            1 & x_n & x_n^2 & \ldots & x_n^n\end{array}
            \right)
            = \text{det}\left(\begin{array}{ccccc}
            1 & 0 & 0 & \ldots & 0\\
                1 & x_1-x_0 & x_1(x_1-x_0) & \ldots & x_1^{n-1}(x_1-x_0)\\
            \ldots & \ldots & \ldots & \ldots & \ldots\\
            1 & x_n-x_0 & x_n(x_n-x_0) & \ldots & x_n^{n-1}(x_n-x_0)\end{array}
            \right)\]
        Этого можно добиться, последовательно заменяя для каждого $i=n..1$ в порядке убывания столбец
        $S_i$ на $S_i-x_1S_{i-1}$, столбцы мы нумеруем с 0, по степени $x_0$. Такая замена очевидно не
        меняет детерминанта. Так как мы привели матрицу к нижнетреугольному блочному виду, то её 
        определитель это произведение определителей блоков, а в нашем случае определитель нижнего блока
        \[
            V(x_0,\ldots,x_n)
            = \text{det}\left(\begin{array}{cccc}
            x_1-x_0 & x_1(x_1-x_0) & \ldots & x_1^{n-1}(x_1-x_0)\\
            \ldots & \ldots & \ldots & \ldots\\
            x_n-x_0 & x_n(x_n-x_0) & \ldots & x_n^{n-1}(x_n-x_0)\end{array}\right)
        \]
        По полилинейности вынесем общий множителей в каждой строке и получим
        \[
            V(x_0,\ldots,x_n) = \prod_{i=1}^n (x_i-x_0)
            \text{det}\left(\begin{array}{cccc}
            1 & x_1 & \ldots & x_1^{n-1}\\
            \ldots & \ldots & \ldots & \ldots\\
            1 & x_n & \ldots & x_n^{n-1}\end{array}\right)
            = \prod_{i=1}^n (x_i-x_0)V(x_1,\ldots,x_n) 
        \]

        Продолжив дальше рекурсивный спуск по индукции мы получим искомую формулу.
        
    \item \textbf{Пусть $A\in\mymat_{m\times n}(K)$. Покажите, что если для любого $X\in\mymat_{n×m}(K)$
        имеем $\textnormal{tr}(AX) = 0$, то $A = 0$.}

        Заметим, что так как это будет в частности верно для матричных единиц, то $\text{tr}(Ae_{i,j})
        =\text{tr}(\sum_k A_{k,i}e_{i,j})=A_{j,i}=0$, так как мы можем выбирать $1\leq i\leq n$ и 
        $1\leq j\leq m$, то все коэффициенты матрицы должны быть нулями.
        
    \item \textbf{Пусть $A\in\mymat_{n\times n}(K)$. Покажите, что если $\textnormal{rk}(A)=n$, то
        $\textnormal{rk}(\widehat A) = n$; если $\textnormal{rk}(A)=n−1$, то $\textnormal{rk}(\widehat A)=1$;
        если $\textnormal{rk}(A)\leq n − 2$, то $\widehat A= 0$.}

        Если $\text{rk}(A)=n$, то $\det(A)!=0$, тогда $A^{-1}=\widehat A/\det(A)$. $\widehat A$ обратима,
        а значит её ранг $n$.

        Ecли $\text{rk}(A)=n-1$, то $A\widehat A= \widehat AA = 0$, первое равенство означает, что
        $\text{im}(\widehat A)\leq\text{ker}(A)$, но так как $\text{dim}\text{ker}(A)=1$, то
        $\text{rk}(\widetilde A)=1\text{ или }0$. Но так как в $A$ есть $n-1$ линейно независимых 
        строк, то в $A$ можно выкинуть лишнюю строку и получить матрицу $A'$, её ранг по прежнему будет
        $n-1$. Тогда в $A'$ найдутся $n-1$ линейно независимых столбцов, выкинув лишний мы получим
        минорную матрицу $A''$, чей ранг по прежнему $n-1$, а значит её детерминант ненулевой, тогда
        матрица $\widehat A$ ненулевая и её ранг не может быть нулевым, а значит он 1.

        Пусть $rk(A)\leq n-2$, так как ранг минорной матрицы не может превышать ранг матрицы,
        то ранг любого минорной будет $\leq n-2$, а значит детерминанта минорной матрицы нуль и
        $\widehat A=0$.

    \item \textbf{Покажите, что $\textnormal{rk}(AB)\leq\textnormal{rk}(A)$; $\textnormal{rk}(AB)\leq
        \textnormal{rk}(B)$; $\textnormal{rk}(A+B)\leq\textnormal{rk}(A)+\textnormal{rk}(B)$.}

        \[\text{rk}(AB)=\text{dim}(AB\mathbb{k}^m)\leq\text{dim}(A\mathbb{k}^n)=\text{rk}(A)\]
        \[\text{rk}(AB)=\text{rk}(B^TA^T)\leq\text{rk}(B^T)=\text{rk}(B)\]
        \[\text{rk}(A+B)=\text{dim}((A+B)\mathbb{k}^n)\leq\text{dim}(A\mathbb{k}^n+B\mathbb{k}^n)\leq
        \text{dim}(A\mathbb{k}^n)+\text{dim}(B\mathbb{k}^n)=\text{rk}(A)+\text{rk}(B)\]


\end{enumerate}

\end{document}
