\documentclass[a4paper, 12pt]{book}
\usepackage{amsmath}
\usepackage{amssymb}
\usepackage{hyperref}
\usepackage[russian]{babel}

\usepackage{tikz-cd}
\usepackage{array}
\usepackage{graphicx}
\newcommand\mapsfrom{\mathrel{\reflectbox{\ensuremath{\mapsto}}}}
\newcommand{\mymat}{\mathcal{M\mkern-3mu a\mkern-0.3mu t}}

\usepackage{fontspec}
\setmainfont{Linux Libertine O}
\usepackage{unicode-math}
\setmathfont{Cambria Math}

\title{
\textit{\huge{НМУ Алгебра I, Константин Логинов}}
}

\date{\vspace{-10ex}}

\begin{document}
\maketitle
\chapter{Векторные пространства}
\section{Жорданова нормальная форма}
Матрица называется жордановым блоком, если она имеет вид

\[J_k(\lambda)=\left(\begin{array}{ccccc}
    \lambda & 1       & 0      & \cdots  & 0\\
    0       & \lambda & 1      & \ddots  & \vdots\\
    \vdots  & \ddots  & \ddots & \ddots  & \vdots\\
    0       & \cdots  & 0      & \lambda & 1\\
    0       & \cdots  & \cdots & 0       & \lambda\\
\end{array}\right)\]

Болок размера $k\times k$ c $\lambda$ на диагонали и с 1 над диагональю. В
прошлый раз мы доказали, что для любого линейного эндоморфизма векторных
конечномерных пространств над алгебраически замкнутым полем есть базис, в
котором матрица имеет болочно диагональный вид, c жордановыми блоками.

\textit{Поле называется алгебраически замкнутым, если каждый многочлен над этим
полем положительной степени имеет корень.}
\[
    \left(\begin{array}{cccc}
        J_{k_1}(\lambda_1) &                    &        & 0\\
                           & J_{k_2}(\lambda_2) &        & \\
                           &                    & \ddots & \\
        0                  &                    &        & J_{k_n}(\lambda_n)\\
    \end{array}\right)
\]

Стоит отметить, что $\lambda_i$ и $k_i$

\textbf{Пример:} Пусть полем будет $\mathbb{k}=\mathbb{R}$, а пространством
$V=\mathbb{R}^2$. Зметим, что $x^2+1$ неприводим в этом поле. Тогда возьмём
оператор поворота на 90 градусов.

\[A=\left(\begin{array}{cc}0 & -1\\ 1 & 0\end{array}\right)\]

Для неё нет жордановой нормальной формы над $\mathbb{R}$, так как у неё нет
собственных значений. Если бы они были, то были бы корнем характеристического
многочлена $\chi_A(t)=t^2+1$, а у него корней нет. Над $\mathbb{C}$, наш
оператор приводим, так как $\pm\sqrt{-1}$ его собственные значения, а тогда

\[A=\left(\begin{array}{cc}\sqrt{-1} & 0\\ 0 & -\sqrt{-1}\end{array}\right)\]

Заметим, что по жордановой нормальной форме легко вычислять инварианты, так как
след – сумма диагональных элементов, $\text{tr}(A)=\sum k_i\lambda_i$.

\textbf{Замечание:} базис, в котором оператор имеет жорданову нормальную форму,
вообще говоря не единственен, например тривиальный оператор $I$.

Тем не менее кое-что определено канонически. Давайте бозначим за $n_{\lambda,k}$
– количество клеток вида $J_k(\lambda)$ в нашей матрице.

\textbf{Утверждение:}
\[\sum_{p=1}^k pn_{\lambda, p}+\sum_{p=k+1}^{\inf}kn_{\lambda, p} =
\text{dimKer}(A-\lambda\text{Id})^k,\;\forall\lambda,k\]
Следовательно, $n_{\lambda, k}$ – инвариаенты $A$.

Для доказательства, давайте запишем матрицу в жордановой нормальной форме и 
посчитаем ядро $\text{dimKer}(A-\text{Id})^\lambda$. В таком виде нас будут
интересовать только клетки, в которых стоит $\lambda$. Тогда можно предполагать,
что оператор состоит только из клеток с $\lambda$. Если посмотреть на то, что
происходит с клетками, то мы увидем
\[J_k(\lambda)-\lambda\text{Id}=\left(\begin{array}{cccc}
    0 & 1      &        & 0\\
      & \ddots & \ddots & \\
      &        & \ddots & 1\\
    0 &        &        & 0
\end{array}\right)\]
И если мы возведем в степень такие клетки, то равенство станет очевидным.

\textbf{Замечание:} Пусть $A\in\text{End}(V)$. Заметим, что задать оператор $A$,
равносильно заданию на $V$ структуры $\mathbb{k}[t]$-модуля. Структура
$\mathbb{k}[t]$-модуля это в точности $\mathbb{k}$-модуль с действием $t$.
Зададим это действие следующим образом $t^l\cdot v=A^l(v),\,v\in V$ и продолжим
его по линейности. В обратную сторону, мы зададим оператор через действие $t$,
то есть $A(v)=t\cdot v$. И это также эквивалентно заданию гомоморфизма (колец?)
$\phi: \mathbb{k}[t]\rightarrow\text{End}(V)$, где образ $t$ будет оператором
$A$. (Скорее всего это работает только в коммутативном случае, когда на $\text{End}(V)$
Есть структура модуля и я бы брал гомоморфизмы модулей!).

Например если $A=J_k(\lambda)$, то $V\cong\mathbb{k}[t]/(t-\lambda)^k$. Давайте
поймём почему этот изоморфизм имеет место. Нам нужно вопервых убедится, что они
изоморфны как $\mathbb{k}$-векторные пространства, а во вторых, что $A$
действует в $V$ также как $t$ умножением в $\mathbb{k}[t]/(t-\lambda)^k$. Первое
верно из наблюдения размерности, в обоих случаях она $k$. Для воторого, нужно
понять как $A-\lambda\text{Id}$ дей ствует на базисные вектора, а именно $e_1
\mapsto 0$ и $e_{i+1}\mapsto e_i$ для $1\leq i \le k$. Заметим, что
$\{(t-\lambda)^i\}_{0\leq i\le k}$ $\mathbb{k}$-базис фактор кольца, и в нём
$t-\lambda$ умножением действует точно также на элементы кольца, а значит у нас
есть изоморфизм $\mathbb{k}[t]$-модулей.

\textbf{Следситвие (из теоремы о существовании ЖНФ)}
Для $A\in\text{End}(V)$, V – $\mathbb{k}[t]$-модуль. То $V\cong_{\mathbb{k}[t]}
\bigoplus_{i=1}^N\mathbb{k}[t]/(t-\lambda_i)^{k_i}$, где действие $A$
соответствует действию $t$, а сумма идёт по жордановым блокам. Это верно, так
как матрица оператора блочно диагональная, а значит пространство раскладывается
в прямую сумму подпространств, так, что на каждом подпространстве наш оператор
деуствует как жорданов блок, а тогда применив предыдущий результат, мы получаем
искомое. Такая формулировка теоремы о жордановой нормальной форме более
правильная, так как она имеет обобщения, то есть на классификацию конечно
порожденных модулей. В частности классификация конечных и конечно порожденных
абелевых групп.

\textbf{Определение:} $A\in\text{End}(V)$ называется полупростым, если
существует базис, в котором матрица $A$ диагональна. $A$ называется
нильпотентом, если $A^m=0$ для $m>1$.

\textbf{Следствие (из ЖНФ):} $A\in\text{End}(V)$, то $A=A_{ss}+A_n$, где
$A_{ss}$ – полупрост, а $A_n$ – нильпотент. И эти два оператора коммутируют.
\[J_k(\lambda)=\lambda\text{Id}+\left(\begin{array}{cccc}
    0 & 1      &        & 0\\
      & \ddots & \ddots &  \\
      &        & \ddots & 1\\
    0 &        &        & 0
    \end{array}\right)\]

\textbf{Теорема (Гамильтона-Кэли):}
 $A\in\text{End}(V)\Rightarrow\chi_A(A)=0$. Поле не обязательно алгебраически
 замкнуто. $\chi_{J_k(\lambda)}(t)|_{t=a}=(t-\lambda)^k|_{t=A}=(A-\lambda)^k=0$.
 А значит в каждом блоке будет 0, теорему доказали, но жульничество в том, что
 нам необходима алгебраическая замкнутость поля, но жульничество можно обойти,
 показав, что каждое поле вложено в алгебраически замкнутое. 

 \textbf{Доказательство:}\\$(tE-A)(\widehat{tE-A})=(\widehat{tE-A})(tE-A)=\chi_A(t)
 \text{Id}$ в кольце $\mymat_{n\times n}(\mathbb{k}[t])=(\mymat_{n\times n}
 (\mathbb{k}))[t]$. Определим отображение \[\phi: R\rightarrow\mymat_{n\times n}
 (\mathbb{k}),\] где $R=Z_A(\mymat_{n\times n}(K)[t])$, а устроено оно
 вычислением в $A$, то есть $\phi(\sum B_it^i)=\sum B_iA^i$, где $B_i\in
 \mymat_{n\times n}(\mathbb{k})$. Заметим, что $\phi$ является гомоморфизмом.

 $\chi_A(A)=\phi(\chi_A(t)E)=\phi((\widehat{tE-A})(tE-A))=\phi(\widehat{tE-A}
 \phi(tE-A)=\phi(\widehat{tE-A})(A-A)=0$.

 \textbf{Замечание:} $A\in\text{End}(V)$ задание эндоморфизма эквивалентно заданию
 гомоморфизма $\phi:\mathbb{k}[t]\rightarrow\text{End}(V)$. По теореме
 Гамильтона-Кэли мы знаем, что $\chi_A(t)\in\text{Ker}(\phi)$. С другой стороны
 $\text{Ker}(\phi)=(m_A(t))$, тогда можно определить $m_A$ минимальный многочлен
 оператора $A$, минимальный многочлен оператора $A$, он определен однозначно,
 если страший коэффициент брать за 1. Заметим, что минимальны многочлен делит
 характеристичесий.

 \textbf{Упражнение:} Существует $N$, что $\chi_a(t)\,|\,m_a(t)^N$.

 \textbf{Пример:}
 \begin{itemize}
     \item $m_A(t)=t-\lambda$, для $A=\lambda E$. Тогда $\chi_A(t)=(t-
         \lambda)^k$.

     \item $m_A(t)=t^k$, тогда $A-нильпотент$ и $\chi_A(t)=t^n$. Можно взять
         нулевой жордановый блок и нулевуюматрицу и соединить их в блочно
         диагональной маньере.

     \item Если $m_A(t)=(t-1)^k$, то $A$ называется унипотентом.
         
     \item Если $m_A(t)=t(t-1)$, то $A$ – проектор. Идемпотентен
 \end{itemize}
 \chapter{Поля и их расширения}
 Пусть $\mathbb{k}$ – поле. Тогда можно рассмотреть гомоморфизм $\varkappa:
 \mathbb{Z}\rightarrow\mathbb{k}, 1\mapsto 1$, у него есть ядро $\text{Ker}
 (\varkappa)\subseteq\mathbb{Z}$, это идеал в $\mathbb{Z}$, он главный, так как
 идеал кольца главных идеалов, пусть он равен $(d)$.
 
 \textbf{Утверждение:} $d$ – простое число или 0.

 \textbf{Доказательство:} Ядро – прообраз простого идеала, а значит ядро просто.

 \textbf{Определение:} $d$ – характериситка $\mathbb{k}$, её мы обозначаем
 $\text{char}(\mathbb{k})=d$, то есть простое число или 0, которое однозначно
 определяется по полю.

 \begin{itemize}
     \item $\text{char}(\mathbb{Q})=0$
     \item $\text{char}(\mathbb{Z}/p\mathbb{Z})=p$
 \end{itemize}

 \textbf{Напоминание:} Если $f:\mathbb{K}\rightarrow\mathbb{L}$ гомоморфиз
 полей, то он инъективен. Так как несобственный идеал только 0.

 $A=\text{Im}(\varkappa)$ – область целостности. Тогда можно рассмотреть поле
 частных $\text{Frac}(A)\leq\mathbb{k}$, подполе в $\mathbb{k}$, оно назаватеся
 простым подполем.
 \[\text{Frac}(A)\cong \left\{\begin{array}{rcl}\mathbb{Q} & \text{char}(\mathbb{k})=0\\
                                                \mathbb{Z}/p\mathbb{Z} &\text{char}(\mathbb{k})=p
                              \end{array}\right.\]
Простое подполе определено однозначно, так как гомоморфизм $\varkappa$ определен
однозначно, канонически. Оно называется простым, так как в нём нет собственных
подполей.

\textbf{Утверждение:} Пусть $f:\mathbb{K}\rightarrow\mathbb{L}$ – гомоморфизм
полей. Тогда $\text{char}(\mathbb{K})=\text{char}(\mathbb{L})$ и $f$ индуцирует
изоморфизм простых подполей в $\mathbb{K}$ и $\mathbb{L}$.

\textbf{Доказательство:}
\begin{center}
\begin{tikzcd}
    \mathbb{Z}\arrow{r}{\varkappa_K} \arrow[bend right]{rr}{\varkappa_L} & \mathbb{K}\arrow{r}{f} & \mathbb{L}
\end{tikzcd}
\end{center}
Давайте тогда заметим, что композиция является гомоморфизмом $\varkappa$ для
$\mathbb{L}$, так как композиция перводит единицу в единицу. Отсюда следует,
что ядро $\varkappa_L$ равно ядру $\varkappa_K$, так как $f$ вложение. Более
того $\text{Im}(\varkappa_K)\cong_f\text{Im}(\varkappa_L)$, а значит простые
подполя изоморфны, а характеристики равны.

\textbf{Определение:} $K\leq L$ называется расширением полей, если $K
\hookrightarrow L$, то есть следующий набор данных, поле $K$, поле $L$ и
вложение. Иногда это обозначается $(L/K)$ и черта читается как "над".

Если $K\leq L$, то $L$ является векторным пространством над $K$. Тогда можно
говорить о размерности $L$ над $K$ и если $\text{dim}_K\,L\le\infty$, то
расширение мы называем конечным, а размерность мы будем писать чуть иначе
$\text{dim}_K\,L=[L:K]$.

$K_1\le K_2\le\ldots\le K_s$ мы называем башней полей, а расширение $K_i\le
K_{i+1}$ – этаж этой башни.

\textbf{Пример:} $\mathbb{Q}\le\mathbb{R}\le\mathbb{C}$ в этой башне только
второй этаж конечен.

\textbf{Утверждение:} Если $F\le K\le L$, то $[L:F]=[L:K][K:L]$

\textbf{Доказательство:} Пусть $K=\langle x_i\rangle_F,\,x_i\in K$, где
$\{x_i\}$ базис $K$ над $L$ и пусть $L=\langle y_j\rangle_K,\,y_j\in L$, где
$\{y_j\}$ базис $L$ над $F$. Тогда мы можем построить базис $L$ над $F$, а
именно $L=\langle x_iy_j \rangle_F$ поверим это. Пусть $a\in L$, тогда его
можно разложить над $\{y_j\}$, то есть $a=\sum a_jy_j,\,a_j\in K$. Но тогда
$a_j$ можно разложить над $\{x_i\}$, то есть $a_j=\sum a_{i,j}x_i,\,a_{i,j}\in
F$, а тогда $a=\sum a_{i,j}x_iy_j$, что означает $\{x_iy_j\}$ порождает $L$ над
$F$.

Пусть теперь 
\end{document}
