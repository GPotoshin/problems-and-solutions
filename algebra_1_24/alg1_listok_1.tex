\documentclass{article}
\usepackage[a4paper,left=3cm,right=3cm,top=1cm,bottom=2cm]{geometry}
\usepackage{amsmath}
\usepackage{amssymb}
\usepackage{hyperref}
\usepackage[russian]{babel}

\usepackage{tikz-cd}
\usepackage{array}
\usepackage{graphicx}
\newcommand\mapsfrom{\mathrel{\reflectbox{\ensuremath{\mapsto}}}}
\setlength{\parindent}{0mm}

\usepackage{fontspec}
\setmainfont{Linux Libertine O}
\usepackage{unicode-math}
\setmathfont{Cambria Math}

\title{
\textit{\small{Георгий Потошин, 2024}}\\
\vspace{0.3ex}
\textit{\huge{Алгебра I, листочек 1}}\vspace{1ex}
}

\date{\vspace{-10ex}}

\begin{document}
\maketitle

\begin{enumerate}
    \item \textbf{Докажите, что всякая циклическая группа изоморфна
        $\mathbb{Z}/n\mathbb{Z}$ для некоторого $n\geq 0$.}\par
        Пусть $g\in G$ порождает $G$, положим $n=\text{ord}(g)$. Тогда построим
        гомоморфизм $\varphi:g^a \mapsto [a]_n$, $\varphi(g^a*g^b)=[a+b]_n=[a]_n+[b]_n$. Он
        инъективен $\varphi^{-1}(0) = \{g^a|[a]_n=[0]_n\} = 1$. То, что он
        сюръективен тоже очевидно, так как мы можем возводить $g$ в любую
        степень. Изоморфизм построен.

    \item \textbf{Какие подгруппы у циклической группы?}\par
        Мы будем рассматривать далее $G=\mathbb{Z}/n\mathbb{Z}$, $n\in\mathbb{N}_0$.
        Пусть $H\leq G$, введем на $G$ стандартный линейный порядок. Тогда
        пусть $d =\min H\setminus\{0\}$. Очевидно, что $\langle d\rangle\leq H$. Пусть
        $a\in H\setminus\langle d\rangle$, тогда $a=kd+r$, $r<d$ и $r\in H$,
        противоречие, а значит $\langle d\rangle=H$. Тогда все подгруппы 
        циклической группы циклические. Причем для $n\neq 0$, легко построить
        группу порядка $d|n$ - $\langle n/d\rangle$, a для $n=0$, $\langle
        a\rangle = a\mathbb{Z}$.

    \item
        \begin{enumerate}
            \item \textbf{Почти группа без ассоциативности:}\\
                \begin{center}
                \noindent\begin{tabular}{c | c c c}
                    $e$ & $a$ & $b$ & $c$  \\
                    \cline{1-4}
                    $a$ & $b$ & $e$ & $b$ \\
                    $b$ & $e$ & $b$ & $c$ \\
                    $c$ & $a$ & $c$ & $e$ \\
                \end{tabular}
                \end{center}
                В этой алгебре есть единица, есть единственный обратный, но операция
                не ассоциативна $(ab)c=ec=c$, но $a(bc)=ac=b$.

            \item \textbf{Почти группа без 1 и обратных (полугруппа)} $(\mathbb{N}, +)$
            \item \textbf{Почти группа без обратных (моноид)} $(\mathbb{N}_0, +)$
        \end{enumerate}

    \item \textbf{Симметрические группы порядка $n\geq 3$ неабелевы.}\par
        Элементы симметрических групп здесь и далее я буду комбинировать
        слева направо, а действовать элементами справа.
        $S_3$ неабелева, так как $(2,3)(2,1)=(1,2,3)$, a $(2,1)(3,2)=
        (3,2,1)$, дальше легко проделать вложение, если оставить запись
        циклами $S_3\hookrightarrow S_n$, $a\mapsto a$, так что
        $S_n$ тоже неабелева.

    \item \textbf{Докажите, что для ассоциативной операции все расстановки
        скобок в $g_1\cdot\ldots\cdot g_n$ дают один и тот же результат.}

        Индукция по длине.

        \textbf{База:} $n=1,2,3$ очевидно

        \textbf{Шаг:} Пусть $V=(g_1\cdot\ldots)\cdot(g_k\cdot\ldots g_n)$ некая
        расстановка скобок. Всегда будет последнее умножение, оно перемножает
        два выражения меньшей длинны, для которых расстановка скобок не играет
        роли по предположению индукции, тогда ориентируем умножения в левой
        скобке влево, а в правой вправо и применим ассоциативность некоторое
        количество раз рекурсивно, переориентируя скобки в одну сторону:
        \begin{align*}
            V&=(((g_1g_2)\ldots g_k)(g_{k+1}(g_{k+2}\ldots g_n))) \\
            &=((((g_1g_2)\ldots g_k)g_{k+1})(g_{k+2}\ldots g_n))\\
            &=\cdots\\
            &=(((g_1g_2)g_3)\ldots g_n)
        \end{align*}
        В итоге любую ориентацию скобок можно свести к левой, а значит все они
        равны.
    \item \textbf{Пусть $G$ – группа, и пусть для любого ее элемента $g$ выполнено
        $g^2=e$. Докажите, что $G$ абелева. Верно ли, что $G$ абелева, если для
        любого элемента $g\in G$ выполнено $g^3=e$?}
        
        Если $G$ – группа инвлюций, то $(ab)(ba)=e=(ab)^2$, а значит $ab=ba$.

        Если у нетривиальных элементов группы порядок равен трём, то она вообще
        говоря не абелева, так как есть группа вращений правильной треугольной
        пирамиды. Она по очевидным соображениям не абелева, но все её
        нетривиальные вращения имеют порядок 3.

    \item \textbf{Опишите все автоморфизмы и все подгруппы в следующих группах:
        $\mathbb{Z}, \mathbb{Z}/n\mathbb{Z}, S_3, S_4$.}
        \begin{itemize}
            \item $\text{Aut}(\mathbb{Z})$. Пусть $n$ порождает $\mathbb{Z}$,
                тогда очевидно, $\langle n\rangle=n\mathbb{Z}$. A значит,
                $n=\pm 1$. Заметим, что порождающий должен переходить в
                порождающий, иначе гомоморфизм не сюръективен. А значит есть
                всего два автоморфизма. $1\mapsto 1$ и $1\mapsto -1$.

            \item $\text{Aut}(\mathbb{Z}/n\mathbb{Z})$. Пусть $g$ порождает
                $\mathbb{Z}/n\mathbb{Z}$ это эквивалентно тому, что $lg=kn+1$,
                некоторая степень $g$ равна 1. Это эквивалентно тому, что
                $g\wedge n=1$. Тогда $(\varphi_m(1)=m)\wedge n=1$ будут
                всеми автоморфизмами группы. Причем они комбинируются по
                следующему закону $(\varphi_n \circ \varphi_m: 1\mapsto m\mapsto
                nm)=\varphi_{nm}$. A значит группа автоморфизмов изоморфна
                мультипликативной группе соответствующего кольца.
                
            \item $\text{Aut}(S_3)$. Перечислим нетривиальные элементы группы:
                $i_1=(3,2)$, $i_2=(1,3)$, $i_3=(1,2)$ – инволюции, $t_1=(1,2,3)$,
                $t_2=(3,2,1)$ - 3-циклы. Очевидно, что при автоморфизме
                инволюции переходят в инволюции, а трициклы в трициклы. По этому
                свойству можно провести классификацию.

                \begin{center}
                \noindent\begin{tabular}{c | c c c c c}
                    $e$ & $i_1$ & $i_2$ & $i_3$ & $t_1$ & $t_2$ \\
                    \cline{1-6}
                    $i_1$ & $e$ & $t_2$ & $t_1$ & $i_3$ & $i_2$\\
                    $i_2$ & $t_1$ & $e$ & $t_2$ & $i_1$ & $i_3$\\
                    $i_3$ & $t_2$ & $t_1$ & $e$ & $i_2$ & $i_1$\\
                    $t_1$ & $i_2$ & $i_3$ & $i_1$ & $t_2$ & $e$\\
                    $t_2$ & $i_3$ & $i_1$ & $i_2$ & $e$ & $t_1$\\
                \end{tabular}
                \end{center}
                \hrule
                \vspace{2ex}
                Перечислим нетривиальные автоморфизмы для которых $t_1$ и $t_2$
                неподвижные точки:
                \begin{align*}
                    t_1&\mapsto t_1,\quad t_2\mapsto t_2,\quad i_1\mapsto i_2\\
                    i_2&=i_1t_2\mapsto i_2t_2 = i_3,\quad i_3\mapsto i_1\\
                \end{align*}
                Это отображение совпадет с сопряжением по $t_2$, так как:
                \begin{align*}
                    S_{t_2}:\quad t_1&\mapsto t_2t_1t_2^{-1}=t_2^{-1}=t_1\\
                    t_2&\mapsto t_2\\
                    i_1&\mapsto t_2i_1t_2^{-1} = i_3t_1 = i_2\\
                    i_2&\mapsto t_2i_2t_1 = i_1t_1 = i_3\\
                    t_3&\mapsto i_1\\
                \end{align*}
                \hrule
                \begin{align*}
                    t_1&\mapsto t_1,\quad t_2\mapsto t_2\\
                    i_1&\mapsto i_3,\quad i_2\mapsto i_1,\quad i_3\mapsto i_2\\
                \end{align*}
                Это отображение совпадет с сопряжением по $t_1$, так как:
                \begin{align*}
                    S_{t_1}:\quad t_1&\mapsto t_1\\
                    t_2&\mapsto t_2\\
                    i_1&\mapsto t_1i_1t_2 = i_2t_2 = i_3\\
                    i_2&\mapsto t_1i_2t_2 = i_3t_2 = i_1\\
                    t_3&\mapsto i_2\\
                \end{align*}
                \hrule
                \vspace{2ex}
                Теперь меняем $t_1$ и $t_2$ местами.
                \begin{align*}
                    t_1&\mapsto t_2,\quad t_2\mapsto t_1,\quad i_1\mapsto i_1\\
                    i_2&=i_1t_2\mapsto i_1t_1 = i_3,\quad i_3\mapsto i_2\\
                \end{align*}
                Это отображение совпадет с сопряжением по $i_1$, так как:
                \begin{align*}
                    S_{t_2}:\quad t_1&\mapsto i_1t_1i_1=i_3i_1=t_2\\
                    t_2&\mapsto t_1\\
                    i_1&\mapsto i_1i_1i_1 = i_1\\
                    i_2&\mapsto i_1i_2t_1 = t_2i_1 = i_3\\
                    t_3&\mapsto i_2\\
                \end{align*}
                \hrule
                \vspace{2ex}
                \begin{align*}
                    t_1&\mapsto t_2,\quad t_2\mapsto t_1,\quad i_1\mapsto i_2\\
                    i_2&\mapsto i_1,\quad i_3\mapsto i_3\\
                \end{align*}
                Это отображение совпадет с сопряжением по $i_3$, так как:
                \begin{align*}
                    S_{t_2}:\quad t_1&\mapsto i_3t_1i_3=i_2i_3=t_2\\
                    t_2&\mapsto t_1\\
                    i_1&\mapsto i_3i_1i_3 = t_2i_1 = i_2\\
                    i_2&\mapsto i_1\\
                    t_3&\mapsto i_3\\
                \end{align*}
                \hrule
                \vspace{2ex}
                \begin{align*}
                    t_1&\mapsto t_2,\quad t_2\mapsto t_1,\quad i_1\mapsto i_3\\
                    i_2&\mapsto i_2,\quad i_3\mapsto i_1\\
                \end{align*}
                Это отображение совпадет с сопряжением по $i_2$, так как:
                \begin{align*}
                    S_{t_2}:\quad t_1&\mapsto i_2t_1i_2=i_1i_2=t_2\\
                    t_2&\mapsto t_1\\
                    i_1&\mapsto i_2i_1i_2 = t_1i_2 = i_3\\
                    i_2&\mapsto i_2\\
                    t_3&\mapsto i_1\\
                \end{align*}

                Так как каждое отображение совпало с сопряжением, то они автоморфизмы.
                Больше автоморфизмов по построению нет. $S_\cdot:\;\text{Aut}
                (S_3)\longleftrightarrow S_3$

            \item $\text{Aut}(S_4)$. Покажем, что группа $S_4$ совершенна, то
                есть, что все её автоморфизмы на самом деле сопряжение. В $S_4$
                6 транспозиций и ещё 3 инволюции, которые $(2,2)$-циклы. Очевидно,
                что инволюции должны отправляться в инволюции. Если
                транспозицию применить к $(3)$-циклу, то может произойти 2
                вещи $(abc)(ab) = (cb)$ или $(abc)(cd)=(abdc)$, мы получим или
                элемент порядка 2 или порядка 3. Если $(2,2)$-цикл применить к
                $(3)$-циклу, то $(abc)(ab)(cd)=(bdc)$ мы получим элемент порядка 3.
                Так что транспозиции не могут быть отправлены в $(2,2)$-циклы,
                а только в транспозиции, за неимением других инволюций.

                Теперь проверим, что если транспозиции отправляются в транспозиции,
                то это сопряжение. Пусть автоморфизм отправляет $(ab)\mapsto(a'b')$.
                Тогда если $(cb)\mapsto(c'b'')$, $c\neq a$ то раз $(ab)$ и $(cb)$
                не коммутируют, то не коммутируют и $(a'b')$ и $(c'b'')$, а
                значит c точностью до перестановок $c'$ и $b''$ либо $b''=a'$,
                либо $b''=b'$, зафиксируем второе. Пусть теперь $a\neq d\neq c$,
                тогда $(db)\mapsto (d'b''')$, так как прообраз не коммутировал
                с $(ab)$ и $(cb)$, то и образ тоже не будет. Тогда $(d'b''')$
                совпадет с $(a'b')$ и $(c'b')$ ровно по 1 букве. А значит будет
                либо $b'\in\{d',b'''\}$ и мы положим $b'''=b'$, либо $\{c',a'\}
                =\{d',b'''\}$, но в этом случае $(c'a')(a'b')(c'b')=(a'b')$,
                a комбинация их прообразов $(db)(ab)(cb)=(dacb)$, так что
                второй случай не возможен. Из этого мы заключаем,
                что во всех остальных транспозициях с $b$ все их образы содержат
                $b'$. Такое соответствие задаст перестановку, так как оно
                инъективно, потому как для случая $\geq 4$ для $b$ и $d$ можно
                найти коммутирующие транспозиции $(ab)$ и $(cd)$, их образы не
                будут пересекаться по действию, а значит $b'\neq d'$ Назовём
                эту перестановку $g$.  Тогда построим автоморфизм $S$ по ней. 
                \begin{center}
                    \begin{tikzcd}
                        S_n\arrow{r}{\cdot h} \arrow[swap]{d}{\cdot g} & S_n \arrow{d}{\cdot g} \\
                        S_n \arrow{r}{\cdot (h)S} & S_n
                    \end{tikzcd}
                \end{center}
                То есть, если нужно понять куда перейдет $a$ под действием
                $(h)S$, $(h)S=(..ab..)$ и $h=(..a'b'..)$, где $a'g=a$ и $b'g=b$,
                то $a(h)S=a(..ab..)=b=b'g=a'hg=ag^{-1}hg$, а это сопряжение по
                $g^{-1}$. Тогда все автоморфизмы - сопряжения. Покажем, что
                сопряжение по разным элементам различны. Две перестановки
                различаются минимум по 2м индексам $ig\neq ih$ и $jg\neg jh$.
                Тогда очевидно, что транспозиция этих 2х индексов $(ij)$ при
                каждом сопряжении о $g$ или $h$ переходит в разные элементы.
                Описано.
        \end{itemize}
    \item \textbf{Приведите пример группы $G$, обладающей автоморфизмом, не
        являющимся внутренним.}

        Пусть $G=\mathbb{Z}/3\mathbb{Z}$ - коммутативна, тогда все сопряжения
        очевидно слипаются. При этом есть нетривиальный автоморфизм:
        \begin{align*}
            0&\mapsto 0\\
            1&\mapsto 2\\
            2&\mapsto 1\\
        \end{align*}
    \item \textbf{Пусть $n$ и $m$ – целые положительные взаимно простые числа.
        Постройте изоморфизм групп $\mathbb{Z}/nm\mathbb{Z}\cong Z/n\mathbb{Z}
        \times\mathbb{Z}/m\mathbb{Z}$.}

        Построим
        \begin{align*}
            \varphi:Z/n\mathbb{Z}\times\mathbb{Z}/m\mathbb{Z}&\longrightarrow\mathbb{Z}/nm\mathbb{Z}\\
            ([k]_n,[l]_m)&\mapsto [km+ln]_{mn}\\
        \end{align*}
        Это морфизм, так как $\varphi([a+b],[c+d])=[(a+b)m+(c+d)n]=[am+cn+bm+dn]
        =\varphi([a],[c])+\varphi([b],[d])$. Он сюрьективен, так как выражение
        $km+ln$ принимает все возможные значения в $\mathbb{Z}$, и так как это
        отображение между множествами одного конечного порядка, то он и биективен.
        Изморфизм построен.
    \item \textbf{Классифицируйте с точностью до изоморфизма все группы $G$
        такие, что $G$ содержит не более 6 элементов.}
        \begin{itemize}
            \item $|G|=1$, очевидно единственная группа $1$.
            \item $|G|=p$, порядок прост, единственная группа $\mathbb{Z}/p\mathbb{Z}$.
            \item $|G|=4$
                \begin{enumerate}
                    \item $a$, $b$, $c$ - инволюции и пусть $ab=c$.\\
                            \noindent\begin{tabular}{c | c c c}
                                $e$ & $a$ & $b$ & $c$  \\
                                \cline{1-4}
                                $a$ & $e$ & $c$ & $b$ \\
                                $b$ & $c$ & $e$ & $a$ \\
                                $c$ & $b$ & $a$ & $e$ \\
                            \end{tabular}\\
                        Видно, что эта группа изоморфна $V_4$.
                    \item Есть элемент порядка 4, тогда это $\mathbb{Z}/4\mathbb{Z}$.
                \end{enumerate}
            \item $|G|=6$
                \begin{enumerate}
                    \item Если мы нашли элемент порядка 6, то это $\mathbb{Z}/6\mathbb{Z}$.
                    \item Если в группе все элементы инволюции. Тогда инволюция
                        своим действием на нетривиальные элементы группы
                        свяжет их в пары, то есть $ba=c$, $ca=b$, $da=f$,
                        $fa=d$.
                        \begin{align*}
                            ba=c&\Rightarrow a=bc&\Rightarrow ac=b\\
                            ca=b&\Rightarrow a=cb&\Rightarrow ab=c\\
                            da=f&\Rightarrow a=fd&\Rightarrow ad=f\\
                            fa=d&\Rightarrow a=df&\Rightarrow af=d\\
                        \end{align*}
                        \begin{center}
                        \noindent\begin{tabular}{c | c c c c c}
                            $e$ & $a$ & $b$ & $c$ & $d$ & $f$ \\
                            \cline{1-6}
                            $a$ & $e$ & $c$ & $b$ & $f$ & $d$\\
                            $b$ & $c$ & $e$ & $a$ & $\cdot$ & $\cdot$\\
                            $c$ & $b$ & $a$ & $e$ & $\cdot$ & $\cdot$\\
                            $d$ & $f$ & $\cdot$ & $\cdot$ & $e$ & $a$\\
                            $f$ & $d$ & $\cdot$ & $\cdot$ & $a$ & $e$\\
                        \end{tabular}
                        \end{center}
                        Но дальше заполнить таблицу нам не удастся, так
                        как $db$, если смотреть на столбец может быть равен или
                        $d$ или $f$, а если смотреть на строчку, то либо $b$,
                        либо $c$.
                    \item Тогда в группе можно найти элемент порядка 3 $b$.
                        Порядок группы четен, тогда в группе найдётся инволюция,
                        так как в противном случая $G=\{e,a,a^{-1},\ldots ,b,b^{-1}\}$
                        и её порядок нечетен. Пусть $a$ - инволюция.
                        Тогда в группе есть как минимум $a$, $b$, $b^2$, $ab$,
                        $ab^2$. $H=<b>$ нормальна, так как у неё всего 2 класса
                        смежности, а значит $H=gH\sqcup H=Hg\sqcup H$, то есть
                        $gH=Hg$ для любого $g$, а значит $gHg^{-1}=Hgg^{-1}=H$.
                        Тогда возможно 2 случая:
                    \begin{itemize}
                        \item $aba=b$. Тогда $ab^2=aabaaba=b^2a$, $ab=aaba=ba$
                            и $ab^2a=aabaabaa=b^2$. 
                        \begin{center}
                        \noindent\begin{tabular}{c | c c c c c}
                            $e$ & $a$ & $b$ & $b^2$ & $ab$ & $ab^2$ \\
                            \cline{1-6}
                            $a$ & $e$ & $ab$ & $ab^2$ & $b$ & $b^2$\\
                            $b$ & $ab$ & $b^2$ & $e$ & $ab^2$ & $a$\\
                            $b^2$ & $ab^2$ & $e$ & $b$ & $a$ & $ab$\\
                            $ab$ & $b$ & $ab^2$ & $a$ & $b^2$ & $e$\\
                            $ab^2$ & $b^2$ & $a$ & $ab$ & $e$ & $b$\\
                        \end{tabular}
                        \end{center}
                            Если приглядеться, то мы получили $\mathbb{Z}/n\mathbb{Z}$,
                            так как например у $ab$ порядок 6.
                        \item $aba=b^2$. Тогда $ab^2a=aabaa=b$, $ba=abbaa=ab^2$ и
                            $bab=ab^2b=a$, $b^2a=abaa=ab$.
                        \begin{center}
                        \noindent\begin{tabular}{c | c c c c c}
                            $e$ & $a$ & $b$ & $b^2$ & $ab$ & $ab^2$ \\
                            \cline{1-6}
                            $a$ & $e$ & $ab$ & $ab^2$ & $b$ & $b^2$\\
                            $b$ & $ab^2$ & $b^2$ & $e$ & $a$ & $b^2$\\
                            $b^2$ & $ab$ & $e$ & $b$ & $ab^2$ & $ab$\\
                            $ab$ & $b^2$ & $ab^2$ & $a$ & $e$ & $b$\\
                            $ab^2$ & $b$ & $a$ & $ab$ & $b^2$ & $e$\\
                        \end{tabular}
                        \end{center}
                        У нас три инволюции $a$, $ab$, $ab^2$. И два элемента
                        порядка 3, что очень похоже на $S_3$. Сопоставив
                        \begin{align*}
                            a&\mapsto(12)\quad ab\mapsto(23)\quad ab^2\mapsto(13)\\
                            b=a(ab)&\mapsto(12)(23)=(321)\quad b^2=a(ab^2)\mapsto(12)(13)=(123)\\
                        \end{align*}
                            мы получим изоморфизм. Так как $aba=(23)(12)=(123)=
                            b^2$.
                    \end{itemize}
                \end{enumerate}
        \end{itemize}
    \item \textbf{Назовем множество элементов $H=\{g1,g2,\ldots\}$ группы $G$
        порождающими, если каждый элемент из $G$ можно записать в виде
        произведения элементов из $H$ и обратных к ним. Будем говорить, что $G$
        конечно порождена, если множество $H$ можно выбрать конечным. Является
        ли группа $(\mathbb{Q},+)$ конечно порожденной?}

        Пусть $H=\{\frac{p_1}{q_1},\ldots,\frac{p_n}{q_n}\}$, что $p_i\wedge
        q_i=1$. Возьмём $q=q_1\ldots q_n+1$. Если $<H>=\mathbb{Q}$, то
        \begin{align*}
            \frac{1}{q}&=n_1\frac{p_1}{q_1}+\ldots+n_n\frac{p_n}{q_n}\\
            1&=q(n_1\frac{p_1}{q_1}+\ldots+n_n\frac{p_n}{q_n})\\
            q_1\cdot\ldots\cdot q_n&=q(\sum_in_ip_i\prod_{j\neq i}q_i)\\
        \end{align*}
        но это противоречит с фактом, что $q\wedge q_1\ldots q_n=1$.
        Так что $<H>\neq\mathbb{Q}$ и группа $\mathbb{Q}$ не конечно порожденная.
    \item \textbf{Элемент симметрической группы $S_n$, то есть группы биекций
        $n$-элементного множества $X$, называется транспозицией, если он меняет
        местами два элемента $X$, а остальные элементы $X$ оставляет на месте.
        Покажите, что транспозиции порождают $S_n$.}

        Пусть $g\in S_n$, тогда $g$ при действии на $\{1,\ldots,n\}$ разобьёт
        его на циклы. Циклы не будут
        пересекаться, так как иначе было бы $ag=bg$ и действие $g$ не было бы
        инъективным. Тогда g разбивается на произведение циклов. Эти циклы
        очевидно коммутируют, так как действуют дизъюнктивно. Каждый цикл
        раскладывается в произведение транспозиций. $(a_1a_2\ldots a_k)=
        (a_ka_{k-1})\ldots(a_2a_1)$.
    \item \textbf{Верно ли, что всякая подгруппа $H$ в прямом произведении
        групп $G_1\times G_2$ имеет вид $H_1\times H_2$, где $H_i$ это подгруппа
        в $G_i$ для $i=1,2$?}

        $V_4=\mathbb{Z}_2\times\mathbb{Z}_2$. Подгруппы множителей $0,
        \mathbb{Z}_2\leq\mathbb{Z}_2$. При этом $<(1,1)>$ имеет порядок 2 и
        не равна $0\times\mathbb{Z}_2$ или $\mathbb{Z}_2\times 0$.
    \item \textbf{(“Обращение теоремы Лагранжа”) Пусть n делится на m.
        Может ли в группе порядка n не быть подгруппы порядка m?}

        $|A_4|=12$. А этой группе нет элементов порядка 6, так что
        $\mathbb{Z}_6$ не $\hookrightarrow A_4$. Другая группа порядка 6 -
        $S_3$. В ней 3 инволюции и они не коммутируют. В группе $A_4$ тоже
        три инволюции, но они коммутируют:
        \begin{align*}
            (12)(34)(13)(24)&=(14)(23)\\
            (13)(24)(12)(34)&=(14)(23)\\
        \end{align*}. Так что в $A_4$ нет подгрупп порядка 6.
    \item \textbf{Модулярной группой назовем множество матриц
        \[
            \textnormal{SL}_2(\mathbb{Z}) = \left\{
                \left(\begin{array}{cc} a & b\\ c & d\end{array}\right)
                |\, a,b,c,d\in\mathbb{Z},\,ad-cd=1
            \right\}
        \]
        с операцией умножения. Проверьте аксиомы группы. Докажите, что 
        модулярная группа порождается матрицами
        \[
            R = \left(\begin{array}{cc} 0 & -1\\ 1 & 1\end{array}\right),\quad
            S = \left(\begin{array}{cc} 0 & -1\\ 1 & 0\end{array}\right).
        \]}
        \begin{itemize}
            \item Умножение любых матриц, подходящего размера ассоциативно, так
                как
                \begin{align*}
                    AB&=\left[\sum_ka_k^ib_j^k\right]_j^i,\\
                    A(BC)&=\left[\sum_ka_k^i\left[BC\right]_j^k\right]_j^i\\
                    &=\left[\sum_ka_k^i\sum_lb_l^kc_j^l\right]_j^i\\
                    &=\left[\sum_k\sum_la_k^ib_l^kc_j^l\right]_j^i\\
                    &=\left[\sum_l\sum_ka_k^ib_l^kc_j^l\right]_j^i\\
                    &=\left[\sum_l\left[\sum_ka_k^ib_l^k\right]c_j^l\right]_j^i\\
                    &=\left(AB\right)C
                \end{align*}
            \item Единицей очевидно будет $\left(\begin{array}{cc}1 & 0\\0 & 1
                \end{array}\right)$.
            \item Обратной к модулярной матрице $\left(\begin{array}{cc}a & b\\
                    c & d\end{array}\right)$ как нетрудно заметить будет
                    $\left(\begin{array}{cc}d & -b\\-c & a\end{array}\right)$.
        \end{itemize}
        Поэтому $\text{SL}_2(\mathbb{Z})$ – группа.

        Пусть $\left(\begin{array}{cc}a & b\\ c & d\end{array}\right)\in
        \text{SL}_2(\mathbb{Z})$. Тогда эту матрицу можно свести к единичной,
        умножая её на $R$ и $S$ справа, действуя по следующему рекурсивному
        алгоритму:

        Если $a$ и $b$ одного знака, то домножим матрицу на $R$
        \[\left(\begin{array}{cc}a & b\\c & d\end{array}\right)\cdot
          \left(\begin{array}{cc}0 & -1\\1 & 1\end{array}\right)=
          \left(\begin{array}{cc}b & b-a\\d & d-c\end{array}\right)\]
        Заметим, что таким образом модуль чисел в первой строчке будет на
        следующем шаге меньше.

        Если $a$ и $b$ разных знаков, то мы домножим матрицу на $S$
        \[\left(\begin{array}{cc}a & b\\c & d\end{array}\right)\cdot
          \left(\begin{array}{cc}0 & -1\\1 & 0\end{array}\right)=
          \left(\begin{array}{cc}b & -a\\d & -c\end{array}\right)\]
        И теперь в первой строке числа одного знака.

        Продолжая эту процедуру, мы получим матрицу вида $\left(\begin{array}{cc}
        a&  0\\c & d\end{array}\right)$, она модулярна, а значит $ad=1$, так как
        числа целые, то $a=\pm1=d$. Дальше, мы повторим ту же процедуру для
        нижней строки, так как в верхней были $\pm1$ и $0$, то они там так и
        останутся. Тогда в конце алгоритма у нас будет матрица вида $\left(
        \begin{array}{cc}0 & \pm1\\\mp1 & 0\end{array}\right)$, но это явно
        либо $S$, либо $S^{-1}$. Домножив на подходящую, мы получим единичную.
        Тогда если перемножить обратные ко всём, на что мы умножали в обратном
        порядке, то мы получим разложение изначальной матрицы.
\end{enumerate}
\end{document}
