\documentclass[a4paper, 12pt]{book}
\usepackage{amsmath}
\usepackage{amssymb}
\usepackage{hyperref}
\usepackage[russian]{babel}

\usepackage{tikz-cd}
\usepackage{array}
\usepackage{graphicx}
\newcommand\mapsfrom{\mathrel{\reflectbox{\ensuremath{\mapsto}}}}
\newcommand{\mymat}{\mathcal{M\mkern-3mu a\mkern-0.3mu t}}

\usepackage{fontspec}
\setmainfont{Linux Libertine O}
\usepackage{unicode-math}
\setmathfont{Cambria Math}

\title{
\textit{\huge{НМУ Алгебра I\\Константин Логинов}}
}

\date{2024}

\author{Потошин Георгий}

\begin{document}
\maketitle
\chapter{Векторные пространства}
\section{Жорданова нормальная форма}
Матрица называется жордановым блоком, если она имеет вид

\[J_k(\lambda)=\left(\begin{array}{ccccc}
    \lambda & 1       & 0      & \cdots  & 0\\
    0       & \lambda & 1      & \ddots  & \vdots\\
    \vdots  & \ddots  & \ddots & \ddots  & \vdots\\
    0       & \cdots  & 0      & \lambda & 1\\
    0       & \cdots  & \cdots & 0       & \lambda\\
\end{array}\right)\]

Болок размера $k\times k$ c $\lambda$ на диагонали и с 1 над диагональю. В
прошлый раз мы доказали, что для любого линейного эндоморфизма векторных
конечномерных пространств над алгебраически замкнутым полем есть базис, в
котором матрица имеет болочно диагональный вид, c жордановыми блоками.

\textit{Поле называется алгебраически замкнутым, если каждый многочлен над этим
полем положительной степени имеет корень.}
\[
    \left(\begin{array}{cccc}
        J_{k_1}(\lambda_1) &                    &        & 0\\
                           & J_{k_2}(\lambda_2) &        & \\
                           &                    & \ddots & \\
        0                  &                    &        & J_{k_n}(\lambda_n)\\
    \end{array}\right)
\]

Стоит отметить, что $\lambda_i$ и $k_i$

\textbf{Пример:} Пусть полем будет $\mathbb{k}=\mathbb{R}$, а пространством
$V=\mathbb{R}^2$. Зметим, что $x^2+1$ неприводим в этом поле. Тогда возьмём
оператор поворота на 90 градусов.

\[A=\left(\begin{array}{cc}0 & -1\\ 1 & 0\end{array}\right)\]

Для неё нет жордановой нормальной формы над $\mathbb{R}$, так как у неё нет
собственных значений. Если бы они были, то были бы корнем характеристического
многочлена $\chi_A(t)=t^2+1$, а у него корней нет. Над $\mathbb{C}$, наш
оператор приводим, так как $\pm\sqrt{-1}$ его собственные значения, а тогда

\[A=\left(\begin{array}{cc}\sqrt{-1} & 0\\ 0 & -\sqrt{-1}\end{array}\right)\]

Заметим, что по жордановой нормальной форме легко вычислять инварианты, так как
след – сумма диагональных элементов, $\text{tr}(A)=\sum k_i\lambda_i$.

\textbf{Замечание:} базис, в котором оператор имеет жорданову нормальную форму,
вообще говоря не единственен, например тривиальный оператор $I$.

Тем не менее кое-что определено канонически. Давайте бозначим за $n_{\lambda,k}$
– количество клеток вида $J_k(\lambda)$ в нашей матрице.

\textbf{Утверждение:}
\[\sum_{p=1}^k pn_{\lambda, p}+\sum_{p=k+1}^{\inf}kn_{\lambda, p} =
\text{dimKer}(A-\lambda\text{Id})^k,\;\forall\lambda,k\]
Следовательно, $n_{\lambda, k}$ – инвариаенты $A$.

Для доказательства, давайте запишем матрицу в жордановой нормальной форме и 
посчитаем ядро $\text{dimKer}(A-\text{Id})^\lambda$. В таком виде нас будут
интересовать только клетки, в которых стоит $\lambda$. Тогда можно предполагать,
что оператор состоит только из клеток с $\lambda$. Если посмотреть на то, что
происходит с клетками, то мы увидем
\[J_k(\lambda)-\lambda\text{Id}=\left(\begin{array}{cccc}
    0 & 1      &        & 0\\
      & \ddots & \ddots & \\
      &        & \ddots & 1\\
    0 &        &        & 0
\end{array}\right)\]
И если мы возведем в степень такие клетки, то равенство станет очевидным.

\textbf{Замечание:} Пусть $A\in\text{End}(V)$. Заметим, что задать оператор $A$,
равносильно заданию на $V$ структуры $\mathbb{k}[t]$-модуля. Структура
$\mathbb{k}[t]$-модуля это в точности $\mathbb{k}$-модуль с действием $t$.
Зададим это действие следующим образом $t^l\cdot v=A^l(v),\,v\in V$ и продолжим
его по линейности. В обратную сторону, мы зададим оператор через действие $t$,
то есть $A(v)=t\cdot v$. И это также эквивалентно заданию гомоморфизма (колец?)
$\phi: \mathbb{k}[t]\rightarrow\text{End}(V)$, где образ $t$ будет оператором
$A$. (Скорее всего это работает только в коммутативном случае, когда на $\text{End}(V)$
Есть структура модуля и я бы брал гомоморфизмы модулей!).

Например если $A=J_k(\lambda)$, то $V\cong\mathbb{k}[t]/(t-\lambda)^k$. Давайте
поймём почему этот изоморфизм имеет место. Нам нужно вопервых убедится, что они
изоморфны как $\mathbb{k}$-векторные пространства, а во вторых, что $A$
действует в $V$ также как $t$ умножением в $\mathbb{k}[t]/(t-\lambda)^k$. Первое
верно из наблюдения размерности, в обоих случаях она $k$. Для воторого, нужно
понять как $A-\lambda\text{Id}$ дей ствует на базисные вектора, а именно $e_1
\mapsto 0$ и $e_{i+1}\mapsto e_i$ для $1\leq i \le k$. Заметим, что
$\{(t-\lambda)^i\}_{0\leq i\le k}$ $\mathbb{k}$-базис фактор кольца, и в нём
$t-\lambda$ умножением действует точно также на элементы кольца, а значит у нас
есть изоморфизм $\mathbb{k}[t]$-модулей.

\textbf{Следситвие (из теоремы о существовании ЖНФ)}
Для $A\in\text{End}(V)$, V – $\mathbb{k}[t]$-модуль. То $V\cong_{\mathbb{k}[t]}
\bigoplus_{i=1}^N\mathbb{k}[t]/(t-\lambda_i)^{k_i}$, где действие $A$
соответствует действию $t$, а сумма идёт по жордановым блокам. Это верно, так
как матрица оператора блочно диагональная, а значит пространство раскладывается
в прямую сумму подпространств, так, что на каждом подпространстве наш оператор
деуствует как жорданов блок, а тогда применив предыдущий результат, мы получаем
искомое. Такая формулировка теоремы о жордановой нормальной форме более
правильная, так как она имеет обобщения, то есть на классификацию конечно
порожденных модулей. В частности классификация конечных и конечно порожденных
абелевых групп.

\textbf{Определение:} $A\in\text{End}(V)$ называется полупростым, если
существует базис, в котором матрица $A$ диагональна. $A$ называется
нильпотентом, если $A^m=0$ для $m>1$.

\textbf{Следствие (из ЖНФ):} $A\in\text{End}(V)$, то $A=A_{ss}+A_n$, где
$A_{ss}$ – полупрост, а $A_n$ – нильпотент. И эти два оператора коммутируют.
\[J_k(\lambda)=\lambda\text{Id}+\left(\begin{array}{cccc}
    0 & 1      &        & 0\\
      & \ddots & \ddots &  \\
      &        & \ddots & 1\\
    0 &        &        & 0
    \end{array}\right)\]

\textbf{Теорема (Гамильтона-Кэли):}
 $A\in\text{End}(V)\Rightarrow\chi_A(A)=0$. Поле не обязательно алгебраически
 замкнуто. $\chi_{J_k(\lambda)}(t)|_{t=a}=(t-\lambda)^k|_{t=A}=(A-\lambda)^k=0$.
 А значит в каждом блоке будет 0, теорему доказали, но жульничество в том, что
 нам необходима алгебраическая замкнутость поля, но жульничество можно обойти,
 показав, что каждое поле вложено в алгебраически замкнутое. 

 \textbf{Доказательство:}\\$(tE-A)(\widehat{tE-A})=(\widehat{tE-A})(tE-A)=\chi_A(t)
 \text{Id}$ в кольце $\mymat_{n\times n}(\mathbb{k}[t])=(\mymat_{n\times n}
 (\mathbb{k}))[t]$. Определим отображение \[\phi: R\rightarrow\mymat_{n\times n}
 (\mathbb{k}),\] где $R=Z_A(\mymat_{n\times n}(K)[t])$, а устроено оно
 вычислением в $A$, то есть $\phi(\sum B_it^i)=\sum B_iA^i$, где $B_i\in
 \mymat_{n\times n}(\mathbb{k})$. Заметим, что $\phi$ является гомоморфизмом.

 $\chi_A(A)=\phi(\chi_A(t)E)=\phi((\widehat{tE-A})(tE-A))=\phi(\widehat{tE-A}
 \phi(tE-A)=\phi(\widehat{tE-A})(A-A)=0$.

 \textbf{Замечание:} $A\in\text{End}(V)$ задание эндоморфизма эквивалентно заданию
 гомоморфизма $\phi:\mathbb{k}[t]\rightarrow\text{End}(V)$. По теореме
 Гамильтона-Кэли мы знаем, что $\chi_A(t)\in\text{Ker}(\phi)$. С другой стороны
 $\text{Ker}(\phi)=(m_A(t))$, тогда можно определить $m_A$ минимальный многочлен
 оператора $A$, минимальный многочлен оператора $A$, он определен однозначно,
 если страший коэффициент брать за 1. Заметим, что минимальны многочлен делит
 характеристичесий.

 \textbf{Упражнение:} Существует $N$, что $\chi_a(t)\,|\,m_a(t)^N$.

 \textbf{Пример:}
 \begin{itemize}
     \item $m_A(t)=t-\lambda$, для $A=\lambda E$. Тогда $\chi_A(t)=(t-
         \lambda)^k$.

     \item $m_A(t)=t^k$, тогда $A-нильпотент$ и $\chi_A(t)=t^n$. Можно взять
         нулевой жордановый блок и нулевуюматрицу и соединить их в блочно
         диагональной маньере.

     \item Если $m_A(t)=(t-1)^k$, то $A$ называется унипотентом.
         
     \item Если $m_A(t)=t(t-1)$, то $A$ – проектор. Идемпотентен
 \end{itemize}
 \chapter{Поля и их расширения}
 Пусть $\mathbb{k}$ – поле. Тогда можно рассмотреть гомоморфизм $\varkappa:
 \mathbb{Z}\rightarrow\mathbb{k}, 1\mapsto 1$, у него есть ядро $\text{Ker}
 (\varkappa)\subseteq\mathbb{Z}$, это идеал в $\mathbb{Z}$, он главный, так как
 идеал кольца главных идеалов, пусть он равен $(d)$.
 
 \textbf{Утверждение:} $d$ – простое число или 0.

 \textbf{Доказательство:} Ядро – прообраз простого идеала, а значит ядро просто.

 \textbf{Определение:} $d$ – характериситка $\mathbb{k}$, её мы обозначаем
 $\text{char}(\mathbb{k})=d$, то есть простое число или 0, которое однозначно
 определяется по полю.

 \begin{itemize}
     \item $\text{char}(\mathbb{Q})=0$
     \item $\text{char}(\mathbb{Z}/p\mathbb{Z})=p$
 \end{itemize}

 \textbf{Напоминание:} Если $f:\mathbb{K}\rightarrow\mathbb{L}$ гомоморфиз
 полей, то он инъективен. Так как несобственный идеал только 0.

 $A=\text{Im}(\varkappa)$ – область целостности. Тогда можно рассмотреть поле
 частных $\text{Frac}(A)\leq\mathbb{k}$, подполе в $\mathbb{k}$, оно назаватеся
 простым подполем.
 \[\text{Frac}(A)\cong \left\{\begin{array}{rcl}\mathbb{Q} & \text{char}(\mathbb{k})=0\\
                                                \mathbb{Z}/p\mathbb{Z} &\text{char}(\mathbb{k})=p
                              \end{array}\right.\]
Простое подполе определено однозначно, так как гомоморфизм $\varkappa$ определен
однозначно, канонически. Оно называется простым, так как в нём нет собственных
подполей.

\textbf{Утверждение:} Пусть $f:\mathbb{K}\rightarrow\mathbb{L}$ – гомоморфизм
полей. Тогда $\text{char}(\mathbb{K})=\text{char}(\mathbb{L})$ и $f$ индуцирует
изоморфизм простых подполей в $\mathbb{K}$ и $\mathbb{L}$.

\textbf{Доказательство:}
\begin{center}
\begin{tikzcd}
    \mathbb{Z}\arrow{r}{\varkappa_K} \arrow[bend right]{rr}{\varkappa_L} & \mathbb{K}\arrow{r}{f} & \mathbb{L}
\end{tikzcd}
\end{center}
Давайте тогда заметим, что композиция является гомоморфизмом $\varkappa$ для
$\mathbb{L}$, так как композиция перводит единицу в единицу. Отсюда следует,
что ядро $\varkappa_L$ равно ядру $\varkappa_K$, так как $f$ вложение. Более
того $\text{Im}(\varkappa_K)\cong_f\text{Im}(\varkappa_L)$, а значит простые
подполя изоморфны, а характеристики равны.

\textbf{Определение:} $K\leq L$ называется расширением полей, если $K
\hookrightarrow L$, то есть следующий набор данных, поле $K$, поле $L$ и
вложение. Иногда это обозначается $(L/K)$ и черта читается как "над".

Если $K\leq L$, то $L$ является векторным пространством над $K$. Тогда можно
говорить о размерности $L$ над $K$ и если $\text{dim}_K\,L\le\infty$, то
расширение мы называем конечным, а размерность мы будем писать чуть иначе
$\text{dim}_K\,L=[L:K]$.

$K_1\le K_2\le\ldots\le K_s$ мы называем башней полей, а расширение $K_i\le
K_{i+1}$ – этаж этой башни.

\textbf{Пример:} $\mathbb{Q}\le\mathbb{R}\le\mathbb{C}$ в этой башне только
второй этаж конечен.

\textbf{Утверждение:} Если $F\le K\le L$, то $[L:F]=[L:K][K:L]$

\textbf{Доказательство:} Пусть $K=\langle x_i\rangle_F,\,x_i\in K$, где
$\{x_i\}$ базис $K$ над $L$ и пусть $L=\langle y_j\rangle_K,\,y_j\in L$, где
$\{y_j\}$ базис $L$ над $F$. Тогда мы можем построить базис $L$ над $F$, а
именно $L=\langle x_iy_j \rangle_F$ поверим это. Пусть $a\in L$, тогда его
можно разложить над $\{y_j\}$, то есть $a=\sum a_jy_j,\,a_j\in K$. Но тогда
$a_j$ можно разложить над $\{x_i\}$, то есть $a_j=\sum a_{i,j}x_i,\,a_{i,j}\in
F$, а тогда $a=\sum a_{i,j}x_iy_j$, что означает $\{x_iy_j\}$ порождает $L$ над
$F$.

Пусть теперь $\sum a_{i,j}x_iy_j=0$, пойдя в обратную сторону и положив $a_j=
\sum a_{i,j}x_i\in K$, мы получи $\sum a_jy_j=0$, а тогда по свойству базиса
$\{y_j\}$ получим $a_j=0$, но тогда и $\sum a_{i,j}x_i=0$, и по свойству базиса
$\{x_i\}$ получим $a_{i,j}=0$, что означает линейную независимость $\{x_iy_j\}$,
тогда это и вправду базис и его кординал равен произведению кординалов базисов
$\{x_i\}$ и $\{y_j\}$.

\textbf{Следствие:} Для конечной башни полей $K_1\le K_2\le\ldots\le K_s$
расширение $K_1\le K_s$ конечно, ттогда $K_i\le K_{i+1}$ конечны $\forall i$.

\textbf{Определение:} Пусть $K\le L$ расширение полей, элемент $0\neq\alpha\in
L$ называется алгебраичным, что $f(\alpha)=0$ для некоторого $0\neq f(x)\in K[x]$.
Расширение $K\le L$ называется алгебраичным, если $\forall\alpha\in L$ оно либо
нуль либо алгебраично.

\textbf{Утверждение:} Для любого конечного расширения $K\le L$ известно, что
оно алгебраично.

\textbf{Доказательство:} Для ненулевого $\alpha\in L$ элементы $1,\,\alpha,\,
\alpha^2,\,\ldots,\,\alpha^n$, где $n=[L:K]$, линейно зависимы, а значит
найдутся коэффициенты $a_i\in L$, что $a_0+a_1\alpha+\ldots+a_n\alpha^n=0$, 
а тогда можно положить $f(x)=a_0+a_1x+\ldots+a_nx^n\in K[x]$ и расширение
алгебраично.

Обратное не верно, так как бывают бесконечные алгебраические расширения.

Пусть $K\le L$ – расширение полей, тогда для любого $\alpha\in L$ можно
устроить гомоморфизм колец 
\begin{align*}
    \phi_\alpha:&\;K[x]\longrightarrow L\\
    &\;g(x)\mapsto g(\alpha)
\end{align*}
Тогда обозначим целостное кольцо $K[\alpha]=\text{Im}\,\phi_\alpha\le L$, а его
поле частных мы обозначим за $K(\alpha)=\text{Frac}\,K[\alpha]$.

Заметим, что если $\alpha$ алгебраичен, то $\phi_\alpha$ не вложение.
Действительно, ядро не будет тривиальным по определению алгебраичного элемента.
Тогда $\text{Ker}\,\phi_\alpha\le K[x]$ является нетривиальным идеалом, но как
мы уже обсуждали многочлены над полем образуют кольцо главных идеалов, а значит
$\text{Ker}\,\phi_\alpha=(p(x))$ и $p(x)\neq 0$ и можем считать, что старший
коэффициент единица. Будем называть $p(x)$ минимальным многочленом $\alpha$ или
ниприводимый, то есть $\text{Irr}_\alpha^K(x)$.

\textbf{Утверждение:} Если $\alpha\in L$ алгебраичен над $K$, то $K(\alpha)=
K[\alpha]$, а также степень расширения полей равна $[K(\alpha):K]=\text{deg}\,
\text{Irr}_\alpha^K(x)$.

\textbf{Доказательство:} Обозначим $f(x)=\text{Irr}_\alpha^K(x)$. Пусть есть
некий ненулевой элемент $\beta\in K[\alpha]$, тогда мы найдем $g(x)\in K[x]$,
что $\beta=g(\alpha)$. Заметим, что $f(x)$ неприводим, так как прост. Тогда
$(f(x),g(x))=1$, так как $f(x)$ не может делить $g(x)$, в протвном случае мы бы
имели $\beta=g(\alpha)=kf(\alpha)=0$. Тогда мы можем найти соотношение Безу
$f(x)h(x)+g(x)s(x)=1$, подставим в него $\alpha$, тогда останется $g(\alpha)
s(\alpha)=0$, а значит $\beta s(\alpha)=1$ обратим, из этого заключаем, что
$K[\alpha]$ - поле и совпадает со своим полем частных $K(\alpha)$.

Заметим, что в $K(\alpha)=K[\alpha]=\langle1,\alpha,...,\alpha^{n-1}\rangle_K$
есть базис. Он порождает, так как старшие степени $\alpha$ могут быть вычислены
из тех, что мы выписали по минимальному многочлену и он линейно независим, так
как иначе мы бы нашли меньший многочлен, зануляющий $\alpha$, а у нас уже
наименьший.

\textit{С расширениями полей такая история, что начав изучать, невозможно остановиться}

\textbf{Следствие:} Пусть $K\le L$, $\alpha_i\in L$ – алгебраичны над $K$ и 
$L=K(\alpha_1,\ldots,\alpha_n)$, тогда степень расширения $[L:K]<\infty$. Под
$K(\alpha_1,\ldots,\alpha_n)$ можно понимать как минимальное поле, содержащее
все элементы в скобках, так и значения рациональных дробей многих переменных,
выичсленных в тех же элементах, без обращения знаменателя в ноль.

\textbf{Доказательство:}  Рассмотрим башню
\[ K\le K(\alpha_1)\le K(\alpha_1, \alpha_2)\le\ldots\le K(\alpha_1,\ldots,\alpha_n)\]

Заметим, что $\alpha_{i+1}$ алгебраичен над $K(\alpha_1,\ldots,\alpha_i)$, а значит
каждый этаж башни конечен, а тогда и $L/K$ конечно.

\textbf{Утверждение:} Пусть $F\le K\le L$ – башня полей, тогда $L/F$
алгебраично равносильно тому, что $K/F$ и $L/K$ алгебраичны.

\textbf{Доказательство} Если $L/F$ алгебраично, то для любого $\alpha\in K$, по
включению $\alpha\in L$, а значит $\alpha$ – корень некого многочлена $f(x)\in
F[x]$ и $K/F$ алгебраично. Точно также так как для любого $\alpha\in L$, есть
его зануляющий многочлен $f(x)\in F[x]$, то так как $F[x]\subseteq K[x]$, он
же является многочленом над $K$, а заначит $L/K$ алгебраично. Покажем теперь
импликацию в обратную стороную. Пусть $K/F$ и $L/K$ алгебраичны, тогда для 
$\alpha\in L$ мы найдем зануляющий многочлен $f(x)=x^n+a_nx^{n-1}+\ldots+a_1$
с коэффициентами в $K$. Тогда построим башню $F\le F(a_n,\ldots,a_1)\le
F(a_n,\ldots,a_1)(\alpha)$, здесь каждый этаж башни конечен, тогда конечна и
вся башня, а тогда $F(a_n,\ldots,a_1)(\alpha)/F$ конечно, а значит алгебраично,
а тогда $a$ алгебраично над $F$, а тогда и расширение $L/F$ алгебраично.

\textbf{Определение:} Поле $L$ алгебраически замкнуто, если для любого $f(x)\in
L[x]$ есть корень.

\textbf{Пример:} $\mathbb{C}$

\textbf{Утверждение:} Любое поле можно вложить в алгебраически замкнутое.

\textbf{План:} Пусть удалось построить башню полей
\[K\le K_1\le K_2\le K_2\le \ldots\]
с условием, что любой многочлен $f(x)\in K_i[x]$ имеет корень в $K_{i+1}$.
Тогда можно взять объединение $L=\bigcup_{i=1}^\infty K_i$. Это поле, так как
если $\alpha,\beta\in L$, то мы найдем $\alpha\in K_i$ и $\beta\in K_j$, то
можно выбрать номер побольше, что $\alpha,\beta\in K_{\max(i,j)}$ и там их уже
можно сложить, умножить, поделить, взять обратные, и так далее. Это поле будет
алгебраически замкнутым, так как если $f(x)=x^n+b_1x^{n-1}+\ldots+b_n\in L[x]$,
то найдутся индексы, что $b_j\in K_{i_j}$, тогда обозначим за $K_l=K_{\max_j(i_j)}$
и $f(x)\in K_l[x]$, а значит имеет корень в $K_{l+1}$.

Теперь давайте явно построим такую башню. Для этого опишем как по полю $F$
построить поле $\widetilde F$, что для любого многочлена над $F$, $f(x)\in
F[x]$ найдется корень в $\widetilde F$, тогда применяя бесконечное число раз
эту конструкцию можно построить эту башню. Рассмотрим $\Lambda=F[\{t_f\}_{f\in F[x]}]$
кольцо многочленов от бесконечного числа переменных, заиндексированных
многочленами от $x$ над $F$.

Давайте построим идеал $I=(f(t_f))_{f\in F[x]\setminus F}$. Покажем, что он собственный, то
есть что $\Lambda\neq I$. Если бы $I=\Lambda$, то $1\in I$, и $g_1f_1(t_{f_1})+
g_2f_2(t_{f_2})+\ldots+g_nf_n(t_{f_n})=1$

\textbf{Лемма:} Если $K$ поле и $f(x)\in K[x]\setminus K$, то всегда есть расширение $L$,
что $f(\alpha)=0,\,\alpha\in L$ многочлен в нем имеет корень.

Можно профакторизовать по неприводимому множителю.

Тогда по лемме есть поле $L$ в котором наудутся $\alpha_1,\ldots,\alpha_n\in L$,
что $f_i(\alpha_i)=0$, тогда подставив $t_{f_i}=\alpha_i$ мы слева получим 0,
а справа 1. Значит $I$ собственный, а значит он вложен в некий максимальный
идеал $\mathfrak{m}$. Тогда можно положить $\widetilde{F}=F[t_f]/\mathfrak{m}$.
В этом поле любой многочлен $f(x)$ имеет корень $[t_f]$. Поэтому у любого поля
есть алгебраически замкнутое надполе.

\textbf{Определение:}  $K\le \overline K$ называется алгебраическим замыканием,
если $\overline K$ алгебраически замкнуто и либой $\alpha\in \overline K$
алгебраичен ($K\leq\overline K$ алгебраическое расширение).

\textbf{Утверждение:} $\overline K$ существует (но и единственно)

\textbf{Доказательство:} $K\le L$, где $L$ - aлгебраически замкнуто, тогда
положим $\overline K= \{\text{Все элементы $\alpha\in L$, что $\alpha$ алгебраичен
над $K$}\}$. Проверим, что $\overline K$ – поле. Пусть $\alpha,\beta\in\overline
K$. Можно посмотреть на расширение $K\le K(\alpha,\beta)$ оно конечно, а значит
алгебраично, это значит, что $\alpha+\beta$, $\alpha\beta$, $\alpha/\beta$
алгебраичны над $K$, а значит лежат в $\overline K$. Теперь давайте увидим, что
$\overline K$ замкнуто, тогда пусть $f(x)\in\overline K[x]$ и мы хотим найти у
него корень в $\overline K$, но на него можно посмотреть как на многочлен над
$L$ и в $L$ у него есть корень $\alpha$, но $f=x^k+a_1x^{n-1}+\ldots+a_n$ и
в нём $a_i$ алгебраичны над $K$. Тогда можно посмотреть на следующую башню
\[K\le K(a_1,\ldots,a_n)\le K(a_1,\ldots,a_n)[\alpha]\]
В этой башне первый этаж конечен, воторой тоже, так как $\alpha$ зануляет $f(x)$,
а значит вся башня тоже конечна, а тогда $K(a_1,\ldots,a_n)[\alpha]/K$ конечно,
а значит алгебраично, а тогда алгебраичен и $\alpha$, то есть $\alpha\in
\overline K$ и $\overline K$ алгебраически замкнуто.

\textbf{Примеры:} $\mathbb{C}=\overline{\mathbb{R}}$, $\overline{\mathbb{Q}}=\{a\in
\mathbb{C}\,|\,a\text{ алгебраичен над }\mathbb{Q}\}$, a что можно сказать о
$\overline{\mathbb{F}_p}$?

\section{Поле разложения многочлена}

\textbf{Определение:} $L$ называется полем разложения многочлена $f(x)\in K[x]$,
если $K\le L$, $f(x)=c\prod_{i=1}^n(x-\alpha_i),\,\alpha_i\in L$ и $K(\alpha_1,
\ldots,\alpha_n)=L$.

Поле $L$ строится по полю $K$ и $f(x)\in K[x]$.

Поле $L$ существует, так как мы можем например найти все корни в $\overline K$,
выпишем эти корни $\alpha_i\in\overline K$. Тогда $L=K(\alpha_1,\ldots,\alpha_n)$.
Проверим однозначность конструкции, пусть $L'=K(\alpha'_1,\ldots,\alpha'_n)$
где $\alpha'_i\in L'$ лежат в каком-то другом поле. Тогда можно устроить
морфизм $\sigma:L'\mapsto\overline K,\,\sigma(\alpha'_i)=\alpha_i$. Проверим,
что это корректно [..?].

Пусть $\mathbb{F}_q$ – конечное поле с $q$ элементами, его характеристика может
быть равна только простому числу $p$, а тогда мы имеем вложение $\mathbb{F}_p
\hookrightarrow\mathbb{F}_q$ и $\mathbb{F}_q$ будет векторным пространством
над $\mathbb{F}_p$, а тогда $q=p^n$ может равнятся только степени $p$. Пусть
теперь есть поле $\mathbb{F}_q$ и посмотрим на многочлен $p(x)=x^q-x\in
\mathbb{F}_p[x]$. Пусть $\mathbb{F}_q^\times$ – мультипликативная группа, её
порядок $|\mathbb{F}_q^ \times|=q-1$, а это означает, что для любого $\alpha\in
\mathbb{F}_q^\times,\,\alpha^{q-1}-1=0$. А тогда нетрудно видеть, что любой
$\alpha\in\mathbb{F}$ является корнем $p(x)$. Тогда по теореме Безу $p(x)$
раскладывается на множители степени 1 над $\mathbb{F}_q$
\[x^q-x=\prod_{\alpha_i\in\mathbb{F}_q}(x-\alpha_i),.\]
Тогда можно посмотреть на вложение $\mathbb{F}_p(\alpha_1,\ldots,\alpha_q)\le
\mathbb{F}_q$ и оно тривиально является равенством, а тогда $\mathbb{F}_q$ –
поле разложения многочлена $x^q-x$.

Чем конечные поля замечательны, в теории полей, если есть расширение $K\le L$,
основной объект, который обычно изучают, это $\text{Aut}_K(L)$ автоморфизмы $L$
над $K$, те изоморфизмы поля $L$, что они сохраняют поле $K$. в конечном случае
автоморфимы легко посчитать $\text{Aut}_{\mathbb{F}_q}(\mathbb{F}_{p^n})=
\mathbb{Z}/n\mathbb{Z}$ циклическая группа, с образующей $\phi:\mathbb{F}_q
\rightarrow\mathbb{F}_q=x\mapsto x^p$.

\textbf{Утверждение:} $\phi$ - гомоморфизм (Фробениуса).

Пусть есть алгебраическое замыкание $\mathbb{F}_p\le\overline{\mathbb{F}_p}$.
Возьмём многочлен $x^{p^n}-x$, у него есть $\alpha_i\in\overline{\mathbb{F}_p}$
все корни, тогда мы возьмём $\mathbb{F}_p(\alpha_1,\ldots,\alpha_{p^n})$.
Осталось проверить, что в $\mathbb{F}_q=\mathbb{F}_p(\alpha_1,\ldots,\alpha_p^n)$
$q$ элементов, для этого перепишем многочлен через фробениуса $\phi(x)=x^p$, а
тогда $\phi^n(x)=x^{p^n}$. Тогда видно, что если $\alpha,\beta$ корни $x^q-x$,
то $\alpha+\beta$, $\alpha\beta$ и $\alpha/\beta$ корни, так как оперции
пропускаются через гомоморфизм Фробениуса. Единственная проблема, что в
$\overline{\mathbb{F}_p}$ может быть кратные корни, но кратность корня
эквивалентна тому, что это корень производнойб. но $(x^(p^n)-x)'=-1$ корней нет,
а значит всего $p^n$ различных корней.

Можно пойти по иному пути и факторизовать многочлены, но как бы мы не старались,
поле всегда будет полем разложения полинома $x^q-x$.

Давайте теперь убедимся, что автоморфизмы $\text{Aut}_{\mathbb{F}_p}(
\mathbb{F}_{p^n})$ пораждены автоморфизмом Фробениуса.
\end{document}
