\documentclass{article}
\usepackage[a4paper,left=3cm,right=3cm,top=1cm,bottom=2cm]{geometry}
\usepackage{amsmath}
\usepackage{amssymb}
\usepackage{hyperref}
\usepackage[russian]{babel}

\usepackage{tikz-cd}
\usepackage{array}
\usepackage{graphicx}
\newcommand\mapsfrom{\mathrel{\reflectbox{\ensuremath{\mapsto}}}}
\setlength{\parindent}{0mm}

\usepackage{fontspec}
\setmainfont{Linux Libertine O}
\usepackage{unicode-math}
\setmathfont{Cambria Math}

\title{
\textit{\small{Георгий Потошин, 2024}}\\
\vspace{0.3ex}
\textit{\huge{Алгебра I, листочек 2}}\vspace{1ex}
}

\date{\vspace{-10ex}}

\begin{document}
\maketitle

\begin{enumerate}
    \item \textbf{Докажите, что подгруппа индекса 2 нормальна.}

        Пусть $H\leq G$ – подгруппа индекса 2. Тогда у неё всего два правых и
        два левых смежных класса. $G=H\sqcup gH=H\sqcup Hg$ так как их всего
        двое и они оба дополнения к подгруппе, то они очевидно совпадут $gH=Hg$.
        Тогда $(gH)g^{-1}=(Hg)g^{-1}=H$, а значит $H$ нормальна в $G$.

    \item \textbf{Пусть дана цепочка подгрупп $K\leq H\leq G$. Покажите, что
        $(G:K)=(G:H)(H:K)$.}

        Обозначим за $\text{L}(G,H)$ множество левых классов смежности $H$ в $G$,
        а за $\text{TL}(G,H)$ eё трансверсаль. Тогда построим отображение:
        \begin{align*}
            \varphi:\;&\text{TL}(G,H)\times\text{TL}(H,K)\longrightarrow \text{L}(G, K)\\
            &(g,h)\mapsto ghK\\
        \end{align*}
        Покажем, что оно сюрьективно. Пусть $g\in G$, тогда $g=g'h$ для
        некоторого $g'\in\text{TL}(G,H)$ и $h\in H$. А $h=h'k$ для $h'\in
        \text{TL}(H,K)$, тогда $gK=g'h'K$. Теперь пусть $ghK=g'h'K$, так как
        $hK,h'K\in H$, то $gH\cap g'H\neq\varnothing$, a значит $g=g'$,
        сократим на него. $hK=h'K$, тогда $h=h'$, а значит $\varphi$ – инъективно. Тогда
        $(G:H)(H:K)=|\text{TL}(G,H)||\text{TL}(H,K)| = |\text{L}(G,K)|=(G:K)$.

    \item \textbf{Пусть дана цепочка нормальных подгрупп $K\trianglelefteq H\trianglelefteq
        G$. Верно ли, что $K$ нормальна в $G$?}

        Это не верно, приведем контр пример. Пусть $G=A_4$, возьмём 2 её абелевы
        подгруппы $H=\{e,(12)(34),(13)(24),(14)(23)\}$ и $K=\{e,(12)(34)\}$. Так
        как $H$ абелева, то очевидно, что $K$ в ней нормальна. Проверим, что
        $H$ нормальна в $G$:
        \begin{align*}
            (abc)&(ab)(cd)(cba)=(ac)(bd)\\
            (abcd)&(ab)(cd)(dcba)=(ad)(cb)\\
            (ab)&(ab)(cd)(ab)=(cd)(ab)\\
            (bc)&(ab)(cd)(bc)=(bd)(ac)\\
            (abcd)&(ac)(bd)(dcba)=(ac)(bd)\\
        \end{align*}
        Но $K$ не нормальна в $G$, так как например $(123)(12)(34)(321)=(13)(42)$.

    \item \textbf{(Теорема фон Дика) Пусть дана цепочка подгрупп $K\leq H\leq 
        G$. Пусть $H$ и $K$ нормальны в $G$. Докажите, что $G/H\cong (G/K)/(H/K).$}

        Обозначим каноничную проекцию $\pi: G\longrightarrow G/H$. Так как $K
        \leq H=\text{Ker}(\pi)$, то $\pi$ индуцирует гомоморфизм $\pi':
        G/K\longrightarrow G/H, gK\mapsto gH$, так как сдвиг $g$ на любой элемент
        из $K$ не изменит значение после $\pi$. Заметим, что $\text{Ker}(\pi')=
        \text{Ker}(\pi)/K = H/K$. Тогда запишем теорему о гомеоморфизме для $\pi'$,
        $G/H=\text{Im}(\pi')\cong(G/K)/\text{Ker}(\pi')=(G/K)/(H/K)$.

    \item \textbf{Пусть $G$ – группа, и пусть $S\subseteq G$ – подмножество.
        Определим нормализатор множества $S$ в $G$ следующим образом:
        \[N_S=\{g\in G\;|\;gSg^{−1}=S\}.\]
        Определим централизатор множества $S$ в $G$ следующим образом:
        \[Z_S ={g\in G\;|\;gs=sg\forall s\in S}.\]
        Проверьте, что централизатор и нормализатор являются подгруппами.
        Нормальны ли они?}

        Нетрудно заметить, что $e\in N_s,Z_s$. Пусть $a,b\in N_S$, тогда
        $abSb^{-1}a^{-1}=aSa^{-1}=S$, а значит $ab\in N_S$.
        Пусть $a,b\in Z_S$, тогда $abs=asb=sab, \forall s\in S$, тогда и $ab\in
        Z_S$. Теперь пусть $g\in N_S$, тогда $g^{-1}Sg=g^{-1}(gSg^{-1})g=S$,
        а значит $g^{-1}\in N_S$ и $N_S$ – подгруппа. Также $g\in Z_S$ и $s\in S$
        $g^{-1}s = g^{-1}sgg^{-1} = g^{-1}gsg^{-1}=sg^{-1}$, а значит $g^{-1}\in Z_S$
        и $Z_S$ – подгруппа. Вообще говоря они не нормальны.
        
        Пусть $S_3=\{e,(12),(23),(31),(123),(321)\}$. $Z_{\{(12)\}}=\{e,(12)\}$,
        что очевидно, и $N_{\{(12)\}}=\{e,(12)\}$, что чуть менее очевидно, но
        можно проверить $(23)(12)(23)=(31)$ и $(123)(12)(321)=(13)$. Так вот
        эта подгруппа не нормальна, так как есть например предыдущее равенство.

    \item \textbf{Центром $Z_G$ группы $G$ называется подмножество элементов,
        которые коммутируют со всеми элементами $G$. Проверьте, что центр
        является нормальной подгруппой. Вычислите центр симметрической группы
        $S_n$.}

        Пусть $h\in Z_G$, тогда $ghg^{-1}=hgg^{-1}=h\in Z_G$, а значит $Z_G$
        нормальна в $G$.

        Пусть $g\in Z_{S_n}$, $n>2$ тогда $g$ раскладывается в произведение дизъюнктных
        циклов $g=(\ldots)\ldots(\ldots)$, тогда если там будет цикл длинной
        большей 2 $(abc\ldots)$, то умножение на $(ab)$ справа выкинет из него
        $a$. $(abc\ldots)(ab)=(bc\ldots)$, а умножение слева выкинет $b$, 
        $(ab)(abc\ldots)=(ac\ldots)$, поэтому эти два элемента не коммутируют,
        a значит ни транспозиция, ни элементы с такими длинными циклами не будут
        в центре. Тогда наверно $g$ содержит хотя бы 2 цикла длинной 2? Но тогда
        найдется некоммутирующий 3 цикл $(abc)(ab)(cd)=(bdc)$, но $(ab)(cd)(abc)
        =(acd)$. Тогда $Z_{S_n}=1$, кроме случая $Z_{S_2}=\mathbb{Z}_2$.

    \item \textbf{Пусть $G_1$, $G_2$ – абелевы группы. Обозначим множество
        гомоморфизмов из $G_1$ в $G_2$ через $\textnormal{Hom}(G_1, G_2)$.
        \begin{itemize}
            \item Определите естественную операцию сложения гомоморфизмов и
                покажите, что $\textnormal{Hom}(G_1,G_2)$ обладает структурой
                абелевой группы.
            \item Вычислите $\textnormal{Hom}(\mathbb{Z}/n,\mathbb{Z}/m)$ для
                $n,m\geq0$.
        \end{itemize}}

        Пусть $f,g\in\text{Hom}(G_1 G_2)$ положим $f+g=(x\mapsto f(x)\cdot g(x))$.
        Очевидно, что эта операция ассоциативна. Единица тоже есть $e=1$. И
        обратный $-f=(x\mapsto f(x)^{-1})$. Проверим корректность $(-f)(xy)=
        f(xy)^{-1}=f(y)^{-1}f(x)^{-1}=f(x)^{-1}f(y)^{-1}=(-f)(x)\cdot(-f)(y)$,
        а значит обратный – гомоморфизм. Проверим корректность суммы $(f+g)(xy)=
        f(xy)g(xy)=f(x)f(y)g(x)g(y)=f(x)g(x)f(y)g(y)=(f+g)(x)\cdot(f+g)(y)$.
        Проверим абелевость, $f+g=(x\mapsto f(x)g(x))=(x\mapsto g(x)f(x))=g+f$.

        Теперь пусть $\varphi\in\text{Hom}(\mathbb{Z}_n,\mathbb{Z}_m)$. Тогда по
        теореме о гомеоморфизме $\varphi$ распадается на каноничный эпиморфизм
        $\pi:\mathbb{Z}_n\twoheadrightarrow \mathbb{Z}_n/\text{Ker}(\varphi)$, на
        изоморфизм $\psi:\mathbb{Z}_n/\text{Ker}(\varphi)\leftrightarrow\text{Im}
        (\varphi)\leq\mathbb{Z}_m$ и мономорфизма $i:\text{Im}(\varphi)
        \hookrightarrow\mathbb{Z}_m$. Заметим, что $\text{Ker}(\varphi)\leq
        \mathbb{Z}_n$, а значит $\text{Ker}(\varphi)=d\mathbb{Z}_n$ и
        $\mathbb{Z}_n/\text{Ker}(\varphi)\cong\mathbb{Z}_d$, где $d|n$. Но так
        как $\mathbb{Z}_d\hookrightarrow\mathbb{Z}_m$, а тогда $d|m$. Поэтому
        задача гомоморфизма на самом деле однозначно сводится к выбору двух
        натуральных чисел $d|n\wedge m$ и $k<d$, $k\wedge d=1$. (все гомоморфизмы
        группы вычетов - умножения на взаимно простые с $d$ числа)
        \vspace{-1ex}
        \begin{center}
        \begin{tikzcd}
            \mathbb{Z}_n\arrow[two heads]{r}{\pi}&
            \mathbb{Z}_d\arrow[leftrightarrow]{r}{\cdot k}&
            \mathbb{Z}_d\arrow[hook]{r}{i}&\mathbb{Z}_m\\
            \text{$[a]_n$}\arrow[mapsto]{r} & \text{$[a]_d$}\arrow[mapsto]{r}&
            \text{$[ka]_d$}\arrow[mapsto]{r} & \text{$\frac{m}{d}[ka]_d$}\\
        \end{tikzcd}
        \end{center}
        \vspace{-5ex}
        Где $m/d[ka]_d=m/d(ka+d\mathbb{Z})=mka/d+m\mathbb{Z}=[mka/d]_m$.
        Оно однозначно, так как для каждого гомоморфизма мы находим эти константы.
        И разные константы дают разные гомоморфизмы. Тогда порядок $|\text{Hom}(
        \mathbb{Z}_n,\mathbb{Z}_m)|=\sum_{d|n\wedge m}\varphi(d)=n\wedge m$.
        при этом для $d=n\wedge m$ и $k=1$, $\varphi(1)=[m/d]_m$, то порядок
        этого элемента очевидно $d$, а значит $\text{Hom}(\mathbb{Z}_n,\mathbb{Z}_m)
        \cong\mathbb{Z}_{m\wedge n}$.

    \item \textbf{Могут ли две неизоморфные группы иметь изоморфные нормальные
        подгруппы и изоморфные фактор-группы по ним? Может ли группа иметь две
        изоморфные нормальные подгруппы, фактор-группы по которым неизоморфны?
        Может ли группа иметь неизоморфные нормальные подгруппы, фактор-группы
        по которым изоморфны?}

        Две неизоморфные группы могут иметь изоморфные нормальные подгруппы и
        изоморфные фактор-группы по ним, например $\mathbb{Z}_4/2\mathbb{Z}_2
        \cong \mathbb{Z}_2\times\mathbb{Z}_2/\langle 0,(1,1)\rangle$.

        Группа может иметь две изоморфные нормальные подгруппы, фактор-группы по
        которым неизоморфны, например $\mathbb{Z}_4\times\mathbb{Z}_2/0\times\mathbb{Z}_2
        \cong\mathbb{Z}_4$, но $\mathbb{Z}_4\times\mathbb{Z}_2/\langle 0,2\rangle\times 0
        \cong\mathbb{Z}_2\times\mathbb{Z}_2$.

        Группа может иметь неизоморфные нормальные группы, фактор-группы по
        которым изоморфны, например $\mathbb{Z}_4\times\mathbb{Z}_4/\mathbb{Z}_4
        \times 0\cong\mathbb{Z}_4\times\mathbb{Z}_2/\langle0,2\rangle\times\mathbb{Z}_2$

    \item \textbf{(Теорема Кэли) Докажите, что любая группа изоморфна подгруппе
        симметрической группы.}

        Пусть $G$ - группа, тогда рассмотрим каноничное левое $G$-действие на $G$.
        Это действие является подгруппой симметрической группы $G$ как множества.
\end{enumerate}
\end{document}
