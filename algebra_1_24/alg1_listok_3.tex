\documentclass{article}
\usepackage[a4paper,left=3cm,right=3cm,top=1cm,bottom=2cm]{geometry}
\usepackage{amsmath}
\usepackage{amssymb}
\usepackage{hyperref}
\usepackage[russian]{babel}

\usepackage{tikz-cd}
\usepackage{array}
\usepackage{graphicx}
\newcommand\mapsfrom{\mathrel{\reflectbox{\ensuremath{\mapsto}}}}
\setlength{\parindent}{0mm}

\usepackage{fontspec}
\setmainfont{Linux Libertine O}
\usepackage{unicode-math}
\setmathfont{Cambria Math}

\title{
\textit{\small{Георгий Потошин, 2024}}\\
\vspace{0.3ex}
\textit{\huge{Алгебра I, листочек 3}}\vspace{1ex}
}

\date{\vspace{-10ex}}

\begin{document}
\maketitle

\begin{enumerate}
    \item \textbf{Есть ли в группе $\mathbb{Z}_2\times\mathbb{Z}_{16}$ подгруппа,
        изоморфная $\mathbb{Z}_4\times \mathbb{Z}_4$? Изоморфная $\mathbb{Z}_2
        \times\mathbb{Z}_2\times\mathbb{Z}_2$?}

        Посчитаем количество элементов порядка 4 в обоих группах. Как нетрудно
        заметить $(a,b)\in\mathbb{Z}_2\times\mathbb{Z}_{16}$ имеет порядок 4, если его
        имеет $b$, в противном случае порядок либо 2, либо 1, таких пар всего 4
        $(0,4)$, $(1,4)$, $(0,12)$ et $(1,12)$. Но $(a,b)\in\mathbb{Z}_4\times
        \mathbb{Z}_4$ имеет порядок 4, когда хотя бы одна компонента его имеет.
        Таких пар после нетрудного подсчета оказывается $4*2+2*4-2*2=12$, и их
        больше, чем в группе куда мы хотим устроить вложение, а значит оно не
        удастся.

        Аналогично в $\mathbb{Z}_2^3$ мы обнаружим $7$ инволюций, а в
        $\mathbb{Z}_2\times\mathbb{Z}_{16}$ их всего $3$ и вложение не удастся.

    \item \textbf{(Теорема Нетер) Если $K,H\leq G$ и $H\leq N_K$, докажите, что
        $H/(H\cap K)\cong HK/K$, где $HK$ – подгруппа, порожденная элементами 
        вида $h\cdot k,h\in H,k\in K$.}

        Покажем, что $HK$ в данном случае совпадает с произведением по
        Минковскому. Пусть $h,h'\in H$ и  $k,k',k''\in K$, тогда $hkh'k'=hh'k''k$,
        так как $h'\in H\subseteq N_K$. Тогда $(hk)^{-1}=k^{-1}h^{-1}=h^{-1}k''$,
        и $e=ee\in HK$ есть нейтральный элемент. 
        Ещё можно заметить, что $H,K\leq N_K$, а тогда $K\trianglelefteq HK
        \subset N_K$. Построим гомоморфизм $\varphi:H\longrightarrow HK/K, h\mapsto hK$.
        Он очевидно сюръективен. Посчитаем его ядро $(h\in\text{Ker}(\varphi))
        \Leftrightarrow (hK=K\;\&\; h\in H)\Leftrightarrow( h\in K\cap H)$.
        Тогда применим теорему о сюрьективном гомеоморфизме, $H/(H\cap K)\cong HK/K$.

    \item \textbf{Опишите все перестановки, которые могут быть разложены в
        произведение циклов длины три.}

        Разобьём перестановку на дизъюнктивные циклы. Назовём дискриминантом
        перестановки сумму длин её дизъюнктивных циклов, уменьшенных на 1.
        Покажем, что умножение на транспозицию меняет четность дискриминанта.
        \begin{itemize}
            \item Если транспозиция дизъюнктивна с существующими индексами, то
                умножение на неё добавит к дискриминанту 1.
            \item Если транспозиция пересекается с 1 циклом по 1 индексу,
                то $(ab)(ac\ldots d)=(abc\ldots d)$ или $(d\ldots ca)(ba)=
                (d\ldots cba)$ один цикл увеличится на 1.
            \item Если транспозиция пересекается с 1 циклом по 2 индексам,
                то $(ab)(ad\ldots ebc\ldots f)=(ac\ldots f)(bd\ldots e)$ или
                $(f\ldots cbe\ldots da)(ba)=(e\ldots db)(f\ldots ca)$ и цикл
                разобьется на 2, а дискриминант уменьшится на 1.
            \item Если транспозиция пересекается с 2 циклами по 1 индексу, то
                 $(ab)(ac\ldots e)(bd\ldots f)=(ad\ldots fbc\ldots e)$ или
                 $(f\ldots db)(e\ldots ca)(ba)=(e\ldots cbf\ldots da)$ циклы
                 слипнутся, а дискриминант увеличится на 1.
        \end{itemize}
        Зная также, что перестановка раскладывается по циклам на транспозиции,
        которых будет ровно дискриминант штук, мы получаем, что домножение
        на четную перестановку не меняет четности. В частности это означает, что
        все четные перестановки образуют знакопеременную группу, так как $()$
        четна, обратная получается переворачиванием циклов, что тоже не меняет
        четность, а произведение четных четно.

        Тогда покажем, что циклы длинны три порождают знакопеременную группу.
        Знакопеременная группа состоит из элементов в которых при разложении
        получается четное число нечетных перестановок. Заметим, что мы можем 
        всегда добавить к циклу 2 любых буквы $(a\ldots bc)(dec)=(a\ldots bdec)$.
        Так же мы можем создать пару любых дизъюнктивных транспозиций $(abc)(bcd)
        =(ac)(bd)$. Из этого мы делаем вывод, что всегда можно из 3-х циклов
        собрать четный цикл или четное количество нечетных. Перемножив их можно
        получить любой четный элемент. Но так как 3-циклы сами по себе четные,
        то их произведение всегда таким и останется. Тогда $\langle\text{3-циклы}
        \rangle=A$.

    \item \textbf{Постройте сюръективный гомоморфизм $S_n\longrightarrow
        \mathbb{Z}_2$ для любого $n\geq 2$.}

        Отправим нечетную перестановку в 1, а четную в 0. Исходя из рассуждений
        прошлой задачи, получится сюрьекивный гомоморфизм, так как четность ведёт
        себя мультипликативно.

    \item \textbf{Постройте сюръективный гомоморфизм $S_4\longrightarrow S_3$.}
        
        Как мы видели в задаче 3 листочка 2 $V_4\trianglelefteq S_4$, где
        нетривиальными элементами $V_4$ являются (2,2)-циклы. Там я проверил
        все возможные сопряжения, то тогда можно построить фактор-группу
        $S_4/V_4$, её порядок равен $24/4=6$, так что она изоморфна либо
        $\mathbb{Z}_6$, либо $S_3$. Но в $S_4$ все элементы были порядка
        меньшего 6, то значит и в фактор-группе их порядок меньше $(gA)^n=g^nA=
        A$. Тогда $S_4/V_4\cong S_3$, осталось построить гомоморфизм.
        Транспозиции должна перейти в инволюцию, которыми в $S_3$ могут быть
        только инволюции. Тогда например $(12)\mapsto(12)$. Так как ядром
        является $V_4$, то $(12)(34)\mapsto(12)(??)=e$, а значит $(34)\mapsto
        (12)$. Также $(12)V_4=\{(12),(43),(1324),(1423)\}$, в нём нет $(23)$,
        a значит её образ должен отличатся от $(12)$, пусть им будет $(23)$,
        аналогично поймём и зададим образ $(13)$ равный $(13)$. Тогда аналогично
        прошлому рассуждению поймём, что дополнение к траноспозиции до 2-цикла
        из ядра должно переходить в тот же элемент, то есть $(24)\mapsto(13)$ и
        $(14)\mapsto(23)$. Так как мы построили отображение порождающих элементов
        в порождающие, то если можно корректно дополнить гомоморфизм, то это
        делается единственным способом через разложение на порождающие. И так
        как гомоморфизм существует, в чем мы убедились, построив фактор-группу,
        то построение продолжится корректно, потому как в этом алгоритме все
        выборы привели бы к симметричной ситуации с точностью до переименования.
        Одна из ситуаций подошла бы нам, но так как они эквивалентны, то подходят
        все.

    \item \textbf{Докажите, что любую перестановку из $S_n$ можно получить,
        перемножая транспозицию $(1,n)$ и цикл $(1,2,3,\dots,n)$.}

        Имея два элемента $(a,b)$ и $(a,a+1,\ldots,b-1,b)$ можно их перемножить
        $(a,b)(a,a+1,\ldots,b-1,b)=(a+1,\ldots,b-1,b)$ или $(a,a+1,\ldots,b-1,b)
        (a,b)=(a,a+1,\ldots,b-1)$. Обратные к элементам равняются некоторой
        степени этого элемента, а значит мы можем брать обратные. Тогда продолжив
        дальше перемножение мы сможем выразить $(a,a+1,\ldots,b-1)(b,b-1,\ldots,a+1)
        =(a,b,b-1)$, а также $(b,b-1,\ldots,a+1)(a,a+1,\ldots,b-1)=(b,a,a+1)$,
        пройдя ещё чуть дальше мы получи $(a,b)(b,a,a+1)=(a+1,b)$ и $(a,b)(a,b,
        b-1)=(a,b-1)$. Тогда мы вновь получили 2 пары $(a,b-1)$ и $(a,a+1,
        \ldots,b-1)$ и вторую пару $(a+1,b)$ и $(a+1,a+2,\ldots,b)$, так что мы
        можем продолжить рекурсию. В итоге мы сможем выразить любую транспозицию,
        а значит и любой элемент симметрической группы.

    \item \textbf{Пусть перестановка $\sigma\in S_n$ разложена в произведение l
        независимых циклов длин $\rho_1\geq\ldots\geq\rho_l\geq 1$. Выпишите
        формулу для:}

        \begin{itemize}
            \item \textbf{четности $\sigma$}

                Как мы видели в 3 задаче этого листочка, это четность её
                дискриминант. Так что формула $(\sum_i(\rho_i-1))\,\text{mod}\,2$.

            \item \textbf{порядка перестановки $\sigma$}

                Степень перестановки это в точности произведение степеней
                её циклов. Каждый цикл редуцируется, если степень делит его
                порядок. Наименьшее число удовлетворяющее этому свойству это в
                точности наименьшее общее кратное, то есть $\rho_1\vee\ldots
                \vee\rho_l$.

            \item \textbf{порядка класса сопряженности $\sigma$}

                Заметим, что сопряжение по транспозиции не меняет численный тип
                разбиения на циклы, так как $(ab)(ad\ldots c)(ab)=(bd\ldots c)$,
                $(ab)(c\ldots eabf\ldots k)(ab)=(af\ldots kc\ldots eb)$
                и $(ab)(ad\ldots c)(bd'\ldots c')(ab)=(bd\ldots c)(ad'\ldots c')$.
                Это достаточно проверить для транспозиций, так как остальные 
                элементы группы раскладываются через них.
                Более того одна перестановка может быть получена из другой того
                же типа через переименование индексов, а это как мы видели в 7
                задаче 1 листочка является сопряжением по перестановке индексов.
                Так что размер класса сопряженности это в точности размер
                типового класса. Сначала выберем упорядоченные наборы, чтобы потом
                на них надеть скобки и получить циклы,
                это можно сделать $n!/(n-(\rho_1+\ldots+\rho_l))!$ количеством
                способов. Дальше расставим скобки в порядке убывания. Сначала
                самые длинные циклы, потом меньшие итд. Это делается 1 способом.
                В итоге для каждого элемента мы получим повторения, одно
                получается из другого через циклическую перестановку внутри одних
                скобок. Тогда у нас будет $\prod_i\rho_i$ перестановок и конечная
                формула равняется $\frac{n!}{(n-(\rho_1+\ldots+\rho_l))!\prod_i\rho_i}$


            \item \textbf{порядок централизатора $\sigma$}

                Зададим действии $*$ $G=S_n$ на $G$ через сопряжение. Тогда
                орбитой $\sigma$ по этому действию будет его класс $G*\sigma=
                [\sigma]$, а стабилизатором – централизатор $G_\sigma=Z_\sigma$.
                Тогда $h*\sigma=
                g*\sigma$ тогда и только тогда, когда $g^{-1}h*\sigma=\sigma$
                тогда и только тогда, когда $g^{-1}h\in G_\sigma$ тогда и только
                тогда, когда $h\in gG_\sigma=gZ_\sigma$. Это означает, что
                каждое значение из орбиты получается через действие элементов
                одного класса, а значит $|G*\sigma|=|G|/|G_\sigma|$, или
                $|Z_\sigma|=|G|/|G*\sigma|$. Тогда формула будет следующей
                $(n-(\rho_1+\ldots+\rho_l))!\prod_i\rho_i$.
        \end{itemize}
    \item \textbf{Найдите центр группы перестановок $S_n$. Найдите центр
        группы невырожденных матриц $\text{GL}_2(\mathbb{C})$.}

        Центр группы $S_n$ уже был мной посчитан в листочке 2 задаче 6.

        Обозначим за $e_{i,j}$ матричные единицы. Тогда очевидно, что матрицы
        вида $k(e_{1,1}+e_{2,2})$ входят в центр. Возьмём тогда матрицу из центра.
        \[\left(\begin{array}{cc}a&b\\c&d\end{array}\right)
          \left(\begin{array}{cc}1&0\\0&2\end{array}\right)
         =\left(\begin{array}{cc}a&2b\\c&2d\end{array}\right)\]
        \[
          \left(\begin{array}{cc}1&0\\0&2\end{array}\right)
          \left(\begin{array}{cc}a&b\\c&d\end{array}\right)
         =\left(\begin{array}{cc}a&b\\2c&2d\end{array}\right)\]
         Тогда мы получим $2b=b$ и $2c=c$, а значит они нули.
        \[\left(\begin{array}{cc}a&0\\0&d\end{array}\right)
          \left(\begin{array}{cc}0&1\\1&0\end{array}\right)
         =\left(\begin{array}{cc}0&d\\a&0\end{array}\right)\]
        \[
          \left(\begin{array}{cc}0&1\\1&0\end{array}\right)
          \left(\begin{array}{cc}a&0\\0&d\end{array}\right)
         =\left(\begin{array}{cc}0&a\\d&0\end{array}\right)\]
        А значит, что $a$=$d$. Тогда $\mathbb{C}^*(e_{1,1}+e_{2,2})$ – весь
        центр.
         
    \item \textbf{Докажите, что подгруппа внутренних автоморфизмов $\textnormal
        {Inn}(G)$ нормальна в группе автоморфизмов $\textnormal{Aut}(G)$.
        Докажите, что $\textnormal{Inn}(G)\cong G/Z_G$.}

        Проверим нормальность, пусть $S_s$ – сопряжение по $s$, а $f$ -
        автоморфизм. Тогда сопряжем $f\circ S_s\circ f^{-1}(x)=f(sf^{-1}(x)s)=
        f(s)xf(s)^{-1}$, получилось сопряжение по $f(s)$. А значит $\text{Inn}
        (G)$ нормальна в $\text{Aut}(G)$.

        Сопоставим элементу группы сопряжение по нему, получится сюрьективный
        гомоморфизм. $g$ лежит в его ядре равносильно $gsg^{-1}=s$ или $gs=sg$
        для любого $s$, что равносильно $g\in Z_G$. Тогда по теореме о
        гомоморфизме $\text{Inn}(G)=\text{Im}(f)\cong G/\text{Ker(f)}=G/Z_G$.

    \item \textbf{Докажите, что если $|G| = p^n$, где $p$ – простое, то $|Z_G|=
        p^k$ для $k > 0$.}

        Центр группы – подгруппа, а значит её порядок делит порядок группы.
        В данном случае её порядок может быть только степенью $p$. Осталось
        показать, что она не тривиальна. Разобьем $G$ на классы сопряженности.
        Как мы видели в задаче 7, порядок класса сопряженности делит порядок
        группы. Классы элементов не из центра имеют размер больший 1, а
        значат должны делится на $p$. Если у элемента класс состоит из него
        самого, то это эквивалентно тому, что он лежит в центре. Тогда
        будет иметь место равенство $|G|=|Z_G|+\sum[g]$, где все слагаемые
        кроме $|Z_G|$ делятся на $p$, а значит должно делится и $|Z_G|$, но
        это и означает, что центр не тривиален.

    \item \textbf{Покажите, что невырожденные $2\times 2$ матрицы с
        коэффициентами из $\mathbb{Z}_2$ образуют группу. Обозначим ее
        $\text{GL}_2(\mathbb{Z}_2)$. Докажите, что она изоморфна $S_3$.}

        Как мы видели на первом листочке умножение матриц ассоциативно. В данном
        случае, $\mathbb{Z}_2$ - поле, а значит обращение матриц определено
        корректно. Над данным полем оно $(ae_{1,1}+be_{1,2}+ce_{2,1}+de_{2,2})^{-1}=
        de_{1,1}-be_{1,2}-ce_{2,1}+ae_{2,2}$. Посчитаем порядки элементов
        \[
            \left(\begin{array}{cc}1&0\\1&1\end{array}\right)^2=
            \left(\begin{array}{cc}1&1\\0&1\end{array}\right)^2=
            \left(\begin{array}{cc}0&1\\1&0\end{array}\right)^2=
            \left(\begin{array}{cc}0&1\\1&1\end{array}\right)^3=
            \left(\begin{array}{cc}1&1\\1&0\end{array}\right)^3=
            \left(\begin{array}{cc}1&0\\0&1\end{array}\right)
        \]
        Такие порядки у элементов группы порядка 6 могут быть только у $S_3$.
        Так что эта группа изоморфна $S_3$.


\end{enumerate}
\end{document}
