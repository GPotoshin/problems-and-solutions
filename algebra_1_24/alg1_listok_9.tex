\documentclass{article}
\usepackage[a4paper,left=3cm,right=3cm,top=1cm,bottom=2cm]{geometry}
\usepackage{amsmath}
\usepackage{amssymb}
\usepackage{hyperref}
\usepackage[russian]{babel}

\usepackage{tikz-cd}
\usepackage{array}
\usepackage{graphicx}
\newcommand\mapsfrom{\mathrel{\reflectbox{\ensuremath{\mapsto}}}}
\setlength{\parindent}{0mm}

\usepackage{fontspec}
\setmainfont{Linux Libertine O}
\usepackage{unicode-math}
\setmathfont{Cambria Math}

\newcommand{\mymat}{\mathcal{M\mkern-3mu a\mkern-0.3mu t}}


\title{
\textit{\small{Георгий Потошин, 2024}}\\
\vspace{0.3ex}
\textit{\huge{Алгебра I, листочек 9}}\vspace{1ex}
}

\date{\vspace{-10ex}}

\begin{document}
\maketitle

\begin{enumerate}
    \item \textbf{Докажите, что $F\in\mathbb{k}[x]$, $F(\alpha)=0,\;\alpha\in\mathbb{k}$ влечет
        $(x−\alpha)|F$ (теорема Безу). Докажите, что многочлен степени $n$ над полем имеет не
        более $n$ различных корней. Докажите, что группа
        \[\mu_n(\mathbb{k})=\{\alpha\in\mathbb{k}|\alpha^n=1\}\]
        содержит не больше, чем n элементов.}

        Многочлены над полем можно делить с остатком. Поделим $F$ на $(x-\alpha)$,
        мы получим следующее равенство $F=Q\cdot(x-\alpha)+\beta$. Если подставить
        в это равенство $\alpha$, то занулится все, кроме $\beta$, тогда $0=\beta$,
        что в точности означает, что $F$ делится на $(x-a)$.

        С другой стороны как мы видели ранее $\mathbb{k}[x]$ целостное кольцо
        главных идеалов, а значит в нём единственно разложение на неприводимые, которыми
        в частности являются многочлены степени 1, так как они просты в кольце.
        Тогда в единственном разложении будет только конечное число множителей
        степени 1, и так как степень многочлена равна сумме степеней его фактора,
        то у нас не может быть больше факторов, чем степень многочлена, в частности
        это касается факторов степени 1, а корней не меньше, чем типов факторов
        степени 1, так как каждому корню потенциально соответствует 1 или несколько
        факторов, как мы показали в предыдущем параграфе.
        
        В частности в группе $\mu_n(\mathbb{k})$ лежат все корни $x^n-1$, а их
        не больше $n$.

    \item \textbf{Докажите, что конечная подгруппа мультипликативной группы поля циклична.}

        Пусть $G$ – конечная подгруппа мультипликативной группы поля порядка $n$, она абелева. Обозначим за
        $\psi(d) = \#\{a\in G | a^d = 1\}$. Так как $x^d=1$ имеет решений в $\mathbb{k}$
        не больше, чем $n$, то $\psi(d)\leq d$. Пусть для некого $d$ есть элемент $a$ этого
        порядка, обозначим за $G_d$ множество элементов $G$ порядка $d$, тогда очевидно, что
        $\langle a\rangle\subseteq\{a\in G | a^d=1\}$, но $\#\langle a\rangle = d$,
        a $\#\{a\in G | a^d = 1\}\leq d$, тогда включение превратится в равенство.
        $\langle a\rangle$ циклическая группа порядка $d$, содержащая все корни
        $x^d-1$. Тогда все элементы порядка $d$ лежат в $\langle a\rangle$ и количество
        таких элементов $\phi(d)$. Тогда
        \[n=\#G=\sum_{d|n}\#G_d\leq\sum_{d|n}\phi(d)=n\]
        A значит $\#G_d=\phi(d)$, в частности это верно для $n$, а значит мы находим элемент порядка $n$.
        Он порождает всю группу $G$, тогда эта группа циклическая.
        
    \item \textbf{Докажите, что если $[\mathbb{L}:\mathbb{K}]=2$, то $\mathbb{L}=\mathbb{K}[\sqrt a]$, где
        $a\in\mathbb{K}$.} 
\end{enumerate}

\end{document}
