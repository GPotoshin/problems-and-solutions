\documentclass{article}
\usepackage[a4paper,left=3cm,right=3cm,top=1cm,bottom=2cm]{geometry}
\usepackage{amsmath}
\usepackage{amssymb}
\usepackage{hyperref}
\usepackage[russian]{babel}

\usepackage{tikz-cd}
\usepackage{array}
\usepackage{graphicx}
\newcommand\mapsfrom{\mathrel{\reflectbox{\ensuremath{\mapsto}}}}
\setlength{\parindent}{0mm}

\usepackage{fontspec}
\setmainfont{Linux Libertine O}
\usepackage{unicode-math}
\setmathfont{Cambria Math}

\newcommand{\mymat}{\mathcal{M\mkern-3mu a\mkern-0.3mu t}}


\title{
\textit{\small{Георгий Потошин, 2024}}\\
\vspace{0.3ex}
\textit{\huge{Алгебра I, листочек 10}}\vspace{1ex}
}

\date{\vspace{-10ex}}

\begin{document}
\maketitle

\begin{enumerate}
    \item \textbf{Положим $q = p^n$. Докажите, что поле $\mathbb{F}_q$ имеет
        единственное расширение степени $k$. Докажите, что $\overline{\mathbb{F}_p}
        =\bigcup_{n\in\mathbb{N}}\mathbb{F}_{p^n}$}

        Расширение степени $k$ поля $\mathbb{F}_q$ будет иметь $q^k$, так как
        оно векторное пространство размерности $k$ над $\mathbb{F}_q$. Точно
        также как на лекции заметим, что для $q'=q^k$, $x^{q'}-x$ имеет корнем
        любой элемент поля, так как для $x\in\mathbb{F}_{q'}$ либо $x=0$, либо
        порядок ненулевого элемента делит порядок мультипликативной группы, а
        тогда $x^{q'-1}=1$. При этом других коней у полинома нет, так как мы
        уже нашли их в количестве, равном его степени. Тогда $\mathbb{F}_{q'}$
        является полем разложения многочлена $x^{q'}-x\in\mathbb{F}_q[x]$, как
        мы видели на лекции оно единственно и существует.

        Пусть $L=\bigcup_{n\in\mathbb{N}}\mathbb{F}_{p^n}$. Покажем, что это
        поле. Пусть $\alpha,\beta\in L$, тогда можно предположить, что $\alpha
        \in\mathbb{F}_{p^m}$ и $\beta\in\mathbb{F}_{p^k}$, тогда на самом деле
        по предыдущему утверждению $\alpha,\beta\in\mathbb{F}_{p^
        {\text{lcm}(k,m)}}$, значит определены их обратные, противоположные,
        сумма и произведение. Расширение $L/\mathbb{F}_q$ алгебраичено, так
        каждый элемент из $L$ лежит в конечном алгебраическом расширении. Теперь
        проверим алгебраическую замкнутость $L$. Пусть $P=a_0+a_1x+\ldots+
        a_nx^n\in L[x]$, тогда каждые коэффициент лежит в каком-то конечном
        расширении $a_i\in\mathbb{F}_{p^{n_i}}$. Положим $m=\text{lcm}(n_0,
        \ldots,n_n)$, тогда на самом деле $P\in\mathbb{F}_{p^m}$, если у $P$
        есть корень $\mathbb{F}_{p^m}$, то победа. Если нет, то $P$ неприводим,
        а значит $\mathbb{F}_{p^m}[x]/(P)=\mathbb{F}_{p^{m(n-1)}}$, по
        единственности расширения. И найдется корень в $\mathbb{F}_{p^{m(n-1)}}$.
        А тогда поле $L$ агебраически замкнуто и $L=\overline{\mathbb{F}_p}$ –
        алгебраическое замыкание.

    \item \textbf{Опишите все автоморфизмы поля $\mathbb{C}$ над $\mathbb{R}$.
        Опишите все автоморфизмы поля $\mathbb{F}_{q^n}$ над $\mathbb{F}_q$,
        где $q=p^k$.}

        Заметим, что $\mathbb{C}=\mathbb{R}(i)$ расширяет $\mathbb{R}$ со
        степенью 2, а тогда у нас не может быть больше автоморфизмов над
        $\mathbb{R}$, чем 2 по следствию с лекции. Мы можем предъявить эти 2,
        а именно тождественный и сопряжение.

        Мы знаем, что $\mathbb{F}_{p^{nk}}$ обладает $nk$ автоморфизмами над
        $\mathbb{F}_p$, тогда $\text{Aut}_{\mathbb{F}_q}(\mathbb{F}_{q^n})$
        образуют подгруппу в $\text{Aut}_{\mathbb{F}_p}(\mathbb{F}_{q^n})$,
        так как если автоморфизм переводит тождественно $\mathbb{F}_q$ в себя,
        то он тем более переводит тождественно $\mathbb{F}_p$ в себя. Давайте
        возьмём какой-нибудь автоморфизм $f$ над простым подполем. Пусть
        $f=x\mapsto x^{p^l}$, так как они все имеют такой вид. Если мы хотим
        $f(a)=a$ для любого $a\in\mathbb{F}_q$, то нам необходимо и достаточно,
        чтобы это было верно для элемента $\zeta$, порождающего мультипликативную
        группу. $\zeta^{p^l-1}=1$ означает, что $p^n-1\mid p^l-1$, это возможно
        если $n\mid l$, так как тогда $l=mn$ и $q-1\mid q^m-1$ верно. Покажем, что
        если $n \nmid l$, то $p^n-1 \nmid p^l-1$. Для этого заметим, что
        $p^n-1=p^{n-l}(p^l-1)+p^{n-l}$ и продолжив так полинмиально делить с
        остатком, мы получим $p^n-1=g(p)(p^l-1)+p^r-1$, где $r$ – остаток при
        делении $n$ на $l$. Если мы хотим, чтобы $p^r-1=0$, то необходимо,
        чтобы $r=0$, а это ровно то, что нам нужно. Тогда все автоморфизмы
        $\mathbb{F}_{q^n}$ над $\mathbb{F}_q$ имеют вид $x\mapsto x^{q^m}$, и
        они порождены $x\mapsto x^{q^m}$. Так как $x\mapsto x^{q^n}$ тождественен,
        то их не более $n$ штук. С другой стороны как мы видели на лекции все
        автоморфизмы $x\mapsto x^{p^{km}}$ различны для $0<m\leq n$, а значит они
        и будут всеми автоморфизмами и они образуют циклическую группу.

    \item \textbf{Опишите все автоморфизмы поля $\mathbb{R}$. Конечно ли множество
        автоморфизмов поля $\mathbb{C}$?}

        Пусть $f$ – автоморфизм поля $\mathbb{R}$. Тогда $f$ отправляет простое
        подполе в простое подполе, так как оно образовано единицей. То есть
        для любого рационального $q$ верно $f(q)=q$. Пусть теперь $a,b$ –
        действительные числа и пусть $a>b$. Заметим, что мы найдём число $x$,
        что $x^2=(a-b)$, а тогда $f(a)-f(b)=f(a-b)=f(x^2)=f(x)^2 > 0$, а тогда
        $f$ строго возрастает. Пусть $s\in\mathbb{R}$ обозначим за $S_-=\{q\in
        \mathbb{Q}\mid q<s\}$ и за $S_+=\{q\in\mathbb{Q}\mid q>s\}$ из курса
        анализа известно, что такое сечение однозначно определяет число $s$.
        После автоморфизма сечения перейдут в сечения, а так как $f$ строго
        возрастает, то $f(s)$ окажется зажат между $S_-$ и $S_+$, а значит
        $f(s)=s$, тогда у $\mathbb{R}$ есть единственный автоморфизм –
        тождественный.

        Множество автоморфизмов $\mathbb{C}$ не конечно. Пусть $a\in\mathbb{C}$
        трансцедентное число над $\mathbb{Q}$, такое есть из соображения о 
        кардиналах. Тогда $\mathbb{Q}(a)$ – счетно, так как изморфно
        $\mathbb{Q}(x)$, тогда вновь по соображению о кардиналах есть
        бесконечность трансцендентных комплексных чисел над $\mathbb{Q}(a)$.
        Давайте покажем, что по каждому такому выбору числа $b$ можно построить
        автоморфизм $\mathbb{C}$, что переставляет $a$ и $b$, причем каждый
        такой выбор даст нам новый автоморфизм $\mathbb{C}$.

        Покажем, что есть автоморфизм $\mathbb{Q}(a,b)$, что переставляет $a$ и
        $b$. Для этого заметим, что если для $f(x,y)\in\mathbb{Q}(x,y)$ верно,
        что $f(a,b)=0$, то в частности левую часть можно переписать как
        $g(b)=0$ для некоторого $g(y)\in\mathbb{Q}(a)(y)$, а тогда $g(y)=0$, но
        тогда $f(x,y)=0$, так как каждый коэффициент $g(y)$ является
        вычислением некого $h(x)\in\mathbb{Q}$ в $a$, и так как вычисление
        нулевое и $a$ трансцендентное, то $h(x)=0$. Более того вычисление
        $f(a,b)$ можно всегда произвести, так как $a$ и $b$ по выбору
        алгебраически независимы и знаменатель не зануляется. Тогда зададим
        авитоморфизм $f(a,b)\mapsto f(b,a)$ для любого $f(x,y)\in
        \mathbb{Q}(x,y)$. Если $f\neq g$ для некоторых $f(x,y),g(x,y)\in
        \mathbb{Q}(x,y)$, то $f(a,b)\neq g(a,b)$ и $f(b,a)\neq g(b,a)$, так
        как $f-g\neq 0$ и по предыдущему наблюдению $(f-g)(a,b)\neq 0$ например.
        Тогда у нас есть автоморфизм и мы его назовём $\phi$. Покажем, что его
        можно продлить до $\mathbb{C}\rightarrow\mathbb{C}$. Мы будем продлевать
        именно автоморфизмы, а не морфизмы, чтобы не получим случаем подполя
        $\mathbb{C}$ изоморфного $\mathbb{C}$.

        Устроим частично упорядоченное множество автоморфизмов подполей
        комплексных чисел $K\rightarrow K$, таких что они продолжают $\phi$ c
        порядком $(f:K\rightarrow K)\le(g:L\rightarrow L)$, если $K\le L$ и
        $g|_K=f$. Покжем, что есть максимальный элемент. Пусть $\{f_i:K_i
        \rightarrow K_i\}$ возрастающая цепь, тогда её точная верхняя грань –
        автоморфизм $f:\bigcup_iK_i\rightarrow\bigcup_iK_i$, отправляющий
        $a\in K_i$ в $f_i(a)$, как нетрудно видеть, он корректен и является
        автоморфимом поля. Тогда применив лемму Цорна мы получим, что есть
        максимальный элемент $m:M\rightarrow M$.

        Если $M=\mathbb{C}$, то победа, иначе мы можем взять $a\in\mathbb{C}
        \setminus M$, и продолжить $m:M\rightarrow M$ до $m':M(a)\rightarrow
        M(a)$, просто отправив $a\mapsto a$ в случае, когда $a$ трансцендентно,
        так как разные $f(x)\in M(x)$ имеют разные значения в точке $a$. Для
        алгебраического $a$ мы можем продлить $m:M\rightarrow M$ до $m'':\overline
        M\rightarrow\overline M$, так как расширение $\overline M/M$ алгебраично,
        а значит есть продолжение $\widetilde m:M\rightarrow\overline M$ до
        $m''$. В обоих случая мы прийдем к противоречию, а значит $M=\mathbb{C}$
        и мы построили новый автоморфизм $\mathbb{C}$.

    \item \textbf{Докажите, что расширение полей степени 2 нормально. Докажите,
        что расширение полей $\mathbb{Q}[\sqrt[4]{2}]/\mathbb{Q}$ не нормально.}

        Пусть $L/K$ расширение степени $2$, тогда на самом деле $L=K[\alpha]$
        для некого $\alpha\in L$. Пусть $p(x)=\text{Irr}_\alpha^K(x)$, его
        степень 2. В $L$ многочлен $p(x)$ имеет один корень $\alpha$, а тогда
        поделив $p(x)$ на $x-\alpha$ мы найдем и второй корень. А значит $L$ –
        поле разложения $p(x)$ на $K$, а тогда расширение нормально.

        Расширение $\mathbb{Q}[\sqrt[4]{2}]/\mathbb{Q}$ не нормально, так
        как например у неприводимого $x^4-2$ есть корень $\sqrt[4]{2}$, но в
        $x^4-2=(x^2-\sqrt{2})(x^2+\sqrt{2})$ правый множитель не раскладывается
        на линейные множители, так как у него нет корней, потому как квадрат
        действительного числа всегда положителен.

    \item \textbf{Пусть $F\le K\le L$ – башня полей, и $L/F$ нормально.
        Докажите, что $L/K$ нормально. Приведите пример башни полей $F\le K\le L$,
        в которой $K/F$ и $L/K$ нормальны, но $L/F$ не нормально.}

        Пусть $f(x)\in K[x]$ неприводим и у него есть корень $\alpha\in L$,
        тогда положим $p(x)=\text{Irr}_\alpha^K(x)\in L[x]$. Устроим морфизм
        $\theta:K(x)\rightarrow L=f(x)\mapsto f(\alpha)$, его ядро
        максимальный идеал, в котором лежат $f(x)$ и $p(x)$. Так как $f(x)$
        неприводим, то $\text{Ker}(\theta)=(f(x))$, а значит $f(x)\mid p(x)$.
        С другой стороны, так как $p(x)\in F(x)$ и $L/F$ нормально и у $p(x)$
        есть корень $\alpha$ в $L$, то $p(x)$ раскладывается на линейные
        множители, а значит $f(x)$ тоже.

        Начнем с $\mathbb{Q}$, нормально расширим его до $\mathbb{Q}(i)$.
        Дальше нормально расширим, добавив корень полинома $x^2+(1+i)x+(1+i)$.
        Найдем его корень $D=(1+i)^2-4(1+i)=-2i-4$, и $\alpha=(-(1+i)+i\sqrt{2}
        \sqrt{i+2})/2$. Покажем, что $\alpha\notin\mathbb{Q}(i)$. Пусть это не так,
        тогда есть $a,b\in\mathbb{Q}$, что $\sqrt{2}\sqrt{i+2}=a+ib$. Покажем, 
        что такого быть не может, так как
        \begin{align*}
            a+ib &= \sqrt{2}\sqrt{i+2}\\
            a^2-b^2 + 2abi &= 4+2i\\
            a^2-b^2=4 &\;\&\; ab=1\\
            a^2-1/a^2&=4\\
            a^4+4a^2-1&=0\\
            a^2&=-2\pm\sqrt{5}\\
        \end{align*}
        Тогда $\mathbb{Q}(i,\alpha)/\mathbb{Q}(i)$ вновь расширение степени 2,
        а значит нормально. Теперь покажем, что $\mathbb{Q}(i,\alpha)/
        \mathbb{Q}$ не нормально. Для начала построим неприводимый над $\mathbb{Q}$
        многочлен. У нас уже был $x^2+(1+i)x+1+i$, возьмём сопряженный к нему
        и перемножим их
        \begin{align*}
            &\quad(x^2+(1+i)x+1+i)(x^2+(1-i)x+1-i)\\
            &=x^4+(1-i)x^3+(1-i)x^2+(1+i)x^3+2x^2+2x+(1+i)x^2+2x+2\\
            &=x^4+2x^3+4x^2+4x+2
        \end{align*}
        Покажем, что у него не корней в $\mathbb{Q}$. Из школьного курса
        алгебры известно, что все возможные рациональные корни можно получить,
        посмотрев на делители старшего и младшего члена. Возможные рациональные
        корни $\pm1,\pm2$. Но очевидно, что ни один не зануляет многочлен
        \begin{align*}
            1^4+2\cdot1^3+4\cdot1^2+4\cdot1+2>0\\
            2^4+2\cdot 2^3+4\cdot 2^2+4\cdot 2+2>0\\
            (-1)^4+2(-1)^3+4(-1)^2+4(-1)+2=1\\
            (-2)^4+2(-2)^3+4(-2)^2+4(-2)+2=10
        \end{align*}
        Покажем, что $x^4+2x^3+4x^2+4x+2$ не раскладывается в произведение 2х
        полиномов степени 2. Пусть он разложился, тогда найдутся $a,b,p,q\in
        \mathbb{Q}$, что
        \begin{align*}
            &\quad x^4+2x^3+4x^2+4x+2\\
            &=(x^2+ax+b)(x^2+px+q)\\
            &=x^4+(a+p)x^3+(ap+q+b)x^2+(aq+bp)x+bq
        \end{align*}
        Тогда мы получим следующую систему уравнений
        \begin{align*}
            \begin{cases}
                a+p=2\\
                ap+q+b=4\\
                aq+bp=4\\
                bq=2
            \end{cases}
        \end{align*}
        Заменим везде $p$ на $2-a$
        \begin{align*}
            \begin{cases}
                a(2-a)+q+b=4\\
                aq+b(2-a)=4\\
                bq=2
            \end{cases}
        \end{align*}
        А теперь заменим везде $q$ на $2/b$, так как $b$ не ноль
        \begin{align*}
            \begin{cases}
                a(2-a)+2/b+b=4\\
                2a/b+b(2-a)=4
            \end{cases}
        \end{align*}
        Домножим оба равенства на $b\neq 0$.
        \begin{align*}
            \begin{cases}
                ba(2-a)+2+b^2=4b\\
                2a+b^2(2-a)=4b
            \end{cases}
        \end{align*}
        Заметим, что второе равенство можно переписать
        \begin{align*}
            &2a+b^2(2-a)=4b\\
            &(b^2-2)(2-a)+4=4b\\
            &2-a=(4b-4)/(b^2-2)\quad\text{можем поделить, так как $\sqrt{2}\notin\mathbb{Q}$}\\
            &a=2-\frac{4b-4}{b^2-2}=\frac{2b^2-4b}{b^2-2}
        \end{align*}
        Подствив выражения для $a$ и $2-a$ от $b$ в первое уравнение системы,
        мы получим уравнение на $b$
        \begin{align*}
            b\frac{2b^2-4b}{b^2-2}\frac{4b-4}{b^2-2}+2+b^2=4b\\
            b(2b^2-4b)(4b-4)+(b^2-4b+2)(b^2-2)^2=0
        \end{align*}
        У нас вновь получилось полиномиальное уравнение со старшим коэффициентом
        $1$ и младшим $8$, тогда возможные рациональные корни только $\pm1,\pm2,
        \pm4,\pm8$. Проверим, что ни один не подходит
        \begin{align*}
            b(2b^2-4b)(4b-4)+(b^2-4b+2)(b^2-2)^2&\\
            1(2\cdot 1^2-4\cdot 1)(4\cdot 1-4)+(1^2-4\cdot 1+2)(1^2-2)^2&=-1\\
            (-1)(2\cdot (-1)^2-4\cdot (-1))(4\cdot (-1)-4)+((-1)^2-4\cdot (-1)+2)((-1)^2-2)^2&=55\\
            2(2\cdot 2^2-4\cdot 2)(4\cdot 2-4)+(2^2-4\cdot 2+2)(2^2-2)^2&=-8\\
            (-2)(2\cdot (-2)^2-4\cdot (-2))(4\cdot (-2)-4)+((-2)^2-4\cdot (-2)+2)((-2)^2-2)^2&=440\\
            4(2\cdot 4^2-4\cdot 4)(4\cdot 4-4)+(4^2-4\cdot 4+2)(4^2-2)^2&=1160\\
            (-4)(2\cdot (-4)^2-4\cdot (-4))(4\cdot (-4)-4)+((-4)^2-4\cdot (-4)+2)((-4)^2-2)^2&=10504\\
            8(2\cdot 8^2-4\cdot 8)(4\cdot 8-4)+(8^2-4\cdot 8+2)(8^2-2)^2&=152200\\
            (-8)(2\cdot (-8)^2-4\cdot (-8))(4\cdot (-8)-4)+((-8)^2-4\cdot (-8)+2)((-8)^2-2)^2&=4222792\\
        \end{align*}
        Тогда наш изначальный многочлен $x^4+2x^3+4x^2+4x+2$ не раскладывается
        в произведение квадратов и не имеет рациональных корней, а значит он
        неприводим над $\mathbb{Q}$. C другой стороны, в $\mathbb{Q}(i,\alpha)$
        он раскладывается в $(x^2+(1+i)x+1+i)(x^2+(1-i)x+1-i)$ и левый
        множитель раскладывается на линейные. Давайте убедимся, что у правого
        множителя нет корней в $\mathbb{Q}(i,\alpha)$. В $\mathbb{C}$ у правого
        множителя корни следующие
        \begin{align*}
            D=(1-i)^2-4(1-i)=2i-4\\
            x_{1,2}=(-(1-i)\pm\sqrt{2i-4})/2
        \end{align*}
        И то что они лежат в $\mathbb{Q}(i,\alpha)$ эквивалентно тому, что
        $\sqrt{2i-4}$ $\mathbb{Q}(i)$-линейно выражается через $1,\sqrt{2i+4}$,
        так как расширение $\mathbb{Q}(i,\alpha)/\mathbb{Q}(i)$ – степени 2.
        Пусть $a,b,c,d\in\mathbb{Q}$ такие, что
        \begin{align*}
            &\sqrt{2i-4}+(ai+b)\sqrt{2i+4}=ci+d\\
            &(\sqrt{2i-4}+(ai+b)\sqrt{2i+4})^2=(ci+d)^2\\
            &2i-4+(b^2-a^2+2iab)(2i+4)+2(ai+b)\sqrt{(2i-4)(2i+4)}=(ci+d)^2\\
            &2i-4+(b^2-a^2+2iab)(2i+4)+2(ai+b)\sqrt{-20}=(ci+d)^2\\
        \end{align*}
        Но так как $\sqrt{-20}\notin\mathbb{Q}(i)$, $ai+b=0$, a тогда
        \begin{align*}
            &2i-4=d^2-c^2+2icd\\
            &cd=1\quad\&\quad c^2-d^2=4
        \end{align*}
        Но как мы уже видели у такой системы нет рациональных решений, а значит
        мы получили неприводимый многочлен, что не раскладывается на линейные
        множители в $\mathbb{Q}(i,\alpha)$, а значит расширение $\mathbb{Q}(i,
        \alpha)/\mathbb{Q}$ не нормально и контр-пример построен.
        
\end{enumerate}

\end{document}
