\documentclass{article}
\usepackage[a4paper,left=3cm,right=3cm,top=1cm,bottom=2cm]{geometry}
\usepackage{amsmath}
\usepackage{amssymb}
\usepackage{hyperref}
\usepackage[russian]{babel}

\usepackage{tikz-cd}
\usepackage{array}
\usepackage{graphicx}
\newcommand\mapsfrom{\mathrel{\reflectbox{\ensuremath{\mapsto}}}}
\setlength{\parindent}{0mm}

\usepackage{fontspec}
\setmainfont{Linux Libertine O}
\usepackage{unicode-math}
\setmathfont{Cambria Math}

\title{
\textit{\small{Георгий Потошин, 2024}}\\
\vspace{0.3ex}
\textit{\huge{Алгебра I, листочек 4}}\vspace{1ex}
}

\date{\vspace{-10ex}}

\begin{document}
\maketitle

\begin{enumerate}
    \item \textbf{Найдите группы обратимых элементов в кольцах (здесь $\mathbb{k}$
        – произвольное поле):}
    \begin{enumerate}
        \item $\mathbb{k}[x]$. В этом кольце элементы частично упорядочены по
            степени максимального ненулевого члена, причем степень многочлена
            ведёт себя аддитивно по умножению, а значит если произведение двух
            полиномов равно $1$, то они должны быть свободны от $x$. C другой
            стороны свободные члены обратимы, когда они не нули, потому как
            поле, тогда мультипликативной группой будут $\mathbb{k}^\times x^{0}$.
        \item $\mathbb{k}[[x]]$. Как мы видели на лекции, если у ряда обратим
            нулевой коэффициент, то обратим и сам ряд по правилу $(a_0+a_1x+
            \ldots)^{-1}=(b_0+b_1x\ldots)$, где $b_0=a_0^{-1}$ и $b_n=-a_0^{-1}
            (a_1b_{n-1}+\ldots+a_nb_0)$. Причем, если $a_0$ не обратим, то тогда
            не будет $a_0b_0=1$.
        \item $\mathit{Mat}_2(\mathbb{k})$. Как мы уже видели, то обратимыми
            будут матрицы с обратимым дискриминантом. Для них верно:
            \[
                \left(\begin{array}{cc}a&b\\c&d\end{array}\right)^{-1}=\frac{1}
                    {ad-bc}\left(\begin{array}{cc}d&-b\\-c&a\end{array}\right)
            \]
            Если это не так, то дискриминант в поле равен нулю, а значит $c=ak$
            и $d=bk$, тогда
            \[
                \left(\begin{array}{cc}a&b\\ka&kb\end{array}\right)
                \left(\begin{array}{cc}\alpha&\beta\\\gamma&\delta\end{array}\right)=
                \left(\begin{array}{cc}\lambda_1&\lambda_2\\k\lambda_1&k\lambda_2\end{array}\right)
                \neq\left(\begin{array}{cc}1&0\\0&1\end{array}\right)
            \]
    \end{enumerate}

    \item \textbf{Докажите изоморфизмы колец:}
    \begin{enumerate}
        \item \textbf{$\mathbb{k}[x]/(f(x))\cong\mathbb{k}$, если $f(x)$ -
            многочлен первой степени.}

\vspace{2ex}
            Выберем из каждого класс элемент степени 0. Такой существует, так
            как мы можем делить с остатком на $f(x)$. Такой элемент единственен,
            так как разница многочленов нулевой степени из одного класса -
            многочлен нулевой степени из идеала $(f(x))$, а это 0.
            Такое соответсвие однозначно сопоставит элементы поля и кольца.
            Причем это соответствие будет гомоморфизмом из-за того, как мы
            перемножаем классы. 

        \item \textbf{$\mathbb{R}[x]/(f(x))\cong\mathbb{C}$, если $f(x)$ –
            многочлен степени 2, не имеющий вещественных корней.}

            Если у нас есть такой многочлен, то идеал приводится к виду $(f(x))=
            ((x-b)^2+c)$, где $c>0$ и вообще любой многочлен можно записать в виде
            $a_0+a_1(x-b)+a_2(x-b)^2+\ldots$. Тогда при факторизации мы на самом
            деле говорим, что $(x-b)^2=-c$, тогда можно отправить $(x-b)\mapsto
            \sqrt ci$, а $1\mapsto 1$. Это задаст гомоморфизм колец. Он очевидно
            будет биективным, потому как это мономорфизм $\mathbb{R}$-векторных
            пространств ранга 2. Изоморфизм построен.
    \end{enumerate}
    
    \item \textbf{Пусть $\mathbb{k}=\mathbb{Z}/(2)$. Докажите, что $\mathbb{K}
        [x]/(x^2+x+1)$ – поле.}

        Вообще многочлены над полем образуют кольцо главных идеалов, так как
        мы можем делить в столбик и упорядочивать многочлены по степени. При
        этом $x^2+x+1$ несепарабелен над $\mathbb{Z}/(2)$, так как у него нет
        корней и он степени 2. Так что идеал $(x^2+x+1)$ является максимальным,
        а фактор по нему имеет только 2 идеала, а значит он поле.

    \item \textbf{Пусть $I_i$ – идеалы в кольце $A$. Докажите, что}

        Здесь всё будет доказано для правых идеалов, но также верно и для левых.
    \begin{enumerate}
        \item \textbf{$I_1+I_2$ – идеал}

        Как мы видели для абелевых групп, $I_2+I_1$ - абелева подгруппа. Также
        проверим замкнутость по умножение справа.
        $i_1\in I_1,\,i_2\in I_2,\,a\in A\Rightarrow i_1a\in I_1,\,i_2a\in I_2\Rightarrow
        i_1a+i_2a\in I_1+I_2$. Доказано
        
    \item \textbf{$I_1I_2=\{\sum x_iy_i\,|\,x_i \in I_1,\,y_i\in I_2\}$ – идеал}
        
        Проверим, стабильность по умножению на скаляры $(I_1I_2)A\subseteq
        I_1(I_2A)\subseteq I_1I_2$.
        Проверим, что это абелева подгруппа по сложению. Очевидно, что сумма
        конечных сумм - конечная сумма, так что есть замкнутость по умножению.
        К тому же так как идеалы не пусты, то и произведение тоже.

    \item \textbf{$\bigcap_i I_i$ пересечение идеалов – идеал}

    $x\in\bigcap_i I_i\Rightarrow\forall i,\,x\in I_i\Rightarrow\forall i,\forall
    a,\,xa\in I_i\Rightarrow\forall a,\, xa\in\bigcap_i I_i$\\
    $x,y\in\bigcap_iI_i\Rightarrow\forall i,\,x,y\in I_i\Rightarrow\forall i,\,
    x+y\in I_i\Rightarrow x+y\in\bigcap_i I_i$
    
    \item $(I_1I_2)I_3=I_1(I_2I_3)$

        $a\in (I_1I_2)I_3\quad\Rightarrow\quad a=\sum_j(\sum_i x_{i,j}y_{i,j})z_j
            =\sum_i\sum_j(x_{i,j}y_{i,j})z_j=\sum_i\sum_jx_{i,j}(y_{i,j}z_j)\quad
            \Rightarrow\\a\in I_1(I_2I_3)$. Где $x_i\in I_1$, $y_i\in I_2$ и
            $z_i\in I_3$. Точно также доказывается в обратную сторону.

    \item $I_1(I_2+I_3)=I_1I_2 + I_1I_3$

        $I_1I_2,I_1I_3\subseteq I_1(I_2+I_3)$, а значит и их сумма тоже. В другую
            сторонуa $a=\sum_ix_i(y_i+z_i)=\sum_ix_iy_i+\sum_ix_iz_i\in I_1I_2
            + I_1I_3$, где $x_i\in I_1$, $y_i\in I_2$ и $z_i\in I_3$.
    \end{enumerate}
    
    \item \textbf{Пусть $I,J,K$ – идеалы в кольце $A$. Определим
        частное идеалов $(I : J)$ следующим образом:
        \[(I : J) = \{a\in A\;|\;aJ\subseteq I\}\]
        Покажите, что это идеал.}

        Пусть $a,b\in (I:J)$ и $\lambda\in A$. Тогда $(a+b)J\subseteq aJ+bJ
        \subseteq I+I=I$. $\lambda aJ\subseteq \lambda I\subseteq I$ и
        $a\lambda J=aJ\subset I$. Заметим, что частное является идеалом.
        
        \textbf{Докажите, что}
        \begin{enumerate}
            \item $I\subseteq (I : J)$

                Пусть $a\in I$ лежит в идеале, тогда $aJ\subseteq I$, а
                значит $a\in(I:J)$.

            \item $(I:J)J\subseteq I$

                Пусть $a\in(I:J)J$, тогда $a=\sum_ix_iy_i$, где $x_i\in(I:J)$
                и $y_i\in J$, тогда $x_iy_I\in I$ по определению частного, тогда
                и сумма там же.

            \item $((I:J):K)=(I:JK)=((I :K):J)$

                Пусть $a\in((I:J):K)$ это равносильно тому, что $aK\subseteq(I:J)$
                , что в точности $aKJ\subseteq I$, а это определение $a\in(I:KJ)$.
                Тогда верно $((I:J):K)=(I:KJ)$ в некоммутативном случае, а в
                коммутативном $((I:J):K)=(I:KJ)=(I:JK)=((I:K):J)$.
        \end{enumerate}

    \item \textbf{Докажите, что прообраз простого идеала при гомоморфизме колец
        является простым идеалом. Является ли прообраз максимального идеала
        максимальным?}

        Пусть $f:A\longrightarrow B$ – гомоморфизм колец и $P\subseteq B$ –
        простой идеал. Пусть $ab\in f^{-1}[P]$, тогда $f(a)f(b)=f(ab)\in P$,
        без потери общности по простоте $P$ положим $f(a)\in P$, тогда $a\in
        f^{-1}[P]$, а значит прообраз прост. Прообраз максимального идиала
        вообще говоря не максимален, так как вкладывая целые числа в рациональные,
        прообразом максимального идиала $(0)$ будет не максимальным.

    \item \textbf{Пусть $I\subseteq A$ – двусторонний идеал в кольце. Докажите,
        что $A/I$ не имеет делителей нуля тогда и только тогда, когда $I$ прост.
        Докажите, что A/I является полем тогда и только тогда, когда I максимален.}

        Пусть $ab\in I$, но $a,b\notin I$, тогда $(a+I)(b+I)=ab+I=I$ мы нашли
        делители нуля. Обратно пусть мы нашли два делителя нуля, тогда $a+I\neq
        b+I$, но $ab+I=I$, тогда $ab\in I$, но $a,b\notin I$, a значит идел не
        простой.

        Пусть мы нашли идеал $I\subseteq J\neq A$, положим $J'=\{a+I\;|\;a\in J\}$
        нетрудно видеть, что это множество замкнуто относительно сложения и
        умножения на элементы кольца. Это нетривиальный идеал фактор кольца, а
        значит фактор кольцо не поле. Обратно пусть $A/I$ не поле, тогда
        найдется нетривиальный идеал $J'=\{a+I\}$. Возьмём объединение всех
        классов $J=\bigcup J'$, это идеал, причем $I\subset J$, и так как $J'$
        не был равен всему фактор кольцу, то мы найдем класс который не лежит в
        $J'$ и возьмём из него элемент, он не будет лежать в $J$. А значит $I$
        не максимально.

    \item \textbf{Пусть $A$ – целостное кольцо. Докажите, что кольца $A[x]$ и
        $A[[x]]$ – целостные. Опишите их поля частных.}

    Eсли $A[x]$ или $A[[x]]$ не целостное, то $(a_nx^{n}+\ldots)(b_mx^{m}+\ldots)
    =(a_nb_mx^{n+m}+\ldots)=0$, то $a_nb_m=0$, а значит $A$ не целостно. Построим
    поля частных. $A[x]=\{\frac{a_0+\ldots+a_nx^n}{b_0+\ldots+b_mx^m}\}$. Причем
    для не полей не будет существовать приведенной формы. Точно также в $A[[x]]$
    $\frac{a_nx^n+\ldots}{b_mx^m+\ldots}=\frac{x^n(a_n+\ldots)}{x^m(b_m+\dots)}=
    \frac{x^n}{x^m}(a_n+\ldots)(b_m^{-1}+\ldots)$ приводится к каноничному виду
    только если $b_m$ обратимо.

    \item \textbf{Докажите, что кольцо $\mathbb{k}[[x]]$ нетерово и факториально.
        Перечислите все простые идеалы в нем.}

        Пусть есть некий идеал $I$, кольцо рядов упорядочено по степени
        наименьшего ненулевого монома, тогда в идеале $I$ можно найти элемент
        минимальной степени $n$, так как этот порядок изоморфен порядку
        $\mathbb{N}$. А этот минимальный элемент лежит в тех же идеалах, что
        и $x^n$, потому как один получается из другого через умножение на единицу
        кольца, а значит $x^n\in I$. Но также $x^n$ делит любой элемент из $I$,
        а значит $I=(x^n)$, в итоге все идеалы образуют линейный порядок
        $(1)\supset (x)\supset\ldots$, а значит что для каждого идеала существует
        только конечное количество идеалов больших него, а значит кольцо нётерово.

        Если $a_nx^n+\ldots$ неприводим, то $n$ очевидно не больше 1. Причем,
        если $n=0$, то элемент обратим, а значит не неприводим, а если $n=1$, то для
        любого разложения в произведение один ряд будет обратим, а у другого
        степень минимального мнома будет равна $1$, что получается из решения
        несложного равенства в $\mathbb{N}_0$ $a+b=1$, а как мы видели главные
        идеалы подходящих элементов $(x)$. Он максимален, а значит прост.
        Тогда все неприводимые элементы просты, а значит кольцо факториально.

    \item \textbf{(Лемма об избегании простых идеалов) Пусть $\mathfrak{p}_1,
        \ldots,\mathfrak{p}_n$ – простые идеалы в кольце $A$, и пусть $I$ –
        идеал в $A$. Пусть $I\subseteq\bigcup_{i=1}^n\mathfrak{p}_i$. Докажите,
        что $I\subset\mathfrak{p}_i$ для некоторого $i$.}

        Пойдём по индукции для эквивалентного утверждения $\forall i,\,I
        \nsubseteq\mathfrak{p}_i\Rightarrow I\nsubseteq\bigcup_{i=1}^{n}
        \mathfrak{p}_i$. Для $n=1$ утверждение очевидно. Пусть теперь оно
        верно для $n-1$. И пусть $I\nsubseteq\mathfrak{p}_i$.
        По предположению индукции для каждого $i$ мы найдем $a_i$ из $I$, что
        $a_i\notin\mathfrak{p}_j$, для всех $i\neq j$. Если к тому же $a_i\notin
        \mathfrak{p}_i$ для некоторого $i$, то победа. В противном случан $\sum_i\prod_{j\neq i}
        a_j\notin\bigcup_i\mathfrak{p}_i$, и в этом случае доказуемое тоже верно.

    \item \textbf{Докажите, что кольцо $\mathbb{k}[x]/(x^{n+1})$ – артиново.}

        Заметим, что из нотации видно, что $\mathbb{k}[x]/(x^{n+1})=\mathbb{k}
        [[x]]/(x^{n+1})$, потому как в идеале от кольца рядов можно выбрать в
        каждом классе многочлен степени меньшей $n$, так что они изоморфны
        через этих представителей. Так как в между идеалами фактора и идеалами
        содержащими идеал, по которому факторизуем, наблюдается соответствие, то
        все идеалы фактор кольца имеют вид $(x^k)$, где $k<n$. Их конечное
        количество, а значит кольцо артиново.

    \item \textbf{Постройте пример артинова кольца, в котором бесконечно много идеалов.}

        Пусть $A=\mathbb{R}[x,y]/(x^2,y^2,xy)$ - кольцо. Пусть $\mathfrak{a}
        \subset A$ его нетривиальный идеал. Тогда он содержит ненулевой элемент
        вида $ax+by+c$. Если $c=0$, то он равен $ax+by$ и нильпотент. Если это
        не так, то $(ax+by+c)(-ac^{-2}x-c^{-2}by+c^{-1})=1$ он обратим. Тогда
        чтобы идеал не был тривиальным, мы будем рассматривать $(ax+by)$ этот
        идеал на самом деле прямая плоскости $(ax+by)(qx+wy+c)=acx+bcx$.
        Причем две такие разные прямые образуют плоскоть $(x,y)$. $(x,y)$
        максимален, так как элементы его дополнения обратимы. А значит, что есть
        4 типа идеалов: нулевой, прямые плоскости $(x,y)$, сама плоскость и всё
        пространство. Идеалы этого кольца $\mathbb{R}$-векторные подпространства
        пространства размерности 3, а значит максимальная длина цепи со сторогим
        включением $0\leq V_1\leq\ldots \leq V$ 4, а значит кольцо артиново, но
        так как количество прямых бесконечно, то в кольце бесконечно много идеалов.

    \item \textbf{Докажите, что в (коммутативном) артиновом кольце
        имеется лишь конечное число простых идеалов.}

        Пусть $A$ - кольцо. Возьмём радикал Джекобсона этого
        кольца $\mathfrak{R}$ - пересечение всех максимальных идеалов.
        Тогда все элементы $\mathfrak{R}$ имеют следующее описание $x\in\mathfrak{R}
        \Leftrightarrow 1-xy$ обратим для всех $y$. Так как если $1-xy$ не
        единица, но $x\in\mathfrak{R}$, то $1-xy\in\mathfrak{m}$ он лежит в
        некотором максимально идеале, но и $x\in\mathfrak{m}$ лежит там же.
        тогда и $1=(1-xy)+xy\in\mathfrak{m}$, чего не может быть. В обратную
        сторону, если $x\notin\mathfrak{m}$ не лежит в некотором идеале, то
        $(\mathfrak{m},x)=(1)$, а значит мы найдем $a\in\mathfrak{m}$ и $y\in A$,
        что $a+xy=1$, тогда $1-xy\in\mathfrak{m}$ не обратим.

        Пусть кольцо артиново. Пусть $a\in\mathfrak{R}$, посмотрим на
        убывающую цепочку идеалов $(a^0)\supseteq (a)\supseteq\ldots\supseteq(a^n)
        \ldots$. Так как кольцо артиново, то она с некоторого шага стабилизируется.
        Пусть $(a^n)=(a^{n+1})$, Тогда $a^n=a^{n+1}k$, тогда $a^n(1-ak)=0$, как
        мы видели ранее $1-ak$ обратим, а значит $a^n=0$ и $a$ - нильпотент.

        В любом коммутативном кольце с единицей
        нильпотенты образуют идеал $\mathfrak{N}$. Так как сумма нильпотентов
        $a^n=0=b^m$ в степени суммы равна $(a+b)^{n+m}=\sum_i c_ia^ib^{n+m-i}=0$,
        так как обе степени не могут быть одновременно меньше $n$ и $m$, а значит
        каждое слагаемое занулится. Стабильность при домножении на элементы
        кольца очевидна. Этот идеал называется нильрадикалом и обозначается
        $\mathfrak{N}$. Он в точности является пересечением всех простых
        идеалов. Пусть $a$ - нильпотент, тогда $a^n=0$ лежит в любом простом
        идеале, а значит по простоте идеала там лежит и $a$. В обратную сторону,
        если $a$ не нильпотент, то возьмём множество $S$ всех идеалов, что
        никакая степень $a$ в них не лежит. Это множесво не пусто, так как есть
        нулевой идеал, и упорядочено по включению. Для любой цепи объединение по
        ней является идеалом и не содержит никакой степени $a$. Тогда любая цепь
        имеет верхнюю грань, а значит есть максимальный элемент $\mathfrak{b}$.
        Если $x,y\notin\mathfrak{b}$, то  $(x)+\mathfrak{b}$ и $(y)+\mathfrak{b}$
        строго больше $\mathfrak{b}$, а значит не лежат в $S$. Тогда в них есть
        степени $a$, тогда степень $a$ лежит и в $(xy)+\mathfrak{b}$ и этот
        идеал не лежит в $S$, а значит $xy\notin\mathfrak{b}$. Поэтому
        $\mathfrak{b}$ прост и не содержит $a$.

        Так как любой максимальный идеал прост, то имеет место включение $
        \mathfrak{N}\subseteq\mathfrak{R}$. Для артиновых колец мы видели, что
        любой элемент из радикала Джекобсона – нильпотент, а значит для артиновых
        колец верно $\mathfrak{N}=\mathfrak{R}$.

        Докажем ещё одно утверждение. Если идеалы взаимнопросты, то их
        персечение сопадает с произведением. Докажем это по индукции. Для 2х
        идеалов всегда верно, что $\mathfrak{a}\mathfrak{b}\subseteq\mathfrak{a}
        \cap\mathfrak{b}$, так как произведение вложено как в один идеал, так и
        в другой. С другой стороны мы имеем $(\mathfrak{a}+\mathfrak{b})(\mathfrak{a}
        \cap\mathfrak{b})=\mathfrak{a}(\mathfrak{a}\cap\mathfrak{b})+\mathfrak{b}
        (\mathfrak{a}\cap\mathfrak{b})\subseteq\mathfrak{a}\mathfrak{b}$ здесь
        уже слогаемые вложены в произведение, а значит и их сумма, но так
        как идеалы взаимнопросты, то $\mathfrak{a}+\mathfrak{b}=(1)$, а значит у
        нас есть включение в 2 стороны. Тогда мы наблюдаем равенство $\mathfrak{a}
        \mathfrak{b}=\mathfrak{a}\cap\mathfrak{b}$. Пусть теперь утверждение верно
        для $n-1$ идеалов. Обозначим за $\mathfrak{b}=\prod_{i=1}^{n-1}\mathfrak{a}_i$.
        И так как $\mathfrak{a}_i$ и $\mathfrak{a}_n$ взаимно просты, то есть
        $x_i\in\mathfrak{a}_i$ и $y_i\in\mathfrak{a}_n$, что $x_i+y_i=1$, тогда
        $\prod_i x_i = \prod_i (1-y_i) = 1 - y$, где $y\in\mathfrak{a}_n$, а
        произведение из произведения, тогда $\mathfrak{b}$ и $\mathfrak{a}_n$
        взаимопросты, так как $\prod_i x_i + y = 1$. А значит $\prod_{i=1}^n
        \mathfrak{a}_i=\mathfrak{b}\times\mathfrak{a}_n=\mathfrak{b}\cup\mathfrak
        {a}_n=(\prod_{i=1}^{n-1}\mathfrak{a}_i)\cup\mathfrak{a}_n=(\bigcup_{i=1}
        ^{n-1}\mathfrak{a}_i)\cup\mathfrak{a}_n=\bigcup_{i=1}^n\mathfrak{a}_i$.
        Тогда произведение взаимнопростых идеалов равно их персечению.

        \textbf{Утверждение.} Пусть $\mathfrak{a}_1,\ldots,\mathfrak{a}_n$ - некоторые
        идеалы, $\mathfrak{p}$ – простой идеал, содержащий $\bigcap_{i=1}^n
        \mathfrak{a}_i$, тогда $\mathfrak{p}\supseteq\mathfrak{a}_i$ для некоторого
        $i$. Если в гипотезе равенство, то $\mathfrak{p}=\mathfrak{a}_i$.
        Предположим, что это не так и $\mathfrak{p}\nsupseteq\mathfrak{a}_i$ для
        всех $i$. Тогда выберем элементы $x_i\in\mathfrak{a}_i$, что $x_i\notin
        \mathfrak{p}$. Тогда по простоте произведение $x_i$ не лежит в
        $\mathfrak{p}$, а значит произведение не лежит в пересечение идеалов
        $\mathfrak{a}_i$, ну а это противоречие. Если наблюдается равенство
        $\mathfrak{p}=\bigcap\mathfrak{a}_i$, то $\mathfrak{p}\subseteq
        \mathfrak{a}_i$, но и так как $\mathfrak{p}\supseteq\mathfrak{a}_i$, то и
        здесь будет равенство.

        Пусть теперь $\mathcal{M}$ – множество идеалов, образованных из
        конечного произведения максимальных. Так как кольцо артиново, то у любой
        убывающей цепи есть нижняя грань, а значит есть минимальный элемент
        $\mathfrak{m}_1\cdot\ldots\cdot\mathfrak{m}_n$. Пусть $\mathfrak{m}$ –
        максимальный идеал, так как $\mathfrak{m}\cdot\mathfrak{m}_1\ldots\mathfrak{m}_n
        \subseteq\mathfrak{m}_1\ldots\mathfrak{m}_n$, то по минимальности получим,
        что $\mathfrak{m}\cdot\mathfrak{m}_1\cdot\ldots\cdot\mathfrak{m}_n=\mathfrak{m}_1
        \cdot\ldots\cdot\mathfrak{m}_n$, а $\mathfrak{m}_1\cdot\ldots\cdot\mathfrak{m}_n=\mathfrak{m}
        \cdot\mathfrak{m}_1\cdot\ldots\cdot\mathfrak{m}_n\subseteq\mathfrak{m}\cap
        \mathfrak{m}_1\cdot\ldots\cdot\mathfrak{m}_n\subseteq\mathfrak{m}$, и
        если убрать промежуточные шаги, то останется $\mathfrak{m}_1\cdot\ldots
        \cdot\mathfrak{m}_n\subseteq\mathfrak{m}$. Но так как различные
        максимальные идеалы взаимнопросты, то их произведение совпадает с
        пересечением, то есть $\mathfrak{m}_1\cap\ldots\cap\mathfrak{m}_n
        \subseteq\mathfrak{m}$, но так как максимальный идеал прост, то можно
        применить предыдущее утверждение и получить, что $\mathfrak{m}\supseteq
        \mathfrak{m}_i$ для некого $i$, но так как они оба максимальны, мы
        наблюдаем равенство. $\mathfrak{m}=\mathfrak{m}_i$, тогда $\mathfrak{m}_i$
        для $1\leq i\leq n$ – все максимальные идеалы и их конечное количество.

        Теперь как мы видели радикал Джекобсона – произведение конечного числа
        максимальных идеалов, они взаимнопросты, а значит фактор по ним изоморфен
        произведению факторов. Фактор по максимальному идеалу - поле. Тогда
        фактор по радикалу Джекобсона – произведение полей, а в нём идеалы –
        это произведения идеалов, так как для каждого идеала можно выписать
        все индексы для которых есть элементы в которых соответсвующие координаты
        ненулевые, а значит у нас идеал – произведение с нулями в индексах для
        которых нет ненулевых координат, и с всеми полями для тех индексов, для
        которых это встречается. Тогда в таком фактор кольце конечное число
        идеалов, а так как к тому же любой простой идеал содержит нильрадикал,
        который в этой задаче совпадает с радикалом Джекобсона, то все простые
        идеалы стоят в однозначном соответствии с неокторыми идеалами из фактор
        кольца коих конечное число, а тогда и простыx тоже конечное число.

    \item \textbf{Пусть $A$ – нётерово кольцо. Докажите, что кольца $A[x]$ и
        $A[[x]]$ нетеровы.}.

        Перед тем как доказать это проанализируем то, как устроены нётеровы
        кольца. Дальше $A$ будет обозначать нётерово коммутативное ассоциатовное
        кольцо с единицей.

        Заметим, что через лемму цорна мы получим, что шнётеоровость эквивалентна,
        тому что любой набор идаелов кольца имеет максимальный элемент.

        Также нётеровость эквивалента тому, что каждый идеал кольца конечно
        порожден. Пусть идеал $\mathfrak{a}$ не конечно порожден.
        Тогда по аксиоме выбора мы найдем последовательность элементов этого
        идеала $(a_i)_i$, что будут строгими следующие включения $(a_1)\subset
        (a_1,a_2)\subset(a_1,a_2,a_3)\subset\ldots$. А тогда кольцо не нётерово.
        В обратную соторону, пусть все идеалы конечно порождены, тогда возьмём
        цепь $\mathfrak{a}_1\subseteq\mathfrak{a}_2\subseteq\ldots$. Возьмём
        объединение по этой цепи, мы получим идеал, что конечно порожден. Так
        как его порождает конечное число элементов, то мы найдем конечно число
        звеньев, в которых лежат эти элементы, максимальное из звеньев содержит
        их все, а значит оно равняется всему объединению и цепь стабилизируется, 
        а тогда кольцо нётерово.

        Назовем идаел $\mathfrak{a}$ \emph{неприводимым}, если для любых идеалов
        $\mathfrak{b},\,\mathfrak{c}$ имеет место следующее соотношение $\mathfrak{a}
        =\mathfrak{b}\cap\mathfrak{c}\Rightarrow\mathfrak{a}=\mathfrak{b}\,\text{или}
        \,\mathfrak{a}=\mathfrak{c}$.

        Покажем, что любой идеал нётерова кольца $A$ представляется через
        конечное пересечение неприводимых. Положим $M$ множество всех идеалов
        $A$, которые не представляются через конечное пересечение неприводимых.
        Пусть оно не пусто. Тогда из нётеровости следует, что в нём есть
        максимальный элемент $\mathfrak{m}$. В частности $\mathfrak{m}$ не
        приводим, а это значит, если мы обернём определение приводимости,
        что мы найдем два идеал, что выполнено следующее $\mathfrak{m}=\mathfrak{a}
        \cap\mathfrak{b}$ и $\mathfrak{a}\neq\mathfrak{m}\neq\mathfrak{b}$. Тогда
        два найденых идеала строго больше максимального элемента, а значит не
        лежат в $M$. Тогда они представимы как конечное пересечение неприводимых,
        это же будет верно и для $\mathfrak{m}$, что ведёт к противоречию, а
        значит $M$ пусто. Тогда все идеалы представимы как конечное пересечение
        неприводимых.

        Теперь покажем, что собственные неприводимые идеалы в нётеровом кольце
        примарны. Пусть $\mathfrak{p}$ – неприводимый идеал. Возьмём фактор по
        нему. Нетрудно видеть, что $A/\mathfrak{p}$ тоже нётерово, а 0 в нём
        неприводим. Покажем, что
        в нём нет делителей нуля. Пусть это не так, тогда мы найдем делителей
        $ab=0$, что $a\neq0\neq b$. Также мы построим возрастающую цепь аннуляторов
        $\text{Ann}(a)\subseteq\text{Ann}(a^2)\subseteq\ldots$. По нёторовости
        оно стабилизируется с некоторого шага $n$. Теперь посмотрем на
        персечение $(b)\cap(a^n)\ni z$. Элемент из пересечения имеет две записи
        $z=xa^n=yb$. Домножим это дело на $a$, $za=xa^{n+1}=yba=y0=0$, тогда $x$
        – аннулятор $a^{n+1}$, а значит и $a^n$ тоже, так как их аннуляторы
        совпадают. Тогда $z=0$ и пересечение нуль. Но нулевой идеал неприводим,
        а значит либо $(a^n)=(0)\Rightarrow a=0$, либо $(b)=(0)\Rightarrow b=0$.
        Тогда примарный и $\mathfrak{p}$.

        Из всего вышесказанного выходит, что идеалы нётеровых колец - конечные
        пересечения примарных. Покажем теперь, что на самом деле каждый идеал
        нётерова кольца содержит некую степень своего радикала. Пусть $\mathfrak{a}$
        – идеал, а $r(\mathfrak{a})$ – его радикал. Так как все идеалы нётерова
        кольца конечно порождены, то мы найдем конечное порождение $r(\mathfrak{a})
        =(a_1,a_2,\ldots,a_k)$. Тогда для каждого порождающего элемента мы найдем
        степень $n_i$ в которой он лежит в $\mathfrak{a}$, то есть $a_i^{n_i}\in
        \mathfrak{a}$. Если положить $m=\sum n_i$, то можно заметить, что
        $r(\mathfrak{a})^m=(\prod a_i^{k_i}\;|\;\sum k_i=m)$, но при этом хотя бы
        для одного индекса в каждом новом порождающем элементе будет верно, что
        $k_i\geq n_i$, а значит каждый порождающий лежит в $\mathfrak{a}$, а
        тогда $(r(\mathfrak{a}))^m\subseteq\mathfrak{a}$. В частности это означает,
        что нильрадикал в некоторой степини равен нулю, то есть $(\mathfrak{N})^n
        =(r((0)))^n=(0)$.

        Пусть $\mathfrak{a}\trianglelefteq A[X]$, множество старших коэффициентов
        из $\mathfrak{a}$ очевидно образует идеал $\mathfrak{b}$ в $A$. Так как
        кольцо $A$ нётерово, то $\mathfrak{b}$ конечно порожден элементами
        $\mathfrak{b}=(b_1,\ldots,b_n)$. Тогда для каждого $1\leq i\leq n$
        найдем полином $p_i(x) = a_ix^{d_i} + (\text{мономы меньшей степени})$
        из $\mathfrak{a}$. Положим $\mathfrak{a}'=(p_1(x),\ldots,p_i(x))$ и $
        d=\max\{d_1,\ldots,d_i\}$. Возьмём $M=\{q(x)\in A[x]\;|\;\text{deg}(q(x))
        \leq d\}$. Тогда я утверждаю, что $\mathfrak{a}=(\mathfrak{a}\cap M)+
        \mathfrak{a}'$. Очевидно, что $(\mathfrak{a}\cap M)+\mathfrak{a}'\subseteq
        \mathfrak{a}$, потому как это верно для каждого из слогаемых. В обратную
        сторону положим $p(x)=ax^m+\ldots\in\mathfrak{a}$. Тогда если $m\leq d$,
        то $p(x)\in\mathfrak{a}\cap M$. Тогда проверим для случая, когда $m>d$.
        По определению получим, что $a\in\mathfrak{b}$. Тогда мы найдем
        многочлен из $\mathfrak{a}'$ с подходящим коэффициентом и тогда $p(x)$
        будет представлено, как сумма многочлена из $\mathfrak{a}'$ и
        многочлена меньшей степени. Продолжая так мы за конечное число шагов
        получим сумму многочленов из $\mathfrak{a}'$ и одного из $\mathfrak{a}
        \cap M$. А значит вложение в обратную сторону также верно.

\end{enumerate}

\end{document}
