\documentclass{article}
\usepackage[a4paper,left=3cm,right=3cm,top=1cm,bottom=2cm]{geometry}
\usepackage{amsmath}
\usepackage{amssymb}
\usepackage{hyperref}
\usepackage[russian]{babel}

\usepackage{tikz-cd}
\usepackage{array}
\usepackage{graphicx}
\newcommand\mapsfrom{\mathrel{\reflectbox{\ensuremath{\mapsto}}}}
\setlength{\parindent}{0mm}

\usepackage{fontspec}
\setmainfont{Linux Libertine O}
\usepackage{unicode-math}
\setmathfont{Cambria Math}

\title{
\textit{\small{Георгий Потошин, 2024}}\\
\vspace{0.3ex}
\textit{\huge{Алгебра I, листочек 5}}\vspace{1ex}
}

\date{\vspace{-10ex}}

\begin{document}
\maketitle

\begin{enumerate}
    \item \textbf{Опишите все простые и максимальные идеалы в кольцах $\mathbb{Z}
        /n\mathbb{Z}$ и $\mathbb{k}[x]$, где $\mathbb{k}$ – поле.}

        Как мы видели оба эти кольца – кольца главных идеалов. Первое, так
        как это фактор кольца главных идеалов, а второе целостное и имеет
        деление с остатком, а значит КГИ, тогда дальше все идеалы главные.

        $\mathbb{k}[x]$ целостно и КГИ, а значит все простые идеалы – главные
        идеалы неприводимых элементов. Тогда это описание сводится к описанию
        всех неприводимых элементов, в $\mathbb{C}$ неприводимы например только
        полиномы степени 1. Так как $\mathbb{k}[x]$ целостное кольцо главных
        идеалов, то простые идеалы максимальны. Так как для простых идеалов верно
        $(a)\subseteq(p)$, а значит $p=au$, но так как $p$ неприводим и
        $a$ не обратим, то обратим $u$, а значит $(a)=(p)$.

        Кольцо $\mathbb{Z}/(n)$ в общем случае не целостно. Посмотрим на
        каноническую проекцию $\pi:\mathbb{Z}\longrightarrow\mathbb{Z}/(n)$.
        Так как прообразы простых идеалов просты, то как минимум нужно
        рассматривать образы простых из $\mathbb{Z}$. Пусть $(p)\subset\mathbb{Z}$
        – простой идеал, тогда $\pi[(p)]=([1])$, если $p\wedge n=1$ и
        $\pi[(p)]=([p])$ – собственный идеал фактор кольца в противном. Покажем,
        что во втором случае мы получим простой идеал. Если $p=ab+m$, то $ab$
        делится на $p$, а значит на простое $p$ делится и один из множителей.
        Это означает, что $([p])$ – простое. Так мы получили, что все простые
        идеалы имеют вид $([p])$, где $p$ прост и делит $n$. Каждый простой
        идеал в данном случаем будет максимальным, так как они образы
        максимальных из $\mathbb{Z}$ при сюрьективном гомоморфизме, то есть,
        если $([a])\subseteq([p])$, то $(a)=(p)$ по максимальности второго, а
        тогда и $([a])=([p])$.

    \item \textbf{Элемент x кольца A называется простым, если порожденный им
        идеал $(x)$ прост. Необратимый элемент $x$ целостного кольца $A$
        называется неприводимым, если его нельзя представить в виде произведения
        двух необратимых элементов $A$.}

        \textbf{Покажите, что в целостном кольце
        ненулевой простой элемент неприводим. Покажите, что обратное, вообще
        говоря, неверно. Покажите, что в факториальных кольцах неприводимые
        элементы просты.}

        Пусть $A$ – целостно и $p\neq 0$ прост. Тогда если $p=ab$, то без
        потери общности положим $a\in (p)$. Тогда $a=pc$ и $p=pcb$. Перенесем
        все в одну сторону $p(1-cb)=1$. Так как кольцо целостно и $p$ не нуль,
        то $1=cb$ и $b$ обратим. А значит $p$ неприводимо. Покажем на примере,
        что обратное не верно.

        Возьмём $\mathbb{Z}[i\sqrt 5]$ оно целостно, так как подкольцо поля.
        Покажем, что 2 в нем неприводим.
        Норма $x=a+i\sqrt5 b$ равна $a^2+5b^2$, если $xy=2$, то будет верно,
        что $|x|^2|y|^2=4$, то есть произведение двух целых чисел равно 4.
        Что бы они не были обратимыми, нужно чтобы оба числа равнялись 2.
        Но $a^2+5b^2=2$ не имеет целых решение, так как $b$ не может быть не
        нулём, а 2 не квадрат. А значит 2 неприводим. С другой стороны
        $(1+i\sqrt 5)(1-i\sqrt 5)=6\in (2)$, но ни одно множимое там не лежит,
        так 2 их не делит, а значит 2 не прост.

        Пусть теперь $x$ не прост, тогда мы найдём $a,b\notin(x)$, но $ab\in(x)$.
        Заметим, что они оба не нули и оба необратимы, в противном случае если
        $a$ обратим, то $ab\in(x)\Rightarrow b=a^{-1}ab\in(x)$, что не верно.
        Тогда в каком-нибудь разложении на неприводимы в $a$ и $b$ неприводимых
        будет не меньше 1, иначе они обратимы, тогда по факториальности кольца
        в $x$ неприводимых не меньше 2, но тогда $x$ не неприводим, так как в
        факториальном кольце в разложении неприводимых всегда только одно
        неприводимое.

    \item \textbf{Пусть A – ненулевое кольцо. Следующие утверждения равносильны:
        \begin{enumerate}
            \item $A$ – поле,
            \item в $A$ нет идеалов, кроме $(0)$ и $(1)$,
            \item любой гомоморфизм из A в ненулевое кольцо инъективен
        \end{enumerate}}

        \textbf{(a)$\Rightarrow$(b):} Ненулевые элементы поля необратимы, а
        значит если идеал содержит что-то помимо нуля, то он всё поле.

        \textbf{(b)$\Rightarrow$(c):} Пусть в кольце $A$ только 2 идеала, и
        $f:A\rightarrow B$ ядро $f$ - идеал в $a$, так как единица переходит в 1,
        то ядро не всё кольцо, а идеал не равный кольцу – только $(0)$, а значит
        $f$ инъективен.

        \textbf{(c)$\Rightarrow$(b):} Пусть $a\in A\setminus\{0\}$,
        $\pi:A\longrightarrow A/(a)$ - нетнъективный гомоморфизм колец, а
        значит кообласть нулевое кольцо, а тогда $(a)=A$, а значит $a$ обратим.
        Это верно для любого ненулевого элемента, а значит $A$ - поле.

    \item \textbf{Элемент $0\neq x\in A$ называется нильпотентом, если $x^n=0$
        для некоторого $n$. Докажите, что множество всех нильпотентов в $A$
        является идеалом. Он называется нильрадикалом кольца $A$ и обозначается
        $\mathfrak{N}(A)$. Покажите, что в фактор-кольце $A/\mathfrak{N}(A)$ нет
        нильпотентов.}
        
        Пусть $x,y\in A$ нильпотенты и $a\in A$ в коммутативном кольце. Тогда
        $x^n=0\Rightarrow(ax)^n=0$ и $x^n=0=y^m\Rightarrow(x+y)^{n+m}=0$, а
        значит нильрадикал идеал. Пусть $[a]\in A/\mathfrak{N}(A)$ такой, что
        $[a]^n=0$, тогда мы $a^n\in\mathfrak{A}$, а значит $(a^n)^m=0$ и $a$ –
        нильпотент, а тогда $a\in\mathfrak{N}$ и $[a]=[0]$. Тогда в фактор
        кольце нет делителей нуля.

    \item \textbf{Докажите, что нильрадикал кольца A совпадает с пересечением
        всех простых идеалов A.}

        Пусть $a$ нильпотент, тогда $a^n=0\in\mathfrak{p}$, но так как идеал
        прост, то значит $a\in\mathfrak{p}$ для любого простого идеала
        $\mathfrak{p}$. Теперь пусть $a$ не нильпотент. Пусть $S$ множество
        всех идеалов, что не содержат никакую степень $a$, это множество
        замкнуто относительно объединений цепей и не пусто, так как содержит
        $(0)$. Тогда любая цепь имеет верхнюю грань, а значит по лемме Цорна
        есть максимальный элемент $\mathfrak{p}$. Пусть $x,y\notin\mathfrak{p}$.
        Тогда $(x)+\mathfrak{p}$ и $(y)+\mathfrak{p}$ строго больше $\mathfrak{p}$,
        а значит не лежат в $S$, тогда в них есть некоторые степени $a$, тогда
        они же есть в их произведении $(xy)+\mathfrak{p}$, а значит и этот идеал
        строго больше $\mathfrak{p}$, а тогда $xy\notin\mathfrak{p}$. Значит
        $\mathfrak{p}$ прост и не содержит $a$. Тогда не нильпотенты не лежат
        в пресечении всех простых идеалов.

    \item \textbf{Радикалом Джекобсона $\mathfrak{R}(A)$ кольца $A$ называется
        пересечение всех максимальных идеалов в $A$. Докажите, что
        $x\in\mathfrak{R}(A)$ эквивалентно тому, что $1 − xy$ является
        обратимым элементов для всех $y\in A$.}
        
        Пусть $x\in\mathfrak{R}(A)$ и $1-xy$ не единица, тогда $1-xy\in\mathfrak{m}$
        в некотором максимальном идеале и $xy\in\mathfrak{m}$, но в таком
        случае $1\in\mathfrak{m}$, что противоречие. В обратную сторону, если
        $x\notin\mathfrak{m}$ не лежит в некотором максимальном идеале, то
        $(x,\mathfrak{m})=(1)$, а значит $xy+u=1$ и $u=1-xy$ не единица.

    \item \textbf{Кольцо $A$ называется конечно порожденным, если существует
        сюръективный гомоморфизм колец $\mathbb{Z}[x_1,\ldots,x_n]\twoheadrightarrow A$.
        Верно ли, что если $A$ – нётерово, то всякое подкольцо $A$ конечно
        порождено?}

        То, что конечно порожденные идеалы и кольца имеют схожее название –
        совпадение так как они порождаются по разному. Например $\mathbb{Q}$ –
        поле, а значит нётерово, но при этом не конечно порождено, так как
        из конечного набора дробей нельзя получить дробь со взаимопростым
        знаменателем.

    \item \textbf{Радикалом $r(\mathfrak{a})$ идеала $\mathfrak{a}\subseteq A$
        назовем множество
        \[ r(\mathfrak{a})=\{x\in A| x^n\in\mathfrak{a}\}\]
        где в определении $n$ зависит от $x$.}
        \begin{enumerate}
            \item \textbf{Покажите, что $r(\mathfrak{a})$ – идеал в $A$, и $\mathfrak{a}
                \subseteq r(\mathfrak{a})$.}

                Если $x^n\in\mathfrak{a}$, то $(ax)^n$ тоже. Если $x^n,y^m\in
                \mathfrak{a}$, то $(x+y)^{m+n}$ тоже, так что радикал – идеал.
                Очевидно, что $\mathfrak{a}\subseteq r(\mathfrak{a})$, так как
                мы берём $n=1$ в определении.

            \item \textbf{Покажите, что $r(r(\mathfrak{a}))=r(\mathfrak{a})$.}

                Из прошлого пункта мы знаем, что $r(\mathfrak{a})\subseteq
                r(r(\mathfrak{a}))$. Обратно, если $x^n\in r(\mathfrak{a})$,
                то $(x^n)^m\in \mathfrak{a}$, а значит $x\in r(\mathfrak{a})$.

            \item \textbf{Покажите, что $r(\mathfrak{a})$ совпадает с пересечением всех
                простых идеалов, содержащих $\mathfrak{a}$.}

                Возьмём $A/\mathfrak{a}$. Так как есть биективное сооветствие
                между идеалами содержащими $\mathfrak{a}$ и идеалами фактор
                кольца, то $r(\mathfrak{a})$ соответствует нильрадикал фактор-кольца.
                Нильрадикал, как мы видели является пересечением всех простых.
                Тогда $r(\mathfrak{a})$ является пересечением всех прообразов
                простых при канонической проекции $\pi:A\twoheadrightarrow A/
                \mathfrak{a}$. Но так как если идеал прост в $A$, то его
                проекция тоже проста, из-за того, что простота может быть
                характеризована только на языке идеалов, то есть $\mathfrak{p}$
                просто согда $\mathfrak{a},\mathfrak{b}\nsubseteq\mathfrak{p}$
                имплицирует $\mathfrak{a}\mathfrak{b}\nsubseteq\mathfrak{p}$
                и очевидно, что при проекции это свойство проецируется. А значит
                радикал – это в точности пересечение всех простых содержащих
                идеал.

            \item \textbf{Пусть $\mathfrak{a},\mathfrak{b}$ –идеалы в $A$.
                Покажите, что $r(\mathfrak{a}\cap\mathfrak{b})=r(\mathfrak{a})
                \cap r(\mathfrak{b}) = r(\mathfrak{a}\mathfrak{b})$.}

                Пусть простой идеал содержит пересечение идеалов $\mathfrak{a}\cap
                \mathfrak{b}\subseteq\mathfrak{p}$, тогда идеал содержит и
                произведение идеалов $\mathfrak{a}\mathfrak{b}\subseteq\mathfrak{p}$.
                Если вспомнить определение радикала через пресечения мы получим
                $r(\mathfrak{a}\mathfrak{b})\subseteq r(\mathfrak{a}\cap\mathfrak{b})$.
                Дальше, если $x^n\in\mathfrak{a}\cap\mathfrak{b}$, то верно
                включение и по отдельности, а значит $r(\mathfrak{a}\cap\mathfrak{b})
                \subseteq r(\mathfrak{a})\cap r(\mathfrak{b})$. Теперь пусть
                $x^n\in\mathfrak{a}$ и $x^m\in\mathfrak{b}$, тогда $x^{n+m}\in
                \mathfrak{a}\mathfrak{b}$, а значит верно и последнее включение
                $r(\mathfrak{a})\cap r(\mathfrak{b})\subseteq r(\mathfrak{a}
                \mathfrak{b})$.

            \item \textbf{Пусть $\mathfrak{a},\mathfrak{b}$ – идеалы в $A$.
                Покажите, что $r(\mathfrak{a}+\mathfrak{b})=r(r(\mathfrak{a})+
                r(\mathfrak{b}))$.}

                Пусть $x^n\in\mathfrak{a}+\mathfrak{b}$, тогда $x^n\in
                r(\mathfrak{a})+r(\mathfrak{b})$, так как идеалы включены в свои
                радикалы. В обратную сторону, пусть $x^n\in\mathfrak{a}$ и
                $y^m\in\mathfrak{b}$, тогда $(x+y)^{n+m}\in\mathfrak{a}+\mathfrak{b}$,
                а значит включение в обратную сторону также верно.

            \item \textbf{Покажите, что в нетеровом кольце $A$ для идеала $\mathfrak{a}$
                существует число $N$ такое, что $r(\mathfrak{a})^N\subseteq
                \mathfrak{a}\subseteq r(\mathfrak{a})$.}

                Включение $\mathfrak{a}\subseteq r(\mathfrak{a})$ уже было нами
                проверено. Теперь мы знаем, что радикал – это идеал, а в нётеровом
                кольце идеалы конечно порождены. Тогда пусть радикал порождается
                следующими элементами $(a_1,\ldots,a_n)$. Тогда $a_i^{n_i}\in
                \mathfrak{a}$ для некоторой степени, тогда положим $m=\sum_i n_i$.
                Тогда нетрудно заметить, что $r(\mathfrak{a})^m=\{\prod a_i^{k_i}
                \;|\;\sum_ik_i=m \}$, а также, что каждый порождающий элемент
                лежит в частности в $\mathfrak{a}$, так как хотя бы одна степень
                $k_i$ будет не меньше $n_i$. Тогда верно и второе включение.
        \end{enumerate}
        
    \item \textbf{Определим кольцо гауссовых чисел (здесь $i=\sqrt{−1}$)
        \[\mathbb{Z}[i]=\{a+ib\;|\;a,b\in\mathbb{Z}\}\subset\mathbb{C}.\]}
        \begin{enumerate}
            \item \textbf{Покажите, что оно целостно, евклидово и факториально.}

                Зафиксируем на этом кольце инволюцию $z\mapsto z^*$ – комплексное сопряжение.
                Зафиксируем на этом кольце норму $|z|=\sqrt{zz^*}$. Она очевидно
                мультипликативна, строго положительна для ненулевых элементов
                и удовлетворяет неравенству треугольника $|a+b|\leq |a|+|b|$.

                Из мультипликативности и строгой положиетельности нормы следует
                что в этом кольце нет делителей нуля и оно целостно. Теперь покажем,
                что можно делить с остатком. Во первых хоть норма и не имеет
                значений в целых числах, но порядок, что она индуцируют
                изоморфен порядку целых чисел. Дальше пусть $a,b$ два гаусовых
                числа, попробуем одно поделить на другое. Посмотрим на множество
                $a-(b)$ найдём в нём минимальный элемент относительно нормы,
                такой существует, так как порядок изоморфен $\mathbb{N}$.
                Назовём его $d$, тогда мы имеем $|d|\leq|a-bk|$, для любого
                Гауссова $k$. Если $|d|\geq|b|$, то если ввести на гауссовых 
                числах структуру $Z$ свободного модуля с каноническим базисом
                и каноническим скалярным произведением, то мы заметим следующее.
                Хотя бы одно из чисел $d+b$, $d-b$, $d+ib$, $d-ib$ будет меньше
                $d$ так как если зафиксировать $b=m+in$ и $d=k+il$, то можно
                посмотреть на произведения:
                \[\langle d,b\rangle=km+nl,\quad\langle d,-b\rangle=-km-nl,\quad
                \langle d,ib\rangle=-kn+lm,\quad\langle d,b\rangle=kn-lm.\] То как
                минимум одно будет положительным, в противном случае они все нулевые
                и мы имеем $lm=kn$ и $km=-nl$. Если $|d|=0$, то победа, мы
                поделили, но это вообще говоря не так, так как $|b|>0$ ненулевое.
                Поэтому пары $(m,n)$ и $(k,l)$ ненулевые. Если без потери
                общности $m=0$, то $n\neq 0$, а значит из равенства $0=kn$
                мы заключаем, что $k=0$, а из $0=-nl$, что $l=0$, чего быть не
                может.

                Теперь пусть мы нашли одно нулевое произведение. Пусть оно
                третье. Тогда $kn=ml$, а значит $b$ пропорционально $d$. Путь
                $d=b*k$ (ну или с $-b$), где $k>1$. Тогда $|d-b|=|bk-b|=|b|(k-1)$
                и мы нашли меньший элемент в $a-(b)$, противоречие. В ином
                случае ни одно из произведений не нулевое. А значит мы найдем
                пару положительных, пусть они без потери общности $\langle d,b\rangle$
                и $\langle d,ib\rangle$. Пусть $d_1,d_2\in\mathbb{R}$ -
                координаты $d$ в $(b,ib)$, они положительны, так как этот базис
                ортогонален, а координаты в точности равны скалярному
                произведению на нужный элемент базиса, поделено на квадрат
                нормы этого элемента. Тогда посмотрим на пару векторов $d-b$ и
                $d-ib$, квадраты из норм равны
                
                \begin{align*}
                    |d-b|^2&=|d|^2-2\langle d, b\rangle+|b|^2=|d|^2+|b|^2(1-2d_1) \\
                    |d-ib|^2&=|d|^2-2\langle d,ib\rangle+|ib|^2=|d|^2+|b|^2(1-2d_2)
                \end{align*}

                Заметим, что так как $|d|\geq|b|$, то в частности $|d|=|b(d_1+id_2)|=
                |b||d_1+id_2|$, а значит $d_1^2+d_2^2\geq1$. Тогда если оба $1-2d_1\geq0$
                и $1-2d_2\geq0$, то $0\leq d_1,d_2\leq1/2$ и $d_1^2+d_2^2\leq 1/4$,
                чего не может быть, а значит одна из разностей отрицательна и
                мы вновь найдем элемент меньший минимального, чего не может быть.
                Тогда наше предположение о том, что $|d|\geq|b|$ не верно и мы
                поделили с остатком, а тогда кольцо евклидово.

                Дальше я буду использовать утверждения из 5 лекции про кольца.
                Наше кольцо целостно и евклидово, а значит это кольцо главных
                идеалов. Наше целостное кольцо главных идеалов, а значит оно
                факториально.

            \item \textbf{Найдите в нем все обратимые элементы.}

                Из-за свойств введенной ранее нормы, её значения никогда не
                меньше нуля и она мультипликативна, а значит у обратного
                элемента норма - обратное число, но из-за ограничения на
                значения нормы мы получаем, что у обратимых элементов норма равна
                1. Такие элементы $i,-i,1,-1$ и они правда обратимы. Других нет.
        \end{enumerate}

    \item \textbf{Докажите, что простое натуральное число $p$ является простым
        числом Гаусса тогда и только тогда, когда уравнение $x^2 + 1 = 0$ не
        имеет решения по модулю $p$, то есть $−1$ не является квадратом по модулю
        $p$.}

        Пусть мы нашли решение $[a]\in\mathbb{Z}/(p),0\leq a<p$ уравнения
        $x^2+1=0$, то есть $(a^2+1)=kp$ тогда построим разложение
        $kp=(a-i)(a+i)$. Очевидно, что ни $(a-i)$, ни $(a+i)$ не лежат в $(p)$,
        но зато лежит их произведение, а значит $p$ не прост. Пусть теперь $p$
        не прост, а так как кольцо факториально, то $p$ приводим, мы найдём разложение
        на необратимые элементы. Так как $p$
        прост в $\mathbb{Z}$, то его разложение будет содержать ненулевую мнимую
        часть, более того аргументы комплексных чисел должны быть противоположны,
        а тогда они будут иметь вид $k(a-bi)$ и $l(a+bi)$ (мы сможем найти
        на прямых, где лежат наши и 0, числа ближайшие к нулю, одно очевидно
        будет получать из другого через сопряжение, обозначим первое за $a-bi$).
        $k,l,a,b\in\mathbb{Z}^*$, $a$ также не нуль, иначе $p=-klb^2$ и один из
        множителей будет обратим, что нам не интересно. Перемножим их $kl(a^2+b^2)=p$
        из-за простоты $p$ в $\mathbb{Z}$, $k=1=l$ без потери общности. Тогда $p=a^2+b^2$, заметим,
        что без потери общности $0<a,b<p$, а значит они не делят $p$. Тогда их
        классы в $[a],[b]\in\mathbb{Z}/(p)$ обратимы и найдем $0<c<p$, что
        $[c]=[a][b]^{-1}$, тогда $[b]^2([c]^2+1)=[b]^2([a]^2[b]^{-2}+1)=[a]^2+[b]^2=0$
        И так как кольцо $\mathbb{Z}/(p)$ целостно и $[b]\neq0$, то $[c]^2+1=0$,
        а тогда мы нашли решение $c$ уравнения $x^2+1=0$ по модулю $p$.

    \item \textbf{Докажите, что простое натуральное число является простым числом
        Гаусса тогда и только тогда, когда оно имеет вид $4k−1$.}

        Заметим, что $2=(1-i)(1+i)$ не прост. Дальше будем под $p$ понимать
        простое натуральное число отличное от 2.

        Для решения этой задачи проанализируем как квадраты устроены в 
        $\mathbb{F}_p=\mathbb{Z}/(p)$. Зафиксируем некоторые гомоморфизмы
        мультипликативной группы $q:x\mapsto x^2$ и $\varphi:x\mapsto x^{(p-1)/2}$.
        Тогда $\text{Ker}(q)=\{1,-1\}$, так как $\mathbb{F}_p$ поле и в нём
        $x^2=1$ имеет не более двух решений, то есть $1$ и $-1$, по теореме об
        гомоморфизме, получаем, что в поле ровно $(p-1)/2$ квадратов. Теперь
        заметим, что все квадраты из $\mathbb{F}_p$ лежат в
        $\text{Ker}(\varphi)$, так как $(x^2)^{(p-1)/2}=x^{p-1}=1$,
        последнее верно по теореме Эйлера. C другой стороны уравнение
        $x^{(p-1)/2}=1$ имеет не более $(p-1)/2$ решений. Но у нас уже есть
        $(p-1)/2$ решение, а именно квадраты, тогда мы заключаем, что $\text{Ker}
        (\varphi)$ – множество всех квадратов. Тогда в частности мы получим, что
        $-1$ квадрат тогда и только тогда, когда $(p-1)/2$ четно, потому что
        возведение в эту степень квадрата даёт 1.

        Теперь, при делении на 4 $p$ можете иметь в остатке только 1 или 3.
        Если $p=4k+1$, $(p-1)/2$ четно, тогда $-1$ – квадрат, а тогда $x^2+1=0$
        имеет решение и
        $p$ не прост в $\mathbb{Z}[i]$. Иначе $p=4k-1$, $(p-1)/2$ не четно,
        $-1$ не квадрат и $x^2+1=0$ не имеет решений и $p$ прост в $\mathbb{Z}[i]$.

    \item \textbf{Опишите множество всех натуральных числа, представимых в виде
        суммы двух квадратов.}
        
        Будем считать, что квадраты тоже раскладываются в сумму, где одно из
        слагаемых нуль. Будем дальше полагать, что в разложение оба слагаемых
        ненулевые.

        Пусть есть число нужного вида $a^2+b^2$. Тогда можно вынести квадрат
        наибольшего общего делителя $a$ и $b$. Так что все представимые числа,
        получаются из всех представимых в виде суммы двух взаимопростых квадратов
        через домножение на некоторый квадрат. Дальше мы положим $a\wedge b=1$.

        Пусть $x$ натуральное число. Оно разложимо единственным образом на
        произведение простых. Тогда если это число раскладывается в сумму
        взаимнопростых квадратов то все его простые делители тоже
        раскладываются в какую-нибудь сумму.

        Пусть $x=a^2+b^2$ представилось как сумма взаимопростых квадратов,
        тогда пусть $p$ прост и $kp=a=x^2+y^2$ очевидно, что $p$ не делит не
        $x$, ни $y$. Тогда аналогично 10 заданию мы получим $[x]^2+[y]^2=0$ и
        найдем в $\mathbb{F}_p$ решение уравнения $x^2+1=0$, а тогда $p$
        разложимо в сумму квадратов.

        С другой стороны если у какого-нибудь числа все его простые делители
        раскладываются в сумму квадратов, то и само число тоже, а именно
        $x=\prod p_i=\prod (a_i+b_ii)(a_i-b_ii)=\prod(a_i+b_ii)\prod(a_i-b_ii)
        =Z*Z^*$, а значит $a$ тоже сумма квадратов.

        Тогда описание всех подходящих нам чисел будет все квадраты и все
        произведения простых вида $4k+1$, умноженные на квадраты.

    \item \textbf{Определим кольцо чисел Эйзенштена (здесь $\rho=\frac{−1+\sqrt3i}{2}$)
        \[\mathbb{Z}[\rho]=\{a+\rho b\;|\;a,b\in\mathbb{Z}\}\subset\mathbb{C}\]}

        \textit{Наблюдение: $\rho^2+\rho+1=0$}

        \begin{enumerate}
            \item \textbf{Покажите, что оно целостно, евклидово и факториально.}
                
                Мы можем опять ввести индуцированную с $\mathbb{C}$ мультипликативную
                норму. Тогда сразу станет ясно, что кольцо целостно. Проверим,
                что оно евклидово.

                Пусть $a,b\in\mathbb{Z}[\rho]$ и $a$ не делит $b$. В множестве
                $a+(b)$ есть минимальный элемент по норме. Назовем его $d$.
                Покажем, что $|d|<|b|$. Пусть это не так и $|d|\geq|b|$, тогда
                поступим также как и для Гауссовых чисел, только применим
                планаметрическое рассуждение. Вектора $b$, $-b$, $\rho b$,
                $-\rho b$, $\rho^2b$, $-\rho^2b$ делят плоскость на углы по 60
                градусов, тогда мы найдем один вектор, до которого от $d$ не
                более 30 градусов, пусть без потери общности это $b$, тогда
                \begin{align*}
                    |b-d|^2&=|b|^2+|d|^2-2\langle b,d\rangle\\
                    &\leq|b|^2+|d|^2-2|d||b|\cos(\pi/6)\\
                    &\leq |d|^2+|b|^2(1-\sqrt(3))\\
                    &\leq |d|^2
                \end{align*}
                Тогда мы получили противоречие о минимальности $d$. А значит мы
                можем делить с остатком. Так как кольцо целостно и евклидово,
                то оно факториально.

            \item \textbf{Найдите в нем все обратимые элементы.}

                Обратимые элементы должны обладать нормой 1, но все такие элементы
                $\pm1,\pm\rho,\pm\rho^2$ обратимы.
        \end{enumerate}

    \item \textbf{Докажите, что простое натуральное число $p$ является простым
        числом Эйзенштейна тогда и только тогда, когда уравнение $x^2−x+1=0$ не
        имеет решения по модулю $p$, то есть либо $p=2$, либо $−3$ не является
        квадратом по модулю $p$.}

         Пусть $a^2-a+1=kp$, $k\neq 0$. Тогда $(a+\rho)(a+\rho^2)=kp$ и так как
         $p$ не делит ни $a+\rho$, ни $a+\rho^2$, то $p$ не прост.

         Обратно, пусть $p$ не простое число эйзенштейна, тогда $p$ сепарабельно,
         так как кольцо факториально. Пусть $p=ab$, тогда как и в прошлый раз
         мы найдем $u$, что $a=ku$ и $p=lu^*$. Но так как $p$ прост в $\mathbb{Z}$,
         то $k,l=1$ без потери общности. Тогда запишем $u=a+\rho b,\;a,b\neq 0$.
         Тогда в $\mathbb{F}_p$ будет верно $[a]^2+[b]^2-[a][b]=0$, если
         поделить на $[b]$, то мы получим $([a]/[b])^2-[a]/[b]+[1]=0$, а значит
         мы нашли решение.

         Для $p=2$ решение нет, дальше $p\neq 2$. Теперь в $\mathbb{F}_p$
         \begin{align*}
             c^2-c+1=0\\
             (c^2-c-1/4)+3/4=0\\
             (4c^-4c+1) = -3\\
             (2c-1)^2=-3
         \end{align*}
         То есть решение есть $\Leftrightarrow$ $p$ не 2 и -3 квадрат.
        
     \item \textbf{Докажите, что простое натуральное число является простым
         числом Эйзенштейна тогда и только тогда, когда оно равно $2$ или имеет
         вид $6k−1$.}

        Случай $p=2$ мы уже рассмотрели, $p=3$ не подходит, тогда рассмотрим
        $p>3$. Введем символ Лежандра для нечетного простого $p$
        \[
            \left(\frac{a}{p}\right) =
            \left\{ \begin{array}{rcl}
                0 & \mbox{если} & [a]_p=[0]_p \\
                1  & \mbox{если} & [a]_p=[b]_p^2\neq[0]_p \\
                -1 & \mbox{если} & \mbox{иначе}
                \end{array}\right.
        \]

        Заметим, что как мы видели в упражнении 11, $\left(\frac{a}{p}\right)\equiv
        a^{\frac{p-1}{2}}(\mbox{mod}\;p)$, а тогда в частности этот символ
        мультипликативен относительно верхнего индекса. Теперь зафиксируем $a$
        не кратное $p$ и положим $p_1=\frac{p-1}{2}$. Для $1\leq i\leq p_1$
        положим $[a\cdot i]=[\varepsilon_i\cdot r_i]$, где $\varepsilon=\pm1$ и
        $1\leq r_i\leq p_1$ и $\varepsilon_0=1,r_0=0$. Тогда так как умножение в мультипликативной
        группе на разные элементы даёт разные результаты, то $-p_1\leq i\leq p_1$
        - представители всех классов, то тогда $[a\cdot i]=
        [\text{sgn}(i)\varepsilon_{|i|}\cdot r_{|i|}]$ все классы для $-p_m\leq i\leq p_m$.
        Тогда $[\prod_{i\in\{-p_1..p_1\}\setminus\{0\}}i]=[\prod_{i\in\{-p_1..p_1\}
        \setminus\{0\}}\text{sgn}(i)\varepsilon_{|i|}r_{|i|}]$, а тогда в мы
        получим, что $[\prod_{i\in\{1..p_1\}}i]=[\prod_{i\in\{1..p_1\}}\varepsilon_i
        r_i]$. Тогда $[a^{p_1}][\prod_{i\in\{1..p_1\}}i]=[\prod_{i\in\{1..p_1\}}
        \varepsilon_ir_i]$, а тогда $[a^{p_1}]=[\prod_{i\in\{1..p_1\}}\varepsilon_i]$.

        Теперь заметим, что для дробей мы имеем:
        \[
            \left\lfloor\frac{2ai}{p}\right\rfloor=
            \left\lfloor2\left\lfloor\frac{ai}{p}\right\rfloor+2\left\{
                \frac{ai}{p}\right\}\right\rfloor=
            2\left\lfloor\frac{ai}{p}\right\rfloor+\left\lfloor2\left\{\frac{ai}{p}
            \right\}\right\rfloor
        \]
        это число четно или нечетно, если наименьший отрицательный вычет меньше
        или больше $p/2$. Отсюда получаем:
        \[
            \varepsilon_i=(-1)^{\left\lfloor\frac{2ai}{p}\right\rfloor}
        \]
        и поэтому символ Лежандра можно выразить следующим образом
        \[
            \left(\frac{a}{p}\right)=(-1)^{\sum_{i=1}^{p_1}\left\lfloor\frac{2ai}
            {p}\right\rfloor}
        \]

        Пусть теперь $a$ нечетно, тогда $a+p$ четно:
        \begin{align*}
            \left(\frac{2a}{p}\right)=\left(\frac{2a+2p}{p}\right)=\left(
            \frac{4\frac{a+p}{2}}{p}\right)=\left(\frac{\frac{a+p}{2}}{p}\right)
            =(-1)^{\sum_{i=1}^{p_1}\left\lfloor\frac{(a+p)i}{p}\right\rfloor}
            =(-1)^{\sum_{i=1}^{p_1}\left\lfloor\frac{ai}{p}\right\rfloor+\sum_{i=1}^{p_1}
            i}
        \end{align*}
        Откуда нетрудно получить
        \[
            \left(\frac{2}{p}\right)\left(\frac{a}{p}\right)=
            (-1)^{\sum_{i=1}^{p_1}\left\lfloor\frac{ai}{p}\right\rfloor+
            \frac{p^2-1}{8}}
        \]
        Если взять $a=1$, то мы получим
        \[
            \left(\frac{2}{p}\right)=(-1)^{\frac{p^2-1}{8}}
        \]
        А тогда для нечетных $a$ будет верно
        \[
            \left(\frac{a}{p}\right)=(-1)^{\sum_{i=1}^{p_1}\left\lfloor\frac{ai}{p}\right\rfloor}
        \]
        Теперь пусть $q,p$ – два простых нечетных числа. Для $x\in\{1..p_1\}$ и
        $y\in\{1..q_1\}$ никогда не будет равенства $qx=py$, так как $[xq]_p
        \neq[0]_p$. Отсюда мы заключим, что $p_1q_1=S_1+S_2$, где $S_1$ – число
        пар $qx<py$ и $S_2$ – число пар $py<qx$. Очевидно, что число пар $x<(p/q)y$
        также равняется $S_1$. При заданном $y$ можно брать $x\in\{1..\left\lfloor
        \frac{p}{q}y\right\rfloor$. А значит
        \[S_1=\sum_{y=1}^{q_1}\left\lfloor\frac{p}{q}y\right\rfloor\]
        Аналогично получим
        \[S_2=\sum_{x=1}^{p_1}\left\lfloor\frac{q}{p}x\right\rfloor\]
        Тогда
        \[\left(\frac{p}{q}\right)=(-1)^{S_1},\quad\left(\frac{q}{p}\right)=(-1)^{S_2}\]
        И
        \[\left(\frac{p}{q}\right)\left(\frac{q}{p}\right)=(-1)^{S_1+S_2}=(-1)^{p_1q_1}\]
        Отсюда заключаем, что
        \[\left(\frac{p}{q}\right)=(-1)^{p_1q_1}\left(\frac{q}{p}\right)\]
        
        \textit{Эти наблюдения были взяты из книги Виноградова "Основы теории чисел"}

        Теперь
        \[\left(\frac{3}{p}\right)=(-1)^{p_1}\left(\frac{p}{3}\right)\]
        А для $-3$ будет
        \[\left(\frac{-3}{p}\right)=\left(\frac{-1}{p}\right)(-1)^{p_1}\left(\frac{p}{3}\right)
        =\left(\frac{p}{3}\right)=(-1)^{\sum_{i\in\{1..(3-1)/2\}}\left\lfloor\frac{pi}{3}\right\rfloor}
        =(-1)^{\left\lfloor\frac{p}{3}\right\rfloor}\]
        Если $p=6k+1$, то
        \[\left\lfloor\frac{6k}{3}+\frac{1}{3}\right\rfloor=2k\text{ - четно}\]
        Если $p=6k+5$, то
        \[\left\lfloor\frac{6k}{3}+\frac{5}{3}\right\rfloor=2k+1\text{ - нечетно}\]
        Других представлений у простых больших 3 очевидно нет. Тогда мы утверждение
        задачи доказано.

\end{enumerate}

\end{document}
