\documentclass{article}
\usepackage[a4paper,left=3cm,right=3cm,top=1cm,bottom=2cm]{geometry}
\usepackage{amsmath}
\usepackage{amssymb}
\usepackage{hyperref}
\usepackage[russian]{babel}

\usepackage{tikz-cd}
\usepackage{array}
\usepackage{graphicx}
\newcommand\mapsfrom{\mathrel{\reflectbox{\ensuremath{\mapsto}}}}
\setlength{\parindent}{0mm}

\usepackage{fontspec}
\setmainfont{Linux Libertine O}
\usepackage{unicode-math}
\setmathfont{Cambria Math}

\newcommand{\mymat}{\mathcal{M\mkern-3mu a\mkern-0.3mu t}}


\title{
\textit{\small{Георгий Потошин, 2024}}\\
\vspace{0.3ex}
\textit{\huge{Алгебра I, листочек 9}}\vspace{1ex}
}

\date{\vspace{-10ex}}

\begin{document}
\maketitle

\begin{enumerate}
    \item \textbf{Докажите, что $F\in\mathbb{k}[x]$, $F(\alpha)=0,\;\alpha\in\mathbb{k}$ влечет
        $(x−\alpha)|F$ (теорема Безу). Докажите, что многочлен степени $n$ над полем имеет не
        более $n$ различных корней. Докажите, что группа
        \[\mu_n(\mathbb{k})=\{\alpha\in\mathbb{k}|\alpha^n=1\}\]
        содержит не больше, чем n элементов.}

        Многочлены над полем можно делить с остатком. Поделим $F$ на $(x-\alpha)$,
        мы получим следующее равенство $F=Q\cdot(x-\alpha)+\beta$. Если подставить
        в это равенство $\alpha$, то занулится все, кроме $\beta$, тогда $0=\beta$,
        что в точности означает, что $F$ делится на $(x-a)$.

        С другой стороны как мы видели ранее $\mathbb{k}[x]$ целостное кольцо
        главных идеалов, а значит в нём единственно разложение на неприводимые, которыми
        в частности являются многочлены степени 1, так как они просты в кольце.
        Тогда в единственном разложении будет только конечное число множителей
        степени 1, и так как степень многочлена равна сумме степеней его фактора,
        то у нас не может быть больше факторов, чем степень многочлена, в частности
        это касается факторов степени 1, а корней не меньше, чем типов факторов
        степени 1, так как каждому корню потенциально соответствует 1 или несколько
        факторов, как мы показали в предыдущем параграфе.
        
        В частности в группе $\mu_n(\mathbb{k})$ лежат все корни $x^n-1$, а их
        не больше $n$.

    \item \textbf{Докажите, что конечная подгруппа мультипликативной группы поля циклична.}

        Пусть $G$ – конечная подгруппа мультипликативной группы поля порядка $n$, она абелева. Обозначим за
        $\psi(d) = \#\{a\in G | a^d = 1\}$. Так как $x^d=1$ имеет решений в $\mathbb{k}$
        не больше, чем $n$, то $\psi(d)\leq d$. Пусть для некого $d$ есть элемент $a$ этого
        порядка, обозначим за $G_d$ множество элементов $G$ порядка $d$, тогда очевидно, что
        $\langle a\rangle\subseteq\{a\in G | a^d=1\}$, но $\#\langle a\rangle = d$,
        a $\#\{a\in G | a^d = 1\}\leq d$, тогда включение превратится в равенство.
        $\langle a\rangle$ циклическая группа порядка $d$, содержащая все корни
        $x^d-1$. Тогда все элементы порядка $d$ лежат в $\langle a\rangle$ и количество
        таких элементов $\phi(d)$. Тогда
        \[n=\#G=\sum_{d|n}\#G_d\leq\sum_{d|n}\phi(d)=n\]
        A значит $\#G_d=\phi(d)$, в частности это верно для $n$, а значит мы находим элемент порядка $n$.
        Он порождает всю группу $G$, тогда эта группа циклическая.
        
    \item \textbf{Докажите, что если $[\mathbb{L}:\mathbb{K}]=2$, то $\mathbb{L}=\mathbb{K}[\sqrt \alpha]$, где
        $\alpha\in\mathbb{K}$.} 

        Это не верно в случае, когда $\text{char}\,K=2$, так как мы можем положить
        $K=\mathbb{F}_2$ и $L=\mathbb{F}_2[x]/(1+x+x^2)$. Для этого нужно проверить
        неприводимость $f(x)=1+x+x^2$, заметим, что никакой элемент $K$ не является
        корнем $f(x)$, так как $f(0)=1$ и $f(1)=1$ и если бы $f(x)$ раскладывался
        в произведение многочленов меньшей степени, то они были бы степени 1 и были
        бы корни. Это означает, что $(f(x))$ максимальный идеал, так как кольцо $K[x]$
        кольцо главных идеалов, а значит нет большего идеала, так как тогда бы
        $f(x)$ равнялся бы произведению двух многочленов меньшей степени и не
        был бы неприводимым. Тогда $L$ поле, как фактор кольца по максимальному
        идеалу. При этом если бы $L=K[\sqrt \alpha]$, то $\sqrt \alpha\in L
        \setminus K$, а это только $x$ и $1+x$, но их квадраты $x^2=x+1$ и
        $(x+1)^2=x$ не лежат в $K$, а значит такая конструкция невозможна.

        Тем не менее если характеристика $K$ не равна 2, то для некоторого $a
        \in L\setminus K$ $\{1,a,a^2\}$ линейно зависимы, а значит найдутся $b_0,
        b_1,b_2\in K$, что $b_0+b_1a+b_2x^2=0$. $b_2$ не может равняться нулю,
        так как иначе бы $a\in K$, но это не так. Тогда поделив на $b_2$ мы
        получим $a^2+pa+q=0$, так как характеристика не равна нулю, то $a^2+pa+q
        =(a+p/2)^2+q-p^2/4=0$, тогда можно положить $\alpha=p^2/4-q\in K$, а $\sqrt
        \alpha=a+p/2\in L$. Так как расширение имеет степень 2, а $\sqrt\alpha$
        и $1$ линейно независимы, то они образуют базис, а тогда они порождают
        $L$ и утверждение доказано.

    \item
    \item
    \item
    \item
    \item \textbf{Пусть $\mathbb{F}$ – конечное поле. Докажите, что любая
        функция $f:\mathbb{F}\rightarrow\mathbb{F}$ является многочленом.
        Приведите пример двух различных многочленов, задающих одинаковую
        функцию.}

        Это так, как можно для каждой функции записать интерполяционный
        многочлен Лагранжа. Пусть $\mathbb{F}=\{a_1,\ldots,a_q\}$, тогда
        для любой $f:\mathbb{F}\rightarrow\mathbb{F}$ мы найдем $\phi:1..q
        \rightarrow1..q$, что $f(a_i)=a_{\phi(i)}$. Запишем многочлен,
        моделирующий эту функцию
        \[F=\sum_{i\in1..q}a_{\phi(i)}\prod_{j\neq i, j\in1..q}\frac{x-a_j}{a_i-a_j}\]
        Для поля $\mathbb{F}_2$ два многочлена $1$, $1+x+x^2$, как мы уже видели,
        моделируют одинаковые функции.

    \item \textbf{Пусть $\mathbb{F}$ – произвольное поле ненулевой характеристики
        $p$ и $\phi:\mathbb{F}\rightarrow\mathbb{F}=x\mapsto x^p$ – отображение.
        Докажите, что это гомоморфизм (гомоморфизм Фробениуса). Приведите два
        примера бесконечных полей характеристики $p$ таких, что в первом случае
        $\phi$ биективен, а во втором – нет.}

        То что $\phi$ переводит произведение в произведение и единицу в единицу
        достаточно очевидно и не требует проверки, убедимся, что $\phi$ переводит
        сумму в сумму. Известно, что $(a+b)^p=a^p+b^p+\sum_{i=1}^{p-1}C_p^ia^ib^{p-i}$,
        но так как $p$ просто и  $C_p^i=p\cdot\ldots\cdot(p-i+1)/i!$, то для
        $i\neq0,p$ верно, что $p\,|\,C_p^i$, а тогда в биноме Ньютона все члены, 
        кроме первого и последнего занулятся, а значит $\phi(a+b)=a^p+b^p$ и
        $\phi$ – гомоморфизм. Так как он гомоморфизм из поля, то он инъективен.
        
        Приведем теперь два примера. $F=\mathbb{F}_2(x)$ является полем, пусть
        $f(x)\in\text{Im}\,\phi$, тогда $f(x)=(p(x)/q(x))^2$, где $p(x)/q(x)$
        несократима, тогда в несократимом виде у $f(x)$ в числителе и знаменателе
        будут стоять многочлены четной степени, это означает, что мы не сможем
        получить например $x$ возведением в квадрат, а значит $\phi$ не биекция.

        Пусть теперь $F=\mathbb{F}_2(\mathbb{Q}^+)=\text{Frac}\,\mathbb{F}_2[\mathbb{Q}^+]$
        Поле частных группового кольца, проверим, что это кольцо целостно. По
        определению $K=\mathbb{F}_2[\mathbb{Q}^+]=\{f:\mathbb{Q}^+\rightarrow
        \mathbb{F}_2\,|\,f\text{ почти всюду 0}\}$. Сложение устроено по
        точечно, а для $f,g\in K$ и $q\in\mathbb{Q}$ $(f\cdot g)(q)=
        \sum_{l\in\mathbb{Q}}f(l)g(q-l)$. Сумма корректна, так как ненулевые
        значения встречаются только конечное число раз, а произведение элементов
        равно нулю почти повсюду по тому же аргументу. Проверим, что
        произведение ненулевых элементов $f,g$ не нуль. Пусть $a\in\mathbb{Q}$ –
        максимальное по условию $f(a)\neq 0$, аналогично $b\in\mathbb{Q}$
        максимально по условию $g(b)\neq 0$, тогда очевидно, что $(g\cdot f)(a+b)
        =f(a)g(b)\neq 0$, а значит кольцо целостно и по нему можно брать поле
        частных. Теперь пусть $f=p/q\in F$, тогда положим $f'=p'/q'$ таким, что
        $p'(x)=p(2x)$, а $q'(x)=q(2x)$. Заметим, что характеристические функции
        $\{\chi_q\}_{q\in\mathbb{Q}}$
        образуют базис $K$ на $\mathbb{F}_2$, а значит например $p'=\chi_{q_1}
        +\ldots+\chi_{q_n}$, где $q_i$ попарно различны, тогда $(p')^2=(\chi_{q_1}
        +\ldots+\chi_{q_n})^2=\chi_{q_1}^2+\ldots+\chi_{q_n}^2=\chi_{2q_1}+\ldots
        +\chi_{2q_n}=p$, а значит $(f')^2=f$ и возведение в квадрат сюръективно
        и в купе с инъективностью биективно.
\end{enumerate}

\end{document}
