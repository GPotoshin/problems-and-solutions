\chapter{Векторные пространства}
\section{Комплексные и действительные структуры}
\subsection{Рационализация}
Пусть $V$ – векторное пространство над полем комплексных чисел $\mathbb C$
комплексной размерности $\text{dim}_{\mathbb C} V=n$. И
скажем у нас есть $\mathbb C$-линейное отображение $\mathcal A:V\rightarrow V$.
Давайте заметим абсолютно тривиальную вещь, то, что оно и $\mathbb R$-линейно,
то есть оно линейно не только над полем комплексных чисел, но, в частности,
полем вещественных чисел. Соответственно на это можно смотреть как на
вещественное векторное пространство и его эндоморфизм $\mathcal A_{\mathbb R}:
V_{\mathbb R}\rightarrow V_{\mathbb R}$, где $V_{\mathbb R}$ - это $V$, но мы
рассматриваем его как вещественное векторное пространство над $\mathbb R$ и
его вещественная размерность равна $\text{dim}_{\mathbb R} V_{\mathbb R} = 2n$. В
частности у нас есть отображение умножения на мнимую единицу $J:V_{\mathbb R}
\rightarrow V_{\mathbb R}=v\mapsto iv$. И нетрудно заметить, что $J^2=-\text{Id}$.

\subsection{Комплексификация}
Обратно, пусть $(W,J)$ – вещественное векторное пространство и $J:W\rightarrow
W$, такое что $J^2=-\text{Id}$. Тогда давайте заметим, что необходимо, чтобы 
вещественная размерность $W$ была четной.

Ну действительно, пусть $m=\text{dim}_{\mathbb R} W$, тогда $\det J^2=(-1)^m=
(\det J)^2>0$. Отсюда мы обязаны иметь четную размерность.

Поэтому имея такое четномерное векторное пространство с таким оператором, мы
можем превратить его в векторное пространство над полем комплексных чисел по
следующему правилу. Пусть $a+bi\in\mathbb C$ и $w\in W$, тогда мы положим
умножение на скаляры как $(a+bi)\cdot w=aw+bJw$. Нетрудно проверить, что операторы
$\mathcal A:W\rightarrow W, \mathcal AJ=J\mathcal A$ – в точности комплексные 
эндоморфизмы $W$ для нововведенное структуры. И в частности если $\det\mathcal A
\neq 0$, то такие операторы образуют группу $\text{GL}(n,\mathbb C)$. Это просто
пересказ куска курса алгебры за первый курс. Вообще если забыть про существование
$J$, то любое векторное пространство $W$ над $\mathbb R$ можно с помощью тензорного
умножения превратить в комплексное векторное пространство следующим образом
$W\otimes_{\mathbb R}\mathbb C=V$ и мы получим $\dim_{\mathbb R}W=\dim_{\mathbb C}V$.
Давайте посмотрим как на $V$ устроен оператор комплексной структуры. Натуральный
способ ввести умножение очевидно $z'\cdot(v\otimes z)=v\otimes(z'z)$, что можно
продлить на все пространство $V$ и если
рассмотреть разложение тензорного произведение $W\otimes_{\mathbb R}\mathbb C
\cong_{\mathbb R} W\oplus W$ относительно действительного базиса $(1,i)$ пространства
$\mathbb C$, то мы получим $J(w_1,w_2)=(-w_2,w_1)$. Эта операция называется комплексификацией
и обычно пишут $W_{\mathbb C}$. Если посмотреть на овеществление комплексификации,
то мы получим $(W_{\mathbb C})_{\mathbb R}\cong W\oplus W$.

\subsection{Комбинации комплексификации и рационализации с наследованием структур}
Пусть теперь $V$ - комплексное пространство. Тогда $(V_{\mathbb R})_{\mathbb C}
\cong V\oplus\overline V$. Где сопряжение показывает способ умножения на скаляры,
а именно $\overline V$ - это пространство над $\mathbb C$ со следующим действием
скаляров $(a+bi)\cdot v=(a-bi)v$.

Над $\mathbb R$ очевиден изоморфизм $(V_{\mathbb R})_{\mathbb C}\cong_{\mathbb R}
V_{\mathbb R}\oplus V_{\mathbb R}$, а комплексная структура, как мы знаем
следующая $J(v_1, v_2)=(-v_2, v_1)$. И так как пары мы можем также умножать на
$i$, но это не тоже самое, что $J$. И умножение на $i$, которое приходи из $V$ оно
очевидно коммутирует c $J$, а именно $iJ(v_1, v_2)=J(iv_1, iv_2)$. И так как
$J^2=-\text{Id}$, то у нас будут собственными числами $i$ и $-i$. И тогда уместно
ввести два пространства
\begin{align*}
    &V^{10}=\{(v_1,v_2)\in (V_{\mathbb R})_{\mathbb C}\,\|\,J(v_1,v_2)=i(v_1,v_2)\}\\
    &V^{01}=\{(v_1,v_2)\in (V_{\mathbb R})_{\mathbb C}\,\|\,J(v_1,v_2)=-i(v_1,v_2)\}
\end{align*}
И нетрудно доказать, что $V^{10}\cong_{\mathbb C} V$ и $V^{01}\cong_{\mathbb C}
\overline V$.
Давайте теперь заметим, что точно такое же разложение верно и для комплексно 
двойственного векторного пространства $(V^*_{\mathbb R})_{\mathbb C}$, где $V^*$
– двойственное к $V$.

\section{Комплексные пространства и формы}
Пусть $\bigwedge^k(V^*_{\mathbb R})_{\mathbb C}$ - простраснство $k$-форм на
$(V_{\mathbb R})_{\mathbb C}$. Тогда имеет место следующее утверждение
\begin{align*}
    \bigwedge^k(V^*_{\mathbb R})_{\mathbb C}=\bigwedge^k(V^*\oplus \overline V^*)=
    \bigoplus_{p+q=k}\bigwedge^p V^*\otimes\bigwedge^q\overline V^*
\end{align*}
Если мы введем мы обозанчи $\bigwedge^{p,q}(V^*):=\bigwedge^p V^*\otimes\bigwedge^q\overline V^*$,
то эта её элементы называются формами типа $(p,q)$. То есть это означает, что любая $k$-форма
$\omega$ раскладывается единственным способом в сумму $k$-форм $\omega =
\omega_{k,0}+\omega_{k-1,1}+\ldots+\omega_{0,k}$, где $\omega\in\bigwedge^k$, a
$\omega_{p.q}\in\bigwedge^{p,q}$.

Давайте вернемся к комплексному пространству $V$. Можно заметить что его
овеществление имеет каноническую ориентацию. Пусть $e_1,\ldots,e_n$ – базис в
$V$, a $f_1,\ldots,f_n$ – двойственный базис в $V^*$.

\utv $\tau:=f_1\wedge if_1\wedge\ldots\wedge f_n\wedge if_n$, форма
типа $(n,n)$, задает ориентацию на $V_{\mathbb R}$. Эту форму можно ещё переписать
как $\tau=i^{q(n)}f_1\wedge\overline f_1\wedge\ldots\wedge f_n\wedge\overline f_n$,
где $q(n)=n$ или $q(n)=n(n+1)/2$.

\doc Пусть у нас есть оператор $\mathcal A: V\rightarrow V$.
Заметим, что $e_1,\ldots,e_n,ie_1,\ldots,ie_n$ – базис $V_{\mathbb R}$. Cопоставим
оператору $\mathcal A$ матрицу $A_{\mathbb R}$ в действительном базисе и матрицу
$A_{\mathbb C}=B+iC$ в комплексном базисе, где $B$ и $C$ – вещественные матрицы.
Тогда нетрудно заметить, что
\[A_{\mathbb R}=\left(\begin{array}{cc}B&-C\\C&B\end{array}\right)\]
и что $\det A_{\mathbb R}=|\det A_{\mathbb C}|^2>0$. И форма при замене базиса
домножается на положительный детерминант, а значит все такие полученые базисы
имеют одинаковую ориентацию. То есть положительность этой формы не зависит от
выбора базиса.

\subsection{Положительность $(p,p)$ форм}
Когда мы говорим о формах надо различать сильно и слабо положительные формы.
Пусть $\eta\in\bigwedge^{p,p}(V^*)$. Это форма положительна, если для любых
$f_1,\ldots,f_q,q=n-p$ форма $\eta\wedge if_1\wedge\overline f_1\wedge\ldots
\wedge if_q\wedge\overline f_q$ положительна, то есть
\[\frac{\eta\wedge if_1\wedge\overline f_1\wedge\ldots\wedge if_q\wedge\overline f_q}{\tau}>0\]

Форма $\eta$ сильно положительна, если $\eta=\sum\gamma_sif_{1,s}\wedge \overline f_{1,s}\ldots
\wedge if_{p,s}\wedge\overline f_{p,s}$ для положительных $\gamma_s$. В итоге 
получится выпуклая линейная комбинация положительных форм. Очевидно, что сильно
положительная форма она положительна, но обратное вообще говоря не верно,
потому что есть про это задача в листке.

В дальнейшем нас будет интересовать положительность $(1,1)$ форм на многообразиях,
но там на самом деле сильные и слабые положительности эквивалентны.
Cильноположительные формы на самом деле образуют конус, и есть нетрудное, но
достаточно муторное утверждение, что конус положительных форм двойственен коносу
сильно положительных форм.

\subsection{Положительные формы типа $(1,1)$}
Пусть $\eta$ – положительная форма типа $(1,1)$, в каком-то базисе она может
быть записана как $\eta = \sum \eta_{j,\overline k} f_j\wedge f_k$. Тогда
$\eta_{j,\overline k}=ih_{j,\overline k}$, где $h=(h_{j,\overline k})$ – эрмитова
матрица.

\utv $(1,1)$ форма положительна тогда и только тогда, когда
она полжительна в ограничении на каждое одномерное пространство.

$\underline{\Leftarrow:}$ Ну действительно, пусть $f_1, f_2,\ldots,f_n$
двойственно $e_1,\ldots,e_n$ и пусть у нас есть одномерное пространство
$L=\langle e_1\rangle$. Тогда $\eta|_L\geq 0$, а точнее $\eta|_L=\eta_{1,1}if_1
\wedge \overline f_1$ c $\eta_(1,1)>0$. A тогда $\eta \wedge if_2\wedge\overline f_2\wedge\ldots
\wedge if_n\wedge\overline f_n=\eta|_L \wedge if_2\wedge\overline f_2\wedge\ldots
\wedge if_n\wedge\overline f_n>0$.

И так как любую положительную 2-форму можно диагонализировать, то она сразу же
также является и сильно положительной формой.

\textbf{Вывод:} Положительные $(1,1)$-формы – это формы вида 
\[\eta=i\sum_{j,k=1}^nh_{j,\overline k}f_j\wedge\overline f_k\]
где $h=(h_{j,\overline k})$ - неотрицательная определенная эрмитова матрицa. То
есть для любого вектора $\xi=\xi^i e_i\in V\setminus{0}$, $h_{j,\overline k}\xi^j\overline{\xi^k}
>0$. А как мы знаем, то если на векторном пространстве есть эрмитова форма,
особенно если она положительно определена, то она даёт вам евклидову метрику и
сиплектическую струкстуру.

Если $(V,h)$ – комплексное векторное пространство, то $h$ определяет евклидову
метрику $g$ и симплектическую структуру $\omega$ на $V_{\mathbb R}$.

Пусть у нас есть векторa $\xi_1,\xi_2\in V$, то $h(\xi_1,\xi_2)=g(\xi_1,\xi_2)
+i\omega(\xi_1,\xi_2)$. Тогда заметим, что $h(\xi_2,\xi_1)=
\overline{h(\xi_1,\xi_2)}=g(\xi_2,\xi_1)-i\omega(\xi_2,\xi_1)$. Отсюда видно,
что как формы на овеществлении $g$ – симметрично, а $\omega$ – кососсиметрична.

Заметим, что $h(i\xi_1,i\xi_2)=h(\xi_1,\xi_2)$, что есть следствие эрмитовости.
А отсюда следует, что $g(J\xi_1,J\xi_2)=g(\xi_1,\xi_2)$ и тоже самое верно для
$\omega$. А дальше $h(\xi,\xi)=g(\xi,\xi)>0$ так как при сопряжении она переходит
в себя же. Также так как $h(i\xi_1,\xi_2)=ih(\xi_1,\xi_2)$, то это показывает,
что $g(J\xi_1,\xi_2)=-\omega(\xi_1,\xi_2)=-g(\xi_1,J\xi_2)$.

\textbf{Вопрос:} Если мы возьмём какое-то гладкое многообразие и рассмотрим его
кокасательное рассмотрение, то там возникает симплектическая струсктура и в
линейном случае просто $V\oplus V^*$, a вот естественная...

\textbf{Ответ:} Ну вот вы рассматриваете кокасательное расслоение как само
многообразие, ну в каком-то смысле, когда мы комплексифицировали у нас возникал
похожий эффект. Как вы знаете, ко касательному расслоению, косательное пространство
– это просто подъём касательного простраства плюс подъем касательного пространства.
И как раз у вам есть симплектическая структура, с ней связана почти комплексная
струсктура всегда, да? Их там может быть много, но они есть. Вот как раз каноничная,
она определяется ровно той же формулой, которую мы определяли при комплектификации.
Или у вас был какой-то другой вопрос?

– У меня был вопрос о том, что в нашем случае мы тоже кое-что канонически определяем
и мне было интерсно...

– Там тоже есть почти комплексная структура, но проблема в том, что почти
комплексная структура, она, вообще говоря, ну то есть вот этот оператор, которы
в квадрате равен минус единицы, когда вы переходите от векторных пространств
к многообразиям, то он не определяет структуру комплексного многообразия. Это
как раз связано с тем, что есть тензор Нинохёйза, который определяет неинтегрируемость,
А неинтегрируемость у вас на самом деле вот с чем связана. Когда мы сначала овеществили,
а потом комплексифицировали, то у вас векторное пространство развалилось в сумму $V$
и сопряженного к $V$. Вот тоже самое происходит с касательным расслоением, когда вы
его овеществлили, а затем комплексифицировали. Проблема в том, что у вас есть
комплексные векторные поля, которые принимают значения, скажем, в комплексном
касательном расслоении. Когда вы овеществлили и комплексифицировали, то вы можете
взять коммутатор двух таких полей, и проблема в том, что если у вас есть просто
почти комплексная структура, то коммутатор дву векторных полей он принимает
значения как в исходном касательном расслоении, так и в его сопряженном. То есть
вообще говоря, линейная алгебра всего того, что мы проговаривали в случае
кокасательного расслоения она работает, а когда мы начнем комплексный анализ,
случится так что вообще говоря комплексный анализ там не работает, потому что не
всегда одного этого оператора достаточно, чтобы определить комплексные координаты
и комплексные функции перехода.

% лекция 2

Пусть $\omega$ – комплексная часть эрмитовой формы, а $\alpha$ – $(1,1)$-форма.
Их тогда можно записать покоординатно, а именно $\omega=\frac{i}{2}
\sum_{j,k}h_{j\overline k}f_j\wedge\overline{f_k}$ и $\alpha=\frac{i}{2}\sum_{j,k}
\alpha_{j\overline k}f_j\wedge\overline{f_k}$. Моральная сторона вопроса такая,
$(h_{j,k})$ – положительно определенная эрмитова матрица, она задаёт эрмитово
скалярное произведение на векторном пространстве, на его двойственном, а значит
просто по стандартным лекалам линейной алгебры оно распространяется на все
тензорные произведения, поэтому
\[\text{Tr}_\omega\alpha=h^{j\overline k}\alpha_{j\overline k}\quad\text{и}\quad|\alpha|^2_\omega=\alpha_{j\overline k}\overline{\alpha_{l\overline m}}
h^{j\overline l}h^{m\overline k} \]
