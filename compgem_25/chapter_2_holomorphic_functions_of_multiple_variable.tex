\chapter{Комплексный анализ многих переменных}
\section{Комплексные функции многих переменных}
\textbf{Определение:} Пусть $\Omega\subset\mathbb C^n$ – открытое и связное
множество и $f:\Omega\rightarrow\mathbb C$ – голоморфная функция, если $f$
непрерывная и голоморфная по каждому переменному $z^1,\ldots,z^n$.

Вообще имеется очень много определений голоморфных функций нескольких переменных
и все они эквивалентны, но мы не будем на этом останавливаться, потому что нас
собственно сам анализ интересует постольку, поскольку надо определять комплексные
многообразия и работать с этими функциями. И если вы были на курсе коплексного
анализа, то знаете, что можно ввести комплексное дифферецируемость, условие
Коши-Римана и так далее.

Давайте зафиксируем обозначения.

\textbf{Оперделение:} Пусть $z_0\in\mathbb\mathbb C^n$ и $\mathbf R\in\mathbb R^n_{>0}$,
тогда определим поликруг
\[P(z_0,\mathbf R):=\{z\in\mathbb C^n\,|\,|z^j-z^j_0|<R_j,\,j=1,\ldots,n\}\]
И мы введем
\[T(z_0,\mathbf R):=\partial [P(z_0,\mathbf R)\]

\utv Пусть $f$ голоморфно на $\Omega$, ($f\in\mathcal O(\Omega)$) и нетрудно понять,
что $\mathcal O(\Omega)$ на самом деле кольцо и пусть $\text{Cl}(P(z_0,\mathbf R))\subseteq\Omega$, то верна
следующая формула
\[f(z_0)=\frac{1}{(2\pi i)^n}\int_{T(z_0,\mathbf R)}\frac{f(\xi)d\xi^1\ldots d\xi^n}{(\xi^1-z^1_0)\ldots(\xi^n-z^n_0)}\]
Этот интеграл определяется просто как повторный интеграл по произведению окружностей.

Для краткости можно ввести следующее обозначение $d^n\xi:=d\xi^1\ldots d\xi^n$ и
$(\xi-z):=(\xi^1-z^1)\ldots(\xi^n-z^b$.

Отсюда следует, что все голоморфные функции аналитические, доказательство
дословно повторяет доказательство для функций одной переменной. Если У нас
есть производные, то их можно также написать через похожий интеграл. И если
есть точка в которой все производные всех порядков равны нулю, то фукция –
тождественный ноль на этой области. Это все факты из курса про одну переменную
и все переходит дословно. Давайте это все зафиксируем.

\sled
\begin{enumerate}
    \item Голоморфная функция аналитична

    \item Пусть $\nu=(\nu_1,\ldots,\nu_n)$ – мультииндекс и пусть
        $f^{(\nu)}:=\frac{\partial^{|\nu|}}{(\partial z^1)^{\nu_1}\ldots(\partial z^n)^{\nu_n}}f$
        Тогда
        \[f^{(\nu)}(z_0)=\frac{\nu!}{(2\pi i)^n}\int_T\frac{f(\xi)d^n\xi}{(\xi-z_0)^{\nu+1}}\]
        где $\nu+1=(\nu_i+1)_i$ и $\nu!=(\nu_1!)\cdot\ldots\cdot(\nu_n!)$.

    \item $|f^{\nu}|\leq\frac{\nu!\sup_{T(z_0,\mathbf R)}(f)}{\mathbf R^\nu}$,
        где $\mathbf R^\nu:=R_1^{\nu_1}\ldots R_n^{\nu_n}$.

    \item Если $f$ – голоморфная функця на $\mathbb C^n$ и ограничена, то $f=const$.
\end{enumerate}

\textbf{Напоминание:} $z^j=x^j+iy^j$, а тогда $\frac{\partial}{\partial z_j}=
\frac{1}{2}(\frac{\partial}{\partial x_j}-i\frac{\partial}{\partial y_j})$ и
$\frac{\partial}{\partial \overline z_k}=\frac{1}{2}(\frac{\partial}{\partial
x_k}-i\frac{\partial}{\partial y_k})$.

\section{Голоморфные отображения}
\textbf{Определение:} Пусть $\Omega\subseteq\mathbb C^n$ – открытое множество.
Отображение $F:\Omega\rightarrow\mathbb C^m$ голоморфно, если оно задаётся
голоморфными функциями, то есть если у нас есть координаты $w^1,\ldots,w^m$ на
$\mathbb C^m$, то оно задаётся $F^l=w^l\circ F$ (мы условимся, что координата
– это функция типа $M\rightarrow \mathbb R$ или $\mathbb C$.

Можно дать алтернативную характеристику

\utv Пусть $f:\omega\rightarrow\mathbb C$ – голоморфная функция,
ну и есть область $\Omega'\subseteq\mathbb C^p$ и $\phi^1,\ldots,\phi^n\in\mathcal
O(\Omega)$ – голоморфные функции в этой области такие, что $(\varphi^1(w),\ldots,
\varphi^n(w))\in\Omega$ для всех $w \in\Omega'$. Тогда $f(\varphi^1(w^1,\ldots,w^p)
,\ldots,\varphi^n(w^1,\ldots,w^p))\in\mathcal O(\Omega')$.

Доказательство – это обычное цепное правило, поэтому можно сказать что гоморфные
отображения это такие отображения, которые (обратный образ голоморфной снова голоморфен?)
при композиции с голоморфными функциями пораждают снова голоморфные. Нетрудно видеть,
что эти определения эквивалентны.

$F:\Omega\rightarrow\mathbb C^m$, если для любой голоморфной функции $g\in
\mathcal O(\mathbb C^m)$, $F^*g$ – голоморфная. $*$ – это операция обратного
образа или побега.

\section{Комплексные многообразия}
\textbf{Определение:} Пусть $X$ – xaycдорфого пространство удолевторяющее второй
аксиоме счетности. $X$ – комплексное многообразие, если существует атлас карт
$(U_\alpha, f_\alpha)_\alpha$ и отображениея $\phi_{\alpha\beta}:U_\alpha\cap U_\beta
\rightarrow U_\alpha\cap U_\beta$ голоморфно и обратимо.

Наверно это немного рано так их определять, комплексные многообразия, потому что
мы ещё некоторое время потопчемся в областях $\mathbb C^n$, и нам на самом деле
нужны были голоморфные отображения, потому что одна из самых важных вещей, которые
есть на комплексных многообразиях, это разложение дифференциала де Рама, сейчас
мы его определим.

Если у нас есть обычное многообразие, у нас есть на нём диффернециальные формы
и есть дифференциал де Рама, который переводит формы степени $k$ в формы степени
$k+1$. Давайте заметим, что если у нас есть $X$ – комплексное многообразие, то
оно автоматически является вещественными многообразием. И поэтому на нём есть
косательное расслоени $T_X$. При чем $T_X$ обладает стректурой комплексного
пространства и любое сечение $T_X$ локально представляется $\xi=\xi^j\frac{\partial}
{\partial z^i}$. И если мы его овеществим и тензорно умножим, то получим
$(T_X)_{\mathbb R}\otimes\underline{\mathbb C}\cong T_X\oplus\overline{T_X}$,
аналогичную конструкцию мы получим для кокасательного расслоения, а именно 
$(T_X^*)_{\mathbb R}\otimes\underline{\mathbb C}\cong T_X^*\oplus\overline{T_X^*}$.
Причем можно заметить, что если у нас есть локальные координаты, то сечения
косательное расслоение пораждаются $dz^j=dx^j+idy^j$ $(1,0)$-формами. А есть
$d\overline z^j=dx^j-idy^j$ – $(0,1)$-формы. Давайте теперь посмотрим на
\[\bigwedge^k((T_X^*)_{\mathbb R}\otimes\underline{\mathbb C})=\bigwedge^kT_X^*
\oplus\overline{T_X^*}=\bigoplus_{p+q=k}\bigwedge^pT_x\otimes\bigwedge^q\overline{T_X}\]
Давайте обозначим эту штуку $\bigwedge^{p,q}T^*_X:=\bigwedge^pT_x\otimes
\bigwedge^q\overline{T_X}$, сечения такого расслоения мы будем называть формами
типа $(p,q)$.

Теперь как мы все помним есть дифференциал де Рама, который можно продолжить
очевидным образом на комплекснозначные формы $d:\Gamma(\bigwedge^k(T^*_X)_{
\mathbb R}\otimes\underline{\mathbb C}\rightarrow\Gamma(\bigwedge^{k+1}(T^*_X)_{
\mathbb R}\otimes\underline{\mathbb C}$. Давайте возьмём дифференциальную форму
$\psi\in\Gamma(\bigwedge^k\ldots)$ и вспомним как дифференцирование выглядит в
локальных координатах
\[\psi=\sum_{m+n=k}\sum_{p_1<\ldots<p_m}\sum_{q_1<\ldots<q_n}\psi_{(p,q)}(dx)^p\wedge(dy)^q,\;\text{для}\;(dx)^p=dx^{p_1}\wedge\ldots\wedge dx^{p_m}\]
тогда мы получим
\[d\psi=\sum d\psi_{(p,q)}\wedge (dx)^p\wedge (dy)^q\]
И если $f$ – гладкая функция на $X$, то
\[df=\sum_{j=1}^n\frac{\partial f}{\partial x^j}dx^j+\sum_{j-1}^n\frac{\partial f}{\partial y_j}dy^j
=\sum_{j=1}^n\frac{\partial f}{\partial z^j}dz^j+\sum_{j=1}^n\frac{\partial f}{\partial\overline z^j}d\overline z^j\]
Ну и мы также имеем $\psi=\sum_{p+q=k}\psi_{p,q}$, где $\psi_{p,q}\in\Gamma(\bigwedge^{p,q}T^*_X)$.
разложение пси в такую сумму. И так как дифференциал де Рама линеен, то мы можем
рассматривать формы типа $(p,q)$. Пусть $J=(j_1,\ldots,j_p)$ и $K=(k_1,\ldots,k_n)$
такие мультииндексы, что $1\leq j_1<\ldots<j_p\leq n$ и $1\leq k_1<\ldots<k_q\leq n$.
Тогда $\psi_{p,q}=\sum_{J,K} \psi_{J\overline K}dz^J\wedge d\overline Z^K$, где
$dz^J=dz^j_1\wedge\ldots\wedge dz^j_p$ и $dz^K=d\overline z^k_1\wedge\ldots\wedge
d\overline z^j_p$. Ну тогда давайте ещё один раз напишем этот координатный ужас.
\begin{align*}
    d\psi_{p,q}&=\sum_{J,K}d\psi_{J\overline K}\wedge dz^J\wedge d\overline z^K\\&=
    \sum_{J,K,j=1}^n\frac{\partial \psi_{J\overline K}}{\partial z^j}dz^j\wedge
    dz^J\wedge d\overline z^K+\sum_{J,K,k=1}^n\frac{\partial \psi_{J\overline K}}
    {\partial\overline z^k}d\overline z^j\wedge dz^J\wedge d\overline z^K
\end{align*}
Теперь мы уверждаем, что есть такое разложение и куски один и два получившиеся
в конце разных $(p,q)$-типов инвариантны относительно голоморфных замен координат,
потому что если у вас есть голоморфное отображение, то в координатах оно задаётся
голоморфными функциями. Если у нас есть дифференция $dz^j$, то пубэк $F^*dz^j$ – это
снова $(1,0)$-форма, тоже самое верно и для $(0,1)$-форм. Эти слагаемые мы обозначим
как
\[\partial\psi_{p,q}:=\sum_{J,K,j=1}^n\frac{\partial \psi_{J\overline K}}{\partial z^j}dz^j\wedge
    dz^J\wedge d\overline z^K\quad\overline\partial\psi_{p,q}:=\sum_{J,K,k=1}^n\frac{\partial \psi_{J\overline K}}
    {\partial\overline z^k}d\overline z^j\wedge dz^J\wedge d\overline z^K
\]

\utv
\begin{align*}
    d:\Gamma\left(\bigwedge^{p,q}T^*_X\right)&\rightarrow\Gamma\left(\bigwedge^{p+1,q}T^*_X\right)\oplus
\Gamma\left(\bigwedge^{p,q+1}T^*_X\right)\\
    \partial:\Gamma\left(\bigwedge^{p,q}T^*_X\right)&\rightarrow\Gamma\left(\bigwedge^{p+1,q}T^*_X\right)\\
    \overline\partial:\Gamma\left(\bigwedge^{p,q}T^*_X\right)&\rightarrow\Gamma\left(\bigwedge^{p,q+1}T^*_X\right)
\end{align*}

И $d=\partial+\overline\partial$ такое разложение единственно и не зависит от координат.
% time - 1:06:35

\sled Функция $f:X\rightarrow\mathbb C$ голоморфная, если и только
если 
\[\overline\partial f=0\]

Так как $d^2=0$, то $\partial^2+\overline\partial^2+\partial\overline\partial+
\overline\partial\partial=0$. И если $\psi_{p,q}\in\Gamma(\bigwedge^{p,q})$, то
\begin{align*}
    \partial^2\psi_{p,q}&\in\Gamma\left(\bigwedge^{p+2,q}\right)\\
    \overline\partial^2\psi_{p,q}&\in\Gamma\left(\bigwedge^{p+2,q}\right)\\
    (\partial\overline\partial+\overline\partial\partial)\psi_{p,q}&\in\Gamma\left(
    \bigwedge^{p+1,q+1}\right)
\end{align*}
A значит $\overline\partial^2=\partial^2=\partial\overline\partial+\overline\partial
\partial=0$. Тогда для оператора $\overline\partial$ можно тоже определить
когомологии, тем же способом, которым определяются когомологии Де Рама.
\[H^{p,q}(X)=\frac{\text{Ker}\,\overline\partial|_{\Gamma\left(\bigwedge^{p,q}\right)}}{\text{Im}\,
\overline\partial|_{\Gamma\left(\bigwedge^{p,q-1}\right)}}\]
Очевидно, что эти группы могут быть достаточно большими. Например даже группа
$H^{0,0}$ может быть громадна в полидиске или в единичном шаре, потому что у
вас просто невероятно много голоморфных функций.

Более общё, нетрудно видеть, что

\utv Пусть $\psi_{p,0}\in\Gamma\left(\bigwedge^{p,0}\right)$. Тогда оно
удолетворяет $\overline\partial\psi){p,0}=0$ тогда и только тогда, когда
её коэффициенты – это голоморфные функции.

То есть форма типа $(p,0)$ у которой в координатной записи есть только
дифференциалы типа $dz_j$ и нет $d\overline z_j$, то она пренадлежит
этой группе, то есть она $\overline\partial$-замекнута тогда и только тогда,
когда её коэффициенты – это голоморфные функции.

\textbf{Вопрос:} А $\mathbb H^n$ оно не раскладывается в сумму $H^{p,q}$?

\textbf{Ответ:} Вообще говоря нет, если взять прямую сумму $H^{p,q}(X)$, то это
будет больше, чем когомологии де Рама.

\textbf{Вопрос:} A это тоже самое, что $H^q\left(\Omega^p\right)$?

\textbf{Ответ:} Да, да, да. Это теорема Дальбо, которую я надеюсь мы через
какое-то достаточно быстрое время докажем. Но да, будте осторожны с $H^{p,q}$,
вообще говоря если мы берём форму оттуда, то она не обязана быть замкнутой
относительно дифференциала де Рама. Из $\overline\partial\psi=0$ не следует, что
$\partial\psi=0$, такое бывает на некэйлеровых многообразиях. Есть многообразия,
которые не доспускают некоторые специальные эрмитовые метрики. В частности из 
$\overline\partial\psi=0$ следует $\partial\psi=0$ на проективных многообразиях,
то есть на подмножествах в $\mathbb C^n$, например если вы возьмёте голоморфную
функцию на поликруге, то если она является замкнутой на относительно
дифференциала де Рама, то она константа.

Ну и собственно на самом деле нас будет интересовать то, как можно решить
уравнение вида
\[\overline\partial\psi=f\]
где $\psi$ – это какая-то форма типа $(p,q)$, а $f$ – это форма типа $(p,q+1)$
или не просто форма, а форма со значение в расслоении. [Хотелось бы расказать
сегодня про фундаметальное решение этого уравнения, хотя бы на $(0,1)$-формах,
но мы видемо не успеем на этой лекции, так что расказ пойдёт о кое-чем другом.

Если говорить о голоморфных векторных полях, то непонятно что это такое, потому
что если у нас есть гладкое многообразие, то на векторных полях у нас есть
дифференциал де Рама на формах, но какого-то аналога естественного и канонического
оператора на векторных полях у нас нет в гладком случае. Но оказывается что в
случае комплексных многообразий можно определить $\overline\partial$ и на
касательном расслоении и вообще на многих других расслоениях.

\opr Пусть $\pi:E\rightarrow X$ – комплексное векторное расслоение наж комплексным
многообразием $X$ (все же занают что такое комплексное векторное расслоение над
гладким многообразием). $E$ называется голоморфным векторным расслоением, если
существует тривиализующее покрытие $\left\{U_\alpha\right\}$, $U_\alpha\subset
\mathbb C^n$, такое что функции перехода $\gamma_{\alpha\beta}$ голоморфны
[запись второй лекции прервалась]
