% лекция 3
\chapter{Потоки (и обобщенные функции)}
И так мы закончили на том, что определили оператор $\overline\partial$, который
действует на формах следующим образом
\[\overline\partial\alpha=\sum_{k=1}^n\frac{\partial_{\overline k}\alpha_{I\overline J}}{\partial
\overline z^K}d\overline{z^k}\wedge dz^I\wedge d\overline{z^J}\]
Наша цель сегодня понять когда существует решение у уравнения $\overline\partial
u=f$. Необходимое условие очевидно $\overline\partial f=0$ исходя из того, что
мы обсуждали в прошлый раз. Сегодня мы попытемся подойти к этому для областей в
$\mathbb C^n$, потому что на многообразиях теория куда многообразней и сложней.
Но локальную теорию мы постараемся разобрать.

\textbf{Цель:}
\begin{enumerate}
    \item Научиться выражать $u$ через $f$. Стоит учесть, что даже в случае,
        когда $f$ – это $(0,1)$-форма, то у нас может быть очень и очень
        много различных решений, потому что ядро $\overline\partial$ оно
        содержит в себе все голоморфные функции, поэтому мы будем искать
        какое-то специфическое решение, и так как мы можем прибавить к $u$
        любой элемент из ядра, то мы получим другое решение, и все решения
        различаются на элемент из ядра, а значит нам надо найти некоторое
        решение, в некотором смысле оптимальное.
    \item Доказать аналог леммы Пуанкаре. То есть лемма для дифференциала де Рама
        она говорит, что любая замкнутая форма локально точна, то есть вы всегда
        можете найти локальное решение.
\end{enumerate}

К пункту один можно подойти поразному, один из подходов изложен в следующей
задаче

\textbf{Задача:}
Пусть $f=f_jd\overline z^j$ и $\overline\partial f=0$, тогда мы можем написать
такую вещь
\[u_j=\frac{1}{2\pi i}\int_{\mathbb C}\frac{f_j(z^1,\ldots,z^{j-1},\zeta,z^{j+1},\ldots)d\zeta\wedge d\overline\zeta}
{\zeta-z^j}
\]
И $f_j$ зависят как от $z^j$, так и от их сопряженных, но мы не написалии обозначение
сопряженных координат, потому что и так всё понятно. И утверждается, что когда форма
$f$ замкнута и с компактным носителем, то мы можем выбрать любое $f_j$ и написать
такой интеграл, то все $u_j$ будут равны и корректно задает функцию $u$. Тоже
самое на самом деле можно проделать и с формами других степеней, но такой подход
он более наивный, менее техничный и как следствие, если мы убираем техничность,
то мы получаем больше нудной работы, при работе с формами степени $(p,q)$, где
$p>1$ и $q>1$. Это как-то занудно, тем более оно везде написано: в Гриферте-Харрисоне,
в Вуазене, в Хуберте и ещё бог знает где, поэтому сегодняшняя лекция будет следовать
подходу, изложеному в Демаи, но для этого надо обсудить обобщенные функции и потоки.
Расказ не будет останавливаться особо на деталях, так как во-первых – сильно не
надо, во вторых вы с этим более-менее знакомы, в-третих нам нужен на самом деле
один кокретный пример мы его в деталях разберем и потом из него всё быстро
выведем.

\section{Обобщенные функции}
Нас в первую очередь будут интересовать банаховы пространства функций, которые
непрерывно дифференцируемы до порядка $k$ и пространства бесконечно
дифференцируемых функций с компактным носителем и соответственно дифференциальные
формы с компактным носителем. На самом деле последнее является нашим основным
интересом.

Пусть $\mathcal D^k(\Omega)$, $\Omega\subseteq\mathbb R^n$ – пространство
$k$-форм со значениями в $\mathbb R$ или $\mathbb C$ с компактными носителями.
Давайте остановимся на этом случае, когда технические детали сводяться к минимум,
когда нам не нужно следить за тем, что происходит на бесконечности и за прочим.

Пусть $L\subseteq\Omega$ – компакт, $\alpha\in\mathcal D^k(\Omega)$, тогд мы
можем определить
\[p_{S,L}(\alpha):=\sup_{x\in L}\sup_{|\nu|<S}
\sup_J|\partial^{\nu}\alpha_J(x)|\]
где $\alpha=\sum_{|J|=K}\alpha_Jdx^{J}$ и $\partial\alpha_J=\frac{\partial^{|\nu|}}
{\partial^{\nu_1}x^1\ldots\partial^{\nu_n}x^n}\alpha_J(X)$. Этот набор полунорм
задает топологию, но нам важно знать, что последовательность норм сходится в
этой топологии тогда и только тогда, когда сходиться по каждой из этих полунорм.

\opr Поток $T$ размерности $k$ (или степени $n-k$) – это линейный функционал на
$\mathcal D^k$, неперывен в вышеописанной топологии.

\textbf{Пример:}
\begin{enumerate}
    \item Пусть $\beta$ – гладкая $(n-k)$-форма на $\Omega$. Тогда $T_\beta(\alpha)
        :=\int_\Omega\beta\wedge\alpha$ – поток, что легко проверить простой
        оценкой интеграла по любой из этих норм.
    \item $\Sigma\subseteq\Omega$ – гладкая ориентируемая поверхность в $\Omega$,
        тогда $T_\Sigma(\alpha):=\int_\Sigma\alpha$. Частный пример этого пункта
        – это знаменитая $\delta$ функция на $0$ формах, то есть на функциях.
\end{enumerate}

На самом деле нетрудно видеть, что все про что мы говорили обобщается на
многообразия, но так как мы будем использовать потоки первого типа, то мы будем
считать, что многообразия ориентируемы.
