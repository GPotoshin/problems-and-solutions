\documentclass[a4paper, 12pt, oneside]{book}
\usepackage{amsmath}
\usepackage{amssymb}
\usepackage{hyperref}
\usepackage[russian]{babel}

\usepackage{tikz-cd}
\usepackage{array}
\usepackage{graphicx}

\newcommand\utv{\textbf{Утверждение:} }
\newcommand\doc{\textbf{Доказательство:} }
\newcommand\sled{\textbf{Следствие:} }
\newcommand\opr{\textbf{Определение:} }
\newcommand\mapsfrom{\mathrel{\reflectbox{\ensuremath{\mapsto}}}}
\newcommand{\mymat}{\mathcal{M\mkern-3mu a\mkern-0.3mu t}}

\usepackage{fontspec}
\setmainfont{Linux Libertine O}
\usepackage{unicode-math}
\setmathfont{Cambria Math}

\title{
\textit{\huge{НМУ Комплексная Геометрия\\Никита Клементин}}
}

\date{2024}

\author{\small{ЗаТеХано Потошином Георгием}}

\begin{document}
\maketitle

\begin{center}\textsc{\Large{Préface}}\end{center}

Этот курс читался НМУ в 2025 году по две лекции в неделю. Обычного курса один раз
в неделю хватает, чтобы начитать только нулевую главу Грифицехариса минус эпсилон
плюс теорема Кадайро о вложении. И так как хотелось бы добраться до чего-то более
содержательного и покрыть больше материала, то вот поэтому курс проводился два
раза в неделю. К курсу прилагаются 6 листочков по итогу решения которых выставлялась
оценка. Пятерка выставлялась с 60\% решенных задач. 

Что касается содержания курса, то он начинается с напоминания базовых фактов из
линейной алгебры. За тем мы обсудим голоморфные функции нескольких переменных.
Следующим объектом изучения будут плюрисубгармоничекие функциию. Они на самом
деле в каком-то смысле отвечают за положительность кривизны метрик на голоморфных
линейных расслоениях и связаны с такой вещью как положительные потоки. Потоки
являются аналогией обобщенных функций, то есть линейных функционалов на гладких
функциях, а есть потоки – это линейные функционалы на гладких дифференциальных
формах. Затем мы определим многообразия, поговорим чуть-чуть про пучки, потому
как некоторые виды пучков тесно связаны с плюрисобгармоническими функциями и ими
определяются. И потом мы обсудим голоморфные векторные расслоения, обсудим
случаи положительной и отрицательной кривизны голоморфного векторного расслоения.
После этого мы обсудим всякие вещи типа теоремы Кадэра о занулении, формулу
Бохнера, и затем мы докажем $L^2$ оценку для дебар уравнения. Из неё можно
теорему Кадаира о вложении для компактных Кэлеровых многообразий. Из неё можно,
на самом деле, сюрприз, можно доказать теорму Ньюлендер Ниренберга. Обычно 
стандартное доказательство в курсах затрагивает только случай, когда ваше
многообразие и почти комплексная структура, они вещественно аналитичные. Мы можем
доказать без этого предположения, просто для гладких многообразий, гладких почти
комплексных структур. Это некоторый сюрприз, что аппарат, который чисто комплексно
-аналитический, может быть использован для доказательства интегрируемости почти
комплексных структур. Затем мы перейдем к уравнению Монжампера. Докажем теорему
Калабияу. Выведем из неё некоторые следствия с помощью ранее обсуждавшихся формул
Бохнера и прочего. И в конце, если останется время, мы обсудим некоторые приложения
теоремы Клабияу. Я бы хотел на самом деле обсудить теорему Дэмэи-Пауна про
характеризацию кэйлерового конуса, компактных кэйлеровых многообразий. То есть
какой когомологический класс на вашем кэлеровом многообразии может содержать
кэлерову метрику. На самом деле оно достаточно небезынтерсно и очень использует
как раз возможность решить уравнение монжанпера. Либо можно будет обсудить решения
вопросов о существовании и несуществовании метрик Кэлера-Эйнштейна на многообразиях
Фана, так как оне существуют там не всегда.

\chapter{Векторные пространства}
\section{Комплексные и действительные структуры}
\subsection{Рационализация}
Пусть $V$ – векторное пространство над полем комплексных чисел $\mathbb C$
комплексной размерности $\text{dim}_{\mathbb C} V=n$. И
скажем у нас есть $\mathbb C$-линейное отображение $\mathcal A:V\rightarrow V$.
Давайте заметим абсолютно тривиальную вещь, то, что оно и $\mathbb R$-линейно,
то есть оно линейно не только над полем комплексных чисел, но, в частности,
полем вещественных чисел. Соответственно на это можно смотреть как на
вещественное векторное пространство и его эндоморфизм $\mathcal A_{\mathbb R}:
V_{\mathbb R}\rightarrow V_{\mathbb R}$, где $V_{\mathbb R}$ - это $V$, но мы
рассматриваем его как вещественное векторное пространство над $\mathbb R$ и
его вещественная размерность равна $\text{dim}_{\mathbb R} V_{\mathbb R} = 2n$. В
частности у нас есть отображение умножения на мнимую единицу $J:V_{\mathbb R}
\rightarrow V_{\mathbb R}=v\mapsto iv$. И нетрудно заметить, что $J^2=-\text{Id}$.

\subsection{Комплексификация}
Обратно, пусть $(W,J)$ – вещественное векторное пространство и $J:W\rightarrow
W$, такое что $J^2=-\text{Id}$. Тогда давайте заметим, что необходимо, чтобы 
вещественная размерность $W$ была четной.

Ну действительно, пусть $m=\text{dim}_{\mathbb R} W$, тогда $\det J^2=(-1)^m=
(\det J)^2>0$. Отсюда мы обязаны иметь четную размерность.

Поэтому имея такое четномерное векторное пространство с таким оператором, мы
можем превратить его в векторное пространство над полем комплексных чисел по
следующему правилу. Пусть $a+bi\in\mathbb C$ и $w\in W$, тогда мы положим
умножение на скаляры как $(a+bi)\cdot w=aw+bJw$. Нетрудно проверить, что операторы
$\mathcal A:W\rightarrow W, \mathcal AJ=J\mathcal A$ – в точности комплексные 
эндоморфизмы $W$ для нововведенное структуры. И в частности если $\det\mathcal A
\neq 0$, то такие операторы образуют группу $\text{GL}(n,\mathbb C)$. Это просто
пересказ куска курса алгебры за первый курс. Вообще если забыть про существование
$J$, то любое векторное пространство $W$ над $\mathbb R$ можно с помощью тензорного
умножения превратить в комплексное векторное пространство следующим образом
$W\otimes_{\mathbb R}\mathbb C=V$ и мы получим $\dim_{\mathbb R}W=\dim_{\mathbb C}V$.
Давайте посмотрим как на $V$ устроен оператор комплексной структуры. Натуральный
способ ввести умножение очевидно $z'\cdot(v\otimes z)=v\otimes(z'z)$, что можно
продлить на все пространство $V$ и если
рассмотреть разложение тензорного произведение $W\otimes_{\mathbb R}\mathbb C
\cong_{\mathbb R} W\oplus W$ относительно действительного базиса $(1,i)$ пространства
$\mathbb C$, то мы получим $J(w_1,w_2)=(-w_2,w_1)$. Эта операция называется комплексификацией
и обычно пишут $W_{\mathbb C}$. Если посмотреть на овеществление комплексификации,
то мы получим $(W_{\mathbb C})_{\mathbb R}\cong W\oplus W$.

\subsection{Комбинации комплексификации и рационализации с наследованием структур}
Пусть теперь $V$ - комплексное пространство. Тогда $(V_{\mathbb R})_{\mathbb C}
\cong V\oplus\overline V$. Где сопряжение показывает способ умножения на скаляры,
а именно $\overline V$ - это пространство над $\mathbb C$ со следующим действием
скаляров $(a+bi)\cdot v=(a-bi)v$.

Над $\mathbb R$ очевиден изоморфизм $(V_{\mathbb R})_{\mathbb C}\cong_{\mathbb R}
V_{\mathbb R}\oplus V_{\mathbb R}$, а комплексная структура, как мы знаем
следующая $J(v_1, v_2)=(-v_2, v_1)$. И так как пары мы можем также умножать на
$i$, но это не тоже самое, что $J$. И умножение на $i$, которое приходи из $V$ оно
очевидно коммутирует c $J$, а именно $iJ(v_1, v_2)=J(iv_1, iv_2)$. И так как
$J^2=-\text{Id}$, то у нас будут собственными числами $i$ и $-i$. И тогда уместно
ввести два пространства
\begin{align*}
    &V^{10}=\{(v_1,v_2)\in (V_{\mathbb R})_{\mathbb C}\,\|\,J(v_1,v_2)=i(v_1,v_2)\}\\
    &V^{01}=\{(v_1,v_2)\in (V_{\mathbb R})_{\mathbb C}\,\|\,J(v_1,v_2)=-i(v_1,v_2)\}
\end{align*}
И нетрудно доказать, что $V^{10}\cong_{\mathbb C} V$ и $V^{01}\cong_{\mathbb C}
\overline V$.
Давайте теперь заметим, что точно такое же разложение верно и для комплексно 
двойственного векторного пространства $(V^*_{\mathbb R})_{\mathbb C}$, где $V^*$
– двойственное к $V$.

\section{Комплексные пространства и формы}
Пусть $\bigwedge^k(V^*_{\mathbb R})_{\mathbb C}$ - простраснство $k$-форм на
$(V_{\mathbb R})_{\mathbb C}$. Тогда имеет место следующее утверждение
\begin{align*}
    \bigwedge^k(V^*_{\mathbb R})_{\mathbb C}=\bigwedge^k(V^*\oplus \overline V^*)=
    \bigoplus_{p+q=k}\bigwedge^p V^*\otimes\bigwedge^q\overline V^*
\end{align*}
Если мы введем мы обозанчи $\bigwedge^{p,q}(V^*):=\bigwedge^p V^*\otimes\bigwedge^q\overline V^*$,
то эта её элементы называются формами типа $(p,q)$. То есть это означает, что любая $k$-форма
$\omega$ раскладывается единственным способом в сумму $k$-форм $\omega =
\omega_{k,0}+\omega_{k-1,1}+\ldots+\omega_{0,k}$, где $\omega\in\bigwedge^k$, a
$\omega_{p.q}\in\bigwedge^{p,q}$.

Давайте вернемся к комплексному пространству $V$. Можно заметить что его
овеществление имеет каноническую ориентацию. Пусть $e_1,\ldots,e_n$ – базис в
$V$, a $f_1,\ldots,f_n$ – двойственный базис в $V^*$.

\utv $\tau:=f_1\wedge if_1\wedge\ldots\wedge f_n\wedge if_n$, форма
типа $(n,n)$, задает ориентацию на $V_{\mathbb R}$. Эту форму можно ещё переписать
как $\tau=i^{q(n)}f_1\wedge\overline f_1\wedge\ldots\wedge f_n\wedge\overline f_n$,
где $q(n)=n$ или $q(n)=n(n+1)/2$.

\doc Пусть у нас есть оператор $\mathcal A: V\rightarrow V$.
Заметим, что $e_1,\ldots,e_n,ie_1,\ldots,ie_n$ – базис $V_{\mathbb R}$. Cопоставим
оператору $\mathcal A$ матрицу $A_{\mathbb R}$ в действительном базисе и матрицу
$A_{\mathbb C}=B+iC$ в комплексном базисе, где $B$ и $C$ – вещественные матрицы.
Тогда нетрудно заметить, что
\[A_{\mathbb R}=\left(\begin{array}{cc}B&-C\\C&B\end{array}\right)\]
и что $\det A_{\mathbb R}=|\det A_{\mathbb C}|^2>0$. И форма при замене базиса
домножается на положительный детерминант, а значит все такие полученые базисы
имеют одинаковую ориентацию. То есть положительность этой формы не зависит от
выбора базиса.

\subsection{Положительность $(p,p)$ форм}
Когда мы говорим о формах надо различать сильно и слабо положительные формы.
Пусть $\eta\in\bigwedge^{p,p}(V^*)$. Это форма положительна, если для любых
$f_1,\ldots,f_q,q=n-p$ форма $\eta\wedge if_1\wedge\overline f_1\wedge\ldots
\wedge if_q\wedge\overline f_q$ положительна, то есть
\[\frac{\eta\wedge if_1\wedge\overline f_1\wedge\ldots\wedge if_q\wedge\overline f_q}{\tau}>0\]

Форма $\eta$ сильно положительна, если $\eta=\sum\gamma_sif_{1,s}\wedge \overline f_{1,s}\ldots
\wedge if_{p,s}\wedge\overline f_{p,s}$ для положительных $\gamma_s$. В итоге 
получится выпуклая линейная комбинация положительных форм. Очевидно, что сильно
положительная форма она положительна, но обратное вообще говоря не верно,
потому что есть про это задача в листке.

В дальнейшем нас будет интересовать положительность $(1,1)$ форм на многообразиях,
но там на самом деле сильные и слабые положительности эквивалентны.
Cильноположительные формы на самом деле образуют конус, и есть нетрудное, но
достаточно муторное утверждение, что конус положительных форм двойственен коносу
сильно положительных форм.

\subsection{Положительные формы типа $(1,1)$}
Пусть $\eta$ – положительная форма типа $(1,1)$, в каком-то базисе она может
быть записана как $\eta = \sum \eta_{j,\overline k} f_j\wedge f_k$. Тогда
$\eta_{j,\overline k}=ih_{j,\overline k}$, где $h=(h_{j,\overline k})$ – эрмитова
матрица.

\utv $(1,1)$ форма положительна тогда и только тогда, когда
она полжительна в ограничении на каждое одномерное пространство.

$\underline{\Leftarrow:}$ Ну действительно, пусть $f_1, f_2,\ldots,f_n$
двойственно $e_1,\ldots,e_n$ и пусть у нас есть одномерное пространство
$L=\langle e_1\rangle$. Тогда $\eta|_L\geq 0$, а точнее $\eta|_L=\eta_{1,1}if_1
\wedge \overline f_1$ c $\eta_(1,1)>0$. A тогда $\eta \wedge if_2\wedge\overline f_2\wedge\ldots
\wedge if_n\wedge\overline f_n=\eta|_L \wedge if_2\wedge\overline f_2\wedge\ldots
\wedge if_n\wedge\overline f_n>0$.

И так как любую положительную 2-форму можно диагонализировать, то она сразу же
также является и сильно положительной формой.

\textbf{Вывод:} Положительные $(1,1)$-формы – это формы вида 
\[\eta=i\sum_{j,k=1}^nh_{j,\overline k}f_j\wedge\overline f_k\]
где $h=(h_{j,\overline k})$ - неотрицательная определенная эрмитова матрицa. То
есть для любого вектора $\xi=\xi^i e_i\in V\setminus{0}$, $h_{j,\overline k}\xi^j\overline{\xi^k}
>0$. А как мы знаем, то если на векторном пространстве есть эрмитова форма,
особенно если она положительно определена, то она даёт вам евклидову метрику и
сиплектическую струкстуру.

Если $(V,h)$ – комплексное векторное пространство, то $h$ определяет евклидову
метрику $g$ и симплектическую структуру $\omega$ на $V_{\mathbb R}$.

Пусть у нас есть векторa $\xi_1,\xi_2\in V$, то $h(\xi_1,\xi_2)=g(\xi_1,\xi_2)
+i\omega(\xi_1,\xi_2)$. Тогда заметим, что $h(\xi_2,\xi_1)=
\overline{h(\xi_1,\xi_2)}=g(\xi_2,\xi_1)-i\omega(\xi_2,\xi_1)$. Отсюда видно,
что как формы на овеществлении $g$ – симметрично, а $\omega$ – кососсиметрична.

Заметим, что $h(i\xi_1,i\xi_2)=h(\xi_1,\xi_2)$, что есть следствие эрмитовости.
А отсюда следует, что $g(J\xi_1,J\xi_2)=g(\xi_1,\xi_2)$ и тоже самое верно для
$\omega$. А дальше $h(\xi,\xi)=g(\xi,\xi)>0$ так как при сопряжении она переходит
в себя же. Также так как $h(i\xi_1,\xi_2)=ih(\xi_1,\xi_2)$, то это показывает,
что $g(J\xi_1,\xi_2)=-\omega(\xi_1,\xi_2)=-g(\xi_1,J\xi_2)$.

\textbf{Вопрос:} Если мы возьмём какое-то гладкое многообразие и рассмотрим его
кокасательное рассмотрение, то там возникает симплектическая струсктура и в
линейном случае просто $V\oplus V^*$, a вот естественная...

\textbf{Ответ:} Ну вот вы рассматриваете кокасательное расслоение как само
многообразие, ну в каком-то смысле, когда мы комплексифицировали у нас возникал
похожий эффект. Как вы знаете, ко касательному расслоению, косательное пространство
– это просто подъём касательного простраства плюс подъем касательного пространства.
И как раз у вам есть симплектическая структура, с ней связана почти комплексная
струсктура всегда, да? Их там может быть много, но они есть. Вот как раз каноничная,
она определяется ровно той же формулой, которую мы определяли при комплектификации.
Или у вас был какой-то другой вопрос?

– У меня был вопрос о том, что в нашем случае мы тоже кое-что канонически определяем
и мне было интерсно...

– Там тоже есть почти комплексная структура, но проблема в том, что почти
комплексная структура, она, вообще говоря, ну то есть вот этот оператор, которы
в квадрате равен минус единицы, когда вы переходите от векторных пространств
к многообразиям, то он не определяет структуру комплексного многообразия. Это
как раз связано с тем, что есть тензор Нинохёйза, который определяет неинтегрируемость,
А неинтегрируемость у вас на самом деле вот с чем связана. Когда мы сначала овеществили,
а потом комплексифицировали, то у вас векторное пространство развалилось в сумму $V$
и сопряженного к $V$. Вот тоже самое происходит с касательным расслоением, когда вы
его овеществлили, а затем комплексифицировали. Проблема в том, что у вас есть
комплексные векторные поля, которые принимают значения, скажем, в комплексном
касательном расслоении. Когда вы овеществлили и комплексифицировали, то вы можете
взять коммутатор двух таких полей, и проблема в том, что если у вас есть просто
почти комплексная структура, то коммутатор дву векторных полей он принимает
значения как в исходном касательном расслоении, так и в его сопряженном. То есть
вообще говоря, линейная алгебра всего того, что мы проговаривали в случае
кокасательного расслоения она работает, а когда мы начнем комплексный анализ,
случится так что вообще говоря комплексный анализ там не работает, потому что не
всегда одного этого оператора достаточно, чтобы определить комплексные координаты
и комплексные функции перехода.

% лекция 2

Пусть $\omega$ – комплексная часть эрмитовой формы, а $\alpha$ – $(1,1)$-форма.
Их тогда можно записать покоординатно, а именно $\omega=\frac{i}{2}
\sum_{j,k}h_{j\overline k}f_j\wedge\overline{f_k}$ и $\alpha=\frac{i}{2}\sum_{j,k}
\alpha_{j\overline k}f_j\wedge\overline{f_k}$. Моральная сторона вопроса такая,
$(h_{j,k})$ – положительно определенная эрмитова матрица, она задаёт эрмитово
скалярное произведение на векторном пространстве, на его двойственном, а значит
просто по стандартным лекалам линейной алгебры оно распространяется на все
тензорные произведения, поэтому
\[\text{Tr}_\omega\alpha=h^{j\overline k}\alpha_{j\overline k}\quad\text{и}\quad|\alpha|^2_\omega=\alpha_{j\overline k}\overline{\alpha_{l\overline m}}
h^{j\overline l}h^{m\overline k} \]


\chapter{Комплексный анализ многих переменных}
\section{Комплексные функции многих переменных}
\textbf{Определение:} Пусть $\Omega\subset\mathbb C^n$ – открытое и связное
множество и $f:\Omega\rightarrow\mathbb C$ – голоморфная функция, если $f$
непрерывная и голоморфная по каждому переменному $z^1,\ldots,z^n$.

Вообще имеется очень много определений голоморфных функций нескольких переменных
и все они эквивалентны, но мы не будем на этом останавливаться, потому что нас
собственно сам анализ интересует постольку, поскольку надо определять комплексные
многообразия и работать с этими функциями. И если вы были на курсе коплексного
анализа, то знаете, что можно ввести комплексное дифферецируемость, условие
Коши-Римана и так далее.

Давайте зафиксируем обозначения.

\textbf{Оперделение:} Пусть $z_0\in\mathbb\mathbb C^n$ и $\mathbf R\in\mathbb R^n_{>0}$,
тогда определим поликруг
\[P(z_0,\mathbf R):=\{z\in\mathbb C^n\,|\,|z^j-z^j_0|<R_j,\,j=1,\ldots,n\}\]
И мы введем
\[T(z_0,\mathbf R):=\partial [P(z_0,\mathbf R)\]

\utv Пусть $f$ голоморфно на $\Omega$, ($f\in\mathcal O(\Omega)$) и нетрудно понять,
что $\mathcal O(\Omega)$ на самом деле кольцо и пусть $\text{Cl}(P(z_0,\mathbf R))\subseteq\Omega$, то верна
следующая формула
\[f(z_0)=\frac{1}{(2\pi i)^n}\int_{T(z_0,\mathbf R)}\frac{f(\xi)d\xi^1\ldots d\xi^n}{(\xi^1-z^1_0)\ldots(\xi^n-z^n_0)}\]
Этот интеграл определяется просто как повторный интеграл по произведению окружностей.

Для краткости можно ввести следующее обозначение $d^n\xi:=d\xi^1\ldots d\xi^n$ и
$(\xi-z):=(\xi^1-z^1)\ldots(\xi^n-z^b$.

Отсюда следует, что все голоморфные функции аналитические, доказательство
дословно повторяет доказательство для функций одной переменной. Если У нас
есть производные, то их можно также написать через похожий интеграл. И если
есть точка в которой все производные всех порядков равны нулю, то фукция –
тождественный ноль на этой области. Это все факты из курса про одну переменную
и все переходит дословно. Давайте это все зафиксируем.

\sled
\begin{enumerate}
    \item Голоморфная функция аналитична

    \item Пусть $\nu=(\nu_1,\ldots,\nu_n)$ – мультииндекс и пусть
        $f^{(\nu)}:=\frac{\partial^{|\nu|}}{(\partial z^1)^{\nu_1}\ldots(\partial z^n)^{\nu_n}}f$
        Тогда
        \[f^{(\nu)}(z_0)=\frac{\nu!}{(2\pi i)^n}\int_T\frac{f(\xi)d^n\xi}{(\xi-z_0)^{\nu+1}}\]
        где $\nu+1=(\nu_i+1)_i$ и $\nu!=(\nu_1!)\cdot\ldots\cdot(\nu_n!)$.

    \item $|f^{\nu}|\leq\frac{\nu!\sup_{T(z_0,\mathbf R)}(f)}{\mathbf R^\nu}$,
        где $\mathbf R^\nu:=R_1^{\nu_1}\ldots R_n^{\nu_n}$.

    \item Если $f$ – голоморфная функця на $\mathbb C^n$ и ограничена, то $f=const$.
\end{enumerate}

\textbf{Напоминание:} $z^j=x^j+iy^j$, а тогда $\frac{\partial}{\partial z_j}=
\frac{1}{2}(\frac{\partial}{\partial x_j}-i\frac{\partial}{\partial y_j})$ и
$\frac{\partial}{\partial \overline z_k}=\frac{1}{2}(\frac{\partial}{\partial
x_k}-i\frac{\partial}{\partial y_k})$.

\section{Голоморфные отображения}
\textbf{Определение:} Пусть $\Omega\subseteq\mathbb C^n$ – открытое множество.
Отображение $F:\Omega\rightarrow\mathbb C^m$ голоморфно, если оно задаётся
голоморфными функциями, то есть если у нас есть координаты $w^1,\ldots,w^m$ на
$\mathbb C^m$, то оно задаётся $F^l=w^l\circ F$ (мы условимся, что координата
– это функция типа $M\rightarrow \mathbb R$ или $\mathbb C$.

Можно дать алтернативную характеристику

\utv Пусть $f:\omega\rightarrow\mathbb C$ – голоморфная функция,
ну и есть область $\Omega'\subseteq\mathbb C^p$ и $\phi^1,\ldots,\phi^n\in\mathcal
O(\Omega)$ – голоморфные функции в этой области такие, что $(\varphi^1(w),\ldots,
\varphi^n(w))\in\Omega$ для всех $w \in\Omega'$. Тогда $f(\varphi^1(w^1,\ldots,w^p)
,\ldots,\varphi^n(w^1,\ldots,w^p))\in\mathcal O(\Omega')$.

Доказательство – это обычное цепное правило, поэтому можно сказать что гоморфные
отображения это такие отображения, которые (обратный образ голоморфной снова голоморфен?)
при композиции с голоморфными функциями пораждают снова голоморфные. Нетрудно видеть,
что эти определения эквивалентны.

$F:\Omega\rightarrow\mathbb C^m$, если для любой голоморфной функции $g\in
\mathcal O(\mathbb C^m)$, $F^*g$ – голоморфная. $*$ – это операция обратного
образа или побега.

\section{Комплексные многообразия}
\textbf{Определение:} Пусть $X$ – xaycдорфого пространство удолевторяющее второй
аксиоме счетности. $X$ – комплексное многообразие, если существует атлас карт
$(U_\alpha, f_\alpha)_\alpha$ и отображениея $\phi_{\alpha\beta}:U_\alpha\cap U_\beta
\rightarrow U_\alpha\cap U_\beta$ голоморфно и обратимо.

Наверно это немного рано так их определять, комплексные многообразия, потому что
мы ещё некоторое время потопчемся в областях $\mathbb C^n$, и нам на самом деле
нужны были голоморфные отображения, потому что одна из самых важных вещей, которые
есть на комплексных многообразиях, это разложение дифференциала де Рама, сейчас
мы его определим.

Если у нас есть обычное многообразие, у нас есть на нём диффернециальные формы
и есть дифференциал де Рама, который переводит формы степени $k$ в формы степени
$k+1$. Давайте заметим, что если у нас есть $X$ – комплексное многообразие, то
оно автоматически является вещественными многообразием. И поэтому на нём есть
косательное расслоени $T_X$. При чем $T_X$ обладает стректурой комплексного
пространства и любое сечение $T_X$ локально представляется $\xi=\xi^j\frac{\partial}
{\partial z^i}$. И если мы его овеществим и тензорно умножим, то получим
$(T_X)_{\mathbb R}\otimes\underline{\mathbb C}\cong T_X\oplus\overline{T_X}$,
аналогичную конструкцию мы получим для кокасательного расслоения, а именно 
$(T_X^*)_{\mathbb R}\otimes\underline{\mathbb C}\cong T_X^*\oplus\overline{T_X^*}$.
Причем можно заметить, что если у нас есть локальные координаты, то сечения
косательное расслоение пораждаются $dz^j=dx^j+idy^j$ $(1,0)$-формами. А есть
$d\overline z^j=dx^j-idy^j$ – $(0,1)$-формы. Давайте теперь посмотрим на
\[\bigwedge^k((T_X^*)_{\mathbb R}\otimes\underline{\mathbb C})=\bigwedge^kT_X^*
\oplus\overline{T_X^*}=\bigoplus_{p+q=k}\bigwedge^pT_x\otimes\bigwedge^q\overline{T_X}\]
Давайте обозначим эту штуку $\bigwedge^{p,q}T^*_X:=\bigwedge^pT_x\otimes
\bigwedge^q\overline{T_X}$, сечения такого расслоения мы будем называть формами
типа $(p,q)$.

Теперь как мы все помним есть дифференциал де Рама, который можно продолжить
очевидным образом на комплекснозначные формы $d:\Gamma(\bigwedge^k(T^*_X)_{
\mathbb R}\otimes\underline{\mathbb C}\rightarrow\Gamma(\bigwedge^{k+1}(T^*_X)_{
\mathbb R}\otimes\underline{\mathbb C}$. Давайте возьмём дифференциальную форму
$\psi\in\Gamma(\bigwedge^k\ldots)$ и вспомним как дифференцирование выглядит в
локальных координатах
\[\psi=\sum_{m+n=k}\sum_{p_1<\ldots<p_m}\sum_{q_1<\ldots<q_n}\psi_{(p,q)}(dx)^p\wedge(dy)^q,\;\text{для}\;(dx)^p=dx^{p_1}\wedge\ldots\wedge dx^{p_m}\]
тогда мы получим
\[d\psi=\sum d\psi_{(p,q)}\wedge (dx)^p\wedge (dy)^q\]
И если $f$ – гладкая функция на $X$, то
\[df=\sum_{j=1}^n\frac{\partial f}{\partial x^j}dx^j+\sum_{j-1}^n\frac{\partial f}{\partial y_j}dy^j
=\sum_{j=1}^n\frac{\partial f}{\partial z^j}dz^j+\sum_{j=1}^n\frac{\partial f}{\partial\overline z^j}d\overline z^j\]
Ну и мы также имеем $\psi=\sum_{p+q=k}\psi_{p,q}$, где $\psi_{p,q}\in\Gamma(\bigwedge^{p,q}T^*_X)$.
разложение пси в такую сумму. И так как дифференциал де Рама линеен, то мы можем
рассматривать формы типа $(p,q)$. Пусть $J=(j_1,\ldots,j_p)$ и $K=(k_1,\ldots,k_n)$
такие мультииндексы, что $1\leq j_1<\ldots<j_p\leq n$ и $1\leq k_1<\ldots<k_q\leq n$.
Тогда $\psi_{p,q}=\sum_{J,K} \psi_{J\overline K}dz^J\wedge d\overline Z^K$, где
$dz^J=dz^j_1\wedge\ldots\wedge dz^j_p$ и $dz^K=d\overline z^k_1\wedge\ldots\wedge
d\overline z^j_p$. Ну тогда давайте ещё один раз напишем этот координатный ужас.
\begin{align*}
    d\psi_{p,q}&=\sum_{J,K}d\psi_{J\overline K}\wedge dz^J\wedge d\overline z^K\\&=
    \sum_{J,K,j=1}^n\frac{\partial \psi_{J\overline K}}{\partial z^j}dz^j\wedge
    dz^J\wedge d\overline z^K+\sum_{J,K,k=1}^n\frac{\partial \psi_{J\overline K}}
    {\partial\overline z^k}d\overline z^j\wedge dz^J\wedge d\overline z^K
\end{align*}
Теперь мы уверждаем, что есть такое разложение и куски один и два получившиеся
в конце разных $(p,q)$-типов инвариантны относительно голоморфных замен координат,
потому что если у вас есть голоморфное отображение, то в координатах оно задаётся
голоморфными функциями. Если у нас есть дифференция $dz^j$, то пубэк $F^*dz^j$ – это
снова $(1,0)$-форма, тоже самое верно и для $(0,1)$-форм. Эти слагаемые мы обозначим
как
\[\partial\psi_{p,q}:=\sum_{J,K,j=1}^n\frac{\partial \psi_{J\overline K}}{\partial z^j}dz^j\wedge
    dz^J\wedge d\overline z^K\quad\overline\partial\psi_{p,q}:=\sum_{J,K,k=1}^n\frac{\partial \psi_{J\overline K}}
    {\partial\overline z^k}d\overline z^j\wedge dz^J\wedge d\overline z^K
\]

\utv
\begin{align*}
    d:\Gamma\left(\bigwedge^{p,q}T^*_X\right)&\rightarrow\Gamma\left(\bigwedge^{p+1,q}T^*_X\right)\oplus
\Gamma\left(\bigwedge^{p,q+1}T^*_X\right)\\
    \partial:\Gamma\left(\bigwedge^{p,q}T^*_X\right)&\rightarrow\Gamma\left(\bigwedge^{p+1,q}T^*_X\right)\\
    \overline\partial:\Gamma\left(\bigwedge^{p,q}T^*_X\right)&\rightarrow\Gamma\left(\bigwedge^{p,q+1}T^*_X\right)
\end{align*}

И $d=\partial+\overline\partial$ такое разложение единственно и не зависит от координат.
% time - 1:06:35

\sled Функция $f:X\rightarrow\mathbb C$ голоморфная, если и только
если 
\[\overline\partial f=0\]

Так как $d^2=0$, то $\partial^2+\overline\partial^2+\partial\overline\partial+
\overline\partial\partial=0$. И если $\psi_{p,q}\in\Gamma(\bigwedge^{p,q})$, то
\begin{align*}
    \partial^2\psi_{p,q}&\in\Gamma\left(\bigwedge^{p+2,q}\right)\\
    \overline\partial^2\psi_{p,q}&\in\Gamma\left(\bigwedge^{p+2,q}\right)\\
    (\partial\overline\partial+\overline\partial\partial)\psi_{p,q}&\in\Gamma\left(
    \bigwedge^{p+1,q+1}\right)
\end{align*}
A значит $\overline\partial^2=\partial^2=\partial\overline\partial+\overline\partial
\partial=0$. Тогда для оператора $\overline\partial$ можно тоже определить
когомологии, тем же способом, которым определяются когомологии Де Рама.
\[H^{p,q}(X)=\frac{\text{Ker}\,\overline\partial|_{\Gamma\left(\bigwedge^{p,q}\right)}}{\text{Im}\,
\overline\partial|_{\Gamma\left(\bigwedge^{p,q-1}\right)}}\]
Очевидно, что эти группы могут быть достаточно большими. Например даже группа
$H^{0,0}$ может быть громадна в полидиске или в единичном шаре, потому что у
вас просто невероятно много голоморфных функций.

Более общё, нетрудно видеть, что

\utv Пусть $\psi_{p,0}\in\Gamma\left(\bigwedge^{p,0}\right)$. Тогда оно
удолетворяет $\overline\partial\psi){p,0}=0$ тогда и только тогда, когда
её коэффициенты – это голоморфные функции.

То есть форма типа $(p,0)$ у которой в координатной записи есть только
дифференциалы типа $dz_j$ и нет $d\overline z_j$, то она пренадлежит
этой группе, то есть она $\overline\partial$-замекнута тогда и только тогда,
когда её коэффициенты – это голоморфные функции.

\textbf{Вопрос:} А $\mathbb H^n$ оно не раскладывается в сумму $H^{p,q}$?

\textbf{Ответ:} Вообще говоря нет, если взять прямую сумму $H^{p,q}(X)$, то это
будет больше, чем когомологии де Рама.

\textbf{Вопрос:} A это тоже самое, что $H^q\left(\Omega^p\right)$?

\textbf{Ответ:} Да, да, да. Это теорема Дальбо, которую я надеюсь мы через
какое-то достаточно быстрое время докажем. Но да, будте осторожны с $H^{p,q}$,
вообще говоря если мы берём форму оттуда, то она не обязана быть замкнутой
относительно дифференциала де Рама. Из $\overline\partial\psi=0$ не следует, что
$\partial\psi=0$, такое бывает на некэйлеровых многообразиях. Есть многообразия,
которые не доспускают некоторые специальные эрмитовые метрики. В частности из 
$\overline\partial\psi=0$ следует $\partial\psi=0$ на проективных многообразиях,
то есть на подмножествах в $\mathbb C^n$, например если вы возьмёте голоморфную
функцию на поликруге, то если она является замкнутой на относительно
дифференциала де Рама, то она константа.

Ну и собственно на самом деле нас будет интересовать то, как можно решить
уравнение вида
\[\overline\partial\psi=f\]
где $\psi$ – это какая-то форма типа $(p,q)$, а $f$ – это форма типа $(p,q+1)$
или не просто форма, а форма со значение в расслоении. [Хотелось бы расказать
сегодня про фундаметальное решение этого уравнения, хотя бы на $(0,1)$-формах,
но мы видемо не успеем на этой лекции, так что расказ пойдёт о кое-чем другом.

Если говорить о голоморфных векторных полях, то непонятно что это такое, потому
что если у нас есть гладкое многообразие, то на векторных полях у нас есть
дифференциал де Рама на формах, но какого-то аналога естественного и канонического
оператора на векторных полях у нас нет в гладком случае. Но оказывается что в
случае комплексных многообразий можно определить $\overline\partial$ и на
касательном расслоении и вообще на многих других расслоениях.

\opr Пусть $\pi:E\rightarrow X$ – комплексное векторное расслоение наж комплексным
многообразием $X$ (все же занают что такое комплексное векторное расслоение над
гладким многообразием). $E$ называется голоморфным векторным расслоением, если
существует тривиализующее покрытие $\left\{U_\alpha\right\}$, $U_\alpha\subset
\mathbb C^n$, такое что функции перехода $\gamma_{\alpha\beta}$ голоморфны
[запись второй лекции прервалась]


% лекция 3
\chapter{Потоки (и обобщенные функции)}
И так мы закончили на том, что определили оператор $\overline\partial$, который
действует на формах следующим образом
\[\overline\partial\alpha=\sum_{k=1}^n\frac{\partial_{\overline k}\alpha_{I\overline J}}{\partial
\overline z^K}d\overline{z^k}\wedge dz^I\wedge d\overline{z^J}\]
Наша цель сегодня понять когда существует решение у уравнения $\overline\partial
u=f$. Необходимое условие очевидно $\overline\partial f=0$ исходя из того, что
мы обсуждали в прошлый раз. Сегодня мы попытемся подойти к этому для областей в
$\mathbb C^n$, потому что на многообразиях теория куда многообразней и сложней.
Но локальную теорию мы постараемся разобрать.

\textbf{Цель:}
\begin{enumerate}
    \item Научиться выражать $u$ через $f$. Стоит учесть, что даже в случае,
        когда $f$ – это $(0,1)$-форма, то у нас может быть очень и очень
        много различных решений, потому что ядро $\overline\partial$ оно
        содержит в себе все голоморфные функции, поэтому мы будем искать
        какое-то специфическое решение, и так как мы можем прибавить к $u$
        любой элемент из ядра, то мы получим другое решение, и все решения
        различаются на элемент из ядра, а значит нам надо найти некоторое
        решение, в некотором смысле оптимальное.
    \item Доказать аналог леммы Пуанкаре. То есть лемма для дифференциала де Рама
        она говорит, что любая замкнутая форма локально точна, то есть вы всегда
        можете найти локальное решение.
\end{enumerate}

К пункту один можно подойти поразному, один из подходов изложен в следующей
задаче

\textbf{Задача:}
Пусть $f=f_jd\overline z^j$ и $\overline\partial f=0$, тогда мы можем написать
такую вещь
\[u_j=\frac{1}{2\pi i}\int_{\mathbb C}\frac{f_j(z^1,\ldots,z^{j-1},\zeta,z^{j+1},\ldots)d\zeta\wedge d\overline\zeta}
{\zeta-z^j}
\]
И $f_j$ зависят как от $z^j$, так и от их сопряженных, но мы не написалии обозначение
сопряженных координат, потому что и так всё понятно. И утверждается, что когда форма
$f$ замкнута и с компактным носителем, то мы можем выбрать любое $f_j$ и написать
такой интеграл, то все $u_j$ будут равны и корректно задает функцию $u$. Тоже
самое на самом деле можно проделать и с формами других степеней, но такой подход
он более наивный, менее техничный и как следствие, если мы убираем техничность,
то мы получаем больше нудной работы, при работе с формами степени $(p,q)$, где
$p>1$ и $q>1$. Это как-то занудно, тем более оно везде написано: в Гриферте-Харрисоне,
в Вуазене, в Хуберте и ещё бог знает где, поэтому сегодняшняя лекция будет следовать
подходу, изложеному в Демаи, но для этого надо обсудить обобщенные функции и потоки.
Расказ не будет останавливаться особо на деталях, так как во-первых – сильно не
надо, во вторых вы с этим более-менее знакомы, в-третих нам нужен на самом деле
один кокретный пример мы его в деталях разберем и потом из него всё быстро
выведем.

\section{Обобщенные функции}
Нас в первую очередь будут интересовать банаховы пространства функций, которые
непрерывно дифференцируемы до порядка $k$ и пространства бесконечно
дифференцируемых функций с компактным носителем и соответственно дифференциальные
формы с компактным носителем. На самом деле последнее является нашим основным
интересом.

Пусть $\mathcal D^k(\Omega)$, $\Omega\subseteq\mathbb R^n$ – пространство
$k$-форм со значениями в $\mathbb R$ или $\mathbb C$ с компактными носителями.
Давайте остановимся на этом случае, когда технические детали сводяться к минимум,
когда нам не нужно следить за тем, что происходит на бесконечности и за прочим.

Пусть $L\subseteq\Omega$ – компакт, $\alpha\in\mathcal D^k(\Omega)$, тогд мы
можем определить
\[p_{S,L}(\alpha):=\sup_{x\in L}\sup_{|\nu|<S}
\sup_J|\partial^{\nu}\alpha_J(x)|\]
где $\alpha=\sum_{|J|=K}\alpha_Jdx^{J}$ и $\partial\alpha_J=\frac{\partial^{|\nu|}}
{\partial^{\nu_1}x^1\ldots\partial^{\nu_n}x^n}\alpha_J(X)$. Этот набор полунорм
задает топологию, но нам важно знать, что последовательность норм сходится в
этой топологии тогда и только тогда, когда сходиться по каждой из этих полунорм.

\opr Поток $T$ размерности $k$ (или степени $n-k$) – это линейный функционал на
$\mathcal D^k$, неперывен в вышеописанной топологии.

\textbf{Пример:}
\begin{enumerate}
    \item Пусть $\beta$ – гладкая $(n-k)$-форма на $\Omega$. Тогда $T_\beta(\alpha)
        :=\int_\Omega\beta\wedge\alpha$ – поток, что легко проверить простой
        оценкой интеграла по любой из этих норм.
    \item $\Sigma\subseteq\Omega$ – гладкая ориентируемая поверхность в $\Omega$,
        тогда $T_\Sigma(\alpha):=\int_\Sigma\alpha$. Частный пример этого пункта
        – это знаменитая $\delta$ функция на $0$ формах, то есть на функциях.
\end{enumerate}

На самом деле нетрудно видеть, что все про что мы говорили обобщается на
многообразия, но так как мы будем использовать потоки первого типа, то мы будем
считать, что многообразия ориентируемы.

\end{document}
