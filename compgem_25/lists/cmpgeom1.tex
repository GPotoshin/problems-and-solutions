\documentclass{article}
\usepackage[a4paper,left=3cm,right=3cm,top=1cm,bottom=2cm]{geometry}
\usepackage{amsmath}
\usepackage{amssymb}
\usepackage{hyperref}
\usepackage[russian]{babel}

\usepackage{tikz-cd}
\usepackage{array}
\usepackage{graphicx}
\newcommand\mapsfrom{\mathrel{\reflectbox{\ensuremath{\mapsto}}}}
\setlength{\parindent}{0mm}

\usepackage{fontspec}
\setmainfont{Linux Libertine O}
\usepackage{unicode-math}
\setmathfont{Cambria Math}

\title{
\textit{\small{Георгий Потошин, 2025}}\\
\vspace{0.3ex}
\textit{\huge{Комплексная геометрия, листочек 1}}\vspace{1ex}
}

\date{\vspace{-10ex}}

\begin{document}
\maketitle

\section{Линейная алгебра}
\begin{enumerate}
\item \textbf{Пусть $M$ – комплексная матрица, а $M_{\mathbb R}$ – её
    овеществление. Докажите, что $\det M_{\mathbb R}=|\det M|^2$.}

    Пусть $M$ раскладывается на вещественную и мнимую часть $M=A+iB$. Тогда
    мы можем написать овеществленную матрицу:
    \[
        M_{\mathbb R} = \left(\begin{array}{cc}A&B\\-B&A\end{array}\right)
    \]
    Суть решения заключается в том, что мы пытаемся блочно триангонализировать
    матрицу, что равносильно домножению на обратимые матрицы с двух сторон.
    Задав коэффициенты боковых матриц и решив систему уравнений, я нашел
    следующее соотношение:
    \[
        \left(\begin{array}{cc}0&-i\\1&i\end{array}\right)
        \left(\begin{array}{cc}A&B\\-B&A\end{array}\right)
        \left(\begin{array}{cc}1&0\\i&-i\end{array}\right)
        =
        \left(\begin{array}{cc}A+iB&-A\\0&A-iB\end{array}\right)
    \]
    Дальше мы воспользуемся блочно-диагональным свойством детерминанта, чем мы
    не могли воспользоваться изначально, потому что вообще говоря такая формула
    совсем может быть не верна. А также мы воспользуемся мультиплетностью, но
    для этого нужно иметь матрицы одного размера:
    \[
        \left|\begin{array}{cc}0&-iI_n\\I_n&iI_n\end{array}\right|
        \left|\begin{array}{cc}A&B\\-B&A\end{array}\right|
        \left|\begin{array}{cc}I_n&0\\iI_n&-iI_n\end{array}\right|
        =
        \left|\begin{array}{cc}A+iB&-A\\0&A-iB\end{array}\right|
    \]
    Теперь нам нужно триангонализировать левую матрицу, переставляя местами
    соответствующие строки верхней и нижней матрицы. Каждая такая перестановка
    изменит знак детерминанта и в итоге мы получим
    \[
        (-1)^n\left|\begin{array}{cc}I_n&iI_n\\0&-iI_n\end{array}\right|
        \left|\begin{array}{cc}A&B\\-B&A\end{array}\right|
        \left|\begin{array}{cc}I_n&0\\iI_n&-iI_n\end{array}\right|
        =
        \left|\begin{array}{cc}A+iB&-A\\0&A-iB\end{array}\right|
    \]
    Теперь мы можем сосчитать все детерминанты:
    \begin{align*}
        (-1)^n(-i)^n\det(M_{\mathbb R})(-i)^n&=\det(A+iB)\det(A-iB)\\
        \det(M_{\mathbb R})&=|\det(A+iB)|^2
    \end{align*}

\item \textbf{Докажите соотношение между вещественным и комплексным
    следами косоэрмитова оператора $A:V\rightarrow V$}

        Пусть у нас есть $\mathbb C$-базис $(e_i)$. В нём мы можем найти
        коэффициенты оператора
        \[A=(a^i_j+\sqrt{-1}b^i_j)e_i\otimes e^{j*}\]
        Так как $A^*=-A$, то на диагонали мы имеем $a^i_i-\sqrt{-1}b^i_i=-(a^i_j
        +\sqrt{-1}b^i_j)$, a значит $a^i_i=0$. Запишем теперь разложение на
        коэффициенты овеществления
        \[A_{\mathbb R}=a^i_je_i\otimes e^{j*}-b^i_j
        e_i\otimes(\sqrt{-1}e^{j*})+b^i_j(\sqrt{-1}e_i)\otimes e^{j*}+a^i_j
        (\sqrt{-1}e_i)\otimes(\sqrt{-1}e^{j*})\]
        Теперь давайте запишим разоложение оператора $i$ в овеществленном базисе.
        \[J=-e_i\otimes(\sqrt{-1}e^{j*})+(\sqrt{-1}e_i)\otimes e^{j*}\]
        Теперь давайте посчитаем произведение операторов
        \begin{align*}
        A_{\mathbb R}J=&(a^i_je_i\otimes e^{j*}-b^i_j
        e_i\otimes(\sqrt{-1}e^{j*})+b^i_j(\sqrt{-1}e_i)\otimes e^{j*}+a^i_j
        (\sqrt{-1}e_i)\otimes(\sqrt{-1}e^{j*}))\\
        &(-e_i\otimes(\sqrt{-1}e^{j*})+(\sqrt{-1}e_i)\otimes e^{j*})=\\
        &(a^i_je_i\otimes(\sqrt{-1}e^{j*})-b^i_je_i\otimes e^{i*}-b^i_j(\sqrt{-1}
        e_i\otimes(\sqrt{-1}e^{j*}) + a^i_j(\sqrt{-1}e_i)\otimes e^{j*}
        \end{align*}
        Теперь мы сосчитаем следы:
        \[\text{Tr}_{\mathbb C} A=\sqrt{-1}b^i_i=-\sqrt{-1}/2\text{Tr}_{\mathbb R}(A_{\mathbb R}J)\]

    \item \textbf{a) Докажите, что пространство эндоморфизмов $I:V_{\mathbb R}\rightarrow V_{\mathbb R}$,
        удовлетворяющих $IJ+JI=0$ может быть отождествлено с $V^{1,0}\otimes(V^∗)^{0,1}$}

        Выберем комплексный базис $e=(e_i)$, его можно дополнить до действительного
        базиса $(e,ie)$. Разложим на блоки $I$ и $J$ в этом базисе.
        \[I=\left(\begin{array}{cc}A&B\\C&D\end{array}\right)\quad\quad J=\left(\begin{array}{cc}0&-1\\1&0\end{array}\right)\]
        Посчитаем теперь произведения
        \[IJ=\left(\begin{array}{cc}B&-A\\D&-C\end{array}\right)\quad\quad JI=\left(\begin{array}{cc}-C&-D\\A&B\end{array}\right)\]
        Теперь их сумму
        \[IJ+JI=\left(\begin{array}{cc}B-C&-A-D\\A+D&B-C\end{array}\right)=0\]
        Отсюда мы делаем вывод, что $B=C$ и $D=-A$, то есть $I$ имеет следующий
        вид
        \[I=\left(\begin{array}{cc}A&B\\B&-A\end{array}\right)\]
        Теперь если считать, что $(e,ie)$ строка векторов, и если в матричном
        умножении договориться, что умножение векторов – это тензорное умножение,
        то $I$ можно переписать как элемент тензорного прострнства следующим образом
        \[I=(e,ie)\left(\begin{array}{cc}A&B\\B&-A\end{array}\right)\left(\begin{array}{c}e^*\\(ie)^*\end{array}\right)\]
        Где правый столбец - это дуальный базис. Так как $V^{0,1}=\overline V$,
        где единственное отличие в том что умножение на $i$ – это умножение на
        $-i$, то мы можем выразить дуальный базис нового пространства следующим
        образом
        \[\left(\begin{array}{cc}1&0\\0&-1\end{array}\right)
        \left(\begin{array}{c}\overline e^*\\i\overline e^*\end{array}\right)=
        \left(\begin{array}{c}e^*\\ie^*\end{array}\right)
        \]
        А тогда осуществив подстановку мы получим
        \[I=(e,ie)\left(\begin{array}{cc}A&B\\B&-A\end{array}\right)
        \left(\begin{array}{cc}1&0\\0&-1\end{array}\right)
        \left(\begin{array}{c}\overline e^*\\i\overline e^*\end{array}\right)=
        (e,ie)\left(\begin{array}{cc}A&-B\\B&A\end{array}\right)
        \left(\begin{array}{c}\overline e^*\\i\overline e^*\end{array}\right)
        \]
        И мы видем, что овеществленная матрица $J$ как отображения из
        $V$ в $\overline V$ является голоморфной, а значит $J\in V^{1,0}
        \otimes_{\mathbb C}(V^∗)^{0,1}$. Более того все переходы выше были
        эквивалентны, а поэтому пространство таких $I$ и $V^{1,0}
        \otimes_{\mathbb C}(V^∗)^{0,1}$ тождествены.

        \textbf{b) Докажите, что если $J(t)$ – гладкая кривая в пространстве комплексных
        структур, такая что $J(0)=J$, то $J(0)J′(0)+J′(0)J(0)=0$. Выведите отсюда,
        что $V^{1,0}\otimes (V^∗)^{0,1}$ являются инфинитезимальными деформациями
        комплексных структур.}

        Про $J$ мы знаем, что $J^2=-\text{Id}$. Давайте теперь возьмём производную,
        тогда мы в точности получим $JJ'+J'J=0$. Тогда мы поймём, что касательное
        пространство векторов $J'$, и есть пространство инфинитеземальных
        деформаций комплексных струкстур. К сожалению из этого следует только
        вложеность касательного пространства в пространство $V^{1,0}
        \otimes_{\mathbb C}(V^∗)^{0,1}$. Для равенства достаточно показать,
        что в окрестности $J$ комплексные структуры образуют гладкое многообразие
        размерности как минимум $2n^2$. [это нужно проверить]

    \item \textbf{Пусть $\alpha$ – вещественная форма типа $(1,1)$ на $V$.
        Докажите, что $n(n−1)\alpha\wedge\alpha\wedge\omega^{n−2}=((\text{Tr}_
        \omega\alpha)^2 −|\alpha|^2_\omega)\omega^n$.}

        Tогда для некого базиса $(e_i)$ над $\mathbb C$ $\alpha$ имеет вид
        \[\alpha=\frac{i}{2}A_{i\overline j}e^{i*}\wedge\overline{e^{j*}}\]
        где $A$ – эрмитова матрица, а симплектическая структура имеет вид
        \[\omega=\frac{i}{2}e^{i*}\wedge\overline{e^{i*}}\]
        Тогда нетрудно сосчитать степень симплектической структуры, так как
        нужно сделать комбинаторные выборы из каждого множителя, из первого – $n$,
        из второго $n-1$ и так далее, все перестановки четные, а тогда мы получим
        \[\omega^n=(\frac{i}{2})n!\tau,\quad\text{для}\quad\tau=e^{1*}\wedge
        \overline{e^{1*}}\ldots e^{n*}\wedge\overline{e^{n*}}\]
        Теперь давайте сосчитаем другую сторону и получим
        \[\alpha\wedge\alpha\wedge\omega^{n-2}=(\frac{i}{2})^2(A_{ii}e^{i*}\wedge
        \overline{e^{i*}}\wedge A_{jj}e^{j*}\wedge\overline{e^{j*}}+A_{ij}e^{i*}
        \wedge\overline{e^{j*}}\wedge A_{ji}e^{j*}\wedge\overline{e^{i*}})\wedge
        (\frac{i}{2})^{n-2}(n-2)!\sum_{i\neq j}\bigwedge_{k\neq i,j}e^{k*}
        \overline{e^{k*}}\]
        В первой скобке я оставил только те слогаемые, у которых есть две пары
        вектор и его сопряженное, потому что только они в произведении с $\omega^{n-2}$
        дадут ненулевой вклад, причем первое слогаемое без коэффициента даст
        вклад $\tau$, a второе – $-\tau$, так как нужно будет осущесвить одну
        транспозицию. Тогда можем это дело упростить, а затем воспользоваться
        эрмитовостью матрицы.
        \begin{align*}
            \alpha\wedge\alpha\wedge\omega^{n-2}&=(\frac{i}{2})^n(n-2)!(A_{ii}A_{jj}-
            A_{ij}A_{ji})\tau\\
            &=(\frac{i}{2})^n(n-2)!(\text{Tr}(\alpha)^2-
            A_{ij}\overline{A_{ij}})\tau\\
            &=(\frac{i}{2})^n(n-2)!(\text{Tr}(\alpha)^2-
            \|\alpha\|^2_\omega)\tau
        \end{align*}
        В итоге нетрудно отсюда видеть, что искомое соотношение верно.Я правда 
        не уверен, что $\text{Tr}\equiv\text{Tr}_\omega$, но формула сошлась с
        искомой.

    \item \textbf{Пусть $(V,g)$ – вещественное векторное пространство размерности 2,
        на котором задана евклидова метрика $g$. Постройте на $(V,g)$ оператор
        комплексной структуры. Выведите отсюда, что на любом ориентируемом
        римановом многообразии $(M^2,g)$ существует поле тензорное поле
        операторов $J$, таких, что $J^2=−Id$.}

        Собственно на $V$ задание структуры просто, мы выбираем ортонормированный
        базис $(e_1, e_2)$ и говорим, что $Je_1=e_2$, а $Je_2=-e_1$. Из курса 
        геометрии известно, что для ортонормированных базисов одной ориентации
        такое задание оператора $J$ совпадает по базисам, а именно это поворот
        на 90 градусов по заданному направлению.

        Для многообразия мы хотим проверить, что заданное поле таких структур
        будет гладким. Так как многообразие ориентируемое, то мы имеем
        ориентированный атлас $(U_\alpha, f_\alpha)$. Тогда в рамках одной карты
        $(U,f=(x^1, x^2))$, мы можем образовать гладкий базиз
        $(\partial_1, \partial_2)$ касательных векторов к линиям координат.
        Затем, так как вектора не занулятся, то при процедуре ортогонализации
        Грамм-Шмидта базис останется гладким, и если мы выберем ориентацию в
        $\text{codom}f$, то мы получим ориентированый ортонормированный гладкий
        базис $(e_1,e_2)$ на $U$. Соответственно зададим $J$ в базисе $(e_1,e_2)$
        cледующей матрицей:
        \[J=\left(\begin{array}{cc}0&-1\\1&0\end{array}\right)\]
        Так как матрица постоянна, а базис гладок, то полученное поле $J$ само
        гладко.

        Осталось проверить, что поле правильно склеивается между картами. Пусть
        у нас есть две пересекающиеся карты $U$ и $V$. Тогда на их пересечении
        в каждой точке есть соответствующие картам ортонормированные базисы,
        но так как ориентация совпадает, то $J$ одинаково определен в разных
        картах на пересечении, как это было в $(V,g)$. Тогда поле $J$ существует
        и согласуется с метрикой, так как в каждом касательном прострастве $J$ 
        – движение.

    \item \textbf{Пусть $(V,g)$ – вещественное векторное пространство вещественной
        размерности $4$, а $g$ – лоренцева метрика сигнатуры $(1,3)$. Покажите,
        что на $\bigwedge^2(V)$ существует оператор комплексной структуры.}

        Пусть $(e_i)$ – ортонормированный базис в котором матрица $g$ равна
        $\text{diag}(1,-1,-1,-1)$, также мы имеем форму объема $\tau$ ассоциированную
        с базисом $e$.

        Тогда пространство $\bigwedge^2(V)$ имеет базис $(e_1\wedge e_2,
        e_1\wedge e_3,e_1\wedge e_4,e_2\wedge e_3,e_2\wedge e_4,e_3\wedge e_4)$
        и имеет следующее скалярное произведение:
        \[\langle a_1\wedge a_2, b_1\wedge b_2\rangle = \det\left(\begin{array}{cc}\langle v_1,w_1\rangle&
        \langle v_1,w_2\rangle\\ \langle v_2,w_1\rangle& \langle v_2,w_2\rangle\end{array}\right)\]
        Заметим, что наш выбранный базис на $\bigwedge^2(V)$ ортонормирован.

        Теперь нам нужно выбрать кандидата на роль оператора $J$, и им будет
        звезда ходжа. По определению мы должны иметь
        \[\phi\wedge(\star\psi)=\langle\phi,\psi\rangle\tau\]
        $\star$ очевидно должен быть линеен, так как скалярное произведение
        линейно, и более того, так как $4-2=2$, то $\text{codom}\star=\bigwedge^2(V)$.

        Так как выбранный базис ортонормален, то $\star(e_i\wedge e_j)$ должен
        содержать только противоположные индексы, так как иначе, мы бы нашили
        другой базисный вектор $f$, чья координата ненулевая в $\star(e_i\wedge e_j)$,
        Взять дополняющий его элемент базиса $g$, такой, чтобы $f\wedge g\neq 0$.
        Заметим, что $g\neq e_i\wedge e_j$, тогда если за $\phi$ взять $g$,
        а за $\psi$ – $e_i\wedge e_j$, то в определяющем звезду ходжа равенстве
        мы бы получили слева неноль, а справа ноль из-за ортонормированности
        базиса.

        Теперь мы можем окуратно посчитать образы дуальных элементов.
        \begin{align*}
            e_1\wedge e_2\wedge(\star(e_1\wedge e_2))&=-\tau\\
            \star(e_1\wedge e_2)&=-e_3\wedge e_4
        \end{align*}
        \begin{align*}
            e_1\wedge e_3\wedge(\star(e_1\wedge e_3))&=-\tau\\
            \star(e_1\wedge e_3)&=e_2\wedge e_4
        \end{align*}
        \begin{align*}
            e_1\wedge e_4\wedge(\star(e_1\wedge e_4))&=-\tau\\
            \star(e_1\wedge e_4)&=-e_2\wedge e_3
        \end{align*}
        \begin{align*}
            e_2\wedge e_3\wedge(\star(e_2\wedge e_3))&=-\tau\\
            \star(e_2\wedge e_3)&=e_1\wedge e_2
        \end{align*}
        \begin{align*}
            e_2\wedge e_4\wedge(\star(e_2\wedge e_4))&=-\tau\\
            \star(e_2\wedge e_4)&=-e_1\wedge e_3
        \end{align*}
        \begin{align*}
            e_3\wedge e_4\wedge(\star(e_3\wedge e_4))&=-\tau\\
            \star(e_3\wedge e_4)&=e_1\wedge e_2
        \end{align*}

        Тогда в выбраном ортонормированном базисе мы можем записать матрицу $\star$.
        \[\text{Mat}(\star)=\left(\begin{array}{cccccc}
            0&0& 0&0& 0&1\\
            0&0& 0&0&-1&0\\
            0&0& 0&1& 0&0\\
            0&0&-1&0& 0&0\\
            0&1& 0&0& 0&0\\
           -1&0& 0&0& 0&0\\
        \end{array}\right)\]
        Из матрици видно, что на $\bigwedge^2(V)$ мы имеем $\star^2=-\text{Id}$.

    \item \textbf{Докажите неравенство Виртингера: если $W\subset V$ – векторное
        подпространство вещественной размерности $2k$. Пусть $\text{Vol}_W$ –
        форма объема на $W$, индуцированная метрикой $g$. Тогда выполнено
        следующее неравенство:
        \[ω^k|_W\leq k!\textnormal{Vol}_W\]
        и равенство достигается тогда и только тогда, когда $W$ – комплексное
        подпространство.}

        Если $W$ комплексное подпространство, то у него есть комплексный
        ортонормированный базис, который можно продлить до комплексного
        ортонормированного базиса $(e_i)$ всего пространства. В нём
        симплектическая форма имеет следующую запись
        \[w=e^{i*}\wedge ie^{i*}\]
        A тогда легко можно сосчитать степень
        \[\omega^k=k!\sum_{i_1,\ldots,i_k\subset \{1..n\}}\bigwedge_l=1^k e^{i_l*}\wedge ie^{i_l*}\]
        А тогда ограничение $\omega^k$ на $W$ очевидно имеет следующий вид
        \[\omega^k|_W=k!\text{Vol}_W\]

        В другую сторону в $W$ мы можем выбрать единичный вектор $p_1$, затем к нему
        там же подобрать вектор $q_1$ такой, что $\omega(p_1,q_1)\neq 0$. Если
        подобрать не получается, то мы берём другой вектор за место $p_1$. Так
        мы находим пару $p_1,q_1$ ортонормальных векторов. Затем продолжаем поиск
        слудующих ортогональных векторов в ортогональном к нему пространстве. В
        итоге мы получим ортонормированную систему векторов $p_1,\ldots,p_l,q_1,
        \ldots p_l$.

    \item \textbf{Пусть $\Psi$ – форма типа $(n−1,n−1)$ на $V$. Покажите, что
        $\Psi$ определяет $(1,1)$-форму на $\bigwedge^{n−1}V$. Покажите также, что
        если $\Psi=\omega^{n−1}$ для некоторой $(1,1)$-формы $ω$, то $\det \Psi
        = (\det\omega)^{n−1}$.}

        Для базиса $(e_n)$ мы можем переписать базис $\bigwedge^{n-1,n-1}$
        как элементы вида $\bigwedge_{i\neq k} e^{i*}\wedge\overline{\bigwedge_{j\neq l}e^{j*}}$.
        что собственно будет базисом $\bigwedge^{1,1}\bigwedge^{n−1}V$.

        Теперь пусть $\omega = f_{ij} z^{i*}\wedge\overline{z^{j*}}$. Тогда если
        мы обозначим за $\Omega^{kl}=\bigwedge_{i\neq k} e^{i*}\wedge\overline
        {\bigwedge_{j\neq l}e^{j*}}$, то мы получим формулу вида
        \[\omega^{n-1}=F_{il}\Omega^{il}\]
        где $F=\text{adj}\;f$. Тогда $\det(\omega^{n-1})=\det(\text{adj}\;f)=\det(f)^{n-1}
        =\det\omega$.

    \item \textbf{Докажите, что для $(1,1)$-форм и $(n−1,n−1)$-форм понятия
        положительности и сильной положительности совпадают.}

        На лекции мы видели, что сильно пложительная форма положительна. Давайте
        докажем импликацию в иную сторону.

    \item \textbf{Придумайте пример положительной формы $β\in\bigwedge^{p,p}(V^∗)$,
        которая не является строго положительной.}
\end{enumerate}

    \section{Голоморфные функции многих переменных}

\begin{enumerate}
    \item \textbf{Докажите, что ограниченная голоморфная функция на $\mathbb C$
        является константой.}

        Пусть $f:\mathbb{C}^n\rightarrow\mathbb{C}$ голоморфная функция. Тогда
        её ограничение на любое комплексное подпространство также является
        голоморфной функцией, так как ограничение линейного оператора по прежнему
        линейно. Тогда на каждом одномерном подпространстве $V$ $f$ является
        ограниченой голоморфной функцией одного переменного. Тогда мы можем
        воспользоваться одномерной теоремой Лиувилля и получить константность
        на каждом одномерном подпространстве. Так как все эти подпространства
        пересекаются в нуле, то вся функция $f$ постоянна.

    \item \textbf{Докажите, что вещественная и мнимая части голоморфной
        функции являются гармоническими функциями. Т.е. если $f=u+iv$, то 
        $\Delta u=\Delta v=0$, где $\Delta$ – стандартный оператор Лапласа на $\mathbb R^{2n}=
        \mathbb C^n$.}

        Из голоморфности мы имеем $\partial_{e_k}u=\partial_{ie_k}v$ и
        $\partial_{e_k}v=-\partial_{ie_k}u$. Тогда мы можем посчитать
        \begin{align*}
            \Delta u&=\sum_k \partial_{e_k}^2u+\sum_k \partial_{ie_k}^2u\\
            &=\sum_k \partial_{e_k}\partial_{ie_k}v+\sum_k \partial_{ie_k}^2u\\
            &=\sum_k \partial_{ie_k}\partial_{e_k}v+\sum_k \partial_{ie_k}^2u\\
            &=\sum_k -\partial_{ie_k}^2u+\sum_k \partial_{ie_k}^2u=0\\
        \end{align*}
        Аналогичные вычисления можно провести и для $v$.

    \item \textbf{Пусть $f\in\mathcal{O}(\mathbb{C}^n)$ – голоморфная функция,
        а $\alpha=(\alpha_1,\ldots,\alpha_n)$ – некоторый мультииндекс.
        Предположим, что существует константа $C$, такая что $|f(z)|\leq C|zα|$.
        Докажите, что $f$ является полиномом степени не выше $\alpha$.}

        Если $C\leq 0$, то задача тривиальна, поэтому мы рассматриваем только
        $C>0$.

        Как мы знаем, $f$ раскладывается в ряд, а его коэффициенты определяются
        интегралами. Возьмём мультикоэффициент $\beta=(\beta_i)$ и мультирадиус
        $R=(R_i)$, тогда для коэффициента ряда мы имеем следующее выражение
        \[c_\beta=\frac{1}{(2\pi i)^n}\int_{\partial B(0,R)}\frac{f(\zeta)}{\zeta^{\beta+1}}d\zeta\]
        Тогда мы можем применить оценочную лемму для интеграла
        \[|c_\beta|\leq\frac{1}{(2\pi)^n}\sup_{\zeta\in\partial B(0,R)}|\frac{f(\zeta)}{\zeta^{\beta+1}}|\text{Area}(\partial B(0,R))\]
        Мы знаем, все заначения $|\zeta_i|=R_i$, а также
        знаем формулу для площади полисферы (нужно просто интеграл разложить в произведение итегралов).
        \[|c_\beta|\leq\frac{1}{(2\pi)^n}\sup_{\zeta\in\partial B(0,R)}|\frac{f(\zeta)}{R^{\beta+1}}|(2\pi)^n\prod R_i
        =\sup_{\zeta\in\partial B(0,R)}|\frac{f(\zeta)}{R^{\beta_i}}|\]
        Теперь мы можем подставить неравенство из условия и получить
        \[|c_\beta|\leq CR^{\alpha-\beta}\]
        и если окажеться, что $\alpha_i-\beta_i<0$, то есть мы будем
        смотреть на коэффициент с не меньшей степенью, то
        \[|c_\beta|\leq CR^{\alpha-\beta}\rightarrow_{R_i\rightarrow+\infty}0\],
        то есть $c_\beta=0$, а значит $f$ – полином степени не выше $\alpha$.


    \item \textbf{Пусть $\Omega⊂\mathbb C^n$ – область, а $f1,\ldots,fk\in\mathcal
        O(\Omega)$ – голоморфные функции в $\Omega$. Докажите, что если
        \[\sum_j |f_j|^2=\textnormal{const}\]
        то каждая из $f_j$ является константой.}

        Заметим, что 
        \begin{align*}
            \frac{\partial^2}{\partial\overline z\partial z}=
            \frac{1}{2}\left(\frac{\partial}{\partial x}+i\frac{\partial}{\partial y}\right)
            \frac{1}{2}\left(\frac{\partial}{\partial x}-i\frac{\partial}{\partial y}\right)=
            \frac{1}{4}\left(\frac{\partial^2}{\partial x^2}+\frac{\partial^2}{\partial y^2}\right)
        \end{align*}
        И если мы этот оператор применим к норме в квадрате от голоморфной функции,
        то мы получим
        \[\frac{\partial^2}{\partial\overline z\partial z}(f\overline f)=
        \frac{\partial}{\partial\overline z}(\frac{\partial}{\partial z}f\overline f+
        f\frac{\partial}{\partial z}\overline f)=
        \frac{\partial}{\partial\overline z}(\frac{\partial}{\partial z}f\overline f)=
        \frac{\partial}{\partial z}f\frac{\partial}{\partial\overline z}\overline f=
        \left|\frac{\partial}{\partial z}f\right|^2\]
        Откуда мы можем переписать лапласиан, мы будем писать лапласиан, чтобы не
        вводить новых обозначений
        \[\Delta=4\sum_l\frac{\partial^2}{\partial\overline z_l\partial z_l}\]
        Применим лапласиан к нашему равенству и получим
        \[\sum_{i,j} |\partial_{z_i} f_j|^2=0\]
        Откуда мы заключаем, что $f_j=\text{const}$.
        
    \item \textbf{Пусть $\Omega\subset\mathbb C^n$ –область, а $f\in \mathcal O(\Omega)$
        - непостоянная голоморфная функция. Докажите, что множество $f(\Omega)⊂C$ открыто.}

        Для каждой точки $x_0\in\Omega$ мы можем выбрать её окрестность-диск $D\subset\Omega$.
        $f$ не может быть постоянной по всем направлениям, так как тогда она постоянна в
        $U$, а значит и на всем $U$. Тогда мы можем выбрать направдение $v$, в
        котором она не постоянна. Тогда $g(t)=f(x_0+tv)$ является непостоянной 
        комплексной функцией одного переменного, определенной на открытом диске
        $D'$. Тогда по теореме комплексного анализа, $g$ – открытая функция и 
        $g(D')\subset\Omega\subset\mathbb C$ – открыто, а тогда открыто и само $\Omega$,
        так как вместе с каждой своей точкой содержит и её какую-нибудь
        окрестность.

    \item \textbf{Пусть $\Omega\subset\mathbb C^n$ – область, а $f\in\mathcal O(\Omega)$
        – непостоянная голоморфная функция. Докажите, что $\Omega\setminus\{f=0\}$
        открыто и связно.}

        Докажем открытость. Так как $f$ непрерывно, то $f^{-1}(0)$ замкнуто, а
        значит $\Omega\setminus f^{-1}(0)=\Omega\cap f^{-1}(0)^c$ открыто как
        пересечение открытых.

        Докжем связность

    \item \textbf{Докажите, что если последовательность органиченных 
        голоморфных функций $(f_j)$ на $\Omega$ содержит сходящуюся 
        подпоследовательность.}

    \item \textbf{Пусть $\Omega\subset\mathbb C^n$, а $f:\Omega\rightarrow\mathbb
        C^n$ – голоморфное отображение. Докажите, что $\det J_{\mathbb R}(f)=
        |\det J_{\mathbb C}(f)|^2$.}

        Очевидно, что $J_{\mathbb R}(f)=J_{\mathbb C}(f)_{\mathbb R}$, так как 
        мы просто переписывеам комплексный оператор над действительными числами, 
        то есть рацианолизируем его, a тогда можно просто применить задачу $1.1$.

    \item \textbf{Пусть $f:\Omega_1\rightarrow\Omega_2$ – голоморфная биекция,
        где $\Omega_i\subset\mathbb C^n$ – ограниченные области. Докажите, что
        обратное отображение $f^{−1}$ голоморфно.}

    \item \textbf{Пусть $\psi=\psi_{\overline k}d\overline z^k$ – гладкая $(0,1)$
        -форма с компактным носителем в $\mathbb C^n$, $n>1$, удовлетворяющая
        уравнению $\overline ∂\psi=0$. Положим
        \[
            u_{\overline k} := \frac{1}{2πi}\int_{\mathbb C}\frac{\psi_{\overline k}
            (z^1,\ldots,z^{k-1},\zeta,z^{k+1},z}{\zeta-z^k}d\zeta\wedge d\overline\zeta
        \]
    Покажите, что $u_{\overline k}=u_{\overline m} = u$ для всех $1\leq k,m\leq n$.
    Покажите, что $\overline\partial u=\psi$ и $u$ имеет компактный носитель.}

\end{enumerate}
\end{document}
