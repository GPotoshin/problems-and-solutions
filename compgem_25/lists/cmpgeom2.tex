\documentclass{article}
\usepackage[a4paper,left=3cm,right=3cm,top=1cm,bottom=2cm]{geometry}
\usepackage{amsmath}
\usepackage{amssymb}
\usepackage{stmaryrd}
\usepackage{hyperref}
\usepackage[russian]{babel}

\usepackage{tikz-cd}
\usepackage{array}
\usepackage{graphicx}
\setlength{\parindent}{0mm}

\usepackage{fontspec}
\setmainfont{Linux Libertine O}
\usepackage{unicode-math}
\setmathfont{Cambria Math}

\title{
\textit{\small{Георгий Потошин, 2025}}\\
\vspace{0.3ex}
\textit{\huge{Комплексная геометрия, листочек 2}}\vspace{1ex}
}

\date{\vspace{-10ex}}

\begin{document}
\maketitle

\section{Голоморфные функции. Подмногообразия и области в $\mathbb C^n$.}
\begin{enumerate}
    \item \textbf{Докажите, что если $X$ – связное и компактное комплексное
        подмногообразие в $\mathbb C^n$, то $X$ – точка.}

        Пусть $p_i=\pi_i|_M:(z_1,\ldots,z_n)\rightarrow z_i$ – голоморфная, а значит
        непрерывная функция на $M$. Тогда $|p_i|$ достигает максимума $m_i$ на
        компакте $M$ в точке $t_i\in M$. Так как $M$ компакт, то на нем есть
        конечный атлас $(\Omega_i,\tau_i)_i$. И пусть $t_i\in\Omega_i$. Тогда мы
        имеем $\tau:\Omega_i\rightarrow U_i$. $p_i\circ\tau_i^{-1}$ имеет
        максимум на области $U_i\in\mathbb C^d$ (мы всегда можем сделать атлас
        из областей). А тогда $p_i\circ\tau_i^{-1}=\text{const}$, то есть $p_i=
        \text{const}$ на $\Omega_i$. Тогда по связности мы получим, что и в каждой
        соседней карте $p_i$ достигает максимума, а значит там он тоже постянен.
        Продолжая по индукции и по связности, $p_i$ оказывается постоянным на
        всем многообразии, и так как это верно для всех $i$, то многообразие – точка.

    \item \textbf{Докажите теорему Римана о продолжении: пусть $P=\{z\in
        \mathbb C^n | |z_j| < ε, j\in\llbracket1;n\rrbracket\}$ – полидиск, а
        $f$ – функция, голоморфная и ограниченая в $P\setminus\{z_1=0\}$. 
        Докажите, что $f$ продолжается до голоморфной функции в $P$.}

        Пусть $(z_1,z')=(z_1,\ldots,z_n)\in\mathbb C^n$. Посмотрим на
        подпространство $(\mathbb C, z')$. В нем функция $f$ ограничена и голоморфна
        в выколотом диске, а значит, как известно из комплексного анализа,
        мы можем продолжить функцию до голоморфной и её значение в выколотой
        точке может быть записано как
        \[\overline f(0,z')=\frac{1}{2\pi i}\oint_{|\zeta|=\varepsilon/2}\frac{
        f(\zeta,z')}{\zeta}d\zeta\]
        По построению это продолжение голоморфно по первой координате, и так
        как функция и её производная ограничены на компакте, и выражение под
        интегралом голоморфно, то мы можем переставить частные производные и
        получить голоморфность по остальным координатам, а значит наша функция
        голоморфна по теорема Хартогса.

    \item \textbf{Докажите,чтоединичный шар $B=\{z∈\mathbb C^n\;|\;\|z\|=1\}$
        и единичный полидиск $P=\{z∈\mathbb C^n\;|\;|z_j|<1,j=1,\ldots,n\}$ не
        биголоморфны.}

        Идея доказательства основывется на двух теоремах, доказанных Картаном,
        которые я нашел в книге \textit{Walter Rudin, Function Theory on the
        Unit Ball of $\mathbb C^n$} на страницах 23-24.

        \textbf{Теорема 1:} \textit{Пусть
        \begin{enumerate}
            \item $\Omega$ – ограниченая область в $\mathbb C^n$
            \item $F:\Omega\rightarrow\Omega$ – голоморфна
            \item для некоторого $p\in\Omega$, $F(p)=p$ и $F'(p)=I$
        \end{enumerate}
        Тогда $F(z)=z$ для всех $z\in\Omega$.}

        \textbf{Доказательство:} Без потери общности, можно считать, что $p=0$.
        И мы можем найти шары $r_1B\subseteq\Omega\subseteq r_2B$. Тогда в шаре
        $r_1B$ мы имеем разложение $F$ в ряд, где вычисление $n$-формы на
        приращении мы будем записывать через $F_n$. Тогда ряд будет следующим
        \[F(z)=z+\sum_{s>1}F_s(z)\]
        Пусть $F^k$ обозначает $k$-ую композици. Тогда мы будем доказываеть
        дальше по индукции для $m\geq 2$ следующий факт, что для $2\leq s<m$ $F_s=0$.
        
        Для $m=2$ утверждение тривиально.

        Пусть утверждение верно для некого $m$. Тогда посмотрим на ряд $F^k$.
        Из композиции рядов очевидно, что в этом ряде слагаемые степени $2\leq s<m$
        нулевые. Осталось посчитать слагаемое степени $m$. Его легко получить опять
        из индукции, так как для степени 2, например мы получим
        \[F(F(z))=F(z)+F_m(F(z))+\ldots = (z+F_m(z))+F_m(z)+\ldots=z+2F_m(z)\]
        a общем случае разложение будет начинаться следующим образом
        \[F^k(z)=z+kF_m(z)+\ldots\]
        Тогда мы можем посчитать следующий интегра в шаре
        \[\frac{1}{2\pi}\int_{-\pi}^\pi F^k(e^{i\theta}z)e^{-im\theta}d\theta=kF_m(z)\]
        Он будет иметь именное такое значение, так как только мономы степени $m$
        дадут ненулевое значение. Так как $F^k$ по условию ограничена, то
        используя неравенство на интеграл, мы получим $k|F_m(z)|<r_2$ для всех 
        натуральных $k$, а значит $F_m=0$ в шаре $r_1B$. А значит мы доказали
        гипотезу для $m+1$.

        В итоге мы получаем, что $F(z)=z$ на шаре $r_1B$, а значит и на всем $\Omega$,
        так как оно связано.
        
        Дальше мы будем называть \emph{круговыми} те подмножества $\mathbb C^n$,
        что замкнуты относительно умножения на $e^{i\theta}$.

        \textbf{Теорема 2:} \textit{Пусть
        \begin{enumerate}
            \item $\Omega_1$ и $\Omega_2$ – круговые область в $\mathbb C^n$, содержащие $0$
            \item $F:\Omega_1\longleftrightarrow\Omega_2$ такой биголоморфизм, что $F(0)=0$
            \item $\Omega_1$ – ограничен
        \end{enumerate}
        Тогда $F$ – линеен.}

        \textbf{Доказательство:} Пусть $G=F^{-1}$ и пусть $A=F'(0)$. Так как
        $G(F(z))=z$, то $G'(0)A=I$, а значит $G'(0)=A^{-1}$. Для фиксированного
        $\theta$ положим $H(z)=G(e^{-i\theta}F(e^{i\theta}))$. Так как области
        круговые, то $H:\Omega_1\rightarrow\Omega_1$ корректно определен и
        голоморфен, $H(0)=0$ и $H'(0)=I$. Применяя предыдущую теорему мы
        получаем $H(z)=Z$, а значит мы имеем
        \[F(e^{i\theta}z)=e^{i\theta}F(z)\]
        Тогда перменив интегрирование из предыдущей теоремы
        \[\frac{1}{2\pi}\int_{-\pi}^\pi F(e^{i\theta}z)e^{-im\theta}d\theta=F_m(z)\]
        мы получим $F_m(z)=\frac{1}{2\pi}\int_{-\pi}^\pi F(z)e^{-i(m-1)\theta}d\theta$,
        то есть единственным ненулевым слагаемым будет линейное, и по связности
        мы получим линейность $F$.

        Теперь давайте перейдем к решению задачи. Предположим, что у нас есть
        биголоморфизм $f:B\rightarrow P$, Тогда $f(0)=(a_1,\ldots,a_n)$. Мы
        можем найти автоморфизм полидиска, который известен с курса геометрии
        \[g(z)=\left(\frac{z_i-a_i}{1-\overline a_iz_i}\right)_i\]
        Тогда $g\circ f:B\rightarrow P$ – биголоморфизм между ограниченными круговымы
        областями и он сохраняет $0$, а значит он по 2 теореме линеен, но такого
        не может быть, так как очевидно, что нельзя сферу линейено преобразовать
        в границу полидиска.

    \item \textbf{Пусть $f_d(z):=z^d_1+\ldots+z^d)n$, $d\in\mathbb N, d\geq 2$.
        \begin{enumerate}
            \item Докажите, что множество $V_{d,c}:=\{z\in\mathbb C^n\;|\;f_d(z)=c\}$
                гладко при $c\neq0$.
            \item  Докажите что $V_{2,1}$ диффеоморфно $TS^{n−1}$ – касательному
                расслоению сферы $S^{n−1}=\{x\in\mathbb R^n\;|\;\|x\|=1\}$.
        \end{enumerate}}
        \begin{enumerate}
            \item Мы имеем следующий набор эквивалентных утверждений
                \begin{align*}
                    f_d'(z)=0\\
                    \forall i, \partial_if_d(z)=0\\
                    \forall i, dz_i^{d-1}=0\\
                    \forall i, z_i=0\\
                \end{align*}
                а значит $V_{d,c}$ гладко при $c\neq0$
                
            \item Пусть у нас будут следующие координат $(x_j+iy_j)$ на $C^n$.
                Тогда $V_{2,1}$ задаётся уравнением
                \begin{align*}
                    \sum(x_j+iy_j)^2=1\\
                    \left\{
                        \begin{array}{rcl}
                            \sum x_i^2-\sum y_i^2=1\\
                            \sum 2x_iy_i=0
                        \end{array}\right.\\
                    \left\{
                        \begin{array}{rcl}
                            \|x\|^2-\|y\|^2=1\\
                            \langle x,y\rangle=0
                        \end{array}\right.
                \end{align*}

                Расслоение имеет гладкую структуру наследованныю из вложения
                \[TS^{n-1}=\coprod_{p\in S^{n-1}}\{p\}\times T_pS^{n-1}\subseteq\mathbb R^n\times\mathbb R^n\]
                Тогда диффеоморфизм можно задать следующими отображенияеми
                \begin{align*}
                    p(x,y)=\frac{x}{\|x\|}\\
                    v(x,y)=y
                \end{align*}
                обратное ему будет задаваться
                \begin{align*}
                    x(p,v)=p\sqrt{\|v\|^2+1}\\
                    y(x,y)=v
                \end{align*}
                Легко видеть, что отображения корректно заданы, взаимообратны и
                $\mathcal C^\infty$.
        \end{enumerate}

    \item \textbf{Докажите локальную $\frac{\sqrt{−1}}{2\pi}\partial\overline\partial$-лемму.
        Пусть в полидиске задана форма $\alpha$ типа $(p,q)$, где $p,q\geq 1$.
        Если $dα=0$, то найдется (возможно в меньшем полидиске) форма $\beta$
        типа $(p−1,q−1)$, такая, что $\alpha=\frac{\sqrt{−1}}{2π}\partial\overline
        \partial\beta$.}

        Здесь мы будем использовать лемму Дольбо-Гротендика. Мы будем считать
        её известной, её доказательство можно найти на странице 28 
        \textit{Jean-Pierre Demaily, Complex Analytic and Differential Geometry}.

        \textbf{Лемма Дольбо-Гротендика:} \textit{Пусть $\Omega\subseteq\mathbb R^m$
        - окрестность нуля в $\mathbb C^n$ и $v\in\Gamma(\bigwedge^{p,q}T_\Omega)$,
        такое что $\overline\partial v=0$. Тогда если $q\geq 1$, то есть окрестность
        $\omega\in\Omega$ нуля и форма $u\in\Gamma(\bigwedge^{p,q-1}T_\Omega)$,
        такое, что $\overline\partial u=v$ на $\omega$.
        }

        \textbf{Следствие:} \textit{Так как у нас есть сопряжение и мы имеем $\overline{
        \partial a}=\overline\partial\overline a$, то мы имеем также аналогичную
        $\partial$-Пуанкаре Лемму.}

        Так как $d\alpha=\partial\alpha+\overline\partial\alpha=0$ и $\alpha$
        имеет тип $(p,q)$, то $\partial\alpha\in\Gamma(\bigwedge^{p+1,q}T_P)$
        и $\overline\partial\alpha\in\Gamma(\bigwedge^{p,q+1}T_P)$. А значит
        $\partial\alpha=\overline\partial\alpha=0$.

        Теперь мы воспользуемся леммой Дольбо-Гротендика, и найдём $\gamma\in
        \Gamma(\bigwedge^{p,q-1}T_P)$, что $\alpha=\overline\partial\gamma$.        
\end{enumerate}

\section{Почти комплексные структуры}
\begin{enumerate}
    \item \textbf{Прямым вычислением покажите, что $N(X,Y)$ действительно
        является тензором.}

        Нам нужно проверить, что для любой гладкой функции $f$ на $M$ мы имеем
        $N(fX,Y)=fN(X,Y)$, $N(X,Y)=N(X,fY)$, так как пропускание сумм очевидно. Так
        как тензор давольно симметричный, то линейность по каждой компоненте
        доказывается схожим способом, а поэтому мы проверим её только для первой
        координаты.
        \begin{align*}
            N(fX,Y)&=[fX,Y]+J([J(fX),Y]+[fX,JY])-[J(fX),JY]\\
            &=[fX,Y]+J([fJX),Y]+[fX,JY])-[fJX,JY]\\
            &=\{f[X,Y]−(Y(f))X\}+J(\{f[JX,Y]−(Y(f))(JX)\}+\{f[X,JY]−((JY)(f))X\})\\
            &-\{f[JX,JY]−((JY)(f))(JX)\}\\
            &=f[X,Y]+fJ([J(fX),Y]+[fX,JY])+f[JX,JY]\\
            &-(Y(f))X+(Y(f))X+(Y(f))X-(Y(f))X\\
            &=f([fX,Y]+J([J(fX),Y]+[fX,JY])-[J(fX),JY])\\
            &=fN(X,Y)
        \end{align*}
    \item \textbf{Пусть $Z_1,Z_2$ – векторные поля типа $(1,0)$, а $π^{0,1}$ –
        проекция на векторные поля типа $(0,1)$. Покажите, что для любой функции
        $f\in\mathcal C^\infty(M)$ верно следующее:
        \begin{align*}
            \pi^{0,1}([fZ_1,Z_2]) =f\pi^{0,1}([Z_1,Z_2])\\
            \pi^{0,1}([Z_1,fZ_2]) =f\pi^{0,1}([Z_1,Z_2])\\
        \end{align*}
        }

        Мы опять докажем только первое равенство, так как второе доказывается схожим
        способом, так как скобка антикоммутирует.

        \begin{align*}
            \pi^{0,1}([fZ_1,Z_2])=\pi^{0,1}(f[Z_1,Z_2]-(Z_2f)Z_1)=f\pi^{0,1}
            ([Z_1,Z_2])-(Z_2f)\pi^{0,1}(Z_1)=f\pi^{0,1}([Z_1,Z_2])
        \end{align*}
        Так как $Z_2\in T^{1,0}M$.

    \item \textbf{Пусть $Z_X=X−iJX$, а $Z_Y=Y−iJY$ – векторные поля типа $(1,0)$.
        Покажите, что
        \[2\pi^{0,1}([Z_X,Z_Y])=N(X,Y)+iJN(X,Y)\]}

        Заметим, что $\pi^{0,1}(X)=\frac{1}{2}(X+iJX)$. Тогда
        \begin{align*}
            2\pi^{0,1}([Z_X, Z_Y])&=[Z_X, Z_Y]+iJ[Z_X, Z_Y]\\
            &=[X−iJX,Y−iJY]+iJ[X−iJX,Y−iJY]\\
            &=[X,Y]-i[JX,Y]-i[X,JY]-[JX,JY]+iJ[X,Y]+J[JX,J]+J[X,JY]-iJ[JX,JY]\\
            &=([X,Y]+J[JX,J]+J[X,JY]-[JX,JY])+iJ([X,Y]+J([JX,Y]+[X,JY])-[JX,JY])\\
            &=N(X,Y)+iJN(X,Y)
        \end{align*}

    \item \textbf{Пусть $\alpha$ является $(1,0)$-формой на почти комплексном
        многообразии $(M,J)$. Докажите, что для $(d\alpha)^{2,0}$ и любых
        векторных полей $Z_1,Z_2$ типа $(0,1)$ верна следующая формула:
        \[(d\alpha)^{2,0}(Z_1,Z_2)=−\alpha(N(Z_1,Z_2))\]}
\end{enumerate}
\section{Комплексные многообразия}
\begin{enumerate}
    \item \textbf{Докажите, что вложение Веронезе и вложение Сегре действительно
        являются вложениями.}

        \textbf{Вложение Веронезе.} $v_d:\mathbb CP^n\rightarrow\mathbb CP^N$,
        ассоциирует точке $[x_0:\ldots:x_n]$ точку c координатами – всеми мономами
        степини $d$ над $x_0,\ldots,x_n$. Очевидно, оно кооректно определено, так
        как домножение координат на $\lambda$ переводиться в домножение координат
        на $\lambda^d$. Отображение голоморфно, так как координаты – полиномы.
        Проверим инъективность. Пусть $v_d([x_0:\ldots:x_n])=v_d([y_0:\ldots:y_n])$.
        Без потери общности можно считать, что $x_0=y_0=1$, так как хотя бы по одной
        координате в образе типа $x_i^d$ у нас будет не нуль. Тогда мы имеем $\lambda$
        – коэффициент перехода, и он обязан быть равен $1$, так как $x_0=\lambda y_0$.
        Тогда их $x_0^{d-1}x_i=y_0^{d-1}y_i$, следует, что $y_i=x_i$, а значит
        мы имеем вложение. Так как вложение голоморфно, то оно является
        биголоморфизм на свой образ.
        
        \textbf{Вложение Сегре.} $\sigma:\mathbb CP^n\times\mathbb CP^m
        \rightarrow\mathbb CP^{(n+1)(m+1)−1}=([x_i]_i,[y_j]_j)\mapsto[x_iy_j]_{i,j}$.
        Оно очевидно голоморфно. Проверим инъективность, пусть $\sigma([x_i],[y_i])
        =\sigma([x_i'],[y_i'])$. Опять же без потери общности по тому же аргументу
        мы можем предположить, что $x_0=y_0=1=x_0'=y_0'$. Тогда опять для $x_0y_0
        =\lambda x_0'y_0'$, мы получаем $\lambda=1$, а тогда $x_0y_i=x_0'y_i'$
        влечет $y_i=y_i'$. Аналогичны получаем $x_i=x_i'$. Так как отображение
        голоморфно и инъективно, то мы получаем биголоморфизм на образ.

    \item \textbf{Докажите, что вложение Плюккера также является вложением.}

        Пусть $\psi:\text{Gr}(k,n)\rightarrow P(\bigwedge^k\mathbb C^n):
        \bigoplus_{i=1}^k v_i\mathbb C\rightarrow \bigwedge_{i=1}^k v_k$.
        Для выбранного базиcа $\mathbb C^n$ мы можем записать координты
        плоскости через матрицу размером $k\times n$, чьи строки – координаты
        базиса плоскости, координаты $P(\bigwedge^k\mathbb C^n)$ также являются
        матрицами координат, участвующих во внешнем произведении, так что
        отображение голоморфно, и так как внешняя степень $k$ над пространство
        размерности $k$ одномерна, то отображение определено корректно. Теперь
        пусть $\psi(W)=\psi(W')$, то есть $v_1\wedge...\wedge v_n=\lambda w_i\wedge...
        \wedge w_n\neq 0$. Но если мы домножим это равенство на $v_i$ слева, то
        мы получим $0$, а значит $W\subseteq W'$ и так как у нас есть равенство
        размерностей, то $W=W'$. Опять же это отображение является биголоморфизмом
        на свой образ.

\end{enumerate}
\section{Пучки}
\begin{enumerate}
    \item \textbf{Пусть $F$ – пучок на многообразии $M$, $C^p(\underline U,F)$ –
        $p$-коцепи. Проверьте, что кограницы являются коциклами, т.е. что оператор
        \[\delta:C^p(U,F)→C^{p+1}(\underline U,F)\]
        удовлетворяет тождеству $\delta^2=0$.}

    Пусть $c\in C^p(\underline U,F)$. Давайте проверим, что $\delta^2c=0$.
    \begin{align*}
        (\delta^2 c)_{i_0...i_{p+2}}&=\sum_{j=0}^{p+2}(-1)^j(\delta c)_{i_0...\hat i_j...i_{p+2}}|_{U_{i_0...i_{p+2}}}\\
        &=\sum_{j=0}^{p+2}(-1)^j\left(\sum_{k=0}^{j-1}(-1)^kc_{i_0...\hat i_k...\hat i_j..i_{p+2}}+
        \sum_{k=j+1}^{p+2}(-1)^{k-1}c_{i_0...\hat i_j...\hat i_k..i_{p+2}}\right)\\
        &=\sum_{j=0}^{p+2}\sum_{k=0}^{j-1}(-1)^{j+k}c_{i_0...\hat i_k...\hat i_j..i_{p+2}}+
        \sum_{j=0}^{p+2}\sum_{k=j+1}^{p+2}(-1)^{j+k-1}c_{i_0...\hat i_j...\hat i_k..i_{p+2}}=0
    \end{align*}
\end{enumerate}
\end{document}
