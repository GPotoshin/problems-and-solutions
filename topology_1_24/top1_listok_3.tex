\documentclass{article}
\usepackage[a4paper,left=3cm,right=3cm,top=1cm,bottom=2cm]{geometry}
\usepackage{amsmath}
\usepackage{amssymb}
\usepackage{hyperref}

\setlength{\parindent}{0mm}

\usepackage{fontspec}
\setmainfont{Linux Libertine O}
\usepackage{unicode-math}
\setmathfont{Cambria Math}

\title{
\textit{\small{Георгий Потошин, 2024}}\\
\vspace{0.3ex}
\textit{\huge{Топология I, листочек 3}}\vspace{1ex}
}

\date{\vspace{-10ex}}

\begin{document}
\maketitle

\begin{enumerate}
    \item \textbf{Докажите, что $\mathbb{R}/\mathbb{Z}\simeq S^1.$}\par
        \textbf{Утверждение 1. Элементы базы топологии на $X$ после
        индуцирования на $Y\subseteq X$ образуют базу топологии на $Y$.}\par
        По определению элемент базы останется открытым после индуцирования.
        Покажем теперь, что все индуцированные элементы базы составят
        базу. Пусть $U\subseteq Y$ – открытое множество. Тогда существует такое
        открытое $V\subseteq X$, что $V\cap Y=U$. Раз $V$ открыто, то
        существуют элементы базы $B_i\in \tau_X,i\in I$, что $\bigcup_{i\in I}
        B_i = V$. Тогда $U=V\cap Y=Y\cap\bigcup_{i\in I}B_i=\bigcup_{i\in I}Y\cap
        B_i$ открытое множество представимо как объединение индуцированных
        элементов базы на топологии $X$, а значит, что множество всех таких
        индуцированных элементов составят базу топологии на $Y$.\par

        \textbf{Утверждение 2. Если в топологии пространства $X/\sim$ образ
        элемента базы топологии на $X$ при канонической проекции открыт, то
        объединение этих образов составит базу топологии на фактор пространства.}
        \par Пусть $U\in X/\sim$ открыто, тогда $\pi_{\sim}^{-1}[U]$ открыто и
        представимо как $\bigcup_{i\in I}B_i$ где $B_i$ – элемент базы топологии
        на $X$. Тогда $U=\pi_{\sim}[\bigcup_{i\in I}B_i]=\bigcup_{i\in I}\pi_\sim
        [B_i]$. А значит образ элементов базы топологии на $X$ составит базу
        топологии на фактор пространстве.\par

        \textbf{Утверждение 3. Если биекция $X\longrightarrow Y$ переводит
        элементы базы в открытые множества и прообразами элементов базы тоже
        являются открытые множества, то биекция является гомеоморфизмом.}\par
        
        Пусть $U\subseteq X$ открыто, тогда существуют такие элементы базы
        $B_i\subseteq X,i\in I$, что $U=\bigcup_{i\in I}B_i$. Тогда $f^{-1}[U]=
        \bigcup_{i\in I}f^{-1}[B_i]$ – объединение открытых, а значит само
        открыто и $f$ непрерывно. В обратную сторону доказывается также.

        Базой пространства $S^1$ являются всевозможные пересечения окружности
        и открытых кругов, то есть открытые дуги. Найдем теперь базу
        пространства $\mathbb{R}/\mathbb{Z}$. Пусть $(a,b)$ – элемент базы
        топологии на $\mathbb{R}$. Прообраз образа этого интервала равен
        $\bigcup_{n\in\mathbb{Z}}(a+n,b+n)$ и открыт, а значит образы
        интервалов составят базу топологии на фактор пространстве. Если классы
        эквивалентности отождествить с точками на $[0,1)$, то образом
        интервалa $(a,b)$ будет $(\{a\},\{b\})$, если изначальный интервал не
        содержал целых точек, $[0,\{b\})\cup(\{a\},1)$, если изначальный
        интервал содержал 1 целую точку и $[0,1)$, если изначальный интервал
        содержал 2 и более целые точки. Пусть $f:[x]\mapsto e^{i2\pi\{x\}}$
        биекция из $\mathbb{R}/\mathbb{Z}$ в $S^1$. Тогда очевидно, что она
        однозначно сопоставляет элементам базы топологии на фактор пространстве
        открытые дуги, а значит пространства гомеоморфны.

    \item \textbf{Докажите, что $\mathbb{D}^n/S^{n−1}\simeq S^n$.}\par

        Пусть $I=(-1,1)$ интервал. Тогда положим $B^n=I^n$, $\mathbb{D}^n=
        \overline{B^n}$ и $S^n=\partial\mathbb{D}^{n+1}$. $\mathbb{D}^n/S^{n-1}$ –
        это диск в котором все точки его границы положили в один класс.
        Построим сюръекцию из диска в шар, что уважает это отождествление:
        Пусть $x=(x_1,...,x_n)$ и пусть $|x|=max_i|x_i|$, тогда
        \[f(x)=
        \begin{cases}
            (-1,4x_1,...,4x_n) &,0\leqslant|x|<1/4\\
            (4|x|-2,x_0/|x|,...,x_n/|x|) &,1/4\leqslant|x|\leqslant 3/4\\
            (1,4(1-|x|)\frac{4}{3}x_0,4(1-|x|)\frac{4}{3}x_n) &,3/4<|x|\leqslant1\\
        \end{cases}\]

        Сюръекция f непрерывно, так как непрерывна каждая композиция $pr_i
        \circ f$. Теперь если объединить все точки $\partial\mathbb{D}^n$ в
        один класс, то $f:\mathbb{D}^n/\partial\mathbb{D}^n\longrightarrow S^n$
        станет биекцией. Причем прообраз открытого не содержащего $f(\partial
        \mathbb{D}^n)$ будет открытым, потому как факторизация ничего не
        поменяла, а прообраз открытого, содержащего эту точку, был открыт до
        факторизации и содержал границу, а значит останется открытым и после
        факторизации. Заметим также, что $f$ – это биекция из компактного
        пространства в хаусдорфово, а значит является гомеоморфизмом.

    \item \textbf{Верно ли, что фактор хаусдорфова пространства является
        хаусдорфовым? Регулярного – регулярным? Нормального – нормальным?}\par
        Возьмём отрезок $[0,1]$ с канонической топологией. Он компактен и
        хаусдорфов, а значит нормален и регулярен. Профакторизуем его так, что
        его внутренность попадёт в один класс эквивалентности, $0$ в другой,
        а $1$ в третий обозначим их за $i,0,1$ соответственно. Тогда из всех
        подмножеств только $\varnothing, \{i\},\{0, i\},\{1,i\},\{0,1,i\}$
        будут открытыми. Заметим, что  $\{0\}$ и $\{1\}$ будут замкнутыми в
        такой топологии, но при этом у этих синглтонов нет непересекающихся
        окрестностей, а значит, что полученное фактор пространства ни
        хаусдорфово, ни регулярно, ни нормально. Тогда ответ на все вопросы – нет.

    \item \textbf{Приведите пример хаусдорфова нерегулярного топологического
        пространства.}\par
        Положим $K=\{1/n|n\in\mathbb{N}\}$. Это множество не открыто в
        стандартной топологии прямой $\mathbb{R}$, так как любая окрестность
        $1$ не лежит в $K$. C другой стороны оно не замкнуто, так как не
        содержит предельную точку $0$. Возьмём множество $S$ всех
        интервалов вместе со всеми интервалами без $K$. Оно покрывает прямую и
        пересечение двух элементов либо интервал, либо интервал без $K$, а
        значит $S$ – база некой топологии, в которой открытые множества - это
        канонические открытые множества без некого подмножества в $K$. Это
        означает, что любая окрестность $0$ содержит отрезок без некого
        количества элементов из $K$. Тогда между границами этого отрезка лежит
        некое число вида $1/p$ и любая окрестность $K$ будет содержать шар
        радиусом меньшим $1/p-1/(p-1)$ вокруг $1/p$ и $1/p$ содержащий. Тогда
        этот шар пересекается с $K$ только по своему центру, а значит это
        шар в привычном нам смысле. Тогда он пересекается с изначальной
        окрестностью $0$. В итоге у $K$ и $0$ нет непересекающихся окрестностей
        и $K$ очевидно замкнуто, а занчит прямая с этой топологией не регулярна,
        но хаусдорфова, так как тоньше стандартной хаусдорфовой топологии.

    \item \textbf{Приведите пример регулярного ненормального топологического
        пространства.}\par
        \textbf{Топология стрелки.} Возьмём прямую $\mathbb{R}$ и снабдим её
        топологией, базой которой являются полуинтервалы вида $[a,b)$. Это
        семейство и вправду является базой, так как если пересечение 2 её
        элементов непусто, то оно тоже будет правым полуинтервалом, а также
        семейство покрывает всё пространство. Назовем получившуюся топологию
        $\tau_l$, а пространства $\mathbb{R}_l:=(\mathbb{R},\tau_l)$. Нетрудно
        видеть, что если $U$ открыто в евклидовом смысле, то вместе с каждой
        своей точкой $x$ $U$ будет содержать некий шар $(x-d,x+d)$, а значит
        содержит и полуинтервал $[x,x+d)$, тогда $U$ открыто в топологии
        стрелки. Это означает что топология стрелки тоньше евклидовой топологии.
        Тогда $\mathbb{R}_l$ хаусдорфово. Пусть теперь $A,B\subseteq
        \mathbb{R}_l$ замкнуты
        и дизъюнктивны. Заметим, что $A\subseteq\mathbb{R}\backslash B$ и $B
        \subseteq\mathbb{R}\backslash A$ открытые окрестности соответствующий
        множеств. Тогда вместе с каждым $a\in A$ есть полуинтервал $U_a:=[a,a+
        d_a),d_a>0$ лежащий в $\mathbb{R}\backslash B$, и вместе с каждым $b\in
        B$ есть полуинтервал $V_b:=[b,b+d_b), d_b>0$ лежащий в $\mathbb{R}
        \backslash A$. Положим $V=\bigcup_B V_b$ и $U=\bigcup_A V_a$, они
        являются открытыми окрестностями $B$ и $A$ соответственно. Покажем
        теперь, что их пересечение пусто. Если $V_b\cap U_a\neq\varnothing$,
        то тогда их пересечение содержит $\max(a,b)$, пусть без потери
        общности $a\in V_b\cap U_a\subseteq V_b\subset \mathbb{R}\backslash A$,
        что есть противоречие. Тогда мы нашли непересекающиеся окрестности 2
        произвольных замкнутых множеств, а значит пространство $\mathbb{R}_l$
        хаусдорфово, регулярно и нормально.

        \textbf{Утверждение 4. Подпространство регулярного пространство
        регулярно.}

        Пусть $X$ – регулярно и $A\subseteq X$. Пусть $a\in A$ точка и $F\subset
        A$ замкнутое, что не содержит $a$. Тогда существует замкнутое $C\subset
        X$, что $F=C\cap A$. $С$ не содержит точки  $a$, а значит по
        регулярности $X$ существуют открытые $C\subseteq U$, $a\in V$, что
        пересекаются по пустому множеству. Тогда $F\subset U\cap A$ и $a\in V\cap
        A$ – открытые в $A$ окрестности точки и замкнутого её не содержащего,
        что не пересекаются, а значит пространство $A$ регулярно.

        \textbf{Утверждение 5. Пространство $X$ регулярно тогда и только тогда,
        когда для любого открытого $O$ и точки $x$ из $O$ существует открытое
        $U$, что верно соотношение $x\in U\subseteq\overline{U}\subseteq O$.}
        
        $\Rightarrow:$ Пусть $U$ – открытое множество и $x\in U$ – точка в нём.
        Тогда $x\notin U^c$ и $U^c$ – замкнуто. Тогда по регулярности мы найдем
        пару открытых $O_1$ и $O_2$, что $x\in O_1$ и $U^c\subseteq O_2$ и
        $O_1\cap O_2=\varnothing$. Тогда будет иметь место следующее соотношение:
        $x\in O_1\subseteq\overline{O_1}\subseteq O_2^c\subseteq U$.

        $\Leftarrow:$ Для точки $x$ и замкнутого $F$ её не содержащего найдём
        открытое $V$, что $x\in V\subseteq\overline{V}\subseteq F^c$. Тогда
        $x\in V$ и $F\subseteq\overline{V}^c$ и $V\cap\overline{V}^c=\varnothing$,
        а значит пространство регулярно.

        \textbf{Утверждение 6. Пространство $\prod_iX_i=X$ регулярно тогда и
        только тогда, когда всякое $X_i$ регулярно.}

        $\Rightarrow$ Если $\prod_i X_i$ регулярно, то регулярнo $\prod_i Y_i$,
        где для $i=i_0$, $Y_{i_0}=X_{i_0}$, а во всех остальных случаях $Y_i=\{
        x_i\}\subseteq X_i$ это произведение гомеоморфно $X_{i_0}$ и регулярно в
        силу утверждения 4. Тогда $X_{i_0}$ тоже регулярно.

        $\Leftarrow$ Пусть теперь всякое $X_i$ регулярно. Пусть $x=(x_i)\in
        \prod_iX_i=X$. Пусть $x\in U\in\tau$ – открытое множество. Тогда есть
        набор открытых $\{W_i\}$, что $x\in W=\prod_iW_i\subseteq U$. Это в частности
        означает, что $x_i\in W_i$ координата лежит в открытом сомножителе.
        Тогда по регулярности пространства $X_i$ найдется открытое $V_i$, что
        $x_i\in V_i\subseteq\overline{V_i}\subseteq W_i$. Тогда $x\in V=\prod_i
        V_i\subseteq\overline{\prod_i V_i}=\overline{\bigcap_i\text{pr}_i^{-1}
        [V_i]}=\bigcap_i\overline{\text{pr}_i^{-1}[V_i]}\subseteq
        \bigcap_i\text{pr}_i^{-1}[\overline{V_i}]=\prod_i\overline{V_i}\subseteq
        W\subseteq U $. Тогда $x\in V\subseteq\overline{V}\subseteq{U}$ и по
        утвердению 5 произведение пространств будет регулярным.

        Положим теперь $\mathbb{S}:=\mathbb{R}_l\times \mathbb{R}_l$. Это
        пространство называется \emph{прямой Зоргенфрея}. Оно является
        произведением регулярных пространств, а значит само регулярно. Также
        топология этой плоскости тоньше топологии евклидовой плоскости. Тогда
        прямая $D=\{(x,-x)|x\in\mathbb{R}\}$ замкнута в топологии Зоргенфрея.
        Базой $\mathbb{S}$ очевидно являются квадраты $[a,a+d)\times[b,b+d)$.
        Я буду дальше под $\mathbb{P}$ подразумевать $\mathbb{R}\setminus
        \mathbb{Q}$. Тогда множества $Q=\bigcup_{q\in\mathbb{Q}}[q,q+1)\times
        [-q,-q+1)$ и $P=\bigcup_{p\in\mathbb{P}}[p,p+1)\times[-p,-p+1)$ будут
        открытыми, а $D\setminus Q$ и $D\setminus P$ замкнутыми.

        \textbf{Утверждение 4. На вещественно прямой с евклидовой топологией $(
        \mathbb{R},\tau)$ если множество $U$ открыто и всюду полотно, то его
        дополнение $U^c$ счётно.}

        Пусть $U$ открыто. Тогда оно представимо как $U=\coprod_{i=0}^{\infty}
        I_i$ дизъюнктивное объединение интервалов. Их счетное количество,
        так как в каждом можно выбрать по рациональной точке и тем самым задать
        вложение в множество действительных чисел. Определим для целого $z$
        $P_z=[z,z+1]$, и для натурального $n$ $A_{n,z}=\{I_k\cap P_z\,|\,k
        \geqslant 1\,\wedge\,\text{diam}(I_k\cap P_z)>1/n\}$ - множество
        интервалов отрезка и $B_{n,z}=\{S\subseteq P_z\,|\,\forall I\in A_{n,z}
        I\cap S=\varnothing\,\wedge\,S\text{ – отрезок}\,\wedge\,(S\subseteq S'
        \text{ вложено в отрезок}\,\wedge\,\forall J\in A_{n,z}S'\cap J=
        \varnothing\Rightarrow S=S')\}$ – множество отрезков, что лежат между
        интервалами. Очевидно что как $A_{n,z}$, так и $B_{n,z}$ оба конечные.
        Заметим, что по построение $A_{n,z}\subseteq A_{n+1,z}$ и $\bigcup
        A_{n,z}\rightarrow U\cap P_z$, когда $n\rightarrow +\infty$. Это
        означает, что $\bigcup B_{n+1,z}\subseteq \bigcup B_{n,z}$ и $\bigcup
        B_{n,z}\rightarrow U^c\cap P_z$.

    \item \textbf{Приведите пример связного, но не линейно связного
        топологического пространства.}
    Обозначим за $L_n$ отрезок между $(0,0)$ и $(1,1/n)$ в $\mathbb{R}^n$. Он
    связен и открыт.\par

    \textbf{Утверждение 4. Если $C_\alpha\subseteq X$ – связные пространства
    для всяких индексов и $\bigcap_\alpha C_\alpha\neq\varnothing$, то
    $\bigcup_\alpha C_\alpha$ связно.}\par

    Пусть $\bigcap_\alpha C_\alpha\neq\varnothing$, но при этом $\bigcup_\alpha
    C_\alpha=U\sqcup V$, где $U$ и $V$ дизъюнктивные открыты непустые множества.
    Если бы ни одно из $C_\alpha$ не одержало одновременно элементы этих двух
    открытых множеств, то тоже было бы справедливым относительно их непустого
    пересечения и тогда все $C_\alpha$ были бы подмножествами одного из
    открытых, а значит второе открытое множество оказалось бы пустым, что
    противоречит с нашим предположением. Пусть $C_{\alpha_0}$ содержит элементы
    из обоих множеств. Тогда $C_{\alpha_0}=(U\cap C_{\alpha_0})\sqcup (U\cap C_
    {\alpha_0})$ – несвязно, а значит мы вновь пришли к противоречию. Тогда
    $\bigcap_\alpha C_\alpha$ обязано быть связным.

    В нашем случае множества $B=\bigcup_{n=1}^{+\infty}L_n$ и $\overline{B}=B
    \cup([0,1]\times\{0\})$ в силу этого утверждения связны, так как их связные
    части-отрезки пересекаются по $(0,0)$.

    \textbf{Утверждение 5. Если множества $C$ и $\overline{C}$ связны, то и
    всякое лежащее между ними тоже связно.}

    Пусть $C$ и $\overline{C}$ связны и $C\subset X\subset\overline{C}$. Если
    бы $X=U\sqcup V$ было несвязно, то если бы оба имели элементы из $C$, то
    $C=(U\cap C)\sqcup (V\cap C)$ было бы несвязно, что ведёт к противоречию.
    Иначе одно из открытых, пусть без потери общности им будет $V$, полностью
    бы находилось в $\overline{C}\backslash C$. Тогда $\overline{C}\backslash V$
    было бы замкнутым в объемлющем пространстве и содержало бы $C$, а значит
    замыкания не было бы минимальным по включению замкнутым надмножеством $C$,
    что опять ведет к противоречию. В итоге $X$ обязано быть связным.


    \item \textbf{Определите естественную топологию на пространства
        невырожденных матриц $\text{GL}_n(\mathbb{R})$. Является ли оно связным?}
    
    Отождествим $M_n(\mathbb{R})$ с $\mathbb{R}^{n^2}$ с топологией проиведения.
    Тогда топологией на пространстве $\text{GL}_n(\mathbb{R})$ будет
    индуцированная с топологии $M_n(\mathbb{R})$. det будет непрерывным
    отображением $\text{GL}_n(\mathbb{R})$ \textbf{на} $\mathbb{R}\backslash\{0\}$.
    Так как образ пространства при непреывном отображении несвязен, то несвязно и
    само пространство.
        
    \item \textbf{Докажите, что функции расстояния $d_1,d_2,d_\infty$ задают
        структуру метрического пространства на $\mathbb{R}^n$. Нарисуйте открытые
        шары $B_0^1$ в метриках $d_i$ при $n=2$.}

    \item \textbf{Докажите, что топология на $\mathbb{R}^n$, индуцированная
        метриками $d_i$ и выше, совпадает с топологией произведения,
        определённой на лекции.}\par
        Обозначим за $\tau_\infty$ топологию порожденную метрикой $d_\infty$
        и за $\tau_\times$ топологию произведения. Базой $\tau_\infty$ являются
        многомерные кубы, то есть множества вида $r(-1,1)^n+a$, где $r\in
        \mathbb{R}$ и $a\in\mathbb{R}^n$. Базой топологии произведения
        являются всевозможные произведения интервалов. Заметим, что база
        метрической топологии вкладывается в базу топологии произведение, а
        значит $\tau_\infty\subseteq\tau_\times$. Пусть теперь $U=\prod_{i=1}^n
        (a_i,b_i)$ – элемент базы $\tau_\times$. Тогда каждая его точка $x=(x_1
        ,...,x_n)\in U$ лежит вместе с шаром $\min\{|a_i-x_i|i\in\{1,...,n\}\}
        \cap\{|b_i-x_i|i\in\{1,...,n\}\}(-1,1)^n+x$, а значит база топологии
        произведения является семейством открытых множеств из $\tau_\infty$.
        Это означает, что $\tau_\times\subseteq\tau_\infty$, и учитывая прошлое
        утверждение $\tau_\times=\tau_\infty$.\par

        Теперь пусть $S_i=\{x\in\mathbb{R}^n|d_i(x,0)=1\}$. $x\mapsto d_i(x,0)$ -
        это непрерывное отображение в смысле $(\mathbb{R}^n,\tau_\times)
        \longrightarrow(\mathbb{R},\tau_c)$, где $\tau_c$ – каноническая
        топология прямой, так как $d_i(\cdot,0)$ является i-м корнем из суммы
        непрерывный отображений. Тогда исходя из 2 задачи 2 листочка множество 
        $S_i$ замкнуто, так как $(\mathbb{R},\tau_c)$ – хаусдорфово.
        Нетрудно также видеть, что $S_i\subset[-1,1]^n$, подмножество
        произведения компактных по лемме Бореля – Лебега отрезков, что само
        компактно. Тогда $S_i$ – замкнутое подмножество компакта, а значит
        $S_i$ компактно в топологии $\tau_\times$. Очевидно, что $d_\infty(
        \cdot, 0):(\mathbb{R}^n,\tau_\times)\longrightarrow(\mathbb{R},\tau_c)$
        тоже является непрерывным отображением. Тогда $d_\infty(S_i,0)$ – образ
        сферы при непрерывном отображении тоже компактен. Более того, так как
        каноническая топология прямой хаусдорфова, то компактный образ сферы
        замкнут, а значит содержит все свои предельные точки. Теперь так, как
        функция расстояния имеет неотрицательные значения и сфера не содержит
        нуль векторного пространства, то она и не может содержать сколь угодно
        близкие к нулю с точки зрения $d_\infty$ точки, в силу замкнутости
        образа. Это означает, что образ имеет ненулевую нижнюю грань $m>0$,
        то есть минимальное расстояния от нуля до некоторой точки сферы. Тогда
        имеет место следующее соотношения для шаров $B_i(a,r)$ метрики $d_i$.
        $B_\infty(a,rm)\subseteq B_i(a,r)\subseteq B_\infty(a,r)$ для любых
        точек $a$ и радиусов $r$. Это значит, что в любой шар пространства с
        метрикой $d_i$ можно вписать куб и вокруг него же можно описать куб,
        а значит открытые множества одного пространства открыты и в другом.
        Тогда $\tau_i=\tau_\infty=\tau_\times$, что и завершает доказательство.

    \item \textbf{Пусть $X,Y$ – метрические пространства. Определите
        естественную метрику на их произведении $X×Y$.}

        Естественной метрикой будет $d_{X\times Y}=\max(d_X,d_Y)$ максимум из
        расстояний между координатами. Она естественная в том смысле, что
        шар будет произведением шаров равного радиуса, а значит топология
        такого пространства совпадет с топологией произведения.

    \item \textbf{Предположим, что в метрическом пространстве X выполнено
        $B^{\varepsilon_1}_x=B^{\varepsilon_2}_y$ для некоторых точек $x,y$ и
        некоторых $\varepsilon_1,\varepsilon_2>0$. Верно ли, что $x=y,
        \varepsilon_1=\varepsilon_2$?}

        Нет, возьмем отрезок $[0,1]$, любые шары радиусом большим 2 являются
        всем пространством, а значит совпадают, при этом их можно рисовать
        вокруг любых точек.

    \item \textbf{Определим топологию Зариского на $\mathbb{C}^n$ следующим
        образом: замкнутыми множествами назовем множества нулей произвольного
        набора многочленов из $\mathbb{C}[x_1,...,x_n]$. Проверьте, что это
        действительно топология. Является ли она хаусдорфовой? Совпадает ли
        топология Зариского на $\mathbb{C}^2$ с топологией произведения,
        полученной из топологии Зариского на $C$?}
\end{enumerate}

\end{document}
