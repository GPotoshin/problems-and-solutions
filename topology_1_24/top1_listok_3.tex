\documentclass{article}
\usepackage[a4paper,left=3cm,right=3cm,top=1cm,bottom=2cm]{geometry}
\usepackage{amsmath}
\usepackage{amssymb}
\usepackage{hyperref}

\setlength{\parindent}{0mm}

\usepackage{fontspec}
\setmainfont{Linux Libertine O}
\usepackage{unicode-math}
\setmathfont{Cambria Math}

\title{
\textit{\small{Георгий Потошин, 2024}}\\
\vspace{0.3ex}
\textit{\huge{Топология I, листочек 3}}\vspace{1ex}
}

\date{\vspace{-10ex}}

\begin{document}
\maketitle

\begin{enumerate}
    \item \textbf{Докажите, что $\mathbb{R}/\mathbb{Z}\simeq S^1.$}\par
        \textbf{Утверждение 1. Элементы базы топологии на $X$ после
        индуцирования на $Y\subseteq X$ становятся базой топологии на $Y$.}\par
        По определению элемент базы останется открытым после индуцирования.
        Покажем теперь, что все индуцированные элементы базы составят
        базу. Пусть $U\subseteq Y$ – открытое множество. Тогда существует такое
        открытое $V\subseteq X$, что $V\cap Y=U$. Раз $V$ открыто, то
        существуют элементы базы $B_i\in \tau_X,i\in I$, что $\bigcup_{i\in I}
        B_i = V$. Тогда $U=V\cap Y=Y\cap\bigcup_{i\in I}B_i=\bigcup_{i\in I}B_i
        \cap Y$ открытое множество представимо как объединение индуцированных
        элементов базы на топологии $X$, а значит, что множество всех таких
        индуцированных элементов составят базу топологии на $Y$.\par

        \textbf{Утверждение 2. Если в топологии пространства $X/\sim$ образ
        элемента базы топологии на $X$ при канонической проекции открыт, то
        объединение этих образов составит базу топологии на фактор пространства.}
        \par Пусть $U\in X/\sim$ открыто, тогда $\pi_{\sim}^{-1}[U]$ открыто и
        представимо как $\bigcup_{i\in I}B_i$ где $B_i$ – элемент базы топологии
        на $X$. Тогда $U=\pi_{\sim}[\bigcup_{i\in I}B_i]=\bigcup_{i\in I}\pi_{sim}
        [B_i]$. А значит образ элементов базы топологии на $X$ составит базу
        топологии на фактор пространстве.\par

        \textbf{Утверждение 3. Если биекция $X\longrightarrow$ устанавливает
        однозначное соответствие между элементами базы двух пространств, то
        она является гомеоморфизмом.}\par
        Пусть $U\subseteq X$ открыто, тогда существуют такие элементы базы
        $B_i\subseteq X,i\in I$, что $U=\bigcup_{i\in I}B_i$. Тогда $f[U]=
        \bigcup_{i\in I}f[B_i]$ – объединение открытых, а значит само открыто.
        В обратную сторону доказывается также.

        Базой пространства $S^1$ являются всевозможные пересечения окружности
        и открытых кругов, то есть открытые дуги. Найдем теперь базу
        пространства $\mathbb{R}/\mathbb{Z}$. Пусть $(a,b)$ – элемент базы
        топологии на $\mathbb{R}$. Прообраз образа этого интервала равен
        $\bigcup_{n\in\mathbb{Z}}(a+n,b+n)$ и открыт, а значит образы
        интервалов составят базу топологии на фактор пространстве. Если классы
        эквивалентности отождествить с точками на $[0,1)$, то образами
        интервалов $(a,b)$ будет $(\{a\},\{b\})$, если изначальный интервал не
        содержал целых точек, $[0,\{b\})\cup(\{a\},1)$, если изначальный
        интервал содержал 1 целую точку и $[0,1)$, если изначальный интервал
        содержал 2 и более целые точки. Пусть $f:[x]\mapsto e^{i2\pi\{x\}}$
        биекция из $\mathbb{R}/\mathbb{Z}$ в $S^1$. Тогда очевидно, что она
        однозначно сопоставляет элементам базы топологии на фактор пространства,
        что мы получили открытые дуги, а значит пространства гомеоморфны.

    \item \textbf{Докажите, что $\mathbb{D}^n/S^{n−1}\simeq S^n$.}\par
        Пусть $I=(-1,1)$ интервал. Тогда положим $B^n=I^n$, $\mathbb{D}^n=
        \overline{B^n}$ и $S^n=\partial\mathbb{D}^{n+1}$. Заметим, что ещё
        $S^n=\{(x_1,...,x_n)\in\mathbb{R}^n|\max_i|x_i|=1\}.
\end{enumerate}

\end{document}
