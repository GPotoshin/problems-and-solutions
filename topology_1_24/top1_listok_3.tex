\documentclass{article}
\usepackage[a4paper,left=3cm,right=3cm,top=1cm,bottom=2cm]{geometry}
\usepackage{amsmath}
\usepackage{amssymb}
\usepackage{hyperref}

\setlength{\parindent}{0mm}

\usepackage{fontspec}
\setmainfont{Linux Libertine O}
\usepackage{unicode-math}
\setmathfont{Cambria Math}

\title{
\textit{\small{Георгий Потошин, 2024}}\\
\vspace{0.3ex}
\textit{\huge{Топология I, листочек 3}}\vspace{1ex}
}

\date{\vspace{-10ex}}

\begin{document}
\maketitle

\begin{enumerate}
    \item \textbf{Докажите, что $\mathbb{R}/\mathbb{Z}\simeq S^1.$}\par
        \textbf{Утверждение 1. Элементы базы топологии на $X$ после
        индуцирования на $Y\subseteq X$ образуют базу топологии на $Y$.}\par
        По определению элемент базы останется открытым после индуцирования.
        Покажем теперь, что все индуцированные элементы базы составят
        базу. Пусть $U\subseteq Y$ – открытое множество. Тогда существует такое
        открытое $V\subseteq X$, что $V\cap Y=U$. Раз $V$ открыто, то
        существуют элементы базы $B_i\in \tau_X,i\in I$, что $\bigcup_{i\in I}
        B_i = V$. Тогда $U=V\cap Y=Y\cap\bigcup_{i\in I}B_i=\bigcup_{i\in I}Y\cap
        B_i$ открытое множество представимо как объединение индуцированных
        элементов базы на топологии $X$, а значит, что множество всех таких
        индуцированных элементов составят базу топологии на $Y$.\par

        \textbf{Утверждение 2. Если в топологии пространства $X/\sim$ образ
        элемента базы топологии на $X$ при канонической проекции открыт, то
        объединение этих образов составит базу топологии на фактор пространства.}
        \par Пусть $U\in X/\sim$ открыто, тогда $\pi_{\sim}^{-1}[U]$ открыто и
        представимо как $\bigcup_{i\in I}B_i$ где $B_i$ – элемент базы топологии
        на $X$. Тогда $U=\pi_{\sim}[\bigcup_{i\in I}B_i]=\bigcup_{i\in I}\pi_\sim
        [B_i]$. А значит образ элементов базы топологии на $X$ составит базу
        топологии на фактор пространстве.\par

        \textbf{Утверждение 3. Если биекция $X\longrightarrow Y$ переводит
        элементы базы в открытые множества и прообразами элементов базы тоже
        являются открытые множества, то биекция является гомеоморфизмом.}\par
        
        Пусть $U\subseteq X$ открыто, тогда существуют такие элементы базы
        $B_i\subseteq X,i\in I$, что $U=\bigcup_{i\in I}B_i$. Тогда $f^{-1}[U]=
        \bigcup_{i\in I}f^{-1}[B_i]$ – объединение открытых, а значит само
        открыто и $f$ непрерывно. В обратную сторону доказывается также.

        Базой пространства $S^1$ являются всевозможные пересечения окружности
        и открытых кругов, то есть открытые дуги. Найдем теперь базу
        пространства $\mathbb{R}/\mathbb{Z}$. Пусть $(a,b)$ – элемент базы
        топологии на $\mathbb{R}$. Прообраз образа этого интервала равен
        $\bigcup_{n\in\mathbb{Z}}(a+n,b+n)$ и открыт, а значит образы
        интервалов составят базу топологии на фактор пространстве. Если классы
        эквивалентности отождествить с точками на $[0,1)$, то образом
        интервалa $(a,b)$ будет $(\{a\},\{b\})$, если изначальный интервал не
        содержал целых точек, $[0,\{b\})\cup(\{a\},1)$, если изначальный
        интервал содержал 1 целую точку и $[0,1)$, если изначальный интервал
        содержал 2 и более целые точки. Пусть $f:[x]\mapsto e^{i2\pi\{x\}}$
        биекция из $\mathbb{R}/\mathbb{Z}$ в $S^1$. Тогда очевидно, что она
        однозначно сопоставляет элементам базы топологии на фактор пространстве
        открытые дуги, а значит пространства гомеоморфны.

    \item \textbf{Докажите, что $\mathbb{D}^n/S^{n−1}\simeq S^n$.}\par

        Пусть $I=(-1,1)$ интервал. Тогда положим $B^n=I^n$, $\mathbb{D}^n=
        \overline{B^n}$ и $S^n=\partial\mathbb{D}^{n+1}$. Заметим, что ещё
        $S^n=\{(x_1,...,x_n)\in\mathbb{R}^n|\max_i|x_i|=1\}$, тогда в силу того,
        что максимум из конечного набора чисел всегда выбирается, то
        $\mathbb{D}^n=\bigcup_{r\in[0,1]}rS^{n-1}$. $\mathbb{D}^n/S^{n-1}$ –
        это диск в котором все точки его границы положили в один класс.
        Построим отображения из диска в шар, что уважает это отождествление.
        Пусть $x=(x_1,...,x_n)$ и пусть $|x|=max_i|x_i|$, тогда
        \[f(x)=
        \begin{cases}
            (-1,4x_1,...,4x_n) &,0\leqslant|x|<1/4\\
            (4|x|-2,x_0/|x|,...,x_n/|x|) &,1/4\leqslant|x|\leqslant 3/4\\
            (1,4(1-|x|)\frac{4}{3}x_0,4(1-|x|)\frac{4}{3}x_n) &,3/4<|x|<1\\
            (1,0,...,0) &x=\partial\mathbb{D}^n\\
        \end{cases}\]
        Обратным ему будет сопоставлять каждому $y=(y_0,...,y_n)$
        \[f^{-1}(y)=
        \begin{cases}
            \partial\mathbb{D}^n &,x=(1,0,...,0)\\
            (x_1/4,...,x_n/4),
        \end{cases}\]


        Отображение f непрерывно, так как непрерывна каждая композиция $pr_i
        \circ f$. Отображение f построено так, что оно делит шар на сферы. Для
        $r\in[0,1/4)$ сферы этих радиусов по возрастающи устилают основания,
        затем для $r\in[1/4,3/4]$ устилают боковые грани, а для $r\in(3/4,1]$
        устилают верхнее основание, причем окружность при приближении к
        границе диска стягивается в точку, а сама граница переходит в середину
        верхней грани. По этому это отображение после факторизации диска
        становится биекцией. До факторизации открытыми множествами диска были
        всевозможные пересечения диска с открытыми объемлющего пространства,
        после факторизации, если открытое не содержало точек границы, то оно
        так и  останется открытым, так как его прообраз он сам. Если некое
        открытое множество профакторезованного диска содержит класс границы,
        то его прообара...

    \item \textbf{Верно ли, что фактор хаусдорфова пространства является
        хаусдорфовым? Регулярного – регулярным? Нормального – нормальным?}\par
        Возьмём отрезок $[0,1]$ с канонической топологией. Он компактен и
        хаусдорфов, а значит нормален и регулярен. Профакторизуем его так, что
        его внутренность попадёт в один класс эквивалентности, $0$ в другой,
        а $1$ в третий обозначим их за $i,0,1$ соответственно. Тогда из всех
        подмножеств только $\varnothing, \{i\},\{0, i\},\{1,i\},\{0,1,i\}$
        будут открытыми. Заметим, что  $\{0\}$ и $\{1\}$ будут замкнутыми в
        такой топологии, но при этом у этих синглтонов нет непересекающихся
        окрестностей, а значит, что полученное фактор пространства ни
        хаусдорфово, ни регулярно, ни нормально. Тогда ответ на все вопросы – нет.

    \item \textbf{Приведите пример хаусдорфова нерегулярного топологического
        пространства.}\par
        Положим $K=\{1/n|n\in\mathbb{N}\}$. Это множество не открыто в
        стандартной топологии прямой $\mathbb{R}$, так как любая окрестность
        $1$ не лежит в $K$. C другой стороны оно не замкнуто, так как не
        содержит предельную точку $0$. Возьмём множество $S$ всех
        интервалов вместе со всеми интервалами без $K$. Оно покрывает прямую и
        пересечение двух элементов либо интервал, либо интервал без $K$, а
        значит $S$ – база некой топологии, в которой открытые множества - это
        канонические открытые множества без некого подмножества в $K$. Это
        означает, что любая окрестность $0$ содержит отрезок без некого
        количества элементов из $K$. Тогда между границами этого отрезка лежит
        некое число вида $1/p$ и любая окрестность $K$ будет содержать шар
        радиусом меньшим $1/p-1/(p-1)$ вокруг $1/p$ и $1/p$ содержащий. Тогда
        этот шар пересекается с $K$ только по своему центру, а значит это
        шар в привычном нам смысле. Тогда он пересекается с изначальной
        окрестностью $0$. В итоге у $K$ и $0$ нет непересекающихся окрестностей.

    \item \textbf{Приведите пример регулярного ненормального топологического
        пространства.}

    \item \textbf{Приведите пример связного, но не линейно связного
        топологического пространства.}
    Обозначим за $L_n$ отрезок между $(0,0)$ и $(1,1/n)$ в $\mathbb{R}^n$. Он
    связен и открыт.\par

    \textbf{Утверждение 4. Если $C_\alpha\subseteq X$ – связные пространства
    для всяких индексов и $\bigcap_\alpha C_\alpha\neq\varnothing$, то
    $\bigcup_\alpha C_\alpha$ связно.}\par

    Пусть $\bigcap_\alpha C_\alpha\neq\varnothing$, но при этом $\bigcup_\alpha
    C_\alpha=U\sqcup V$, где $U$ и $V$ дизъюнктивные открыты непустые множества.
    Если бы ни одно из $C_\alpha$ не одержало одновременно элементы этих двух
    открытых множеств, то тоже было бы справедливым относительно их непустого
    пересечения и тогда все $C_\alpha$ были бы подмножествами одного из
    открытых, а значит второе открытое множество оказалось бы пустым, что
    противоречит с нашим предположением. Пусть $C_{\alpha_0}$ содержит элементы
    из обоих множеств. Тогда $C_{\alpha_0}=(U\cap C_{\alpha_0})\sqcup (U\cap C_
    {\alpha_0})$ – несвязно, а значит мы вновь пришли к противоречию. Тогда
    $\bigcap_\alpha C_\alpha$ обязано быть связным.

    В нашем случае множества $B=\bigcup_{n=1}^{+\infty}L_n$ и $\overline{B}=B
    \cup([0,1]\times\{0\})$ в силу этого утверждения связны, так как их связные
    части-отрезки пересекаются по $(0,0)$.

    \textbf{Утверждение 5. Если множества $C$ и $\overline{C}$ связны, то и
    всякое лежащее между ними тоже связно.}

    Пусть $C$ и $\overline{C}$ связны и $C\subset X\subset\overline{C}$. Если
    бы $X=U\sqcup V$ было несвязно, то если бы оба имели элементы из $C$, то
    $C=(U\cap C)\sqcup (V\cap C)$ было бы несвязно, что ведёт к противоречию.
    Иначе одно из открытых, пусть без потери общности им будет $V$, полностью
    бы находилось в $\overline{C}\backslash C$. Тогда $\overline{C}\backslash V$
    было бы замкнутым в объемлющем пространстве и содержало бы $C$, а значит
    замыкания не было бы минимальным по включению замкнутым надмножеством $C$,
    что опять ведет к противоречию. В итоге $X$ обязано быть связным.


    \item \textbf{Определите естественную топологию на пространства
        невырожденных матриц $\text{GL}(n,\mathbb{R})$. Является ли оно связным?}
        
    \item \textbf{Докажите, что функции расстояния $d_1,d_2,d_\infty$ задают
        структуру метрического пространства на $\mathbb{R}^n$. Нарисуйте открытые
        шары $B_0^1$ в метриках $d_i$ при $n=2$.}

    \item \textbf{Докажите, что топология на $\mathbb{R}^n$, индуцированная
        метриками $d_i$ и выше, совпадает с топологией произведения,
        определённой на лекции.}\par
        Обозначим за $\tau_\infty$ топологию порожденную метрикой $d_\infty$
        и за $\tau_\times$ топологию произведения. Базой $\tau_\infty$ являются
        многомерные кубы, то есть множества вида $r(-1,1)^n+a$, где $r\in
        \mathbb{R}$ и $a\in\mathbb{R}^n$. Базой топологии произведения
        являются всевозможные произведения интервалов. Заметим, что база
        метрической топологии вкладывается в базу топологии произведение, а
        значит $\tau_\infty\subseteq\tau_\times$. Пусть теперь $U=\prod_{i=1}^n
        (a_i,b_i)$ – элемент базы $\tau_\times$. Тогда каждая его точка $x=(x_1
        ,...,x_n)\in U$ лежит вместе с шаром $\min\{|a_i-x_i|i\in\{1,...,n\}\}
        \cap\{|b_i-x_i|i\in\{1,...,n\}\}(-1,1)^n+x$, а значит база топологии
        произведения является семейством открытых множеств из $\tau_\infty$.
        Это означает, что $\tau_\times\subseteq\tau_\infty$, и учитывая прошлое
        утверждение $\tau_\times=\tau_\infty$.\par

        Теперь пусть $S_i=\{x\in\mathbb{R}^n|d_i(x,0)=1\}$. $x\mapsto d_i(x,0)$ -
        это непрерывное отображение в смысле $(\mathbb{R}^n,\tau_\times)
        \longrightarrow(\mathbb{R},\tau_c)$, где $\tau_c$ – каноническая
        топология прямой, так как $d_i(\cdot,0)$ является i-м корнем из суммы
        непрерывный отображений. Тогда исходя из 2 задачи 2 листочка множество 
        $S_i$ замкнуто, так как $(\mathbb{R},\tau_c)$ – хаусдорфово.
        Нетрудно также видеть, что $S_i\subset[-1,1]^n$, подмножество
        произведения компактных по лемме Бореля – Лебега отрезков, что само
        компактно. Тогда $S_i$ – замкнутое подмножество компакта, а значит
        $S_i$ компактно в топологии $\tau_\times$. Очевидно, что $d_\infty(
        \cdot, 0):(\mathbb{R}^n,\tau_\times)\longrightarrow(\mathbb{R},\tau_c)$
        тоже является непрерывным отображением. Тогда $d_\infty(S_i,0)$ – образ
        сферы при непрерывном отображении тоже компактен. Более того, так как
        каноническая топология прямой хаусдорфова, то компактный образ сферы
        замкнут, а значит содержит все свои предельные точки. Теперь так, как
        функция расстояния имеет неотрицательные значения и сфера не содержит
        нуль векторного пространства, то она и не может содержать сколь угодно
        близкие к нулю с точки зрения $d_\infty$ точки, в силу замкнутости
        образа. Это означает, что образ имеет ненулевую нижнюю грань $m>0$,
        то есть минимальное расстояния от нуля до некоторой точки сферы. Тогда
        имеет место следующее соотношения для шаров $B_i(a,r)$ метрики $d_i$.
        $B_\infty(a,rm)\subseteq B_i(a,r)\subseteq B_\infty(a,r)$ для любых
        точек $a$ и радиусов $r$. Это значит, что в любой шар пространства с
        метрикой $d_i$ можно вписать куб и вокруг него же можно описать куб,
        а значит открытые множества одного пространства открыты и в другом.
        Тогда $\tau_i=\tau_\infty=\tau_\times$, что и завершает доказательство.

    \item \textbf{Пусть $X,Y$ – метрические пространства. Определите
        естественную метрику на их произведении $X×Y$.}

    \item \textbf{Предположим, что в метрическом пространстве X выполнено
        $B^{\varepsilon_1}_x=B^{\varepsilon_2}_y$ для некоторых точек $x,y$ и
        некоторых $\varepsilon_1,\varepsilon_2>0$. Верно ли, что $x=y,
        \varepsilon_1=\varepsilon_2$?}

    \item \textbf{Определим топологию Зариского на $\mathbb{C}^n$ следующим
        образом: замкнутыми множествами назовем множества нулей произвольного
        набора многочленов из $\mathbb{C}[x_1,...,x_n]$. Проверьте, что это
        действительно топология. Является ли она хаусдорфовой? Совпадает ли
        топология Зариского на $\mathbb{C}^2$ с топологией произведения,
        полученной из топологии Зариского на $C$?}
\end{enumerate}

\end{document}
