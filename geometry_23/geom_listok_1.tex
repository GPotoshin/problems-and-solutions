\documentclass{article}
\usepackage[a4paper,left=3cm,right=3cm,top=1cm,bottom=2cm]{geometry}
\usepackage{amsmath}
\usepackage{amssymb}
\usepackage{hyperref}

\setlength{\parindent}{0mm}

\usepackage{fontspec}
\setmainfont{Linux Libertine O}
\usepackage{unicode-math}
\setmathfont{Cambria Math}

\title{
\textit{\small{Георгий Потошин, 2023}}\\
\vspace{0.3ex}
\textit{\huge{Геметрия, листочек 1}}\vspace{1ex}
}

\date{\vspace{-10ex}}

\begin{document}
\maketitle

\begin{enumerate}
    \item \textbf{Докажите, что формула $d(x,y)= max_{i=1,...,n} |x_i−y_i|$,
        где $x=(x_1,...,x_n),y=(y_1,...,yn)\in\mathbb{R}^n$, задаёт метрику в
        $\mathbb{R}^n$. Нарисуйте примеры открытых и замкнутых шаров радиуса
        $\varepsilon$ с центром в $(0,0)$ на $\mathbb{R}^2$ с этой метрикой.}\par
        ($i$) Покажем, что $\forall x,y\in\mathbb{R}^n$ $d(x,y)\geqslant 0$. Мы
        знаем, что $\forall x,y\in\mathbb{R}^n$ $\exists l\in$
\end{enumerate}

\end{document}
