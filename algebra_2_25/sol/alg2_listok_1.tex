\documentclass{article}
\usepackage[a4paper,left=3cm,right=3cm,top=1cm,bottom=2cm]{geometry}
\usepackage{amsmath}
\usepackage{amssymb}
\usepackage{hyperref}
\usepackage[russian]{babel}

\usepackage{tikz-cd}
\usepackage{array}
\usepackage{graphicx}
\newcommand\mapsfrom{\mathrel{\reflectbox{\ensuremath{\mapsto}}}}
\setlength{\parindent}{0mm}

\usepackage{fontspec}
\setmainfont{Linux Libertine O}
\usepackage{unicode-math}
\setmathfont{Cambria Math}

\newcommand{\mymat}{\mathcal{M\mkern-3mu a\mkern-0.3mu t}}


\title{
\textit{\small{Георгий Потошин, 2025}}\\
\vspace{0.3ex}
\textit{\huge{Алгебра II, листочек 1}}\vspace{1ex}
}

\date{\vspace{-10ex}}

\begin{document}
\maketitle

\begin{enumerate}
    \item \textbf{Пусть $f(x),g(x)\in K[x]$. Пусть $h(x)\in\overline K[x]$ – НОД
        многочленов $f(x),g(x)$, рассмотренных как многочлены над
        алгебраическим замыканием $\overline K$. Докажите, что $h(x)\in K[x]$.}

        Пусть $\widetilde h(x)\in K[x]$ – НОД многочленов $f(x),g(x)$ в $K[x]$.
        Тогда $\widetilde h(x)\,|\,f(x),g(x)$, тогда $\widetilde h(x)\,|\,h(x)$ в
        $\overline K[x]$ по свойству НОДа. C другой стороны есть соотношение
        Безу в $K[x]$, а именно мы найдём $u(x)$ и $v(x)$ в $K[x]$, что $\widetilde
        h(x)=u(x)f(x)+v(x)g(x)$, и так как $h(x)\,|\, f(x),g(x)$ в $\overline K$,
        то $h(x)\,|\,\widetilde h(x)$ в $\overline K[x]$. У нас есть делимость в
        обе стороны в $\overline K[x]$ и так как оба многочлена приведены, то
        они совпадают и $h(x)=\widetilde h(x)\in K[x]$.

    \item \textbf{Докажите, что если расширение $L/K$ сепарабельно и чисто
        несепаребельно, то $L = K$.}

        Так как $L/K$ сепарабельно, то каждый элемент $\alpha\in L$ имеет сепарабельный
        неприводимый минимальный многочлен $\text{Irr}_\alpha^K(x)$. С другой
        стороны $\alpha^{p^n}\in K$ для $p=\text{char}\,K>0$ и для $n\geq 0$.
        Возьмём такое наименьшее $n$. Тогда $x^{p^n}-\alpha^{p^n}\in(\text{Irr}
        _\alpha^K(x))_{K[x]}$, а значит $\text{Irr}^K_\alpha(x)\,|\,x^{p^n}-
        \alpha^{p^n}=(x-\alpha)^{p^n}$. Это означает, что $\text{Irr}^K_\alpha(x)
        =(x-\alpha)^m$, но так как это многочлен неприводим и сепарабелен, то
        $m=1$. А значит $x-\alpha\in K[x]$ и $\alpha\in K$, откуда получаем, что
        $L=K$.

    \item \textbf{Докажите, что любое конечное поле совершенно, то есть любое
        его алгебраическое расширение сепарабельно.}
    
        Пусть $K$ – конечное поле характеристики $p$, а $\overline K$ его
        алгебраическое замыкание. Пусть $\alpha\in\overline K$, тогда $K(\alpha)$
        конечное поле порядка $q$. Тогда полином $x^q-x$ очевидно зануляется на
        всех элементах $K(\alpha)$ и раскладывается в произведение различных
        мономов вида $x-a$, где $a\in K(\alpha)$ и коих ровно $q$ штук. Тогда
        этот полином не имеет кратных корней и зануляет $\alpha$, а тогда сепарабелен
        $\alpha$. Это верно для всех элементов $\overline K$, а значит $
        \overline K/K$ сепарабельно и $K$ идеально.

    \item \textbf{Докажите, что если расширение $L/K$ нормально, то расширение
        $L^\text{sep}/K$ нормально.}

        Пусть $K<L<L^\text{sep}<\overline L=\overline K$ – башня полей. Оба
        замыкания совпадают, так как $K<L$ в частности алгебраично. Пусть $
        \sigma:L^\text{sep}\rightarrow\overline K=\overline L$ гомоморфизм над
        $K$. Тогда по нормальности $K<L$, $\sigma[L]=L$. Тогда для $\alpha\in
        L^\text{sep}$ будет неприводимый сепарабельный многочлен $f(x)$. Тогда
        $f(x)=(x-\alpha_1)\ldots(x-\alpha_n)$ не имеет кратных корней, тогда
        по инъективности $\sigma$, $f^\sigma(x)=(x-\alpha_1^\sigma)\ldots(x-
        \alpha_n^\sigma)$ тоже и так как $f^\sigma(x)\in L[x]$ и $f^\sigma(\alpha
        ^\sigma)=0$, то $\alpha^\sigma$ сепарабелен над $L$ и $\sigma[L^\text{sep}]
        \subseteq L^\text{sep}$.

        Для включения в обратную сторону заметим, что $f^{\sigma^{-1}}(x)\in L[x]$,
        так как $\sigma[L]=L$ и он однозначно определен, так как $\sigma$ инъективен.
        Тогда мы знаем, что существуют $u,v\in L[x]$, что $fu+f'v=1$. Тогда верно
        и $f^{\sigma^{-1}} u^{\sigma^{-1}}+(f^{\sigma^{-1}})'v^{\sigma^{-1}}=1$,
        а значит $f^{\sigma^{-1}}$ не имеет кратных корней в $\overline L$, а
        тогда $f^{\sigma^{-1}}=(x-\alpha_1)\ldots(x-\alpha_n)$, где очевидно
        $\alpha_i\in L^\text{sep}$, так как они корни многочлена без кратных
        корней. Тогда по предыдущему наблюдению $\alpha_i^\sigma\in L^\text{sep}$
        тоже. Но так как $\alpha$ один из корней $f(x)$, то $\alpha=\alpha_i^\sigma$
        для какого-то сигма, а тогда $L\subseteq \sigma[L]$, а значит
        $L^\text{sep}=\sigma[L^\text{sep}]$ и $K<L^\text{sep}$ нормально. 

    \item \textbf{Докажите, что расширение $\mathbb{F}_p(x, y)/\mathbb{F}_p(x^p,
        y^p)$ чисто несепарабельно; проверьте, что $[F_p(x,y) : F_p(x^p,y^p)] =
        p^2$; убедитесь, что существует бесконечное количество промежуточных
        полей $K$ таких, что
        \[\mathbb{F}_p(x^p, y^p)<K<\mathbb{F}_p(x,y)\]
        }

        Пусть $Q\in\mathbb{F}_p(x,y)$, тогда очевидно, что $Q^p\in\mathbb{F}_p(x^p,y^p)$,
        так как гомоморфизм фробениуса и каждое слагаемое будет возведено в
        степень $p$.

        Заметим, что в башне $\mathbb{F}_p(x^p,y^p)<\mathbb{F}_p(x,y^p)<
        \mathbb{F}_p(x,y)$ первый этаж является расширением по многочлену
        $t^p-x^p=(t-x)^p$, у которого единственный корень $x$ и очевидно, что
        $(t-x)^i\notin\mathbb{F}_p(x^p,y^p)[t]$ для $0<i<p$, так как свободным
        коэффициентом будет $x^i\notin\mathbb{F}_p(x^p,y^p)$. Тогда $t^p-x^p$
        неприводим и $\mathbb{F}_p(x,y^p)$ – расширение по $t^p-x^p$ над $
        \mathbb{F}_p(x^p,y^p)$. И его степень расширения – $p$. Аналогично
        получим, что степень расширения второго этажа также $p$. Тогда степень
        $[F_p(x,y) : F_p(x^p,y^p)]=p^2$ равна произведению степеней.

        Заметим, что для любого $\alpha\in\mathbb{F}_p(x,y)\setminus\mathbb{F}_p
        (x^p,y^p)$. Мы можем аналогично построить расширение по неприводимому
        многочлену $t^p-\alpha^p$, будем называть такое расширение $K_\alpha$.
        Теперь осталось сделать правильный выбор таких $\alpha$. Положим
        $\alpha_i=x^{ip+1}+y$. Пусть для краткости $K_{\alpha_i}=K_i$. Если
        $K_i=K_j$ для разных $i$ и $j$, то
        \[(x^{ip+1}+y)-x^{jp+1}+y=x^{ip+1}-x^{jp+1}\in K_i\]
        тогда $x(x^{ip}-x^{jp})\in K_i$, и так как $0\neq x^{ip}-x^{jp}\in K_i$,
        то $x\in K_i$. Но тогда $y\in K_i$, a значит $K_i=\mathbb{F}_p(x,y)$,
        чего не может быть, так как тогда расширение будет степени $p^2$, а оно
        степени $p$.

    \item \textbf{(Теорема о примитивном элементе) Пусть $K$ – бесконечное поле,
        и $K(\alpha, \beta)/K$ – сепарабельное расширение, причем $[K(\alpha,
        \beta) : K] = n$ и $\text{Aut}(K(\alpha,\beta)/K) = G$. Докажите, что}
    \begin{enumerate}
        \item \textbf{существует элемент $c\in K$ такой, что $|G(\alpha+c\beta)|
            = n$, то есть $G$-орбита $G(\alpha+c\beta)$ элемента $\alpha+c\beta$
            содержит ровно $n$ элементов}

            Как мне кажется в этом задании есть ошибка, так как вообще не факт,
            что в группе $G$ найдется $n$ различных элементов, так как каждый
            автоморфизм переставляет корни минимальных многочленов элементов
            $\alpha$ и $\beta$ и этой перестановкой определен. Но у нас могут
            быть не все корни, и тогда элементов не хватит на $n$ перестановок.
            Например есть расширение $\mathbb{Q}(\sqrt[3]{2},\sqrt[3]{2})/
            \mathbb{Q}$. Поэтому нужно заменить автоморфизмы на вложения в поле
            разложения $\text{Irr}_\alpha^K(x)$ и $\text{Irr}_\beta^K(x)$,
            назовём это поле $F$.

            Так как расшерение сепарабельно, то существует ровно $n$ вложений.
            Вообще вложения обычно рассматриваются в алгебраическое замыкание,
            но так как корни можно отправить только в корни того-же многочлена,
            то достаточно рассмотреть поле разложения. Если бы расширение было
            к тому же нормальным, то вложения были бы автоморфизмами, как в
            задаче и спрашивается. Назовем эти вложения $\sigma_i$
            для $1\leq i\leq n$. Теперь пусть $c\in K$ такое, что $\sigma_i(\alpha)
            +c\sigma_i(\beta)=\sigma_j(\alpha)+c\sigma_j(\beta)$ для $i\neq j$.
            Тогда $c=(\sigma_i(\alpha)-\sigma_j(\alpha))/(\sigma_j(\beta)-\sigma_i(\beta))$.
            Выкинем все такие элементы, коих не больше $n$. Тогда возьмём
            какой-нибудь оставшийся ненулевой. Он всегда будет, так как поле $K$
            бесконечно. Тогда для такого $c$, вложения $\sigma_i$ дадут нам
            $n$ различных образов элемента $\alpha+c\beta$.

        \item \textbf{если $|G(\alpha+c\beta)|=n$, то $[K(\alpha+c\beta):K]_{sep}\geq n$}

            Так как мы получили $n$ различных образов $\alpha+c\beta$, то у
            $\text{Irr}_{\alpha+c\beta}^K(x)$ есть как минимум $n$ корней и они
            различны. Тогда степень расширения равна степени полином, которая
            больше или равна $n$, в сепрабельном случае степень расширения
            совпадает с сепарабельной степенью.

        \item \textbf{если $|G(\alpha+c\beta)|=n$, то $K(\alpha+c\beta)=K(\alpha,\beta)$}

            Но $K<K(\alpha+c\beta)\leq K(\alpha,\beta)$, а значит $[K(\alpha+c
            \beta):K]\leq n$, но тогда там равенство и по мультипликативности
            степени будет $[K(\alpha+c\beta):K(\alpha,\beta)]=1$, то есть
            $K(\alpha+c\beta)=K(\alpha,\beta)$.

        \item \textbf{если $L/K$ конечно и сепарабельно, то $L=K(\alpha)$.}

            Как обычно построим башню.
            \[ K<K(\alpha_1)<\ldots<K(\alpha_1,\ldots,\alpha_m)=L \]
            Пусть гипотезой индукции будет $K(\alpha_1,\ldots,\alpha_i)=K(\alpha)$
            для некоторого $\alpha$. Тогда для $i=1$ она очевидно верна. Пусть
            она верна для $i=k$, тогда $K(\alpha_1,\ldots,\alpha_k,\alpha_{k+1})=
            K(\alpha_1,\ldots,\alpha_k)(\alpha_{k+1})=K(\alpha, \alpha_{k+1})$.
            Тогда по предыдущему пункту мы получим $K(\alpha, \alpha_{k+1})=
            K(\alpha')$. По индукции это будет верно и для $L$.
    \end{enumerate}
\end{enumerate}

\end{document}
